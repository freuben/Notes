\chapter*{Abstract}
\addcontentsline{toc}{chapter}{Abstract}

This thesis comprises a portfolio of creative musical work and a written commentary. The creative work seeks to rework musical strategies through technology by challenging aspects of how music is traditionally performed, composed and presented. 

The portfolio of submitted work is divided into five main projects. The first project is \emph{E-tudes}, a set of four compositions for live electronics and six keyboard players. The second project is a composition called \emph{On Violence}, for piano, live electronics, sensors and computer display. The third project is \emph{\v{Z}i\v{z}ek!?}, a computer-mediated-performance for three improvisers that serves as an alternative soundtrack to a documentary about Slovenian philosopher Slavoj \v{Z}i\v{z}ek. The fourth project is a collection of small experimental pieces for fixed media called \emph{FreuPinta}. The fifth project consists of a selection of different improvisations that I devised or participated in using a computer environment I developed for live improvisation. 

Throughout the portfolio recent technological advancements are considered not for their use in implementing pre-existing models of music-making but rather for their potential to challenge preconceived notions about music. The written commentary gives the theoretical tools necessary to understand the underlying reasoning, preoccupations and concerns behind the submitted work as well as providing supplementary information about the musical results and the computer programmes developed as part of this research.