\chapter{Preface}

\pagenumbering{arabic}

Little intro...

For this reason, that I believe \emph{radical music} today can not be defined by neither the anti-mimetic notion of \emph{modernism} nor its reaction that now---under the \emph{postmodern} banner---challenges the emancipatory potential of music, its autonomy and its capacity to accomplish. 

It is then of utmost importance that \emph{radical music} is redefined again positively, given the inability that the two main aesthetic proposals in music have to inspire radical thought. Today we live in a world facing unprecedented problems of global magnitude:

%CAREFULL! THIS SOUNDS LIKE MODERNIST VISION OF LINKING HIsTORY TO A PARTICULAR ARTISTIC MOVEMENT and POLICs, eTC... I believe this to be a critical step given that today we live in a crucial historical moment that I claim needs radical change on many levels. 

Another obstacle in redefining \emph{radical music} has been the recent trend to find new terms which apply to practices relating to sound that defy the conventional definitions and functions of music. New terms such as Sonic Arts, Sound Art and Audio Arts have emerged in an attempt to justify these new practices. It has been precisely the cultural resistance and unwillingness toward accepting \emph{radical music} that has motivated the invention of new definitions that try to identify these sonic practices as `other' arts and not as music. The reluctance to widening the definition of \emph{what music is} has motivated some to search for new definitions that they believe will give some acceptance and legitimacy to their practices. Instead of embracing this approach, I propose one should struggle more with the concept of \emph{music}, and in my opinion, one should strive to redefine \emph{what music is} rather than following the fashionable rebranding of sound related practices. 

%This is important, I think because. . . .
\begin{quote}
One problem I have had in my own musical career is the rejection by some musicians and musicologists of my work on the grounds that `it is not music.' To avoid getting into semantic quibbles, I have therefore entitled this book \emph{On Sonic Art} and wish to answer the question what is, and what is not, `sonic art.' We can begin by saying that sonic art includes music and electro-acoustic music. At the same time, however, it will cross over into areas which have been categorized distinctly as \emph{text-sound} and as \emph{sound-effects}. Nevertheless, focus will be upon the structure and structuring of sounds themselves. I personally feel there is no longer any way to draw a clear distinction between these areas. This is why I have chosen the title \emph{On Sonic Art} to encompass the arts of organizing sound-events in time. This, however, is merely a convenient fiction for those who cannot bear to see the use of the word `music' extended. For me, all these areas fall within the category I call `music'.\footnote{Trevor Wishart, ``What is Sonic Art?'', in \emph{On Sonic Art}, Amsterdam: Harwood Academic Publishers, 1996, p. 4.}
\end{quote}

Is it possible today to rethink a musical \mbox{avant-garde} that can inspire new forms of political thought? Can a connection be established at present time between music and other forms of radical thought?

In the first two chapters, I will attempt to tackle different concerns regarding a single question that I consider to be central to my approach in recent years to the way I compose, perform, listen and think about music. The question being: what is radical music today? The idea of \emph{radical music} has fundamentally changed in recent years and today it is hard to think of any music as being radical. This is partly due to the fact that for some years now the prevailing ideology in thinking and writing about music has been one of skepticism and indifference towards radical ideas and innovation in how we make, present and perceive music. By radical ideas and innovation, I do not mean music that is technologically ground breaking or innovative only in specific considerations to a particular set of musical parameters or new ideas that are relevant only to the specialist's theoretical interest. What I mean is music that is perceived as radical within our contemporary culture and redefines what \emph{music is} in the community, what it means to us, how it is perceived and defined.

It is in my attempt to rethink what \emph{radical music} is today, that I believe Jaques Ranci\`{e}re's work proves to be helpful, particularly in establishing a relationship between \emph{music} and \emph{radical thought}, and more specifically \emph{radical thought} as it relates to politics. Therefore, in \hyperlink{chapter2}{Chapter 2}, I will give theoretical background based on Ranci\`{e}re's views on the relationship between aesthetics and politics. Additionally, I undertake the task of trying to apply his concepts as they pertain to music. 

In \hyperlink{chapter3}{Chapter 3}---motivation

In \hyperlink{chapter4}{Chapter 4}---musical strategies and practices

In \hyperlink{chapter5}{Chapter 5}---computer applications

In \hyperlink{chapter6}{Chapter 6}---compositions: Etudes, On Violence, Zizek, FreuPintaconclusions

In \hyperlink{chapter7}{Chapter 7}---conclusions

I would like to thank:

Parents 

Eszter Takacs

Family: specially bro

Supervisors: Richard Barrett and Christopher Fox

Aleksander Kolkowski

Nicholas Peters

Computer: SuperCollider community, David Plans Casal, Gabriel Montagn\'e

Impro: Javier Carmona, Dominic Lash, Alexander Hawkins

Performers: Sarah Nicols, Piano Circus for their time and effort devoted

Commentary: Philip Somervell, Adam de la Cour

Conversations: Steve Potter, Ed McKeon, Mauricio Pauly, Assaf Gidron

The Hague Colleagues: Jasna Velickovic, Ronald Boersen, Dganit Elyakim, Ophir Ilzetzki

\label{ch:preface}
