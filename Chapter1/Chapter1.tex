\chapter{Preface}

\pagenumbering{arabic}

Writing about music can be an overwhelming and complex endeavor. Music can be examined from many angles and by many disciplines given that music carries within itself a complex content of meaning and that within the act of music-making lies in essence a collective experience in which complex human relationships are formed. Music can be analyzed as a starting point by the complex---social, cultural, political, economic and historical---conditions in which it is created and received. However, these conditions at the same time are revealed within music and therefore it is possible to examine music by studying how its wider context may be traced through what happens within music itself. Additionally, music can also be studied as a vital human act that is deep-rooted in our evolutionary past and which performs important functions that are closely related to human behavior and emotions. Consequently, music has also been the subject of extensive research within the scientific community for its impact on human experience: how it affects the human body and the brain.\footnote{In the last fifteen years there has been a notable increase of interest surrounding music within neuropsychology, cognitive and computational sciences for the remarkable reaction that the human brain undergoes when it experiences music.} At the same time, music has also gained status as an autonomous discipline---music can be studied independently as a subject that has developed its own music theory based on technical and aesthetic considerations regarding how music is conceived, performed and received from a musician's perspective. Therefore, music can be assessed according to different criteria: music can be evaluated for the (social, economic, evolutionary, etc.) functions it performs, for the way in which it follows certain considerations and models that are exclusive to music, and for its potential to inspire new forms of thought in other areas of human endeavor based on the reconfiguration and redefinition of its own considerations and models. The last principle by which music can be evaluated only clearly emerged and gained visibility during the last century and in my opinion is more suitable to be considered through philosophical inquiry and critical reflection. As a result, attempting to write about music while acknowledging and embracing music's large content of meaning is therefore a humbling task given the complexity of meanings that are revealed within music, the diverse criteria by which music can be evaluated and the variety of disciplines that it might entail.

Writing about ones own music. Problematic...

, one can take two approaches when writing about music. One can take a reductionist approach where one focuses only on a limited area of enquiry through rigorous empirical investigation with the risk of ignoring music's larger context of meaning or one might proceed through an associative, holistic, multidisciplinary and subjective approach to critical enquiry, which while recognizing and attempting to explore the interconnections between music's multiple meanings, it might loose accuracy, methodological rigor and empirical validity.


The last is associative . . .

and considering music's associative and subjective nature,

 only emerged more explicitly in the last century

---the criteria for evaluation is independent from the direct (social, economic, etc.) functions it performs), and for its potential to inspire new forms of thought in other areas of human activity (based on the reconfiguration and redefinition of its own considerations and models). The last form of appreciation only emerged more explicitly in the last century

Subjective, associative, political.




Writing about one's own music might be even riskier.

Writing as self-reflection... on ones own work.

Little intro... 

Critical Theory - approach, etc.

This commentary takes as a starting point that music can not simply be defined as the resulting sounds of a musical performance. Music is much more than that... it includes the relationships, social act, etc... "Music" as a definition that can change (it already has historically)...

Multidisciplinary approach...

On methodology... Meta-commentary, etc. The nature of the submitted work calls for meta-commentary...

Chapters are independent but refer to each other.

In \hyperlink{chapter2}{Chapter 2}, I will give theoretical background based on Ranci\`{e}re's views on the relationship between aesthetics and politics. Additionally, I undertake the task of trying to apply his concepts as they pertain to music. 

In \hyperlink{chapter3}{Chapter 3}---motivation

In \hyperlink{chapter4}{Chapter 4}---technology and musical strategies 

In \hyperlink{chapter5}{Chapter 5}---appropriation

In \hyperlink{chapter6}{Chapter 6}---computer applications

In \hyperlink{chapter7}{Chapter 7}---compositions

I would like to thank:

Family: Parents, bro, extended family 

Eszter Takacs

Supervisors: Richard Barrett and Christopher Fox

Performers: 

Impro: Javier Carmona, Dominic Lash, Alexander Hawkins

Composers - playing: Aleksander Kolkowski, Nicholas Peters, Adam de la Cour, Neil Luck, 

Sarah Nicols, Piano Circus for their time and effort devoted

Commentary and Advice: Steve Potter, Ed McKeon, Philip Somervell

Colleagues and Conversations throughout the years:  Assaf Gidron, Jasna Velickovic, Ronald Boersen, Dganit Elyakim, Ophir Ilzetzki, Mauricio Pauly, 

Hague Teachers: Louis Andriessen, Richard Ayres, Gilius van Bergeijk, Martijn Padding

Computer: SuperCollider community, David Plans Casal, Gabriel Montagn\'e

\label{ch:preface}