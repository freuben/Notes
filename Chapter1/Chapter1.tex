\chapter{Preface}

\pagenumbering{arabic}

%Give a few sentences in general introduction to why you have undertaken the research that you have � mention your reasons for embarking, your reasons for continuing and your hopes for how successful you have been.
%Thank your supervisor. Think about it, he or she has been your greatest support throughout the course of your thesis and you probably would not even have a thesis if it were not for them: the least they deserve is a �thank you� in black and white!
%Thank your family as without their support the journey would have been tougher.
%Thank your typist, if you used one, and/or those who helped and advised you along the way.

There are various reasons why I consider writing about music a difficult and overwhelming challenge. First, I acknowledge that music can be examined from many different angles and by many disciplines given that music carries within itself a complex content of meaning. Furthermore, I accept that music can be analyzed by the elaborate---social, cultural, political, economic and historical---conditions in which it is created and received. These conditions at the same time are revealed within music and therefore music can also be studied by looking at how its wider context may be traced through what happens within music itself. At the same time, I recognize that within the act of music-making lies in essence a collective experience in which intricate human relationships are formed. Therefore, music can be studied as a vital human act that is deep-rooted in our evolutionary past and which performs important functions that are closely related to human behavior and emotions. I am convinced that it is for this reason that music has drawn a lot of interest from the scientific community that has led to extensive research about music's impact on human experience: how it affects the human body and the brain.\footnote{In the last fifteen years there has been a notable increase of interest surrounding music within neuropsychology, cognitive and computational sciences for the remarkable reaction that the human brain undergoes when it experiences music. For a comprehensive survey of these research see \hyperlink{musmind}{Levitin (2006)}.}  Moreover, music is broadly considered as an autonomous discipline---music is studied independently as a subject that has developed its own music theory based on technical and aesthetic considerations (regarding how music is conceived, performed and received) from a musician's perspective. Therefore, music can be examined according to very different criteria: music can be studied for the (social, economic, evolutionary, etc.) functions it performs as well as for the way in which it follows certain considerations and models that are exclusive to music. Moreover, I believe that during the last century it has become apparent that music can also be studied for its potential to inspire new forms of thought in other areas of human endeavor based on the reconfiguration and redefinition of its own considerations and models. I believe this criteria for examining music can be more effectively scrutinized through philosophical inquiry and critical reflection. In my view, writing about music can be a daunting task considering the diverse criteria by which music can be examined, the variety of disciplines that can be involved in studying music and the complexity of meanings that might be revealed within music.

This commentary involves not only writing about music but writing about ones own music, which inevitably leads to an even greater challenge. Having experienced the creative process that resulted in the submitted musical output, I am almost too aware of its highly associative, multilayered, holistic and subjective nature.  Moreover, as a musician that is absorbed in music practice, I am also extremely aware that music has a multiplicity of meanings---not only the definition of music is contingent to social, historical and philosophical conditions, but also music as it is perceived and experienced conveys multiple meanings---a self-evident fact that for me is too difficult to ignore. For that reason, I will not attempt to embrace a reductionist approach (which would focus only on a limited area of musical enquiry through rigorous empirical investigation) to writing this commentary, to avoid the risk of ignoring the intuitive character of my own creative process and the complex nature and meaning of music. Instead, I decided to write in a style that combines self-reflective, speculative and multidisciplinary arguments with specialized and technical information regarding to how music is created and performed. I consider this approach to writing better suited to describe the type of critical reflection that goes on as part of my creative process, which deals with problems that at times can be concerned only with the technical aspects of how music is created and performed at times attempts to situate the creative work and the musical result within a wider---cultural, political, philosophical, etc.---context. At the same time, I also recognize that while attempting to explore the interconnections between music's multiple meanings, I am running the risk that my arguments might loose accuracy, methodological rigor and empirical validity and that my claims at times may sound overambitious or as sweeping statements. However, I am convinced that this style of writing represents at best my aesthetic and musical concerns and that considering the difficulty that lies in writing about ones own creative work . . .

Writing as self-reflection... on ones own work.

On methodology... Meta-commentary, etc. The nature of the submitted work calls for meta-commentary...

Chapters are independent but refer to each other.

First 4 chaps are meta-commentary

First 2 chaps are introductory

In \hyperlink{chapter2}{Chapter 2}, I will give theoretical background based on Ranci\`{e}re's views on the relationship between aesthetics and politics. Additionally, I undertake the task of trying to apply his concepts as they pertain to music. 

In \hyperlink{chapter3}{Chapter 3}---motivation

In \hyperlink{chapter4}{Chapter 4}---technology and musical strategies 

In \hyperlink{chapter5}{Chapter 5}---appropriation

In \hyperlink{chapter6}{Chapter 6}---computer applications

In \hyperlink{chapter7}{Chapter 7}---compositions

I would like to thank:

Family: Parents, bro, extended family 

Eszter Takacs

Supervisors: Richard Barrett and Christopher Fox

Performers: 

Impro: Javier Carmona, Dominic Lash, Alexander Hawkins

Composers - playing: Aleksander Kolkowski, Nicholas Peters, Adam de la Cour, Neil Luck, 

Sarah Nicols, Piano Circus for their time and effort devoted

Commentary and Advice: Steve Potter, Ed McKeon, Philip Somervell

Colleagues and Conversations throughout the years:  Assaf Gidron, Jasna Velickovic, Ronald Boersen, Dganit Elyakim, Ophir Ilzetzki, Mauricio Pauly, 

Hague Teachers: Louis Andriessen, Richard Ayres, Gilius van Bergeijk, Martijn Padding

Computer: SuperCollider community, David Plans Casal, Gabriel Montagn\'e

\label{ch:preface}