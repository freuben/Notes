\chapter{Preface}

\pagenumbering{arabic}


A little introduction to my work before PhD. Concerns at the begining of the PhD an how they evolved in time. Explanation of the way I'm writing the commentary and why. etc. etc.

In the first chapters, I will attempt to tackle different concerns regarding a single question that I consider to be central to my approach in recent years to the way I compose, perform, listen and think about music. This question being: \emph{what is radical music today?} The idea of \emph{radical music} has fundamentally changed in recent years and today it is hard to think of any music as being radical. This is partly due to the fact that for some years now the prevailing ideology in thinking and writing about music has been one of skepticism and indifference towards radical ideas and innovation in how we make, present and perceive music. By radical ideas and innovation, I do not mean music that is technologically ground breaking or innovative only in specific considerations to a particular set of musical parameters or new ideas that are relevant only to a specialist's theoretical interest. What I mean is music that is perceived as radical in our contemporary culture and redefines what \emph{music is} to the community, what it means to us, how it is perceived and defined.
Another obstacle in redefining \emph{radical music} has been the recent trend to find new terminology for practices that relate to sound that defy conventional definitions and functions of music. New terms such as Sonic Arts, Sound Art, Audio Arts, etc., have emerged in an attempt to justify these new practices. It has been precisely the cultural resistance and unwillingness toward accepting \emph{radical music} that has motivated the invention of new definitions that try to identify these sonic practices as `other' arts and not as music. The reluctance to widening the definition of \emph{what music is} has motivated some to search for new definitions that they believe will give some acceptance and legitimacy to their practices. Instead of embracing this approach, I prefer to struggle a bit more with the concept of music and I am of the opinion that one should strive to redefine \emph{what music is} rather than following the recent trend to find new names for recent practices relating to sound. This is important, I think because. . . .

-Division of Music and Music Tech/Sonic Arts, etc.
-Technological Innovation is not equal to Radical or Innovation in Music!

\label{ch:intro}
