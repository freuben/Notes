\chapter{Preface}

\pagenumbering{arabic}

Little intro... 

This commentary takes as a starting point that music can not simply be defined as the resulting sounds of a musical performance. Music is much more than that... it includes the relationships, social act, etc... "Music" as a definition that can change (it already has historically)...

Multidisciplinary approach...

On methodology... Meta-commentary, etc.

In the first two chapters, I will attempt to tackle different concerns regarding a single question that I consider to be central to my approach in recent years to the way I compose, perform, listen and think about music. The question being: what is radical music today? The idea of \emph{radical music} has fundamentally changed in recent years and today it is hard to think of any music as being radical. This is partly due to the fact that for some years now the prevailing ideology in thinking and writing about music has been one of skepticism and indifference towards radical ideas and innovation in how we make, present and perceive music. By radical ideas and innovation, I do not mean music that is technologically ground breaking or innovative only in specific considerations to a particular set of musical parameters or new ideas that are relevant only to the specialist's theoretical interest. What I mean is music that is perceived as radical within our contemporary culture and redefines what \emph{music is} in the community, what it means to us, how it is perceived and defined.

It is in my attempt to rethink what \emph{radical music} is today, that I believe Jaques Ranci\`{e}re's work proves to be helpful, particularly in establishing a relationship between \emph{music} and \emph{radical thought}, and more specifically \emph{radical thought} as it relates to politics. Therefore, in \hyperlink{chapter2}{Chapter 2}, I will give theoretical background based on Ranci\`{e}re's views on the relationship between aesthetics and politics. Additionally, I undertake the task of trying to apply his concepts as they pertain to music. 

In \hyperlink{chapter3}{Chapter 3}---motivation

In \hyperlink{chapter4}{Chapter 4}---technology and musical strategies 

In \hyperlink{chapter5}{Chapter 5}---computer applications

In \hyperlink{chapter6}{Chapter 6}---appropriation

In \hyperlink{chapter7}{Chapter 7}---compositions

I would like to thank:

Parents 

Eszter Takacs

Family: specially bro

Supervisors: Richard Barrett and Christopher Fox

Aleksander Kolkowski

Nicholas Peters

Computer: SuperCollider community, David Plans Casal, Gabriel Montagn\'e

Impro: Javier Carmona, Dominic Lash, Alexander Hawkins

Performers: Sarah Nicols, Piano Circus for their time and effort devoted

Commentary: Philip Somervell, Adam de la Cour

Conversations: Steve Potter, Ed McKeon, Mauricio Pauly, Assaf Gidron

The Hague Colleagues: Jasna Velickovic, Ronald Boersen, Dganit Elyakim, Ophir Ilzetzki

\label{ch:preface}
