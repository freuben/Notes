\hypertarget{chapter2}{}
\chapter{Motivation}

The present chapter formulates a theoretical framework in which concepts described in \hyperlink{chapter1}{Chapter 1} are elaborated further in an attempt to establish a discourse that clarifies the motivations behind the submitted creative work. Taking into consideration the philosophical and historical background previously elaborated, I will therefore undertake the difficult task of proposing a new attitude towards music creation that at once takes into consideration the shortcomings ascribed to the notion of \emph{modernism} and simultaneously acknowledges the importance of the original vision of the avant-garde. The musical stance I propose, also recognizes the misguided intentions of the modernist anti-mimetic position and the consequences it brought to musical discourse---a criticism now credited to the first generation of artists that became associated with the label of \emph{postmodernism}. At the same time, I acknowledge that so called \emph{postmodern music} has recently started to signify an artistic approach which encourages false notions of plurality and open-mindedness and---by aimlessly questioning notions of progress and universality in music---promotes a deceiving impression that nothing new can be achieved through musical creativity. I will contend this position first, by introducing Ranci\`{e}re's idiosyncratic notion of the avant-garde and by pointing to the relationship that exists between music and other forms of subjectivity. I will therefore explain Ranci\`{e}re's concepts of the \emph{strategic} and \emph{aesthetic} types of avant-garde with the purpose of suggesting that the confusion between these two kinds of avant-gardes is what has led to the ideas behind the development of the notions of modernism and postmodernism in music. Moreover, I will attempt to apply Ranci\`{e}re's concepts regarding the types of avant-garde to music with the purpose of clarifying misunderstandings regarding the relationship between politics and music. Additionally,  by looking at the ethical functions of music and the role they take in basic human endeavors, I will propose that there is an implicit ethical core in the definition of music. I will further argue that the shared purpose that music and language have---which is to convey emotion and meaning through sound---makes music a vital human act that is deep-rooted in our evolutionary past. This points to the understanding that music conveys knowledge, thoughts and feelings that are not exclusive to music, but relate to other forms of human action and experience. Moreover, the ethical functions attributed to music also compromise the attempts at expanding the definition of \emph{what music is} that is characteristic of the \emph{aesthetic regime of art}. As a consequence of the relationship that exists between new types of music and new forms of human experience, innovation in music is often seen skeptically by most people if it ceases to perform its ethical functions. Nevertheless, I will propose that there is an ethical function \emph{in itself} in music that lies within the \emph{aesthetic regime}, that is: to inspire new sensible forms that relate to other aspects of human activity. I will therefore argue that an important role of the \emph{musical avant-garde}\footnote{From now on I will use italics to refer to the \emph{musical avant-garde} as group of individuals dealing with music that have a joint \emph{strategic} and/or \emph{aesthetic} purpose, in contrast with Ranci\`{e}re's idiosyncratic notion of the avant-garde which refers rather to two different ideas that constitute two kinds of `avant gardes'. I will explain Ranci\`{e}re's notion of the avant-garde in more detail \hyperlink{rethinkavant}{later} in this chapter.} today is to reestablish an agreement of trust with a wider range of contemporary society by demonstrating through music that the purpose of new musical forms, concepts and definitions is to inspire new ideas, opinions, desires and emotions and not to undermine the ethical function music already performs. 

Finally, I will argue that if an agreement of trust is to be regained between the \emph{musical \mbox{avant-garde}} and a wider range of society, it is important to recognize the `basic' \emph{ethical} functions that music performs at the same time as acknowledging the significance of music that lies within the \emph{aesthetic regime}. I will therefore propose that this can be achieved by reworking musical strategies from the past to create new \emph{aesthetic} forms. That is to say, after critically examining and reevaluating musical strategies from past traditions, we should consider altering and modifying the fundamental aspects by which they function with the purpose of creating new musical forms and structures as well as challenging conventional notions and definitions of music.

\hypertarget{eventdis}{}
\section{Redefining the Musical Subject?}

I will start explaining the basic motivations surrounding the submitted musical output by considering a position I believe to be prevalent today between people concerned with music. This dominant position is characterized by a skeptical and often cynical attitude towards new forms of thought in music. However, this attitude is dominant not without a reason: it has to do with the notion that today music is---as Alain Badiou has stated---`negatively defined'. Badiou expresses this view in his essay entitled `Scholium: A Musical Variant of the Metaphysics of the Subject'.
\begin{quote}
Today, the music-world is negatively defined. The classical subject and its romantic avatars are entirely saturated, and it is not the plurality of `musics'---folklore, classicism, pop, exoticism, jazz and baroque reaction in the same festive bag---which will be able to resuscitate them. But the serial subject is equally unpromising, and has been for at least twenty years. Today's  musician, delivered over to the solitude of the interval---where the old coherent world of tonality together with the hard dodecaphonic world that produced its truth are scattered into unorganized bodies and vain ceremonies---can only heroically repeat, in his very works: `I go on, in order to think and push to their paradoxical radiance the reasons that I would have for not going on'.\footnote{\hyperlink{badioumus}{Badiou (2009)}, p. 89.} 
\end{quote}
Here, Badiou precisely delineates the situation in which music is created and received today, where the only two main options for the \emph{musical avant-garde} seem to embrace either the joyful and permissive attitude towards mixing genres and styles now commonly ascribed to \emph{postmodern music} or the desolate notion of modernism that for over thirty years, has heroically stood in `life support'. These options also seem unable at the present time to inspire profound change in the way in which we create, perform, perceive and think about music. In order for the music-world to be positively redefined, I believe it needs to discover a new musical subject---a set of musical notions, theories and subjectivities that will constitute a new `system of believes' in music. Only by following a new musical subject, will the \emph{musical avant-garde} regain confidence in itself and produce \emph{events}\footnote{Here, I am referring to an \emph{event} as conceptualized by  Badiou's philosophy of the event. In very simple terms, an \emph{event} as described by Badiou consists of a perfunctory force that happens as a combination of chance and a structural fragility in a situation that enables its temporary manifestation. It may take place for example as a political revolution, two people falling in love, or an artistic invention, which might trigger subsequent fidelity amongst people who understand the implications of such an \emph{event}. According to Badiou, an \emph{event} reveals the inconsistency of being within a situation and staying faithful and true to an \emph{event} can motivate and generate actual change to the situation prior to the \emph{event}. An \emph{event} therefore provides the possibility of rethinking and redistributing the elements of a situation considering the new truth revealed by the \emph{event}. See \hyperlink{hallward}{Hallward (2008)}.} that will constitute far-reaching change in the way we make and experience music. To become a true alternative from the two other options that currently dominate the music-world, I also believe that this new musical subject needs to reintroduce as its core principles notions of emancipation, logic, universality and risk.\footnote{This argument stems from the philosophical position Badiou puts forward in his article `Philosophy and Desire'. See \hyperlink{infthought}{Badiou (2006)}, pp. 30--35.}.

The skepticism regarding innovation in relationship to music creation today is related not only to the perceived notion of failure associated with the aesthetics of modernism in music, but to the argument put forward by the so called \emph{postmodern} position, which questions the idea that it is possible to achieve something new through music. However, the problem with the position associated with postmodernism is that it reduces music only to a mediation of already existing musical styles and forms and to a multiplicity of musical `games' that aimlessly \emph{mix} and \emph{remix} past notions of music and musical thought.\footnote{This position is also inspired by Badiou. See Ibid. pp. 35--42.} This concept of music also ceases to respond to the original premise of the modernist vision of the avant-garde, which establishes a connection between new forms of music and new types of subjectivity. I think Ranci\`{e}re's analysis gives us strong theoretical tools to imagine an alternative which would involve reinvigorating the modernist idea of the avant-garde in music without falling back to the misunderstandings that led to the `crisis of modernity'. Nevertheless, Ranci\`{e}re's notion of the \emph{avant-garde} is considerably different from the conventional one, and in order to understand his definition and relate it to music, it is important to separate it from its former association to a particular movement in music history. Even though the idea of the avant-garde in music emerged as it became associated to a group of `modernist' composers, the concept remains useful to us now only as a way of understanding the importance of the \emph{aesthetic regime} in the relationship between music and other types of subjectivity and forms of thought. Additionally, Ranci\`{e}re's idiosyncratic notion of the avant-garde is also at the center of his attempt to establish a link between aesthetics and politics. 

\hypertarget{rethinkavant}{}
\section{Rethinking the Avant-garde}

Ranci\`{e}re has persuasively argued that if there is a connection to be established between the aesthetic and the political, it is suggested by the original modernist vision of the avant-garde.  The basis for this association is not the connection between artistic innovation and politically motivated change, but the suggestion of a link between two different kinds of `avant-gardes'. The first kind being characterized by an abstract and militant notion of a movement that symbolizes a force that chooses a historical direction and ideological position---the embodiment of a type of subjectivity (political or artistic) to a specific form (a party or an artistic movement). The second kind of avant-garde is rooted in Schiller's model of \emph{aesthetics} as a projection of the future. The meaning of the avant-garde in the \emph{aesthetic regime of art} is therefore not that of artistic innovation as seen by a particular movement that links artistic subjectivity to a determinate form, but the idea of ``the invention of sensible forms and material structures for a life to come''.\footnote{\hyperlink{ranpoli}{Ranci\`{e}re (2004)}, p. 29.} This is where the aesthetic avant-garde may inform, inspire and encourage the political avant-garde and bring about transformations in the anticipation of the future. Moreover, Ranci\`{e}re makes a very interesting theoretical observation when he draws a parallel between these two kinds of avant-garde and two forms of political philosophy:
\begin{quote}
The history of the relations between political parties and aesthetic movements is first of all the history of confusion, sometimes complacently maintained, at other times violently denounced, between these two ideas of the avant-garde, which are in fact two different ideas of political subjectivity: the archi-political idea of a party, that is to say the idea of a form of political intelligence that sums up the essential conditions for change, and the meta-political idea of global political subjectivity, the idea of the potentiality inherent in the innovative sensible modes of experience that anticipate a community to come.\footnote{Ibid., p. 30.}
\end{quote}

The ideas that have led to the notions of modernism and postmodernism in the arts---as well as to the `crisis of art' as ascribed by many---have therefore developed as a consequence of the confusion caused by a division which exists between the \emph{strategic} and \emph{aesthetic} conceptions of the avant-garde as manifested in art. This division of the avant-garde is also to be found within the political sphere and is in fact considered as two different forms of political philosophy, which not only clarifies the presence of aesthetics in politics, but also the inherent politicity within the artistic disciplines.\footnote{See \hyperlink{ranaesth}{Ranci\`{e}re (2009), \emph{Aesthetics and Its Discontents}}, `Aesthetics as Politics', pp. 19--44, for a further discussion about the relationship between the `aesthetics of politics' and the `politics or aesthetics'.}

Here, I would like to attempt to explain how these to kinds of `avant-gardes' can be found in music with the purpose of conceptualizing not only the differences between the two, but also to point at how one might relate to the other. I think that the link between the two avant-gardes might also help to understand the importance collectivity and relationships may have on the \emph{aesthetic} result in music. Moreover, the distinction between the two types of avant-garde can also be useful clearing certain confusions that might arise when thinking about the relationship between music and politics.  

\subsection{\emph{Strategic} and \emph{Aesthetic} Types of Avant-garde in Music}

The \emph{strategic} type of avant-garde as manifested in music is an idea that serves as a driving force that leads particular individuals involved in music (composers, performers, critics and other people who make, think and/or listen to music) to consolidate themselves as a group (musical institution, movement, ensemble, etc.). It is important to remember that this idea holds a core set of values that have a common ideological position which sums up a type of subjectivity which triggers the conception of this group.\footnote{Slovoj \v{Z}i\v{z}ek has repeatedly emphasized how ideology is not an abstract notion or theory one simply ascribes to, but a type of subjectivity that is reflected in the way we act, on how we behave and carry ourselves on a day-to-day basis. Therefore, a musical `movement' doesn't necessarily have to be one in which there is a `conscious' or openly declared agenda that follows a particular position of objectified consensus. See \hyperlink{zizekreader}{\v{Z}i\v{z}ek (2006), \emph{The \v{Z}i\v{z}ek Reader}}, `The Spectacle of Ideology', for \v{Z}i\v{z}ek's own examination of the concept of \emph{ideology}.} On the other hand,  the \emph{aesthetic} type of avant-garde as manifested in music is an idea that---through new ways of thinking and making music as expressed by the creation of new musical forms and structures---has the capacity to inspire and encourage new forms of thought about the life to come.  Furthermore, it is crucial that the \emph{strategic} type of avant-garde is not confused with the \emph{aesthetic} type in as much as it will lead to further misunderstandings within the music-world. 

It is important to notice that we can find these two kinds of ideas or `types of avant-gardes' both in the musical and political spheres (as well as in the other artistic disciplines). Additionally, as they manifest themselves in music, the \emph{aesthetic} and \emph{strategic} types of avant-garde are intrinsically related; but only in as much as music is concerned.  This relationship becomes evident in the causality that exists between musical groups, institutions and movements; and the creation and reception of music that lies within the \emph{aesthetic regime}. In other words, the ideas that prompt the formation of a group relate to the idea of the aesthetic type of avant-garde only in as much as they may contribute in the consolidation of the conditions of exchange necessary for the creation of new forms of subjectivity for the future. The \emph{strategic} avant-garde as manifested in music is therefore useful to the political sphere only as much as it contributes to the \emph{aesthetic} avant-garde---specifically as it provides a platform for the creation of `new sensible forms and structures'. Hence, the way in which the two types of avant-gardes dwell within music can not be directly compared to the way in which they reside in politics. Here lies another vital point we can induce from Ranci\`{e}re's enquiry: the \emph{strategic} type of avant-garde manifests itself \emph{differently} in music as it does in politics. In other words, the ideas that give rise to the establishment of groups in music and politics are different from each other and do not reflect a relationship between the two disciplines. From this, we can conclude for example that the activism of a musician or group of musicians as they become directly involved in politics does not reflect a relationship between music and politics, but only the involvement of a group of people---which happen to have the same occupation---in a political movement. The true relationship between music and politics is rather reflected in the \emph{aesthetic} type of avant-garde. This argument makes evident why it is misleading to attempt to identify a movement with concerns that are specific to music with a particular political affiliation or party. The position put forward by some critics of modernism in music---which concludes that the emancipatory project which seeks the autonomy of music leads to totalitarianism---is therefore flawed.  

Moreover, I will claim that it is very important to consider the intrinsic relationship between the two types of avant-gardes, exclusively as they manifest themselves within music. The basis of this way of thinking stems from the assertion that the \emph{strategic} type of avant-garde has a considerable effect on the \emph{aesthetic} type precisely because the subjective force that brings a group together influences the aesthetic forms it produces. Furthermore, the \emph{strategic} type of avant-garde is responsible for the creation of specific forms of collectivity which might serve as a platform for the creation of new \emph{aesthetic} forms in music. Put briefly, the relationship between the two types of avant-garde stems from the fact that the \emph{strategic} type of avant-garde motivates the creation of the groups necessary for the development of the \emph{aesthetic} type of avant-garde. This argument evidently assumes that the impact musical movements, institutions, ensembles and other organized groups of musicians and people dealing with music, have on the actual musical results, is significant. This assumption however is often underrated by people involved in creating (particularly composers in my experience) and experiencing music who avoid or forget how these forms of collectivity condition and influence the aesthetic result. Contrary to this position, I will go as far as to suggest that in music, the type of subjectivity that is synthesized in these groups is reflected or `embodied' in the musical result. In other words, the ideology of the people involved in the creation, presentation and dissemination of music is expressed in the musical modes of action, production, perception and thought. Furthermore, the notion that the composer is the only person whose ideology is reflected in the music and that the \emph{musical work}\footnote{See \hyperlink{goer}{Goehr (2007)}, for a thorough discussion on the philosophy of musical works.} is the only carrier of meaning---an idea that up to this moment is still widespread in western culture---is also misleading. Thus, to avoid misunderstandings I will introduce the notion of a \emph{musical result} (as opposed to the more limited concept of \emph{musical work}) as that which describes the complex set of percepts given by all aspects of a musical experience. These include for example: all sorts of aural and visual elements in music performance; the space and time in which music is performed; the way in which music is presented to the audience (including their role and participation in the musical experience); different modes of action in performance (performance practice) and composition (act of composing); the relationships established between composer, performer and audience; the context (cultural, sociological, political) in which music is presented; the way music is created, consumed and distributed; etc. A particular kind of \emph{musical result} consequently discloses a type of collective subjectivity which encompasses the ideology of the people involved in the music.\footnote{I am not implying however that the ideology of \emph{all} the people is represented \emph{equally} in the \emph{musical result}. The question of how much an individual is represented widely depends on the role they take within the \emph{musical result} and the audience's interpretation of it.} Additionally, within the \emph{musical result} lies a system of elaborate symbols that synthesizes the relationships between the people involved in the collective act of music-making.

\subsubsection{\emph{Musicking}}
\hypertarget{musicking}{}

According to Christopher Small, the set of complex relationships that are formed between people involved in music is that which gives meaning to music. His interest lies particularly on the collective action surrounding music and defines this activity as \emph{musicking}. 
\begin{quote}
The act of \emph{musicking} establishes in the place where it is happening a set of relationships, and it is in those relationships that the meaning of the act lies. They are to be found not only between those organized sounds which are conventionally thought of as being the stuff of musical meaning but also between the people who are taking part . . . relashionships between person and person, between individual and society, between humanity and the natural world.\footnote{\hyperlink{small}{Small (1998)}, p. 13.}
\end{quote}
By giving priority to the verb \emph{to music}, as opposed to the noun \emph{music}, he also questions the notion of the \emph{musical work} and gives emphasis to the human action of \emph{musicking}. Small argues that music is not an object and that \emph{musical works} only give material for the musicians to perform, in contrast to the notion (developed as a consequence of western concert music) of performance only as a presentation of a \emph{musical work}. He also defines the verb \emph{to music} to include any type of action that contributes to a musical performance, which includes performing, listening, practicing, composing and dancing. He goes as far as to include actions such as selling and collecting tickets and cleaning the concert hall after a performance within his notion of \emph{musicking}. Therefore, \emph{musicking} encompasses all social relationships and actions that are related to music-making. Furthermore, he argues that \emph{musicking}, together with speaking, are characteristics that are at the very core of what makes us human.
 \begin{quote}
I am  certain, first, that to take part in a music act is of central importance to our very humanness, as important as taking part in the act of speech, which it so resembles (but from which it also differs in important ways), and second, that everyone, every normally endowed human being, is born with the gift of music no less than with the gift of speech.\footnote{Ibid., p. 8.}
\end{quote} 
Recent scientific studies in a variety of specialities including neuroscience, psychology, archaeology, anthropology and cognitive musicology have also pointed towards the same hypothesis. The idea put forward by Steven Pinker that music is `auditory cheesecake'---that it is only a byproduct of evolution and has no biological value for humans\footnote{See \hyperlink{pinker}{Pinker (1998)}, pp. 528--538.}---has been challenged recently within the scientific community.  These studies have shown how music plays an important role, amongst other things, in human communication, social bonding, cooperation, sexual selection, conveying emotions, phycological \mbox{well-being}, development of coordination and motor skills, expression of empathy, communication between infants and parents and exercising intelligence.\footnote{See \hyperlink{mithen}{Mithen (2006)}, for an overview of these studies.} In addition, various theories have emerged regarding the relationship between music and language; some of them even suggesting that \mbox{`proto-language'}\footnote{Mithen prefers the term `Hmmmmm' over `proto-language' as it better describes the system of communication of Early Humans, which he claims was---\underline{h}olistic, \underline{m}ulti-\underline{m}odal, \underline{m}anipulative, \underline{m}usical and \underline{m}imetic. See Ibid., p. 172.} (the predecessor of language) was a pre-linguistic, non-verbal form of communication that was a `musical' form of action and thought.\footnote{Ibid., pp. 147--150.} It appears that language and music have a similar evolutionary \mbox{starting-point} and the common purpose of communicating emotion and meaning through sound. From this research we can infer that Small is correct in suggesting that \emph{musicking}, like speaking, is at the core of being human and performs important social, cultural and biological functions. 

\hypertarget{musdef}{}
\section{The Definition of Music and the \emph{Ethical Regime}}

The important functions music performs in the development of individuals and the way in which they establish and nurture relationships within a community is what defines music as a vital human act. Perhaps this is the reason why in the musical domain---going back to Ranci\`{e}re's notion of `the regimes of art'\footnote{See \hyperlink{artregimes}{pp. 4--8}.}---music is still defined as such within the \emph{ethical regime}. In other words, if we go back to the question of why within music there is no change of identification with the break between the \emph{ethical} and \emph{poetic} regimes; I will suggest that it is because there is a strong ethical core implicit in the very meaning of \emph{what music is}. That is to say, as opposed to the definition of the other arts, the definition of music has been tied to the ethical functions that it performs for individuals and their communities. It is worth mentioning that only dance, like music, can also be defined as such within the \emph{ethical regime}, which points towards the deep-rooted relationship between both disciplines. On the contrary, other artistic disciplines including `fine' art, poetry and theater are identified as such only with the break between the \emph{ethical} and \emph{poetic} regimes.   

The ability that human beings have to communicate and perceive emotion and meaning through music is also tied to music's identification and to the ethical functions it performs. It is by no coincidence that already in Ancient Greece, Aristotle observed that music has an immense power to change people's state of character and that different types of music affect audiences in different ways. According to Aristotle, music represents various types of emotions and actions that closely resemble those that the listener undergoes in reality as a result of the performance.\footnote{See \hyperlink{aristotle}{Aristotle (1995)}, `The Aims and Methods of Education in Music', pp. 309--310.} It is as a consequence of this link between music and human experience, emotion and action that communities have attempted to regulate and evaluate music according to the ethical functions it performs. One could consequently argue that music that lies within the \emph{ethical regime} is evaluated for its ability to affect people in a way that is considered appropriate by the community, given a particular situation. This argument also points towards one of the reasons why labeling music as different `styles' or `genres' seems to be a dominant practice within communities: by knowing what kind of music to expect from a specific `style', it is possible to anticipate the type of experience the audience will go through. This is also one of the reasons why innovation in music has been discouraged and even censured by communities for centuries. The modification of musical styles within the perspective of the \emph{ethical regime} implies an unexpected change in our experience and a potential threat to the community's consensus of what is considered to be the appropriate way in which people are to be affected by music. Furthermore, innovation in music has been perceived as a political threat in the past since new forms of music produce new experiences that might stimulate behavior outside the political order. 

Plato, in his \emph{Republic} already warns about the danger that innovation in music might pose to the order of the State:
\begin{quote}
Put briefly, then, those charged with care of the city must hold fast to this, so that the city may not be corrupted unawares; but beyond all else, they must guard against innovation in gymnastic and music contrary to the established order, and to the best of their ability be on guard lest when someone says that people care more ``for the newest song on the singer's lips'', the poet may be understand to mean not new songs but a new style of singing, and to comment it. One must not praise such a thing, nor so interpret the poet, but guard against changing to a new form of music, as endangering the whole. For styles of music are nowhere disturbed without disturbing the most important laws and customs of political order---as Damos says and I believe.\footnote{\hyperlink{plato}{Plato (2006)}, `Music and the Constitution', p. 117.}
\end{quote}
Therefore, the Platonic view regarding innovation in music is that it is threatening to the social agreements and political organization of the State. Even though the idea that innovation in music might endanger the political and social contracts of the community today might seem hard to imagine, it still gives us a clue towards an attitude that up to this day is still widespread, that is: that innovation in music regarding its own rules, hierarchies, subject matter and genres is still received with reservation, suspicion and even fear amongst the community (if compared to the visual arts for example). In my opinion, this is due in the most part for to two main reasons. First, considering the implication that music performs certain ethical functions, innovation can be seen with skepticism as it could lead to confusion, uncertainty and even irritation, if music ceases to perform the functions expected by the community successfully or does so less efficiently. Secondly, given the immersive and participatory (either by listening or performing) aspects implied by the definition of music that establishes a link between music and human action and experience, innovation in music can be associated with new and unpredictable experiences and behavior. Therefore, it is not surprising that some people would be distrustful in allowing themselves experience something they are not familiar with or are uncertain about.\footnote{On a related note: according to recent research in cognitive science, most people stop acquiring new musical tastes by the time they are around twenty years old. This might be as a result that as people grow older, they seem less open to new experiences. See \hyperlink{musmind}{Levitin (2006)}, `My Favorite Things', pp. 231--233.}

\section{An Ethical Function within the \emph{Aesthetic Regime}?}

Going back to Ranci\`{e}re's notion of the regimes of art, if we consider the ethical core implicit in the definition of music simultaneously with music that falls within the \emph{aesthetic regime}, we might run into a deadlock: if music is to be evaluated \emph{only} by the functions it already performs within the community (and innovation in music is seen as a disruption of these functions), music that lies within the \emph{aesthetic regime} appears as having no apparent noble purpose. To resolve this problem we need to point towards the relationship that exists between music and other forms of human endeavor. If music is evaluated and appreciated for its capacity to inspire new ideas, opinions, beliefs and desires, then one can argue that there is an ethical position implicit in music that falls within the \emph{aesthetic regime}. In other words, their is an ethical function \emph{in itself} in breaking with previous models of music making and in questioning the very notion of \emph{what music is}. This function is precisely that of imagining and experiencing through music, new forms of action, production, perception and thought.

The ethical function within the \emph{aesthetic regime} in music is also related to the notion of `autonomous' music having a \emph{secondary social function}. Adorno, in his \emph{Introduction to the Sociology of Music} claims that autonomy in music---music that is independent from a primary or immediate social function\footnote{Music that has a `primary social function' would be close to music that lies within the \emph{ethical regime}, using Ranci\`{e}re's terminology.}---has another type of social function.
\begin{quote}
In a society that has been functionalized virtually through and through, totally ruled by the exchange principle, lack of function comes to be a secondary function. In the function of functionlessness, truth and ideology entwine. What results from it is the autonomy of the work of art itself: in the context of social effects, the man-made in-itself of a work that will not sell out to that context promises something that would exist without defacement by the universal profit.\footnote{\hyperlink{adornointro}{Adorno (1976)}, p. 41.}
\end{quote}
Put briefly, functionless music has a social function in itself. According to Adorno, the social function of autonomous music (and \emph{art} in general) lies in its own fetish character, which sets music `against its bourgeois functionalization, which is perpetuated in art's undialectical social condemnation.'\footnote{\hyperlink{adornoaesth}{Adorno (1997)}, p. 227.} In other words, autonomous music may serve as subjective social criticism by opposing functionalization. Even though there is a clear relationship between Adorno's position and the one I am attempting to put forward in that both ascribe a secondary function to `autonomous' music, there are also key differences between the two. There are two main distinctions between the two positions, the first being what the secondary function is in itself. Adorno argues that a secondary function of functionless music is that it provides social criticism by opposing bourgeois functionalization. On the other hand, the position I put forward is that the secondary function of autonomous music follows Schiller's model of \emph{aesthetics} as a projection of the future (which in itself is an \emph{ethical} function). In other words, my argument is that the secondary function of autonomous music is not that of social criticism but of delivering new sensible modes of experience. The second distinction between the two positions is how this secondary function is achieved. Adorno argues that it is achieved through the fetish character of functionless music that subjectively denounces functionalization brought by modernist society. Nevertheless, I will argue that as a consequence of the development of modernism in music it was realized that the resistance of autonomous music towards functionalization might also become a futile gesture as a denouncing strategy if the avant-garde isolates itself extensively from the wider society, instead of rendering itself visible as an opposing force to bourgeois and consumer culture. Therefore, autonomous music through its own fetishism might just become another suppressed and unheard voice in a society dominated by entertainment culture. Through the institutionalization of musical modernism and the seclusion of autonomous music into self-exile only within academic and other exclusive circles, functionless music---instead of opposing consumer culture---reassures it by strengthening capitalist values of individualism and diversity and becoming just another specialized product in a liberal and diversified market. Therefore, what I will suggests is that the secondary function of autonomous music is not achieved through its own fetish character, but through the wider understanding of different modes of music appreciation. That is to say, autonomous music achieves its secondary function, not by standing in direct opposition to consumer society by isolating itself from it, but instead by giving visibility to a specific form of sensibility and musical appreciation that encourages the interpretation of new forms of music in relationship to new forms of thought, experience, production and action which can be associated to other types of human endeavor.

Nevertheless, the establishment of the \emph{aesthetic regime} in music---which redefines the `musician' as a practitioner of whatever falls into the category of `music'---and the understanding of its secondary ethical function, have still not been spread out through a wider range of contemporary society. The reason, I believe, is that the agreement of trust between the wider public and the \emph{musical avant-garde} has been weakened as a consequence of the practice of some musicians that can be associated with the notion of modernism (mainly, those seeking music's `purity'  in composition through a militant anti-mimetic attitude and those who only advocate `authenticity' and `sterility' in performance practice). These practices have also generated an attitude commonly held by many musicians today, which avoids addressing the most basic ethical functions that the community associates with music while pursuing only their individual (and sometimes fetishist) musical priorities. If the \emph{aesthetic regime} in music is to be acknowledged and appreciated widely, an agreement of trust needs to reestablished between the \emph{musical avant-garde} and the wider public. Considering the ethical core implicit in music's definition, it is likely that the community will be unwilling to be open to new musical experiences if they fear that the ethical functions music already performs within the community will be disrupted or negatively altered. Therefore, this agreement needs to demonstrate that the purpose of creating new music is not to betray its ethical functions, but to inspire and experience new forms of subjectivity---\emph{and this in itself has an underlying ethical function}. Additionally, if this agreement with a wider range of contemporary society is to be reached, it needs to be embedded within the \emph{musical result} and cannot only be expressed theoretically through verbal and written forms of public dissemination.

\section{Reworking Musical Strategies}

In my view, if the agreement of trust between the \emph{musical avant-garde} and the community is to be regained, it is important to consider the ethical core implicit in the definition of music in parallel with a strong desire towards innovation and change in the fundamental aspects of music-making. In other words, while acknowledging the audience and their perception of what the ethical functions of music are (by providing an experience that they would associate with their own understanding of music-making), at the same time I believe we should attempt to challenge the very notions implied by traditional definitions of music. Consequently, if we ascribe to this position, we should strive to examine and rethink the most basic notions of how music is created, experienced and evaluated in combination with an awareness of what is expected from music. Moreover, I think that it is particularly important to contemplate how the strategies dealing with fundamental aspects of music-making (both in western and non-western musical traditions) can be reconsidered to produce new \emph{aesthetic} forms. In other words, while considering the \emph{ethical} functions performed by music, I propose it is vital for the \emph{musical avant-garde} to engage in rethinking and modifying past traditions of music-making that sometimes come to be viewed as immovable, standardized, authoritarian and unchangeable. Furthermore, in addition to the more conventional model of the avant-garde where musical innovation is seen as a continuation or development of past traditions---in which one attempts to stretch the boundaries within a given musical practice---I suggest we should also consider an alternative `bottom to top' model, where after a significant reexamination of past traditions one seeks innovation by altering fundamental aspects of a given musical practice (including the most elementary ways in which music is created, performed and experienced) in order to produce a new \emph{musical result}. However, this approach requires a fair amount of \emph{unlearning the already learned}, in order to \emph{relearn it}. That is, having thoroughly studied and practiced past musical traditions and considered their underlying \emph{ethical} function, we should step back and up to a certain point forget the acquired musical knowledge and practices in order to reevaluate them. At this point, we should evaluate how can we accomplish something new by reusing, modifying and relearning the strategies of past traditions. Put briefly, we should contemplate the possibility of reworking musical strategies of the past to produce new \emph{aesthetic} forms. 

In my view, this approach should actively seek to creatively rework different modes of performance, composition, presentation and dissemination of music and rethink the relationships between composer, performer and audience. I believe that this approach should also attempt to provide alternative musical experiences at the same time as providing certain \emph{ethical} functions that audience expects from music. In attempting to do so, it is vital to consider the type of audience as well as the context, time and space where music is to be presented as this too has a direct influence in the \emph{musical result} and its visibility, and plays a significant part in the disclosure of a particular type of experience. Furthermore, I accept that it is fundamental to take under consideration how people relate \emph{with} and \emph{through} music and that music is a collective experience in which elaborate human relationships are formed. I also acknowledge the significance that musical groups (ensembles, institutions, industry, movements, etc.) have over the \emph{musical result}, and the considerable potential that exists in devising musical strategies that deal explicitly with reshaping the way that these forms of collectivity function. These strategies, in my view should consider how we can achieve new forms of music-making by rethinking, altering and modifying how musical groups operate traditionally. I believe this also involves a thorough evaluation of the role musical groups might have in the \emph{musical result}. Once we have evaluated the way these groups function, we should determine whether and how they can contribute in the creation of new \emph{aesthetic} forms. At this point we should contemplate the possibility of reworking how these groups operate if we believe that by doing so we might achieve a new \emph{aesthetic} forms. I am convinced that changing the dynamics and relationships within musical groups can have a profound effect on the \emph{musical result} in interesting and innovative ways if the underlying purpose is to challenge past notions of music-making while acknowledging the \emph{ethical} functions of music. 

The arguments previously elaborated have been the motivating force that resulted in the submitted work. The ideas and notions previously examined, influenced the direction of my creative process and informed my aesthetic decisions and choices. The following two chapters also stem form the preoccupations discussed in this chapter and attempt to develop the notion of reworking, modifying and relearning previous musical practices as a creative tool. This chapters also describe certain approaches towards thinking about and creating music that I consider embrace the `bottom to top' model described earlier. They also give several examples of how we can rethink and alter certain fundamental aspects of how we make and experience music. In \hyperlink{chapter3}{Chapter 3}, I contemplate the possibility of reworking musical strategies through digital technologies to challenge traditional notions of how we create, perform and experience music. I also give some ideas of how through technology we might challenge the relationships between composer, performer and audience as well as how musical groups conventionally operate. In \hyperlink{chapter4}{Chapter 4}, I examine the possibility of using appropriation as a strategy to rethink and modify past traditions of music-making. I particularly focus on strategies that explicitly appropriate existing music and in my opinion accomplish something new within music by reworking past notions of music-making. In my view, the way in which I have approached the rest of the subjects I will examine in the next chapters in one way or another relate to the rationale elaborated in this chapter. I also hope that the reader by understanding the concerns of this chapter can better grasp the motivations and justification of the submitted work.

\label{ch:motivation}