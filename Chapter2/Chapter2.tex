\hypertarget{chapter2}{}
\chapter{Background}

In this chapter, I will attempt to give the philosophical and historical background necessary to understand the aesthetic preoccupations and ideas behind my work.\footnote{These ideas and concepts will be introduced, elaborated and discussed in \hyperlink{chapter3}{Chapter 3}, \hyperlink{chapter4}{Chapter 4} and \hyperlink{chapter5}{Chapter 5}.} I will endeavor to do so by closely examining the theoretical edifice of French philosopher Jacques Ranci\`{e}re. I have chosen Ranci\`{e}re's work as I think it successfully rethinks the relationship between art and politics as well as invigorating the concept of \emph{aesthetics}. It does so by clarifying crucial concepts, explaining important aesthetic questions and demystifying concepts that are too often misused (or misunderstood) in discussions about art. My central interest is in how Ranci\`{e}re's concepts relate to music and more specifically to the musical discourse of western avant-garde composers. 

I will start by addressing some concerns and questions regarding the notion of \emph{modernity} and how it manifests in music as compared to other artistic disciplines, particularly that of the fine arts. Then, I will attempt to explain Ranci\`{e}re's idiosyncratic and revealing view on aesthetics and its relationship to politics---later going into a more in-depth analysis of what he calls the `regimes of art'. Having given the theoretical tools necessary examination, I will attempt to clarify some of the misunderstandings and misconceptions that are usually ascribed to modernism in music. In doing so, I will discuss certain elements about the work of early twentieth-century composers, whose innovations shook up the musical status-quo---focusing on Sch\"{o}nberg's departure from tonality. I will analyse these developments in relationship to the initial premises of the modernist project that later would come to be simplified and misunderstood by the next generation of avant-garde composers who embraced the rejection of tonality and references to other music as one of their central premises. In addition, I will argue that a link was established between `modernist' composers and the idea of a political revolution. As the concepts of emancipation and utopia became scrutinized as a result of the fall of the communist block, this link would contribute to the `decline' of the modernist aesthetic in music. Finally, I will discuss the musical stance (sometimes attributed to the term \emph{postmodern music}) which encouraged a break with everything that modernism stood for, but more recently, has become associated with something more than a criticism of musical modernism.

The aim of this chapter is therefore to contextualize the situation in which the music that is being submitted was conceived. The ideas that are presented actively informed the composition of the works but most importantly encouraged reflection regarding the urgency to find new approaches to some of the problems that are exposed by Ranci\`{e}re's analysis.

\section{Ranci\`{e}re and the Re-evaluation of the Notion of Modernity}

Jacques Ranci\`{e}re in his book \emph{The Politics of Aesthetics} examines the relationship between the concept of \emph{modernity} and the break from figurative representation in the visual arts. He argues that aesthetic modernity---which according to him is specific to a single regime of the arts---is often confused with the departure from representation of images through figurative means. Ranci\`{e}re defines a single regime of the arts as ``a specific type of connection between ways of producing works of art or developing practices, forms of visibility that disclose them, and ways of conceptualizing the former and the latter''.\footnote{\hyperlink{ranpoli}{Ranci\`{e}re (2004)}, `The Distribution of the Sensible', p. 20.} If one is to think about the confusion that is associated with modernism in the realm of music, some questions come into mind: Does this confusion apply to the musical domain when compared to the other arts and if so how does it manifest itself? Is it possible to talk about representation in music and if so within what context? Could one compare the breaking from figurative representation to the departure from tonality at the beginning of the twentieth-century? Has `the musician' gone through a corresponding redefinition of \emph{what is expected} from her/him by the community the same way as `the fine artist' has through the process of modernisation?

In the following discussion, I will attempt to read Ranci\`{e}re's text as applied to music not only with the purpose of tracing parallels and discrepancies between music and fine art, but to try to find out something particular about music itself. Also, I will venture to examine the limitations of the notion of modernity within music and its relationship to the wider modernist political project.

\subsection{The Distribution of the Sensible}

Before starting the discussion on the notion of modernity and its political and aesthetic consequences, I will first try to examine the relationship of aesthetics and politics in the work of Ranci\`{e}re. According to Ranci\`{e}re, the political and the aesthetic spheres are intrinsically linked through what he calls `The distribution of the sensible'. The distribution of the sensible refers to an abstract notion that describes a system of division of spaces, times and forms of activity that defines aesthetics and is also at the heart of politics. Here though, Ranci\`{e}re points out, in order to make the relationship between politics and aesthetics, one must understand aesthetics ``in a Kantian sense---re-examined perhaps by Foucault---as the system of \emph{a priori} forms determining what presents itself to sense experience''.\footnote{Ibid., p. 13.} Aesthetics therefore should be seen here beyond the conventional view as strictly belonging to the confines of art and should not be seen merely as the `aesthetic practices' manifested in different artistic disciplines.  In order to think of aesthetics in a context that could be applied outside of the arts, it requires its abstraction as modes of action, production, perception and thought; a system of ``delimitation of spaces and times, of the visible and the invisible, of speech and noise, that simultaneously determines the place and the stakes of politics as a form of experience''.\footnote{Ibid.} Consequently, aesthetics takes part in the political act of governing and in determining who the rulers are and how they come to power; as well as how the commons are distributed within a community. Therefore, through the work of Ranci\`{e}re, it is possible to think of aesthetics in politics with a broader understanding of aesthetics as the distribution of the sensible. The notion of the distribution of the sensible therefore implies a commonality between different ways of distributing existing forms that one perceives.
\begin{quote}
I call the distribution of the sensible the system of self-evident facts of sense perception that simultaneously discloses the existence of something in common and the delimitations that define the respective parts and positions within it. A distribution of the sensible therefore establishes at one and the same time something common that is shared and exclusive parts. This apportionment of parts and positions is based on a distribution of spaces, times, and forms of activity that determines the very manner in which something in common lends itself to participation and in what way various individuals have a part in this distribution.\footnote{Ibid., p. 12.}
\end{quote}
Moreover, for Ranci\`{e}re, `aesthetic practices' that disclose visibility in artistic practices reveal `ways of doing and making' that exist and have visibility within the community. There are different manifestations of these practices that confine an aesthetic distribution.
\begin{quote}
These forms define the way in which works of art or performances are `involved in politics', whatever may otherwise be the guiding intentions, artists' social modes of integration, or the manner in which artistic forms reflect social structures or movements. . . . In this way, a sensible politicity exists that is immediately attributed to the major forms of aesthetic distribution such as theater, the page, or the chorus. There `politics' obey their own proper logic, and they offer their services in very different contexts and time periods.\footnote{Ibid., pp. 14-15.} 
\end{quote}
Consequently, it could be argued that there is an inherent political core in the way these artistic forms are constituted. Moreover, within each major aesthetic discipline there lies a political project that renders a distribution of `ways of doing and making', an internal mode of organization and a delimitation of what remains visible or invisible.    

\hypertarget{artregimes}{}
\subsection{The Regimes of Art}

In order to understand Ranci\`{e}re's reevaluation of the notion of modernity one must first understand what he calls the three `regimes of art', which are modes of identification and articulation between `ways of doing and making' and forms of visibility, as well as their conceptualization. In other words, the `regimes of art' simply distinguish different ways in which societies are organized with respect to the arts.

\subsubsection{The Ethical Regime of Images and the Poetic Regime of Art}

To begin with, Ranci\`{e}re defines the \emph{ethical regime of images} as the Platonic notion of the use and distribution of images in relationship to the community's \emph{ethos}. This regime therefore uses images as `true' imitations of the original and are distributed and valued by their purpose of educating the community in accordance to its social order. Therefore, within this regime `art' is not evaluated by qualities within itself but by their purpose in the community. He goes on to define a \emph{poetic regime of art} (also referred to as \emph{representative regime of art}) as that which breaks away from the \emph{ethical regime of images} and values the arts in terms of their own \emph{substance}.
\begin{quote}
I call this regime \emph{poetic} in the sense that it identifies the arts---what the Classical Age would later call the `fine arts'---within a classification of `ways of doing and making', and it consequently defines proper `ways of doing and making' as well as means of assessing imitations. I call it \emph{representative} insofar as it is the notion of representation or \emph{mim\={e}sis} that organizes these ways of doing, making, seeing and judging. Once again, however, \emph{mim\={e}sis} is not the law that brings the arts under the yoke of resemblance. It is first of all a fold in the distribution of `ways of doing and making' as well as in social occupations, a fold that renders the arts visible. It is not an artistic process but a regime of visibility regarding the arts.\footnote{Ibid., p. 22.}
\end{quote}

If one is to apply Ranci\`{e}re's notion of the `regimes of art' to music and understand the difference between the \emph{ethical regime of images} and the \emph{poetic regime of art} outside the domain of the visual and fine arts, one must first remember that music not only has different social functions and visibility, but within its unique organization, it has particular `ways of doing and making' that are specific to its own discipline. Even though music occupies a different and particular position in the ways of distributing the sensible, I will continue to argue that it is still possible to refer to the \emph{ethical} and the \emph{poetic} regimes in music. 

Following Ranci\`{e}re definition, I will refer to music within the \emph{ethical regime} as music that is made, heard and judged for its purpose within the community. By this, I mean music that is not assessed by it own qualities---or as Ranci\`{e}re would say `by its own \emph{substance}'---but by the purpose it performs within the community. Examples of this in western tradition would include church, court and military music, to mention just a few. It is easy to find music that falls within the \emph{ethical regime} in other cultures where in some cases music is not even differentiated from other disciplines, like dance or storytelling, and is performed (in some cultures everyone partakes in music-making) and valued by members of the group by its communal and ceremonial purposes (celebration, mourning, war, etc). Of course, one can still find many examples of the \emph{ethical regime} today in music for theater, dance, television, films and religious purposes. Here, I want to make clear that I am not attempting to devalorize or make a value judgment about music that falls within the \emph{ethical regime}. Furthermore, some music might also be considered within more than one regime simultaneously.

Music that falls within the \emph{poetic regime} is that which is appreciated for its own \emph{substance} but still follows or imitates a model.\footnote{By model I not only mean the written but also the unwritten rules in music performance and composition. The written rules could be for example treatises of harmony and orchestration whereas the unwritten rules could be performance practices and conventions in composition and improvisation, to name a few.} Namely, music that is judged by its own `musical' qualities, and that is made with the main purpose of being listened to and evaluated according to its own subject matter. This music would be \emph{representative} insofar as it imitates or resembles a musical model (for example rules of harmony, counterpoint or sonata form, to mention just a few). A lot of western `concert music' falls in this modality in that it is made, heard and valued for its `musical' qualities and judged as good or bad, adequate or inadequate, satisfactory or not, dependent on how the performer or composer follows certain models---in the case of the performer, models of performance practice, and in the case of the composer, compositional models such as chord progressions, \mbox{voice-leading}, musical themes, variations, etc. 

It is interesting to note that within the visual arts the breaking from the \emph{ethical regime of images} and the establishment of the \emph{poetic regime of art} is what now separates the `fine arts' from other modes and techniques of production (of images, shapes, objects, etc), whereas within music there is not such a change in definition. That is to say, in the visual arts this break between \emph{ethical} and \emph{poetic} regimes identifies the arts as such but in music it does not change its identification. Why is it that in the musical domain it is still plausible to call the `ways of doing and making' in both regimes \emph{music}? At this moment, I will not draw any conclusions about this enquiry as one needs first to examine other aspects of Ranci\`{e}re's postulation in order to fully understand the consequences of this difference. However, in the following chapter I will come back to this question and look at the possible reasons and implications of this disparity.\footnote{See \hyperlink{musdef}{pp. 28-29}.} Nevertheless, for the moment I will continue the discussion by examining the \emph{aesthetic regime of art} to have a better understanding of Ranci\`{e}re's thesis.

\subsubsection{The Aesthetic Regime of Art and the Shortcomings of the Notion of Modernity}

Ranci\`{e}re calls the \emph{aesthetic regime of art} that which liberates art from the \mbox{\emph{poetic regime}} by breaking with its identification as the division of `ways of doing and making'. The \emph{aesthetic regime} therefore puts an end to the models used by the \emph{poetic regime} and breaks the barriers of identification in the arts. It does so by distinguishing art as an occupation that establishes, questions and alters the concept of what art is, its hierarchies, subject matter and genres. 
\begin{quote}
The aesthetic regime of the arts is the regime that strictly identifies art in the singular and frees it from any specific rule, from any hierarchy of the arts, subject matter, and genres. Yet it does so by destroying the mimetic barrier that distinguished `ways of doing and making' affiliated with art from other `ways of doing and making', a barrier that separated its rules from the order of social occupations. The aesthetic regime asserts the absolute singularity of art and, at the same time, destroys any pragmatic criterion for isolating this singularity.\footnote{Ibid., p. 23.}
\end{quote}
Hence, the \emph{aesthetic regime} establishes the autonomy of art and at the same time makes art independent of its own forms. As a result, the artist becomes a practitioner of a discipline specific to whatever falls into the category of art. 

At this point, I want to examine the \emph{aesthetic regime} in the domain of music. I will propose that music that falls within this regime is music that challenges the \emph{poetic regime} and the very notion of \emph{what music is} at a given point in time. It should also be thought as a regime that makes music independent from its own subject matter, rules, conventions and genres, and frees it from specific `ways of doing and making'. It changes music's visibility and makes it autonomous from the very notion of itself, from its expected `musical' and social functions.\footnote{Here, I refer to `social functions' not as in the purpose or use of music within the \emph{ethical regime}, but the social functions it performs within the \emph{poetic regime}.} In the history of music, it is easy to think of examples of music that breaks with the musical practices of its time and redefines itself.\footnote{There are too many examples for me to list them here.} It is even possible to think of brief historical periods before the twentieth-century where one can observe some form or manifestation of the \emph{aesthetic regime} in music. Nevertheless, it is difficult to think of music as an autonomous discipline, freed from its own \emph{substance}. That is to say, even though the definition of music has changed and was challenged on several occasions, it was not until the twentieth-century that the concept fully emerged of `the musician' as someone who creates music as whatever he considers suitable and is not expected to follow traditional formulas of music-making. To this day, this concept of music and `the musician' is not widely accepted in any contemporary society.\footnote{See p. 23-24 for a further discussion on the possible reasons for this problem.}

Ranci\`{e}re goes further to examine the limitations of the notion of modernity and its relationship to the \emph{aesthetic regime of art}. He describes what is commonly referred to as modernism in art as an `incoherent' label applied to what truly should be referred to as the \emph{aesthetic regime of art}. There is a sort of simplicity ascribed to the notion of modernity that is viewed as a clear line of transition or rupture from the old to the new and in the case of the visual arts between figurative and non-figurative representation. Ranci\`{e}re argues that the break from figurative representation is a confusion that emerged from the simplistic view that this break would mean a rupture from the \emph{poetic regime of art}.
\begin{quote}
The basis for this simplistic historical account was the transition to non-figurative representation in painting. This transition was theorized by being cursorily assimilated into artistic `modernity's' overall anti-mimetic destiny. . . . However, it is the starting point that is erroneous. The leap outside of \emph{mim\={e}sis} is by no means the refusal of figurative representation.\footnote{Ibid., p. 24.}
\end{quote}

Therefore, the break from figurative representation does not mean the establishment of a new visibility for art nor a break from the mimetic barrier. Moreover, Ranci\`{e}re asserts that the contradiction of the \emph{aesthetic regime of art}---which on the one hand establishes the autonomy of art and on the other hand questions the distinction between art and other activities---leads to two big misunderstandings of the notion of modernity. The first confusion was to simply associate the modernist movement with the autonomy of art. The modernist project was therefore reduced only to an anti-mimetic\footnote{From now on, I will use the term `anti-mimetic' as referring to the \emph{erroneous} modernist notion that associates \emph{mim\={e}sis} with figurative representation in the visual arts and tonal music as well as references to other musical styles and traditions in music} movement that concentrates on the idealistic concept of stripping away from all references to previous art forms and works in order to reveal art's `purity' of form and reach its `essence'. They attempted this by exploring only the formal aspects of art by focusing on the capabilities of its own medium. The second big confusion, according to Ranci\`{e}re, is the idea that the forms of the \emph{aesthetic regime of art} were somehow related to other forms that would materialize by accomplishing a task or fulfilling a destiny specific to modernity; in other words, the revolution that rendered autonomy to art became the example for the Marxist revolution. The failure of both the anti-mimetic principles of modernism and the political revolution resulted in a `crisis of art' caused by these paradigms of modernism. Modernism in art therefore ``became something like a fatal destiny based on a fundamental forgetting''.\footnote{Ibid., p. 27.} 

\section{Modernity and Music: Misconceptions and Misunderstandings}

I will propose that a similar confusion has taken place in western music, which leads to analogous misunderstandings regarding the so called \emph{modernist} project. However, in order to avoid simplifications, one should first remember certain aspects about the state of western music at the end of the nineteenth and beginning of the twentieth centuries. It is important first of all to realize that by the end of the nineteenth-century there was a clear specialization of musicians---some were trained specifically as performers and others as composers. This division of occupations in music led to a greater dichotomy in the `ways of doing and making' music. The specificity of the performer's creative decisions therefore became mostly linked to the realization of a given score. The composer's role, on the other hand, was to provide a score to the performers and establish certain directions and instructions on parameters such as pitch, rhythm, musical form and instrumentation. During this time, the role of the composer became more prominent concerning music innovation and therefore these developments are mostly attributed to composers in western music. Hence, I will mostly refer to composers when attempting to explain the limitations of the notion of modernity in music. Nevertheless, by no means am I attempting to discredit or ignore the performers' role---I am just referring to the more widespread view of these developments. Later in this chapter, I will explain how this division of occupations in western music has been questioned and how performers have also attempted to establish themselves within the \mbox{\emph{aesthetic regime}}; but first, I will analyse the work of some composers that reflect the misunderstandings usually ascribed to the modernist project. 

At the end of the nineteenth-century, composers such as Wagner, Mahler and Debussy were already expanding the tonal system through what became widely known as the `emancipation of dissonance', signaling what was to become a radical break in western music---that is, Sch\"{o}nberg's moving away from the tonal system altogether and starting to compose freely. This \emph{event}---as Alain Badiou would describe it\footnote{See \hyperlink{badiouethics}{Badiou (2001)}, p. 41-42, and \hyperlink{badioumus}{Badiou (2009)}, p. 46, 79-85, for a further discussion on what he calls the Sch\"{o}nberg-event. See also \hyperlink{eventdis}{pp. 24-26} for a further discussion about Badiou's philosophy of the \emph{event}.}---signals a step towards the \emph{aesthetic regime} in that this gesture attempts to free music from previous models thus venturing to unleash music from its own \emph{substance}. Sch\"{o}nberg, in his period of so called `free atonality'\footnote{The period between 1908 and 1923 in which Sch\"{o}nberg abstained from using tonality and did not adhere to a systematic method of pitch organization.} and later with his twelve-tone method\footnote{Devised by Sch\"{o}nberg in 1921 and first described to his inner circle in 1923.}, breaks away from the convention that a composer should follow previous models of composition and starts to define a new notion of the composer as someone who decides what he considers music to be and chooses how it is to be organized. Therefore, the rupture from the tonal system at the beginning of the twentieth-century challenges the definition of music in western society and contributes to redefine `the musician' as someone who does not follow existing models, but can invent his own modes and systems of music-making. However, it is important to note that the break from tonality by no means represents the establishment of an \emph{aesthetic regime} in music nor a leap outside representation and the \emph{poetic regime}.  Stravinsky's \emph{Le Sacre du Printemps} (1913), is a clear example of a work that points towards the \emph{aesthetic regime} but does so not by abandoning tonality, but by breaking with other models of concert music. The radicality of \emph{Le Sacre du Printemps} comes from developments in musical parameters such as rhythm, tonality (polytonality, etc), timbre and form, but not from a complete renunciation of tonality. Stravinsky's use of folk-music, primitive rhythms, asymmetric structures and orchestral textures was music never heard before and stretched the definition of concert music as well as proposing new ways of organizing its subject matter, freeing music from specific `ways of doing and making'. At the same time Stravinsky invents new rules and defies traditional genres and styles, which are all characteristics of the \emph{aesthetic regime}. 

Sch\"{o}nberg's importance in the establishment of the \emph{aesthetic regime} is also not to be discredited and I believe that by departing from tonality, he certainly redefined \emph{what music is} and questioned music's subject matter. Moreover, through his revolutionary shock on the community's notion of music, he certainly contributed to changing the notion of `the musician' as someone who produces what \emph{he considers music to be}. It is also compelling to see that Sch\"{o}nberg's use of dissonance was not with the purpose of centering his musical discourse around pitch organization or being non-referential to previous styles and genres. Paradoxically, even though his way of organizing pitches was radically new, he was fairly traditional in his use of other musical parameters such as form\footnote{He constantly used traditional forms such as sonata form, suite and theme and variations.}, timbre and gesture. In his essay entitled \emph{A Self-Analysis}, Sch\"{o}nberg describes how his methods to organize notes or achieve atonality were not very important elements in his work: ``I personally do not find that atonality and dissonance are the outstanding features of my works. They certainly offer obstacles to the understanding of what is really my musical subject''.\footnote{\hyperlink{schoen}{Sch\"{o}nberg (1984)}, p. 77.} This attitude clearly separates him from the next generation of composers who embraced his twelve-tone system and whose main compositional objective focused on the organization of these twelve pitches. 

\subsection{Anti-mimetic Tendencies and the Influence of Serialism}

It is by trying to understand this next generation of serialist composers' work that Ranci\'{e}re's analysis of the confusion of the notion of modernity becomes useful. It is crucial to remember the first confusion, which is simply to seek the autonomy of art through anti-mimetic strategies. In the case of music, this was attempted by focusing on formal aspects of music such as how to organize pitches, rhythms, dynamics and all other possible `musical' parameters. By giving importance to the formal aspects of the compositional medium, they sought to stretch music's capabilities and to seek music's autonomy by stripping away all references to other musics. It is fascinating to read that when Sch\"{o}nberg showed his twelve-tone method to his associates in 1923, he already noticed the potential problems of looking at music only in terms of the formal techniques implemented to compose it. 
\begin{quote}
What I feared, happened. Although I had warned my friends and pupils to consider this as a change in compositional regards, and although I gave them the advice to consider it only as a means to fortify the logic, they started counting the tones and finding out the methods with which I used the rows. Only to explain understandably and thoroughly the idea, I had shown them a certain number of cases. But I refused to explain more of it, not the least because I had already forgotten it and had to find it myself. But principally because I thought it would not be useful to show technical matters which everybody had to find for himself and could do so. This is also the error of Mr. Hill. He also is counting tones and wants to know how I use them and whether I do it consequently.\footnote{Ibid., p. 214.} 
\end{quote}

Sch\"{o}nberg's use of the twelve-tone method did not have an anti-mimetic purpose and he devised it to be able to have a systematic approach to form and to compose melodies, themes, phrases and chords. He also made clear his abandonment of the tonal system was not more important than other aspects of his work. It is important to note as well that after the invention of his method, he relied on gestures, orchestration and structures that were related to traditional styles and genres---especially those of the Germanic tradition. Therefore, Sch\"{o}nberg's invention of the twelve-tone method was mostly pragmatic and did not have the purpose of not referring to other musics or focusing only on music's formal aspects. It is precisely these aspects of Sch\"{o}nberg's use of dodecaphony that later Boulez would criticize in his article `Sch\"{o}nberg is dead'.
\begin{quote}
From Sch\"{o}nberg's pen flows a stream of infuriating clich\'{e}s and formidable stereotypes redolent of the most wearily ostentatious romanticism: all those endless anticipations with expressive accent on the harmony note, those fake appoggiaturas, those arpeggios, tremolandos, and note-repetitions, which sound so terribly empty and which so utterly deserve the label `secondary voices'; finally, the depressing poverty, even ugliness, of rhythms in which a few tricks of variation on classical formulae leave a disheartening impression of bonhomous futility.\footnote{\hyperlink{boulez}{Boulez (1991)}, pp. 212-213.}
\end{quote}
For what interested Boulez in the twelve-tone system were the formal aspects of the \emph{series}---an approach closer to Webern's dodecaphony. One can already see here in Boulez's position an anti-mimetic preoccupation to avoid clich\'{e}s and references to pervious traditional music as well as a modernist concern towards the formalization of music through the capabilities of serialism.
\begin{quote}
It has to be admitted that this ultra-thematicization is the underlying principle of the \emph{series}, which is no more than its logical outcome. Moreover, the confusion between theme and series in Sch\"{o}nberg's serial works is sufficiently expressive of his inability to envisage the world of sound brought into being by serialism. For him dodecaphony is nothing more than a rigorous means for controlling chromaticism; beyond its role as regulator, the serial phenomenon passed virtually unnoticed by Sch\"{o}nberg.\footnote{Ibid., p. 212.}
\end{quote}

It was through the development of serialism in the fifties and sixties---led by Boulez and Stockhausen---that composers would seek music's pure form through the serialization of all conceivable `musical' parameters, thus focusing only in an exploration of the formal capabilities of music and sound. The confusion caused by the establishment of the \emph{aesthetic regime} that identifies \emph{modernity} only with the autonomy of art and which led to an anti-mimetic revolution became endemic in postwar European avant-garde music. Serialism thus would seek through its self-contained system an ideal of music that would avoid any external or `impure' elements and would attempt to escape any reference to other existing music. The scope of the serialist movement and its influence over the avant-garde and \emph{modernist} composers across the world should not be overlooked. Even composers who did not adhere to the serialist camp were influenced by the leading focus on the abstract organization of sound and `musical' parameters and they too adopted the anti-mimetic ideal as an important aesthetic principle.\footnote{Some examples of composers who were influenced by these ideals at some point in their career include John Cage, Morton Feldman, Alvin Lucier and Earle Brown in America and Pierre Schaeffer, Iannis Xenakis, Gy\"{o}rgy Ligeti, Helmut Lachemann and Cornelius Cardew in Europe.}

\subsection{The Political Revolution and the Crisis of Modernism in Music}

Another misconception of the notion of modernity in music  was the association of the \emph{aesthetic regime} with the fulfillment of a Marxist revolution.\footnote{It is important to note here that this association was only made by a number of composers (such as Luigi Nono, Stephan Wolpe, Hans Werner Henze, Frederic Rzewski, Cornelius Cardew, Christian Wolff and Alvin Curran). Many dominant figures of modernism in music remained indifferent or critical towards this idea. In some cases, important `modernist' composers were known to be apolitical (most notably Boulez) and in some cases even politically conservative.} The aesthetic revolution was confused with its materialization in the social and political domains. Therefore, the revolution that attempted autonomy for music was identified with the modernist political project and the social application of its ideals of egalitarianism, solidarity and liberty. Leftist politics were associated with the artistic avant-garde and a misleading link was formulated between modernism in music and the political revolution. Curiously enough, Sch\"{o}nberg again detected the fallacy of establishing a direct relationship between serialism and leftist politics---in fact, with any other political association---and like Ranci\`{e}re,\footnote{See \hyperlink{ranpoli}{Ranci\`{e}re (2004)}, `Politicized Art', pp. 60-66.} makes the point that progressive artistic innovation can produce developments within art but bears no direct correspondence in the political sphere.
\begin{quote}
It has become a habit of late to qualify aesthetic and artistic subjects in terms borrowed from the jargon of politics. Thus mildly progressive works of art, literature or even music might be classified as `revolutionary' or `left-wing', when they only evolve artistic possibilities. . . . No wonder, then, that there are people who call the method of composing with twelve tones `bolshevik'. They pretend that in a `set of twelve tones', upon which such compositions are founded, since there is no tonic nor dominant, every tone is considered independent, and consequently exerts equal functions. This is wrong in every respect. . . . Whether this concept is an advantage or a handicap to the composer or to the listener, certainly it has nothing in common with `Liberty, Equality and Fraternity', neither with the bolshevik, fascist, nor any other totalitarian brand.\footnote{\hyperlink{schoen}{Sch\"{o}nberg (1984)}, pp. 249-250.}
\end{quote}

Despite Sch\"{o}nberg's warning, many associations were made between modernism in music and the Marxist revolution. This notion was also fueled by the political affiliation of many composers and by their general plea for revolution in both the aesthetic and political spheres. Marxist themes were also incorporated in music identified as \emph{modernist} using leftist texts, images and sounds based on the struggle of the proletariat, student demonstrations and other revolutionary events. Luigi Nono most notably was engaged with political activism and at the same time used Marxist dialectics and other themes related to leftist ideology in his compositions. Nono viewed music as a form of activism and at the same time embraced strategies related to the aesthetic revolution. Many of his works use titles and texts that are politically engaged and at the same time reject musical representation. He viewed his work as a continuation of the developments of the Second Viennese School and his approach to musical material can be closely linked with serialism and the Darmstadt School---despite certain differences he had with Boulez and Stockhausen.\footnote{Nono was against Boulez and Stockhausen's interest in the music of John Cage and the use of indeterminism and chance operations. See \hyperlink{nono}{Nono (1975)}, pp. 34-40.} Consequently, there is an interesting contradiction inherent in Nono's \emph{oeuvre}: on the one hand his work uses many `extra-musical' references to address political concerns; on the other hand his music fits within the modernist aesthetic that was on the most part anti-mimetic and avoided `musical' references that could have been used to appeal to the proletariat and identify music with the class struggle and political revolution. 

Other composers that followed a leftist political affiliation but used strategies that were considerably different to the serialist approach were a group whose most prominent figures included Rzewski, Cardew, Wolff and Curran. Some of their compositions rejected the modernist notion of an anti-mimetic ideal with the purpose of introducing political themes as musical material in their compositions and others questioned the division of occupations imbedded in western music-making. Georgina Born argues that these composers were more politicized than what she calls the `postserialist camp'.
\begin{quote}
Beginning in the later 1960s, inspired in part by Marxist-Leninism or Maoism, there emerged out of this a set of experimental composers, including Wolff, Cardew, \mbox{Frederic} Rzewski, and their followers, who were more frankly politicized than those in the \mbox{postserialist} camp. In some cases they attempted to produce political effects through the use of or by reference to, revolutionary popular musical material or lyrics. Another strategy, developed \hyphenation{de-ve-loped} by some of the same composers but more widely influential, extended the critiques of the musical division of labor. Composers such as Cardew, Wolff, and groups such as the Italian-American MEV \emph{(Musica Elettronica Viva)}, the British Scratch Orchestra, and AMM, emphasized changes in the social relations of music production and performance in their attempt at a new interactive, collective, and nonhierarchical group practice. The social dimension of music was seen as a crucible for experiments in collective and democratic social relations.\footnote{\hyperlink{born}{Born (1995)}, pp. 58-59.}
\end{quote}
According to Born, the later strategy as implemented by these groups questioned the power structures and division of occupations in western music through collective compositional strategies based on group improvisation as a method of creating music. By avoiding hierarchical forms of composition and performance these groups attempted to challenge the traditional roles of composer, conductor and performer. Their purpose was to pursue an ideal of an egalitarian division of the group and democratic relations between musicians. Born suggests that there was a conscientious attempt by these groups to invigorate the principle of equality and freedom within the politicity of western music production and performance. Nevertheless, a counter-argument could easily be raised against Born's position if one just questions the effectiveness of these two approaches within the political and aesthetic spheres.\footnote{Isn't the way in which this group improvisation was implemented more characteristic of our liberal democratic model than a true form of egalitarianism? The idea that everyone in the group can improvise and play `freely' giving the appearance of a permissive mode of performance is highly questionable. Even though it is implied that every improviser could play whatever they want, in practice there are many unwritten rules in this kind of group performance. For example, in many of these groups anti-mimetic principles dominate the improvisational setting and it is not allowed to play a recognizable tune or musical quotation. Therefore, within an apparent freedom these improvisers might actually have many prohibitions that are imposed by the unwritten rules of each group. Another problem of this position is that it presupposes that each player will have an equal voice in the group and that no structures of power will emerge. To assume that a collective form of organization will be egalitarian just by giving the appearance that everyone within the group has an equal voice is deceiving and the idea that these improvisations are `free' is naive and misleading.} Despite the ineffectiveness of their strategies, the contribution of this group of composers to the association between a leftist political revolution and modernism in music should not be underestimated.

\subsubsection{The Fall of Communism and the Critique of Utopian Thinking}

Given the association between musical modernism and the Marxist revolution, the result of `the fall of Communism' was that modernist aesthetics, too, was called into question. The aesthetic revolution in music and its ontological model came under scrutiny and close examination. The corruption and abuses that came with the implementation of Marxist ideals in communist countries brought disillusionment and skepticism toward utopian ideals in politics and contributed to a further examination of utopia as it manifests itself in other aesthetic practices. Richard Taruskin, one of the prominent critics of utopianism in music, asserts that the fundamental problem of utopia is that it imagines a `perfect world' instead of a `better world'.
\begin{quote}
But what utopians envision is not a better world. It is a perfect world---or in Kant's two-centuries-old formulation, ``a perfectly constituted state''---that utopians wish to bring about. And that is what makes them dangerous, because if perfection is the aim, and compromise taboo, there will always be a shortfall to correct---a human shortfall. . . . When communism ``fell'', the intellectual world divided into two camps: those who said it was time to go back to the drawing board and those who said it was time to get rid of drawing boards. I am utterly of the latter persuasion.\footnote{\hyperlink{taruskin}{Taruskin (2009)}, `Against Utopia', p. xii.}
\end{quote}
According to Taruskin, there is a gap between the imagined state of perfection and its implementation in reality. It is this gap that is dangerous as it depends on a deficit that has to be corrected and that may result in human casualties and suffering. 

He argues that one of the shortfalls of utopian thinking has been the decline in popularity and dominance of classical  music in contemporary culture. This has been partly attributed to the dominance of utopian ideals in modern performance-practice that has been the governing attitude of professional performers in their rendition of classical music's `masterpieces'. Edward Said has written about how musical performance, with the specialization of musicians and the division of labour in western classical music during the twentieth-century, has become what he calls an `extreme occasion'.\footnote{See \hyperlink{said}{Said (1992)}, `Performance as an Extreme Ocation', pp. 1-34.} The phenomenon of viewing an abstract piece of music as represented in a score as a `utopia' gives the performer the `heroic' opportunity to display their virtuosity and physical dexterity in their attempt at a `perfect' rendition of the composition. This extreme musical practice in classical music, Said suggests, has gone so far as to displace the composer from the center of classical music. Despite the dominance and relative popularity of these `superstar' performers, the influence of classical music in western culture has declined, even within the intellectual elite.\footnote{Said refers to an annecdote about Michel Foucault commenting to Pierre Boulez about the ignorance that contemporary intellectuals have about popular and classical music. See Ibid., p 15.} According to Taruskin, it is precisely the unyielding and militant attitude towards utopianism that has caused the classical establishment's loss of relevance to contemporary culture. The futile search for autonomy and authenticity in classical music has consequently resulted in a musical practice based for the most part on correctness and sterility and an attitude where performance is assessed for its historical value and not for its social functions. Therefore, this attitude has had a negative impact on performance practice as it has sacrificed music's ethical functions for utopian aesthetic considerations. For this reason, most people can not relate to these practices and classical music has lost relevance to the current condition.

In twentieth-century composition, utopian thinking may be associated with the other main misunderstanding of musical modernism that I have previously discussed. That is, the utopian ideal of an aesthetic revolution that would seek music's autonomy by stripping it away from all possible references to other types of music.\footnote{This was mostly true in regard to making reference to other existing western music as some modernist composers looked for alternatives to the western aesthetic by researching non-western musical traditions.} This was attempted by focusing on music's formal aspects and the capacities of its own medium in order to attempt music's `perfect' construction. One of the shortfalls of this utopian way of thinking was that contemporary composition became extremely unappealing to a general public that was not educated in the formal aspects of music and found this music extremely difficult as it also lacked any reference to any other music that was familiar to them. This resulted in an unfortunate seclusion of the musical avant-garde that found its main refuge in academia, which became a comfortable yet alienated new home for composers to test their musical `experiments'---composition at universities consequently became an academic specialization\footnote{Here, one can not avoid making reference to Babbitt's famous article `Who cares if you listen'. See \hyperlink{babbitt}{Babbitt (2003)}, pp. 48-54.} which for the most part focused on technical aspects of music. 

The failure of the anti-mimetic principles of modernity in combination with the `fall of Communism' resulted in a major crisis in music that was caused by the decline of modernist aesthetics and the loss of confidence in utopian thinking. After this crisis, musical modernist tendencies remain to this day on `life support' and one cannot but avoid noticing their nostalgic attitude and unyielding acceptance of defeat---they remain as vigilant victims of a lost utopia, endlessly waiting for a comeback that will never take place. Taruskin points that this attitude of continuing new music's `quiet' presence in contemporary culture in the hope that one day it becomes more widely recognized as important or relevant---an attitude according to him dominant in academic circles---is yet another consequence of utopian thinking that he associates with communist revolutionary ideals and to the Soviet order.\footnote{See \hyperlink{taruskin}{Taruskin (2009)}, p. xiv.} 
\hypertarget{postmodernmusic}{}

\section{Postmodern Music}

The musical stance that later would become associated with the term \emph{postmodern music},\footnote{Here, I am not going to attempt to determine whether this term is appropriate or not in relationship to this musical stance, as I believe it is out of the scope of this commentary. I will be using the term only inasmuch as it is widely used by scholars and music critics to refer to the attitude here described.} came as a reaction to everything that modernist composers stood for: the formalization of music's subject matter, the quest for non-resemblance, the desire for musical progress and emancipation, the association of the aesthetic and political revolutions, and the search for music's `essence' and `purity' of construction. Therefore, at the beginning, composers who were identified as \emph{postmodern} pointed to the confusion ascribed to the notion of modernity and the \emph{aesthetic regime} and attempted to rectify it by reversing all modernist ideals in music. Ranci\`{e}re attributes postmodernism, at first to ``the name under whose guise certain artists and thinkers realized what modernism had been: a desperate attempt to establish a `distinctive feature of art' by linking it to a simple teleology of historical evolution and rupture''.\footnote{\hyperlink{ranpoli}{Ranci\`{e}re (2004)}, p. 28.} In other words, these thinkers and artists detected that there was no necessity to link the realization of a fundamental characteristic of art as represented by the \emph{aesthetic regime} to a historical break or a beginning of a new era. Consequently, postmodernism at first aimed to give an alternative to the drawbacks of the modernist position. This was first attempted in music by breaking away from the `abstract' treatment of musical parameters by reintroducing tonality and references to other traditional and popular music either by quotation or resemblance. 

Luciano Berio was one of the first European avant-garde composers who started to reintroduce references to other existing music in his work. Most notably in the third movement of \emph{Sinfonia} (1968-1969), Berio uses quotations as well as different treatments of material by other composers as a driving force for his compositional discourse. In this movement, Berio uses most prominently the scherzo from Mahler's Second Symphony against quotations and transformations from excerpts of works by many composers including: Bach (First Brandenburg Concerto), Beethoven (Sixth and Ninth Symphonies),  Berg (Violin Concerto and \emph{Wozzeck}),  Berlioz (\emph{Symphonie fantastique}), Boulez (\emph{Pli selon pli}), Brahms (Violin Concerto and Fourth Symphony), Debussy (\emph{La Mer}), Globokar (\emph{Voie}), Hindemith (\emph{Kammermusik Nr.4}), Mahler (Second, Fourth and Ninth Symphonies), Pousseur (\emph{Couleurs crois\'{e}es}), Ravel (\emph{La Valse} and \emph{Daphnis et Chlo�}), Sch\"{o}nberg (\emph{F\"{u}nf Orchesterst\"{u}cke, Op.16}), Stockhausen (\emph{Gruppen}), Strauss (\emph{Der Rosenkavalier}), Stravinsky (\emph{La Sacre du printemps} and \emph{Agon}), Webern (\emph{Kantate}), as well as other unknown sources and Berio's own music.\footnote{See \hyperlink{ossmith}{Osmond-Smith (1985)}, `In ruhig fliessender Bewegung', pp. 39-71, for a detailed analysis of the third movement of \emph{Sinfonia}.} The material derived from the variety of scores is treated carefully by Berio taking into consideration its `musical' qualities, such as pitch and rhythm, as well as its semantic characteristics---all the quotations are related to Berio's own interpretation of L\'{e}vi-Strauss's \emph{Le cru et le cuit}.\footnote{\hyperlink{ossmith}{Osmond-Smith (1985)}, `\emph{Sinfonia} and its Precursors', p. 7.} It is precisely the semiotic value of the musical references that attracted Berio to use already existing music as material for his own work and this itself was a step against the principles of so called `modernist' composers. In his book \emph{The Future of the Image}, Ranci\`{e}re has discussed a similar phenomenon that is usually ascribed to the \emph{postmodernist} label in the visual arts, that is, the reintroduction of images and representation.
\begin{quote}
And the time came when the semiologist discovered that the lost pleasure of images is too high a price to pay for the benefit of forever transforming mourning into knowledge. Especially when this knowledge itself loses its credibility, when the real movement in history that guarantied the traversal of appearances itself proved to be an appearance.\footnote{\hyperlink{ranimg}{Ranci\`{e}re (2007)}, pp. 21-22.}
\end{quote}
Similarly in music, for composers who were interested in semiotics like Berio, the price to pay for only focusing on `abstract' musical thought and anti-mimetic ideals was too high. 

Other strategies were also attempted by so called \emph{postmodern} composers who wanted to break away from everything that modernism stood for: the reintroduction of melody, ornamentation and intervalic consonance that violated the consistency and functionality of serial techniques; the use of improvisatory elements which blurred the line between composer and performer;  the crossing between artistic disciplines, which challenged the integrity of each one; the break from notation which disturbed the focus on abstract musical models which depend on notation; the search for alternatives to the concert hall by presenting work in different venues not designed for contemporary music concerts attacked the ideal of musical performance in a sterile and specifically designed acoustic environment that would be perfect for listening to the intricacies of crafted compositions. 
\hypertarget{postmodernmusicnow}{}

Nevertheless, very quickly \emph{postmodern music} started to signify something more than a criticism of the modernist aesthetic. The music created by the next generation of composers labeled as \emph{postmodern} started to be characterized by a permissive attitude in the mixing of all different musical styles and genres, the hybridization between pop, world, jazz and classical music, the disregard for stylistic consistency and the joy of simulacra, the glorification of music primarily as a path for entertainment and primal enjoyment. The permissive attitude of the postmodern composer produced in some cases results that reinvigorated the idea of the musical performance only as entertainment. That is to say, the avant-garde attitude towards achieving something new within music itself was ignored, in favor of music that is created only to entertain and please its audience.\footnote{This type of music has also become a commodity in a consumer society in which the musician produces with the aim of seducing the consumer to buy a product and make profits.} This is precisely why Ranci\`{e}re argues that art under the label of \emph{postmodernism} ``came to challenge the freedom or autonomy that the modernist principle conferred---or would have conferred---upon art the mission of accomplishing''.\footnote{\hyperlink{ranpoli}{Ranci\`{e}re (2004)}, p. 28.} 

Postmodern music thus embraces Lyotard's notion of the `decline of grand narratives' by questioning the modernist concept of achieving an ideal of emancipation.
\begin{quote}
In the course of the past fifty years, each grand narrative of emancipation---regardless of the genre it privileges---has, as it were, had its principle invalidated. \emph{All that is real is rational, all that is rational is real}: ``Auschwitz'' refutes the speculative doctrine. At least this crime, which is real, is not rational. \emph{All that is proletarian is communist, all that is communist is proletarian}: ``Berlin 1953'', ``Budapest 1956'', ``Czechoslovakia 1968'', ``Poland 1980'' (to name a few) refute the doctrine of historical materialism: the workers rise up against the Party. . . . \emph{Everything that promotes the free flow of supply and demands is good for general prosperity, and vice versa}: the ``crisis of 1911 and 1929'' refute the doctrine of economic liberalism. . . . The investigator records the names of these events as so many signs of the failing of modernity. The grand narratives have become scarcely credible. One is then tempted to give credence to a grand narrative of the decline of the grand narratives.\footnote{\hyperlink{lyotard}{Lyotard (1992)}, p. 40.}
\end{quote}
The so called \emph{postmodern} position is therefore one of mourning metanarratives as identified in scientific postulations, theology, the ideas of self-emancipation and utopia in politics and aesthetics. For this reason, postmodernism\hyphenation{post-modernism} became a celebration of that which is unattainable and impossible to reduce, identify, rationalize or define. The establishment of the \emph{aesthetic regime}---which signifies the emancipation or autonomy of art---consequently comes under scrutiny under Lyotard's viewpoint. Nevertheless, Lyotard also links the recognition of the impossibility of emancipation to a historical break, in a similar fashion to the modernist association of the autonomy of art to a particular historical period; it is precisely for this reason that his argument loses legitimacy as one can interpret the `end of grand narratives' as a `grand narrative' in itself.

The concepts and historical background that I have elaborated in this chapter are crucial in understanding the motivations behind the submitted work, which I attempt to discuss in the following chapter. Moreover, the notions examined previously not only informed the arguments elaborated in the remaining chapters of this commentary, but also inspired, influenced and intrigued me during the creative process which led to the music here presented. I hope that Ranci\`{e}re's theoretical work is as exciting and fascinating to the reader as it certainly is to me and I am holding to the conviction that in attempting to explain some of its postulations within a musical context, it will reveal some of the problems inherent in holding too close to the `anti-mimetic' ideals of \emph{modernism} and to simplistic notions of utopianism, and at the same time disclose that behind what \emph{postmodern music} has recently become, lies a cynical and even `conservative' attitude which diminishes and undermines the significance of accomplishing something new or radical through music.

\label{ch:background}