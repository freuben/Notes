\hypertarget{chapter3}{}
\chapter{Theoretical Framework}

In this chapter...

\section{Redefining the Musical Subject?}

It appears that the dominant position at this moment regarding music is characterized by a skeptical and often cynical attitude towards new forms of thought in music. However, this attitude is dominant not without a reason: it has to do with the notion that today music is---as Alain Badiou has stated---`negatively defined.' Badiou clearly expresses this view in his essay entitled `Scholium: A Musical Variant of the Metaphysics of the Subject.'
\begin{quote}
Today, the music-world is negatively defined. The classical subject and its romantic avatars are entirely saturated, and it is not the plurality of `musics'---folklore, classicism, pop, exoticism, jazz and baroque reaction in the same festive bag---which will be able to resuscitate them. But the serial subject is equally unpromising, and has been for at least twenty years. Today's  musician, delivered over to the solitude of the interval---where the old coherent world of tonality together with the hard dodecaphonic world that produced its truth are scattered into unorganized bodies and vain ceremonies---can only heroically repeat, in his very works: `I go on, in order to think and push to their paradoxical radiance the reasons that I would have for not going on.'\footnote{Badiou,  `Scholium: A Musical Variant of the Metaphysics of the Subject', p. 89.} 
\end{quote}
Here, Badiou precisely delineates the situation in which so called `art music' or `contemporary music' is created and received today, where the only two main options seem to embrace either the joyful and permissive attitude towards mixing genres and styles now commonly ascribed to \emph{postmodernism} or the desolate notion of \emph{modernist} aesthetics that to this day heroically stands in `life support' for more than thirty years. These two positions also seem unable at present time to inspire a profound change in the way we create, perform, perceive and think about music; nor to respond to the original premise of the \emph{modernist} vision of the musical avant-garde, which establishes a connection between new forms of musical and political subjectivity. 

Ranci\`{e}re's analysis gives us strong theoretical tools that can help imagining new ways of reinvigorating the \emph{modernist} idea of the avant-garde in music without falling back to the misunderstandings that led to the `crisis of modernity.' Nevertheless, Ranci\`{e}re's notion of the \emph{avant-garde} is considerably different from the conventional one, and in order to understand his definition and relate it to music, it is important to separate it from its former association to a particular movement in music history. Even though the idea of the avant-garde in music emerged as it became associated to a group of `modernist' composers, the concept remains useful to us now only as a way of understanding the importance of the \emph{aesthetic regime} in the relationship between music and other types of subjectivity and forms of thought. To avoid further misunderstandings, one also needs to take special care and remember the clear differentiation Ranci\`{e}re makes between the \emph{strategic} and \emph{aesthetic} types of avant-garde.

\subsection{The \emph{Strategic} and \emph{Aesthetic} Types of Avant-garde in Music}

The \emph{strategic} type of avant-garde as manifested in music is one that can be associated to a particular group of people (composers, performers, critics and other people who make, think and/or listen to music), musical institution or movement that consolidates a type of subjectivity. It is important to remember that a common ideological position is what triggers the conception of this type of group.\footnote{Slovoj Zizek has repeatedly emphasized how ideology is not an abstract notion or theory one simply ascribes to, but a type of subjectivity that is reflected in the way we act, on how we behave and carry ourselves on a day-to-day basis. Therefore, a musical `movement' doesn't necessarily have to be one in which there is a `conscious' or openly declared agenda that follows a particular position of objectified consensus.} On the other hand,  the \emph{aesthetic} type of avant-garde as manifested in music is that which---through new ways of thinking and making music as expressed by the creation of new musical forms and structures---has the capacity to inspire and encourage new forms of thought about the life to come.  Furthermore, it is crucial that the \emph{strategic} type of avant-garde is not confused with the \emph{aesthetic} type in as much as it will lead to further misunderstandings within the music-world. 

It is important to note that one can find these two types of avant-gardes both in the musical and political spheres (as well as in the other artistic disciplines). Additionally, as they manifest themselves in music, the \emph{aesthetic} and \emph{strategic} types of avant-garde are intrinsically related; but only in as much as music is concerned.  This relationship becomes evident in the causality that exists between musical groups, institutions and movements; and the creation and reception of music. The \emph{strategic} avant-garde as manifested in music is therefore useful to the political sphere only as much as it contributes to the \emph{aesthetic} avant-garde---specifically as it provides a platform for the creation of `new sensible forms and structures.' Hence, the way in which the two types of avant-gardes dwell within music can not be directly compared to the way in which they reside in politics. Here lies another vital point one can induce from Ranci\`{e}re's enquiry: the \emph{strategic} type of avant-garde manifests itself \emph{differently} in music as it does in politics. From this, one can conclude for example that the activism of a musician or group of musicians as they become directly involved in politics does not reflect a relationship between music and politics, but only the involvement of a group of people---which happen to have the same occupation---in a political movement. The true relationship between music and politics is rather reflected in the \emph{aesthetic} type of avant-garde. This argument makes evident why it is misleading to attempt to identify a movement with concerns that are specific to music with a particular political affiliation or party. The position put forward by some critics of \emph{modernism} in music---which concludes that the emancipatory project which seeks the autonomy of music leads to totalitarianism---is therefore flawed.  

Moreover, I will claim that it is very important to consider the intrinsic relationship between the two types of avant-gardes, exclusively as they manifest themselves within music. The basis of this way of thinking stems from the assertion that the \emph{strategic} type of avant-garde has a considerable effect on the \emph{aesthetic} type in numerous significant ways. The impact that musical movements, institutions, ensembles and other organized groups of musicians and people dealing with music, have on the actual musical results, is often underrated. Too often, people involved in creating (particularly composers in my experience) and experiencing music avoid or forget how these strategic forms of collectivity condition and influence the aesthetic result. I will even go as far as to suggest that, in music, the type of subjectivity that is synthesized in the \emph{strategic} avant-garde is reflected or `embodied' in the \emph{aesthetic} avant-garde. That is to say, the ideology of the people involved in the creation, presentation and dissemination of music is expressed in the musical modes of action, production, perception and thought. Furthermore, the notion that the composer is the only person whose ideology is reflected in the music and that the \emph{musical work}\footnote{See Lydia Goehr, \emph{The Imaginary Museum of Musical Works: An Essay in the Philosophy of Music}, Oxford: Oxford University Press, 2007, for a thorough discussion on the philosophy of musical works.} is the only carrier of meaning---an idea that up to this moment is still widespread in western culture---is also misleading. To avoid misunderstandings, I will introduce the notion of a \emph{musical result} (as opposed to the more limited concept of \emph{musical work}) as that which describes the complex set of percepts given by all aspects of a musical experience. These include for example: all sorts of aural and visual elements in music performance; the space and time in which music is performed; the way in which music is presented to the audience (including their role and participation in the musical experience); different modes of action in performance (performance practice) and composition (act of composing); the relationships established between composer, performer and audience; the context (cultural, sociological, political) in which music is presented; the way music is created, consumed and distributed; \emph{etc}. A particular kind of \emph{musical result} consequently discloses a type of collective subjectivity which encompasses the ideology of the people involved in the music.\footnote{I am not implying however that the ideology of \emph{all} the people is represented \emph{equally} in the \emph{musical result}. The question of how much an individual is represented widely depends on the role they take within the \emph{musical result} and the audience's interpretation of it.} Additionally, within the \emph{musical result} lies a system of elaborate symbols that synthesizes the relationships between the people involved in the collective act of music-making.

\subsubsection{\emph{Musicking}}

According to Christopher Small, the set of complex relationships that are formed between people involved in music is that which gives meaning to music. His interest lies particularly on the collective action surrounding music and defines this activity as \emph{musicking}. 
\begin{quote}
The act of \emph{musicking} establishes in the place where it is happening a set of relationships, and it is in those relationships that the meaning of the act lies. They are to be found not only between those organized sounds which are conventionally thought of as being the stuff of musical meaning but also between the people who are taking part . . . relashionships between person and person, between individual and society, between humanity and the natural world.\footnote{Christopher Small, \emph{Musicking: The Meanings of Performing and Listening}, Middletown, Connecticut: Wesleyan University Press, 1998, p. 13.}
\end{quote}
By giving priority to the verb \emph{to music}, as opposed to the noun \emph{music}, he also questions the notion of the \emph{musical work} and gives emphasis to the human action of \emph{musicking}. Small argues that music is not an object and that \emph{musical works} only give material for the musicians to perform, in contrast to the notion (developed as a consequence of western concert music) of performance only as a presentation of a \emph{musical work}. He also defines the verb \emph{to music} to include any type of action that contributes to a musical performance, which includes performing, listening, practicing, composing and dancing. He goes as far as to include actions such as selling and collecting tickets and cleaning the concert hall after a performance within his notion of \emph{musicking}. Therefore, \emph{musicking} encompasses all social relationships and actions that are related to music-making. Furthermore, he argues that \emph{musicking}, together with speaking, are characteristics that are at the very core of what makes us human.
 \begin{quote}
I am  certain, first, that to take part in a music act is of central importance to our very humanness, as important as taking part in the act of speech, which it so resembles (but from which it also differs in important ways), and second, that everyone, every normally endowed human being, is born with the gift of music no less than with the gift of speech.\footnote{Ibid., p. 8.}
\end{quote} 
Recent scientific studies in a variety of specialities including neuroscience, psychology, archaeology, anthropology and cognitive musicology have also pointed towards the same hypothesis. The idea put forward by Steven Pinker that music is `auditory cheesecake'---that it is only a byproduct of evolution and has no biological value for humans\footnote{See Steven Pinker, How the Mind Works, details missing...}---has been challenged recently within the scientific community.  These studies have shown how music plays an important role, amongst other things, in human communication, social bonding, cooperation, sexual selection, conveying emotions, phycological \mbox{well-being}, development of coordination and motor skills, expression of empathy, communication between infants and parents and exercising intelligence.\footnote{See Steven Mithen, \emph{The Singing Neanderthals: The Origins of Music, Language, Mind and Body}, London: Phoenix, 2006, for an overview of these studies.} In addition, various theories have emerged regarding the relationship between music and language; some of them even suggesting that \mbox{`proto-language'}\footnote{Mithen prefers the term `Hmmmm' over `proto-language as ...} (the predecessor of language) was a pre-linguistic, non-verbal form of communication that was a `musical' form of action and thought.\footnote{Ibid., pp. 147-150.} It appears that language and music have a similar evolutionary \mbox{starting-point} and the common purpose of communicating emotion and meaning through sound. From this research one can infer that Small is correct in suggesting that \emph{musicking}, like speaking, is at the core of being human and performs important social, cultural and biological functions. 

\hypertarget{musdef}{}
\subsection{The Definition of Music and the \emph{Ethical Regime}.}

The important functions music performs in the development of individuals and the way in which they establish and nurture relationships within a community is what defines music as a vital human act. Perhaps this is the reason why in the musical domain---going back to Ranci\`{e}re's notion of `the regimes of art'\footnote{See \hyperlink{artregimes}{pp. 4-8}.}---music is still defined as such within the \emph{ethical regime}. In other words, if one goes back to the question of why within music there is no change of identification with the break between the \emph{ethical} and \emph{poetic} regimes; I will suggest that it is because there is a strong ethical core implicit in the very meaning of \emph{what music is}. That is to say, as opposed to the definition of the other arts, the definition of music has been tied to the ethical functions that it performs for individuals and their communities. It is worth mentioning that only dance, like music, can also be defined as such within the \emph{ethical regime}, which points towards the deep-rooted relationship between both disciplines. On the contrary, other artistic disciplines including `fine' art, poetry and theater are identified as such only with the break between the \emph{ethical} and \emph{poetic} regimes.   

The ability that human beings have to communicate and perceive emotion and meaning through music is also tied to music's identification and to the ethical functions it performs. It is by no coincidence that already in Ancient Greece, Aristotle observed that music has an immense power to change people's state of character and that different types of music affect audiences in different ways. According to Aristotle, music represents various types of emotions and actions that closely resemble those that the listener undergoes in reality as a result of the performance.\footnote{See Aristotle, `The Aims and Methods of Education in Music' in \emph{Politics}, Trans. Ernest Barker, Oxford: Oxford University Press, 1995, pp. 309-310.} It is as a consequence of this link between music and human experience, emotion and action that communities have attempted to regulate and evaluate music according to the ethical functions it performs. One could consequently argue that music that lies within the \emph{ethical regime} is evaluated for its ability to affect people in a way that is considered appropriate by the community, given a particular situation. This argument also points towards one of the reasons why labeling music as different `styles' or `genres' seems to be a dominant practice within communities: by knowing what kind of music to expect from a specific `style', it is possible to anticipate the type of experience the audience will go through. This is also one of the reasons why innovation in music has been discouraged and even censured by communities for centuries. The modification of musical styles within the perspective of the \emph{ethical regime} implies an unexpected change in ones experience and a potential threat to the community's consensus of what is considered to be the appropriate way in which people are to be affected by music. Furthermore, innovation in music has been perceived as a political threat in the past since new forms of music produce new experiences that might stimulate behavior outside the political order. 

Plato, in his \emph{Republic} already warns about the danger that innovation in music might pose to the order of the State:
\begin{quote}
Put briefly, then, those charged with care of the city must hold fast to this, so that the city may not be corrupted unawares; but beyond all else, they must guard against innovation in gymnastic and music contrary to the established order, and to the best of their ability be on guard lest when someone says that people care more ``for the newest song on the singer's lips,'' the poet may be understand to mean not new songs but a new style of singing, and to comment it. One must not praise such a thing, nor so interpret the poet, but guard against changing to a new form of music, as endangering the whole. For styles of music are nowhere disturbed without disturbing the most important laws and customs of political order---as Damos says and I believe.\footnote{Plato, `Music and the Constitution' in \emph{The Republic}, Trans. R.E. Allen, New Haven: Yale University Press, 2006, p. 117}
\end{quote}
Therefore, the Platonic view regarding innovation in music is that it is threatening to the social agreements and political organization of the State. Even though the idea that innovation in music might endanger the political and social contracts of the community today might seem hard to imagine, it still gives us a clue towards an attitude that up to this day is still widespread, that is: that innovation in music regarding its own rules, hierarchies, subject matter and genres is still received with reservation, suspicion and even fear amongst the wider community (if compared to the visual arts for example). In my opinion, this is due in the most part for to two main reasons. First, considering the implication that music performs certain ethical functions, innovation can be seen with skepticism as it could lead to confusion, uncertainty and even irritation, if music ceases to perform the functions expected by the community successfully or does so less efficiently. Secondly, given the immersive and participatory (either by listening or performing) aspects implied by the definition of music that establishes a link between music and human action and experience, innovation in music can be associated with new and unpredictable experiences and behavior. Therefore, it is not surprising that some people would be distrustful in allowing themselves experience something they are not familiar with or are uncertain about.\footnote{On a related note: according to recent research in cognitive science, most people stop acquiring new musical tastes by the time they are around twenty years old. This might be as a result that as people grow older, they seem less open to new experiences. See Daniel Levitin, `My Favorite Things' in \emph{This is Your Brain on Music: Understanding a Human Obsession}, London: Atlantic Books, 2006, pp. 231-233.}

\subsection{An Ethical Function within the \emph{Aesthetic Regime}?}

Going back to Ranci\`{e}re's notion of the regimes of art, if one considers the ethical core implicit in the definition of music simultaneously with music that falls within the \emph{aesthetic regime}, one might run into a deadlock: if music is to be evaluated \emph{only} by the functions it already performs within the community (and innovation in music is seen as a disruption of these functions), music that lays within the \emph{aesthetic regime} appears as having no apparent noble purpose. To resolve this problem one needs to point towards the relationship that exists between music and other forms of human endeavor. If music is evaluated and appreciated for its capacity to inspire new ideas, opinions, believes and desires, then one can argue that their is an ethical position implicit in music that falls within the \emph{aesthetic regime}. In other words, their is an ethical function in-itself in breaking with previous models of music making and in questioning the very notion of \emph{what music is}. This function is precisely that of imagining and experiencing through music, new forms of action, production, perception and thought.

Nevertheless, the establishment of the \emph{aesthetic regime} in music, which redefines the `musician' as a practitioner of whatever falls into the category of `music', has still not been spread out through the wider community. The reason, I believe, is that the agreement of trust between the wider community and the musical avant-garde has been weakened as a consequence of the practice of some musicians that can be associated with the notion of \emph{modernism} (mainly, those seeking music's `purity'  in composition through a militant anti-mimetic attitude and those who only advocate `authenticity' and `sterility' in performance practice). These practices have also generated an attitude commonly held by many musicians today, which avoids addressing the most basic ethical functions that the community associates to music while pursuing only their individual musical priorities. If the \emph{aesthetic regime} in music is to be acknowledged and appreciated widely, an agreement of trust needs to reestablished between the musical avant-garde and the community. Considering the ethical core implicit in music's definition, it is likely that the community will be unwilling to be open to new musical experiences if they fear that the ethical functions music already performs within the community will be disrupted or negatively altered. Therefore, this agreement needs to demonstrate that the purpose of creating new music is not to betray its ethical functions, but to inspire and experience new forms of subjectivity---\emph{and this in-itself has an underlying ethical function}. Additionally, if this agreement with the wider community is to be reached, it needs to be embedded within the \emph{musical result} and cannot only be expressed theoretically through verbal and written forms of public dissemination.

\subsection{Strategic Views on Aesthetic Forms}

If a positive redefinition of music is to take place, and an agreement of trust to be reestablished between the musical avant-garde and the wider community, it is crucial to examine the fundamental aspects of how music is created, performed, presented and disseminated today. This includes a significant revision and modification of the \emph{strategic} forms of collectivity in music. In other words, in order to reinvigorate (within the musical sphere) the \emph{aesthetic} type of avant-garde, the \emph{strategic} type of avant-garde also needs to be rethought and reworked. Furthermore, if the agreement of trust between the musical avant-garde and the community is to be regained, I believe it is important to consider the ethical core implicit in the definition of music in parallel with a strong desire towards innovation and change in all aspects of music-making. In other words, while acknowledging the audience and their perception of what the fundamental ethical functions of music are---by making them experience something that they would associate with their idea of music-making within the \emph{musical result}---at the same time challenging these very notions and putting into question the fundamental aspects of music-making. If one subscribes to this position, one should also consider the role musical groups, institutions, ensembles, industry and movements might have in the \emph{musical result} one is involved with, in order to determine whether these groups might help in the establishment of new \emph{aesthetic} forms. Moreover, it is vital to consider the audience as well as the context, time and space where the music is to be presented as this too has a direct causality with the \emph{aesthetic} result and its visibility, and plays a significant part in the disclosure of a particular type of experience. 

Additionally, I believe that the creative process in music should also involve devising and composing these \emph{strategic} aspects of music-making into the \emph{musical result} by creatively reworking the modes of performance, composition, presentation and dissemination of music and rethinking the relationships between composer, performer and audience. Innovation within the \emph{strategic} avant-garde can be achieved through many different approaches and might involve a diverse set of practices. Some examples of how these strategies may be used as a creative tool are herewith described:
\\Composers and performers may have a role determining the type of musicians and ensembles they collaborate with---one may contemplate the possibility of collaborating with musicians from different backgrounds and traditions, whether they are trained to read notation or  they come from an improvisation background. One might work with traditional (already existing group of musicians from a specific tradition) or mixed (musicians from different backgrounds) ensembles and devise how its members may interact with each other (making decisions on the performance dynamic). Communicating with other musicians through a musical score or by other forms of transmission could also be considered as creative decisions. The use of different types and combinations of scores (traditional, graphic, etc), practicing and rehearsal strategies and the use of aural and visual cues might also be taken into account. New strategies in the act of composing a score (using new compositional processes, algorithms or tricks) or an electro-acousitc composition (different ways of recording, triggering and processing audio) may also be examined. The specific venue (considering its historical and social context as well as its acoustic and visual characteristics), clothes, styling, lightning and other theatrical elements in a performance as well as the role of audience and its relationship to the \emph{musical result}, are decisions that one might contemplate as belonging to the musician's creative process. Finally, choosing the way in which music is documented, produced and distributed---by choosing for example the type of media (video/audio recordings, printed CDs, MP3s, digital/printed scores, etc), how it is produced (low/high budget, produced at a professional or home studio, low/high definition, etc), combined (audio/visual, acoustic/electronic, printed/digital) and disseminated (through the internet, printed through a publisher or CD label, self-published, etc)---may also be seen as part of the musician's creative output. 

If these strategies are used creatively, they can be instrumental in radically changing the ways in which we make and experience music. Moreover, they have a direct impact on the \emph{musical result} and might contribute---if carefully examined and put into practice---to the emergence of new aesthetic forms. I believe `art music' or `contemporary music' can be positively redefined through the sensible use of these strategies and without falling back to the anti-mimetic stand commonly ascribed to \emph{modernism} or the permissive attitude which doesn't seek to achieve anything new that is associated to \emph{postmodernism}. At the same time if these strategies are used reasonably, they can also help strengthening the agreement of trust between the musical avant-garde and the wider community.

\section{The Role of Technology in New Musical Strategies}

New technologies may have a vital role in the creative use of the \emph{strategic} aspects of music-making and consequently in the creation of new \emph{aesthetic} forms. However, technology is not often used with the purpose of redefining \emph{what music is}, its own rules and subject matter. Today, technology is more often used to create music---going back to Ranci\`{e}re's terminology---that lies within the \emph{ethical} and \emph{poetic} regimes. Put briefly, new technologies are often produced as tools that facilitate the creation of music that fits with previous models of \emph{what music is supposed to be} and preconceived notions of the functions it performs. It would be pointless, however, to ignore the efforts that have been previously attempted to use and create technology with the purpose of breaking with already-existing-models of music-making and therefore contributing with the establishment of the \emph{aesthetic regime} in music. The technological developments of the twentieth and twenty-first centuries have also inspired and motivated the musical avant-grade to get involved and work with these new technologies. Nevertheless, the same misunderstandings and misconceptions that are ascribed to the notions of \emph{modernism} and \emph{postmodernism} have been embraced in thinking about and implementing technology in music.\footnote{A survey of how these notions have influenced the thought behind the use of technology in music is beyond the scope of this commentary. However, I think this topic deserves an extended study of its own.}  For this reason, I will attempt to put forward some ideas of how to think about, create and use technology in music, considering the aesthetic preoccupations I have previously elaborated.

\subsection{Technological vs. Musical Innovation}

Before discussing my views on how technology might have an important function in rethinking musical strategies, I would like to examine some problems that might arise regarding the use of recent technology in music. As a musician, one of my concerns regarding the relationship between technology and music is that on many occasions scientific innovation and technological curiosity are given priority over musical creativity and aesthetics. Luciano Berio has eloquently expressed the same position:
\begin{quote}
If in the past---even the distant past---music was often the testing bench and the stimulus for scientific research, and thus music tended to draw scientific knowledge to it, in more recent years you get the impression that it's now science that draws music to it and takes possession of it. Indeed, you often get the impression that a scientific creativity applicable to music has substituted itself for musical creativity, and that musical thought has regressed to the level of the (invariably squalid) opinions that an electronic engineer from Bell Telephone or a Stanford ``software man'' may have about music.\footnote{Luciano Berio, \emph{Two Interviews with Rossana Dalmonte and B\'{a}lint Andr\'{a}s Varga},  Ed. and Trans. David Osmond-Smith, London: Marion Bowars, 1985, p. 121.} 
\end{quote}
The attitude of giving more importance to technological (as opposed to musical) innovation while creating music has also increased with the complexity and development of the tools themselves. Scientists and technologists often create music with the sole purpose of demonstrating new developments in music technology. Additionally, musicians that are interested in using technology to a higher level of sophistication very often need to immerse themselves in intricate technological subjects. These circumstances can be misleading for the musician if his priorities shift from a position in which technology is researched and developed for its creative potential in music, to a position in which technological innovation becomes the driving force behind musical creativity.  The shift of attention might even happen without the musician's awareness as a consequence of the effort one needs to go through in understanding the complexity of the technological tools and research developed in this field. This can be deceiving and even `dangerous' if music becomes just a showcase of new technological advancements. 

The experience gained by musicians during the second half of the twentieth century who worked closely with technology can also be very valuable to us today as a warning of the possible problems that working with technology might lead to. Looking back at Berio's account of his experience on this issue, one can grasp how the notion that new technological developments lead to important musical progress is erroneous. On his account, Berio describes how the advancements which permitted the creation of new sounds with electronic means did not in-itself produce any meaningful musical results. 
\begin{quote}
Thus many of the more sensitive musicians quickly realized that it was as easy as it was superfluous to produce new sounds that were not the product of musical thought, just as it's easy nowadays to develop and `improve'' the technologies of electronics music when there are devoid of any real and profound \emph{raison d'\^{e}tre}.\footnote{Ibid., p. 122.}
\end{quote}
He goes on to describe how music that was motivated by technological developments instead of musical thought resulted in a spectacle that did not address the complex set of relationships and conventions that take place in music.
\begin{quote}
It was recognized, for example, that the spectacle of a public gathered together to listen to loudspeakers was not a particularly cheerful one, and that, yet again, the experience of public musical listening was made up of many different conventions, and was rooted in many different aspects of social and cultural life: it was not made up merely of a piece, a musical object to listen to, even if it proposed ``new sounds''. By its very nature, a piece of music by itself cannot easily transform listening conventions and socio-musical relations in general.\footnote{Ibid., pp. 122,123.}
\end{quote}
The lesson to learn from Berio's statement is clear: musical and technological innovation are inherently different from each other and if one's interest lies in creating music, one needs to guide technological interests and development with priorities that will be relevant to the desired \emph{musical result}. That is not to say of course, that scientific research or technological development regarding music is not valuable. On the contrary, my position is that technology can have a vital role in musical innovation if it is developed with a critical approach and considering the complex social, cultural and philosophical aspects inherent in music's definition. Moreover, if technology is developed imaginatively with the purpose of creating new musical strategies for the future, it might help reshaping the way in which we make and experience music.

\subsection{Reshaping Musical Strategies Through Technology}

Even though technology may play a key role in rethinking many aspects that form part of a \emph{musical result}, here I will focus specifically on new strategies concerning the relationships between composer, performer and audience. Therefore, I am not going to go into detail in subjects that are not reated to this specific area of interest as this would be out of the scope of this commentary. Nevertheless, I belive that there is huge potential and work to be done in these areas, which include concerns such as how technology may radically change the way in which musical institutions operate; the visual elements related to the performative aspects of music; how music is recorded, distributed, advertised and consumed. However, what I will concentrate on here is how technology brings a unique opportunity to envision new compositional and performative strategies based on reshaping relationships that have been established traditionally through compositional and performance-practice conventions. I will start by examining the possibilities technology could bring in revising the way in which musical knowledge is transfered by imagining a new type of score that would combine oral and visual traditions within a multimedia experience. Then, I will...

\subsubsection{Scoring Strategies for the Digital Age}

By now, much has been written about the limitations and advantages of the traditional score as a form of communication between composer and performer in western music.\footnote{Authors like Christopher Small, Trevor Wishart, Simmon Emmerson and many others have written extensively about the limitations of the western notation, questioning the score as the only mode of communication between composer and performer and the notion that within the score lies the musical work's ontology.} Through research in \mbox{ethno-musicollogy} and other music practices that incorporate improvisation, an increasing attention has been given to other forms of knowledge transfer in performance-practice that do not include a written score. These might include oral traditions in which music is transferred from one generation to another through a master-apprendice relationship or the increasing practice of studying recordings as a method of learning a particular song, style, genre or performance-practice. It has also been argued that the score is a medium that is highly individual and `isolates' the performer not only from the audience but also when playing within a group of musicians.\footnote{} On the other hand, the idea of using notation has been defended as well for its capacity of capturing complex musical ideas and thoroughly worked structures, providing points of reference and facilitating synchronicity.\footnote{} My position regarding this problem is that the score is still a valuable tool in communicating with musicians trained within western tradition and it is worth expanding the notion of the score to include new strategies that can be developed through technology that might enhance or facilitate communication between composer and performer. In this respect, I completely agree with Simon Emmerson, who argues that technology can serve a tool in generating new forms of notation that can encapsulate different forms of transferring musical knowledge. 
\begin{quote}
But we have one new invention which may hinder and help our endeavor: the computer. Its power was rapidly applied to wester music in all the forms we have discussed. Composition, analysis, transcription, sound production, processing, storage and distribution are all now in one way or another within its domain. . . . An unaddressed need remains: the development of more flexible notation systems; these may also be stimulated by the development of a new generation of music interfaces. . . . We should dream of a technology which bypasses some of theses constraints: a combination of ear and eye---a new `superscore.' . . .\footnote{Simon Emmerson, `Crossing cultural boundaries through technology?'  in \emph{Music, Electronic Media and Culture}, Aldershot: Ashgate, 2000, pp. 121-122.} 
\end{quote}
Emmerson's idea of a `superscore' combines oral and visual forms of communication within an multimedia object combining traditional notation, extended notation, recordings of example material from the live performer, electroacoustic materials, software for performance, patches for live electronic treatment, examples of live electronic treatment, an example recorded performance, written and spoken commentary, video performance material, video example material and graphical material.\footnote{Ibid., pp. 128-129} 

Taking Emmerson's idea further, one could easily imagine the `superscore' as a package that combines performance materials with documentation (including video tutorials, audio examples (sampled mock-up performances or real performances), recordings, interviews, \emph{etc.}) residing on the internet. Additionally, with the increasing accessibility of laptops, one could easily imagine replacing a score that is printed on paper, with one that is displayed on a computer monitor. This would bring the opportunity of exploring the potential to communicate musical meaning through a computer display, which would add movement to the expressive palette of a conventional score. By using animated graphics, scores, pictures, as well as other types visual cues and timed written directions, the composer could enhance the way in which he communicates musical ideas and knowledge through the computer display. In addition, the performer could receive other types of audio information through headphones complimenting the visual input with an `aural score.' This could comprise from spoken directions and sounding cues (click tracks, reference pitches, etc) to recordings of acoustic or electroacoustic music that the performer would have to or react to or improvise with. With the development of real-time processing technologies and generative algorithms, the notion of a \emph{fixed} score could also be contested by a score that is \emph{dynamic}, thereby creating a composition that may change its content (pitches, rhythms, etc) each time it is performed. Real-time scoring could be explored further by combining elements of real-time animation and graphics display with new advancements in machine listening technologies, thereby generating a score that responds to the sonic and acoustic context of a specific performance and space.

With the increasing popularity of new types of interfaces and gadgets, the `superscore' could be implemented in more portable devises like the iPad or iPhone. 

In addition  to the visual elements of the score, aural elements could also be

I would even go as far as to suggest that through technology, new strategies in music performance can be imagined by which the performer through might produce a result that not only would create new musical forms but also would be unique to the particular way in which the technological tools are implemented. In other words, the composer by imagining . . . .
 
Real-Time computer processing allows the possibility of using the audio signal (as well as other information - like MIDI) from several live performances simultaneously as building blocks for a composition.

Mention strategies: Algorithmic Score, Headphones, etc.

\subsubsection{Crossing Cultural Borders?}

The use of computer technology has also fostered new collaborative possibilities between performers of different cultures.


Musicians of different backgrounds (improvisation and notated music) and traditions (Western and non-Western) may now share the stage simultaneously and productively through technology; in spite of previously incompatible performance conventions.

In his article \emph{Crossing Cultural Boundaries through Technology}, Simon Emmerson already gives us many clues of how technology can be used and developed to change the way in which we create, perform and listen to music. His concern is specific to...

A discussion of Simon Emmerson's Crossing Cultural Boundaries through Technology. 
\v{Z}i\v{z}ek's view of Multiculturalism. 

\subsection{Interactivity and Interpassivity?}

\begin{quote}
Interpassivity, like interactivity, thus subverts the standard opposition between activity and passivity: if in interactivity (or the cunning of Reason), I am passive while being active through another, in interpassivity, I am active while being passive through another. More precisely, the term interactivity is currently used in two senses: (1) interacting with the medium, that is, not being just a passive consumer: (2) acting through another agent, so that my job is done, while I sit back and remain passive, just observing the game. While the opposite of the first mode of interactivity is also a kind of interpassivity, the mutual passivity of two subjects, like two lovers passively observing each other and merely enjoying each others presence, the proper notion of interpassivity aims at the reversal of the second meaning of interactivity: the distinguishing feature of interpassivity is that, in it, the subject is incessantly (frenetically even) active, while displacing on to another the fundamental passivity of his or her being.\footnote{From The Fantasy in Cyberspace by Slavoj \v{Z}i\v{z}ek}
\end{quote}



This can be achieved for example through new ways of engaging in the act of composition, new dynamics in improvisation, new ways of presenting music and new spaces in which it can be presented, new ways of documenting the act of composition and performance, new forms of scoring

How? Appropriation, Technology

How to reestablish the agreement? Appropriation/Ideology 

\section {Appropriation and Ideology in Music}

\begin{quote}
The contemporary era constantly proclaims itself as post-ideological, but this denial of ideology only provides the ultimate proof that we are more than ever embedded in ideology. Ideology is always a field of struggle---among other things, the struggle for appropriating past traditions.\footnote{Slavoj Zizek, ``It's Ideology, Stupid!'', in \emph{First As Tragedy, Then as Farce}, London: Verso, 2009, p. 37.}
\end{quote}
Start on appropriation and past traditions...

My approach to appropriation??
\begin{quote}
Characters on stage should be flat, like clothes in a fashion show: what you get should be no more than what you see. Psychological realism is repulsive, because it allows us to escape unpalatable reality by taking shelter in the ``luxuriousness'' of personality, losing ourselves in the depth of individual character. The writer's task is to block this manoeuvre, to chase us off to a point from which we can view the horror with a dispassionate eye.\footnote{Elfriede Jelinek, quoted in Slavoj Zizek, \emph{First As Tragedy, Then as Farce}, p. 40.}
\end{quote}

\begin{quote}
Consumption is simultaneously also production, just as in nature the production of a plant involves the consumption of elemental forces and chemical material\footnote{Karl Marx}
\end{quote}

\begin{quote}
Starting with the language imposed upon us (the system of production), we construct our own sentences (acts of everyday life), thereby reappropriating for ourselves, through these clandestine microbricolages, the last word in the productive chain.\footnote{Nicolas  Bourriaud, \emph{Postproduction. Culture as Screenplay: How Art Reprograms the World}, New York: Lukas and Sternberg, 2005.}
\end{quote}

\subsection{Appropriation and Postproduction in the Digital Age} 

\begin{quote}
By listening to music or reading a book, we produce new material, we become producers. And each day we benefit from more ways in which to organize this production: remote controls, VCRs, computers, MP3s, tools that allow us to select, reconstruct, and edit. Postproduction artists are agents of this evolution, the specialized workers of cultural reappropriation.\footnote{Ibid. p. ?}
\end{quote}

\begin{quote}
Throughout the eighties, the democratization of computers and the appearance of sampling allowed for the emergence of a new cultural configuration, whose figures are the programmer and DJ. The remixer has become more important than the instrumentalist, the rave more exciting than the concert hall. The supremacy of cultures of appropriation and the reprocessing of forms calls for an ethics: to paraphrase Philippe Thomas, artworks belong to everyone. Contemporary art tends to abolish the ownership of forms, or in any case to shake up the old jurisprudence. Are we heading toward a culture that would do away with copyright in favor of a policy allowing free access to works, a sort of blueprint for a communism of forms?\footnote{Ibid. p. ?}
\end{quote}

%\subsection{critisisms} 
\subsection{The liberal-comunists: Open Source, etc.} 

There is no music by John Oswald on the net free to download. Hypocrisy from the appropriator? Or does he fall into the logic of late-capitalism - �no communism of forms�? �I plunder but don�t plunder me. Or, at least not for free��? 

I propose an attitude towards music appropriation similar to that of hacker communities and the open source initiative. Not with the purpose of suggesting a communist utopia, but of being consequent with my creative process. By giving away my music, recorded sounds and experiments, code, etc, through the net, I will hopefully instigate others to do so as well. If this attitude is followed, it could promote the organization of music cyber communities that would plunder, engage with and promote each other, hopefully producing more subversive types of music.

We are far from the Bourriaud�s utopia. The only people how have access to (artistic) shareware are commoditized people, mostly in western countries. Isn�t the DJ approach towards plunderphonics one that appropriates to make more profit and diminish costs only to thereafter feed back their product into the music industry system?

\subsection{Musical Appropriation through Technology}

I will continue by examining different strategies and practices used in my work that use technology as means to appropriate, derive from and transform existing music by other musicians. It is only logical, considering that music is not an object but a complex set of actions, productions, perceptions and thoughts,\footnote{See pp 13-45 for a discussion regarding my preference of the notion of a \emph{musical result} versus the more widely use concept of \emph{musical work}.} that the act of appropriation of existing music can manifest itself in many different ways and take lots of unexpected guises. Therefore, I will propose that the appropriation of existing music \emph{does not} refer exclusively to `borrowing' or `stealing' from \emph{musical works} by other composers but to . . . . Moreover, when dealing with appropriation, I will claim that there are certain fundamental questions that both music creators and listeners should ask themselves. According to David Mezter, Stockhausen (while referring to \emph{Hymnen}) emphasized the importance of asking the questions of ``what'' and ``how''  regarding the practice of `borrowing' or `quoting' from other music.
\begin{quote}
According to him \emph{[Stockhausen]}, the practice involves a rich exchange between the ``what'' and the ``how,'' that is, the gesture has us hear ``what'' music has been borrowed and ``how'' it has been changed. The more familiar and obvious the ``what,'' the more we are drawn into the ``how,'' and the more captivating the ``how,'' the more we can appreciate anew the ``what.'' It is the ways in which quotation handles the ``what'' and the ``how'' that make it so effective a  cultural agent.\footnote{David Metzer, \emph{Quotation and Cultural Meaning in Twentieth-Centure Music}, Cambridge: Cambridge University Press, 2003, p. 6.} 
\end{quote}
I agree with Stockhausen's claim because . . . 
Nevertheless, I would also add : 
the difference between ``what'' and ``who''
also ``from where''. 
but most importantly ``why,''  
Why = motivations.
The motivations regarding musical appropriation can be very varied and also reflect ideological positions that in many cases reflect more the beliefs and feelings of the appropriator that the appropriated. Therefore, I will attempt to explain my viewpoint regarding the motivations and ways in which I use other music within my own work. In doing so, I will also examine other composers work that deals with musical appropriation in ways that I consider valid, interesting and intriguing.

Technology

Will do so by examining other composers work dealing with this issues... that I find valid, interesting, intriguing, stimulating(?)...  

\subsubsection{Copyrights Violation}
 
\subsection{Scores}


The first strategy considered is Clarence Barlow's concept of \emph{Musica Derivata}, which refers to the idea of transforming existing music with Computer Aided Composition (CAC) tools to create ``music that is compositionally based on other music''\footnote{Clarence Barlow, \emph{Musica Derivata} [CD], hat[now]ART 126, Hat Hut Records Basle, 2000.} This approach seems to take as a starting point mostly notated material (but in some occasions spectral information from recordings) from music by other composers. 

MIDI

\subsubsection{me}

\subsection{Recordings}

	
\subsubsection{Plunderphonics}

Plunderphonics:

John Owald, 1985. ``Plunderphonics, or Audio Piracy as Compositional Prerogative''

Use of audio samples as a technique for composition. 

Different from Musica Derivata in that it appropriates the recording of the original musical source. Information from recording (tibre, rhythm, performace practice, etc) is plundered from the original source to create a new composition.

``As a listener my own preference is the option to experiment. My listening system has a mixer instead of a receiver, an infinitely variable speed turntable, filters, reverse capability, and a pair of ears. An active listener might speed up a piece of music in order to perceive more clearly it�s macrostructure, or slow it down to hear articulation and detail more precisely.''\footnote{John Oswald, ``Plunderphonics, or Audio Piracy as a Compositional Prerogative,'' in \emph{Wired Society Electro-Acoustic Conference}, Toronto, 1985. URL: \href{http://www.plunderphonics.com/xhtml/xplunder.html}{http://www.plunderphonics.com/xhtml/xplunder.html}.}

\subsubsection{Sound Transformations} 					


``With the power of the computer, we can transform sounds in such radical ways that we can no longer assert that the goal sound is related to the source sound merely because we have derived one from the other.'' (T. Wishart)

In my work, sound transformations are used for the transformation of existing music. 

Why transformation of musical sources? Because they may carry complex cultural symbolism. 

The amount of processing can affect our ability to recognize the source sound or musical sample. Therefore, there is a wide palette of derivative music available to us: from the radically processed � less recognizable source � more `abstract' extreme; to the less processed � more recognizable source � more `referential'  and quotation type music.

Performance practice and other sonic characteristics of many original musical sources is lost in the transcription to a fully notated score for ensembles of western classically trained musicians. Many aspects of sound production (intonation, groove, spectral characteristics of instruments/voices, etc) is lost via this process.

Process of derivation and sound transformation is not directly apparent to the audience. The act of appropriation is not transparent.


\subsection{Spectral Information} 

\subsubsection{To generate sc}

\subsection{Computer Code}

Max patches, Computer Code.

\subsection{Real Performances} 


\subsection{Real-Time Plunderphonics}

Appropriation of audio signals from live music performances as material for a new composition

Creates a cognitive dissonance between audio and visuals.

The amount of processing of the audio signals is visible. The more processed the performances are, the more contrasting they will look in relationship with what is heard through speakers.

In contrast to acousmatic tradition, Real-Time Plunderphonics makes the process of appropriation transparent to the audience through the cognitive association between audio and visuals.

Changes relationship with the appropriated Other: The performer becomes an accomplice in the process of appropriation (or themselves). 

Deals with the problematic of the lack of visual clues and theatrical elements in electronic music performance by introducing a dynamic group of live performers and an interesting and unusual visual scenario.  

\subsubsection{Some ideas of how to plunder}

Get to know what and who you are plundering and figure why your are doing so before you decide how to plunder.(Know your performers, their music and why you want to work with them)

Appropriate and plunder yourself. 

Plundering not as central purpose of the creative process, but rather a tool for creating new idiosyncratic audio/visual result. 

Use ``from raw to cooked'' (L\'{e}vi-Strauss) techniques to create a narrative that navigates, in literary terms, between the �real� (actual performance) and the `surreal' (extreme processed audio).

Combinations of Real-Time Plunderphonics, (Real-Time) Musica Derivata and Sound Transformations

Use plunderphones as data: reprogram, not just remix.

Micro and macro plundering.

Use also Non Real-Time tools (Scores, Samples, etc.) if suitable. 

Using plunderphones as data

An example: Use FFT data of your plunderphone to trigger samples of recorded instruments.

\subsubsection{Micro and Macro Plundering}

Microplunderphonics

Plundering just microelements of sound. Not the whole spectrum of the original sound file. 

Generate noise with your plunderphones and use it instead of white noise for sound synthesis


Macroplundering

Appropriate a composition�s form. Use the structure as blueprint for a new composition. 

Use variables of the appropriated piece (pitch, dynamics, etc.) as control structures for new output.



\subsection{Musical postmodernism in the digital age}

resurgence of image / music quotations/references - first as reaction to the anti-mimetic
later with digital technology, easy reproduction, etc, etc => the use of images becomes the same as before the establishment aestetic regime : commodification, capitalism, DJ culture, digital quotations (in hip-hop, sound libaries, etc, etc)

I propose an attitude towards music appropriation similar to that of hacker communities and the open source initiative. Not with the purpose of suggesting a communist utopia, but of being consequent with my creative process. By giving away my music, recorded sounds and experiments, code, etc, through the net, I will hopefully instigate others to do so as well. If this attitude is followed, it could promote the organization of music cyber communities that would plunder, engage with and promote each other, hopefully producing more subversive types of music.

We are far from the Bourriaud�s utopia. The only people how have access to (artistic) shareware are commoditized people, mostly in western countries. Isn�t the DJ approach towards plunderphonics one that appropriates to make more profit and diminish costs only to thereafter feed back their product into the music industry system?

\subsection{Musica Derivata and Plunderphonics}

``A good composer does not imitate; he steals''       I. Stravinsky

Musica Derivata:

``music that is compositionally based on other music'' (K. Barlow) 


\subsection{plunderphomes, ideology and the use of references}

\begin{quote}
While some start up a prolonged lamentation for the lost image, others reopen their albums to rediscover the pure enchantment of images- that is, the alterity of the \emph{was}, between the pleasure of pure presence and the bit of the absolute Other.
\end{quote}
\begin{quote}
Evidence of exhibitions devoter to `images', but also the dialectic that affects each type of image and mixes its legitimations and powers with those of the other tow.
\end{quote}
Plunderphones reflect ideology . . . \v{Z}i\v{z}ek/Adorno but. . . . The artist can present their own view of these references by rearranging them modifying them. The plunderphonics artist doesn't necessarily adheres to the ideology of the appropriated material, but reflects it by the use of the plunderphones - how are they presented, modified, etc?  

\subsection{On Musical Appropriation}

What? 

Code, compositional tecniques, what piece of music? 
Do we plunder from the ``flea market or (the) airport shopping mall''? (N. Bourriaud). From the top 20 list - J. Oswald approach-, or from the hidden CDs at the back of the music store?

Who?

Music Industry? Pop/commercial? Historical (dead composers)? Music from different cultures? 

Appropriation of the Other. What relationship do we want to establish with the Other? Impersonal like the 1st/3rd World relationships?

Liberal multiculturalists approach? ``Other deprived of its Otherness (the idealized Other who dances fascinating dances and has an ecologically sound holistic approach to reality, while features like wife beating remain out of sight�)?'' (Slavoj \v{Z}i\v{z}ek, 2003)

Why?

For the meaning of the cultural object you are appropriating? For it�s symbolism? To suggest a metaphor?

For it�s use? ``Don�t look for the meaning, look for the use'' - L. Wittgenstein - for example for the sonic qualities of the appropriation (intonation, groove, etc.)

How? �

\label{ch:motivation}