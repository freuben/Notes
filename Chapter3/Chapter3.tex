\hypertarget{chapter3}{}
\chapter{Introduction}

In this chapter, I will attempt to examine...

\section{The \emph{strategic} and \emph{aesthetic} types of avant-garde in music}

%re-write this not as different avant-gardes (musical and political) but different manifestations of the two types of avant-garde in music and politics

A central concern that derives from Ranci\`{e}re's analysis is how to reinvigorate the \emph{modernist} idea of the avant-garde in music without falling back to the misunderstandings that led to the `crisis of modernity.' However, in order to understand Ranci\`{e}re's notion of the \emph{avant-garde} and relate it to music, it is important to rethink this concept and to separate it from its former association to a particular movement in music history. Even though the idea of the avant-garde in music emerged as it became associated to a group of `modernist' composers, the concept remains useful to us now only as a way of understanding the importance of the \emph{aesthetic regime} in the relationship between new forms of musical and political subjectivity. To avoid further misunderstandings, one also needs to take special care and remember the clear differentiation Ranci\`{e}re makes between the \emph{strategic} and \emph{aesthetic} types of avant-garde. The \emph{strategic} type of avant-garde as manifested in music is one that can be associated to a particular group of people (composers, performers, critics and other people who make, think and/or listen to music), musical institution or movement that consolidates a type of subjectivity. It is important to remember that the possession of a common ideological position is what triggers the conception of this type of group.\footnote{Slovoj Zizek has repeatedly emphasized how ideology is not an abstract notion or theory one simply ascribes to, but it is a type of subjectivity that is reflected in the way we act, on what we do and how we carry on living our lives. Therefore, a musical `movement' doesn't necessarily have to be one in which there is a `conscious' or openly declared agenda that follows a particular ideology.} On the other hand,  the \emph{aesthetic} type of avant-garde as manifested in music is that which---through new ways of thinking and making music as expressed by the creation of new musical forms and structures---has the capacity to inspire and encourage new forms of thought about the life to come.  Furthermore, it is crucial that the \emph{strategic} type avant-garde should not be confused with the \emph{aesthetic} type in as much as it will lead to further misunderstandings within the music-world. 

It is important to note that one can find these two types of avant-gardes both in the musical and political spheres (as well as in the other artistic disciplines). Additionally, as they manifest themselves in music, the \emph{aesthetic} and \emph{strategic} types of avant-garde are intrinsically related; but only in as much as music is concerned.  This relationship becomes evident in the causality that exists between musical groups, institutions and movements; and the creation and reception of music. The \emph{strategic} avant-garde as manifested in music is therefore useful to the political sphere only as much as it contributes to the \emph{aesthetic} avant-garde---specifically as it provides a platform for the creation of `new sensible forms and structures.' Hence, the two types of avant-gardes abide differently in music as they do in politics. Here lies another vital point in Ranci\`{e}re's enquiry: the \emph{strategic} type of avant-garde manifests itself differently in music as it does in politics. Therefore, the activism of a musician or group of musician as they become directly involved in politics does not express the relationship between music and politics, but only the involvement of a group of people---which happen to have the same occupation---in a political movement. The true relationship between music and politics is rather reflected in the manifestation of the \emph{aesthetic} type of avant-garde in music. This point also explains why it is misleading to attempt to identify a movement with concerns that are specific to music with a particular political affiliation or party.



 the notion put forward by some critics of `modernism' that the equate the emancipation or autonomy of music in the {aesthetic} regime would lead to totalitarianism looses credibility.  

Moreover, I will claim that even though the division that Ranci\`{e}re makes between the \emph{aesthetic} and \emph{strategic} types of avant-garde is useful, it is also important to consider the intrinsic relationships these two types avant-garde might have in music.

 
\begin{quote}
The contemporary era constantly proclaims itself as post-ideological, but this denial of ideology only provides the ultimate proof that we are more than ever embedded in ideology. Ideology is always a field of struggle---among other things, the struggle for appropriating past traditions.\footnote{Slavoj Zizek, ``It's Ideology, Stupid!'', in \emph{First As Tragedy, Then as Farce}, London: Verso, 2009, p. 37.}
\end{quote}
Start on appropriation and past traditions...



My approach to appropriation??
\begin{quote}
Characters on stage should be flat, like clothes in a fashion show: what you get should be no more than what you see. Psychological realism is repulsive, because it allows us to escape unpalatable reality by taking shelter in the ``luxuriousness'' of personality, losing ourselves in the depth of individual character. The writer's task is to block this manoeuvre, to chase us off to a point from which we can view the horror with a dispassionate eye.\footnote{Elfriede Jelinek, quoted in Slavoj Zizek, \emph{First As Tragedy, Then as Farce}, p. 40.}
\end{quote}


- Zizek on Fukuhiama. Crisis of Capitalism. Ecological Catastrophy. Unsustainability. Need for Radical Reform in Politics? Critique of liberal democracy - need for new alternatives.
- ethical regime - defined as music - elaborate. . . Close relationship to culture and musiking.
- aesthetic regime in music - still pending - emancipatory potential?

I will also argue that music has a particular emancipatory potential given its particular position within the artistic regimes.\footnote{See \hyperlink{artregimes}{pp. 4-8} for a discussion on Ranci\`{e}re's ideas regarding the artistic regimes.} 

the confusion Ranci\`{e}re has described caused by the  the two different ideas of artistic subjectivity. 


Therefore, the aim of the body of work here presented is very modest: it is to put forward a set of propositions\footnote{The set of propositions are presented both as musical output and written commentary.} that I hope can be used to radicalize the \emph{strategic} as well as the \emph{aesthetic} idea of the avant-garde in music.


\hypertarget{musdef}{}
\section{The definition of music in the \emph{ethical regime}.}

It is interesting to note that within the visual arts the breaking from the \emph{ethical regime of images} and the establishment of the \emph{poetic regime of art} is what separates the `fine arts' from other modes and techniques of production (of images, shapes, objects, etc), whereas within music there is not such a change in definition. That is to say, in the visual arts this break between \emph{ethical} and \emph{poetic} regimes identifies the arts as such but in music it does not change its identification. Why is it that in the musical domain it is still plausible to call the `ways of doing and making' in both regimes \emph{music}? Why within western culture someone who designs billboards is not considered to be a \emph{fine artist} (it probably would fall into graphic design) while someone who writes jingles for television commercials is still a \emph{musician}? In Chapter 3, I will come back to these question and look at the possible reasons and implications of this difference. However, before drawing any conclusions about the consequences of this disparity, first I will examine the \emph{aesthetic regime of art} to have a better understanding of Ranci\`{e}re's enquiry.

Also Ranciere: Did Music break with the mimetic regime to establish itself in the aesthetic regime? Is it still expected for a musician or composer to do something?


- \emph{aesthetic regime} to performers?? confusion with aesthetic regime again... it is not necessarily that performers have a role in composition...or improvisation...they are still in the representative regime...they still use modes and resemble performance conventions (in improv) ... that does not solve the `problem' and they are not going into the aesthetic regime., for that to happen performers have to reinvent what they do, they should do whatever they consider they should do and what ever they think performers should do.

\section{Technology, Appropiation and Postproduction} 

``Consumption is simultaneously also production, just as in nature the production of a plant involves the consumption of elemental forces and chemical material'' K. Marx

Sound Transformations: 					

``With the power of the computer, we can transform sounds in such radical ways that we can no longer assert that the goal sound is related to the source sound merely because we have derived one from the other.'' (T. Wishart)

In my work, sound transformations are used for the transformation of existing music. 

Why transformation of musical sources? Because they may carry complex cultural symbolism. 

The amount of processing can affect our ability to recognize the source sound or musical sample. Therefore, there is a wide palette of derivative music available to us: from the radically processed � less recognizable source � more `abstract' extreme; to the less processed � more recognizable source � more `referential'  and quotation type music.

Performance practice and other sonic characteristics of many original musical sources is lost in the transcription to a fully notated score for ensembles of western classically trained musicians. Many aspects of sound production (intonation, groove, spectral characteristics of instruments/voices, etc) is lost via this process.

Process of derivation and sound transformation is not directly apparent to the audience. The act of appropriation is not transparent.

Nicolas Bourriaud: Postproduction, 2002.

``Starting with the language imposed upon us (the system of production), we construct our own sentences (acts of everyday life), thereby reappropriating for ourselves, through these clandestine microbricolages, the last word in the productive chain.''\footnote{Nicolas  Bourriaud, \emph{Postproduction. Culture as Screenplay: How Art Reprograms the World}, New York: Lukas and Sternberg, 2005.}

``By listening to music or reading a book, we produce new material, we become producers. And each day we benefit from more ways in which to organize this production: remote controls, VCRs, computers, MP3s, tools that allow us to select, reconstruct, and edit. Postproduction artists are agents of this evolution, the specialized workers of cultural reappropriation.''

``Throughout the eighties, the democratization of computers and the appearance of sampling allowed for the emergence of a new cultural configuration, whose figures are the programmer and DJ. The remixer has become more important than the instrumentalist, the rave more exciting than the concert hall. The supremacy of cultures of appropriation and the reprocessing of forms calls for an ethics: to paraphrase Philippe Thomas, artworks belong to everyone. Contemporary art tends to abolish the ownership of forms, or in any case to shake up the old jurisprudence. Are we heading toward a culture that would do away with copyright in favor of a policy allowing free access to works, a sort of blueprint for a communism of forms?'' (N. Bourriaud)

\subsection{The postmodern condition in the digital age}

resurgence of image / music quotations/references - first as reaction to the anti-mimetic
later with digital technology, easy reproduction, etc, etc => the use of images becomes the same as before the establishment aestetic regime : commodification, capitalism, DJ culture, digital quotations (in hip-hop, sound libaries, etc, etc)

%\subsection{critisisms} 
\subsection{The liberal-comunists: Open Source, etc.} 

There is no music by John Oswald on the net free to download. Hypocrisy from the appropriator? Or does he fall into the logic of late-capitalism - �no communism of forms�? �I plunder but don�t plunder me. Or, at least not for free��? 

I propose an attitude towards music appropriation similar to that of hacker communities and the open source initiative. Not with the purpose of suggesting a communist utopia, but of being consequent with my creative process. By giving away my music, recorded sounds and experiments, code, etc, through the net, I will hopefully instigate others to do so as well. If this attitude is followed, it could promote the organization of music cyber communities that would plunder, engage with and promote each other, hopefully producing more subversive types of music.

We are far from the Bourriaud�s utopia. The only people how have access to (artistic) shareware are commoditized people, mostly in western countries. Isn�t the DJ approach towards plunderphonics one that appropriates to make more profit and diminish costs only to thereafter feed back their product into the music industry system?

 
\section{Computer-Mediated Musicking}
Christopher Small argues that music is not a thing or an abstract concept, but a human activity that he calls \emph{musicking}, meaning all individual and collective endeavors in the process of music making. Moreover, Small questions the notion that a musical work is what gives meaning to \emph{musicking}. 

\begin{quote}
The act of \emph{musicking} establishes in the place where it is happening a set of relationships, and it is in those relationships that the meaning of the act lies. They are to be found not only between those organized sounds which are conventionally thought of as being the stuff of musical meaning but also between the people who are taking part . . . relashionships between person and person, between individual and society, between humanity and the natural world. . . . (Small, 1998)
\end{quote}

The music we compose and perform can convey our thoughts and express our feelings. As listeners we interpret . . . make us feel and think. Empathy. Exchange.  

\subsection{Compositional Stratergies based on reshaping relationships in music making}

\subsection{Reshaping relationships in music making through technology?}

The introduction of electroacoustic resources into live musical performance has changed the relationship between the composer and the performer. 

The use of computer technology has also fostered new collaborative possibilities between performers of different cultures.

Musicians of different backgrounds (improvisation and notated music) and traditions (Western and non-Western) may now share the stage simultaneously and productively through technology; in spite of previously incompatible performance conventions.

Real-Time computer processing allows the possibility of using the audio signal (as well as other information - like MIDI) from several live performances simultaneously as building blocks for a composition.

\subsection{Musica Derivata and Plunderphonics}

``A good composer does not imitate; he steals''       I. Stravinsky

Musica Derivata:

``music that is compositionally based on other music'' (K. Barlow) 

Plunderphonics:

John Owald, 1985. ``Plunderphonics, or Audio Piracy as Compositional Prerogative''

Use of audio samples as a technique for composition. 

Different from Musica Derivata in that it appropriates the recording of the original musical source. Information from recording (tibre, rhythm, performace practice, etc) is plundered from the original source to create a new composition.

``As a listener my own preference is the option to experiment. My listening system has a mixer instead of a receiver, an infinitely variable speed turntable, filters, reverse capability, and a pair of ears. An active listener might speed up a piece of music in order to perceive more clearly it�s macrostructure, or slow it down to hear articulation and detail more precisely.''\footnote{John Oswald, ``Plunderphonics, or Audio Piracy as a Compositional Prerogative,'' in \emph{Wired Society Electro-Acoustic Conference}, Toronto, 1985. URL: \href{http://www.plunderphonics.com/xhtml/xplunder.html}{http://www.plunderphonics.com/xhtml/xplunder.html}.}

\subsection{plunderphomes, ideology and the use of references}

\begin{quote}
While some start up a prolonged lamentation for the lost image, others reopen their albums to rediscover the pure enchantment of images- that is, the alterity of the \emph{was}, between the pleasure of pure presence and the bit of the absolute Other.
\end{quote}
\begin{quote}
Evidence of exhibitions devoter to `images', but also the dialectic that affects each type of image and mixes its legitimations and powers with those of the other tow.
\end{quote}
Plunderphones reflect ideology . . . \v{Z}i\v{z}ek/Adorno but. . . . The artist can present their own view of these references by rearranging them modifying them. The plunderphonics artist doesn't necessarily adheres to the ideology of the appropriated material, but reflects it by the use of the plunderphones - how are they presented, modified, etc?  

\subsection{On Appropriation}

What? 

Code, compositional tecniques, what piece of music? 
Do we plunder from the ``flea market or (the) airport shopping mall''? (N. Bourriaud). From the top 20 list - J. Oswald approach-, or from the hidden CDs at the back of the music store?

Who?

Music Industry? Pop/commercial? Historical (dead composers)? Music from different cultures? 

Appropriation of the Other. What relationship do we want to establish with the Other? Impersonal like the 1st/3rd World relationships?

Liberal multiculturalists approach? ``Other deprived of its Otherness (the idealized Other who dances fascinating dances and has an ecologically sound holistic approach to reality, while features like wife beating remain out of sight�)?'' (Slavoj \v{Z}i\v{z}ek, 2003)

Why?

For the meaning of the cultural object you are appropriating? For it�s symbolism? To suggest a metaphor?

For it�s use? ``Don�t look for the meaning, look for the use'' - L. Wittgenstein - for example for the sonic qualities of the appropriation (intonation, groove, etc.)

How? �

\subsection{Real-Time Plunderphonics}

Appropriation of audio signals from live music performances as material for a new composition

Creates a cognitive dissonance between audio and visuals.

The amount of processing of the audio signals is visible. The more processed the performances are, the more contrasting they will look in relationship with what is heard through speakers.

In contrast to acousmatic tradition, Real-Time Plunderphonics makes the process of appropriation transparent to the audience through the cognitive association between audio and visuals.

Changes relationship with the appropriated Other: The performer becomes an accomplice in the process of appropriation (or themselves). 

Deals with the problematic of the lack of visual clues and theatrical elements in electronic music performance by introducing a dynamic group of live performers and an interesting and unusual visual scenario.  

\subsubsection{Some ideas of how to plunder}

Get to know what and who you are plundering and figure why your are doing so before you decide how to plunder.(Know your performers, their music and why you want to work with them)

Appropriate and plunder yourself. 

Plundering not as central purpose of the creative process, but rather a tool for creating new idiosyncratic audio/visual result. 

Use ``from raw to cooked'' (L\'{e}vi-Strauss) techniques to create a narrative that navigates, in literary terms, between the �real� (actual performance) and the `surreal' (extreme processed audio).

Combinations of Real-Time Plunderphonics, (Real-Time) Musica Derivata and Sound Transformations

Use plunderphones as data: reprogram, not just remix.

Micro and macro plundering.

Use also Non Real-Time tools (Scores, Samples, etc.) if suitable. 

Using plunderphones as data

An example: Use FFT data of your plunderphone to trigger samples of recorded instruments.

\subsubsection{Micro and Macro Plundering}

Microplunderphonics

Plundering just microelements of sound. Not the whole spectrum of the original sound file. 

Generate noise with your plunderphones and use it instead of white noise for sound synthesis


Macroplundering

Appropriate a composition�s form. Use the structure as blueprint for a new composition. 

Use variables of the appropriated piece (pitch, dynamics, etc.) as control structures for new output.

\subsection{Crossing Cultural Borders?}

A discussion of Simon Emmerson's Crossing Cultural Boundaries through Technology. 
\v{Z}i\v{z}ek's view of Multiculturalism. 


\subsection{Interpassivity}

\begin{quote}
Interpassivity, like interactivity, thus subverts the standard opposition between activity and passivity: if in interactivity (or the cunning of Reason), I am passive while being active through another, in interpassivity, I am active while being passive through another. More precisely, the term interactivity is currently used in two senses: (1) interacting with the medium, that is, not being just a passive consumer: (2) acting through another agent, so that my job is done, while I sit back and remain passive, just observing the game. While the opposite of the first mode of interactivity is also a kind of interpassivity, the mutual passivity of two subjects, like two lovers passively observing each other and merely enjoying each others presence, the proper notion of interpassivity aims at the reversal of the second meaning of interactivity: the distinguishing feature of interpassivity is that, in it, the subject is incessantly (frenetically even) active, while displacing on to another the fundamental passivity of his or her being.\footnote{From The Fantasy in Cyberspace by Slavoj \v{Z}i\v{z}ek}
\end{quote}



\label{ch:compamp}