\hypertarget{chapter4}{}
\chapter{Appropriation as Strategy}

This chapter examines musical practices that take an explicit and formalized approach to the use of appropriation as a conceptual, performative and compositional strategy.
However, I will look at musical appropriation first for its wider context and meaning. I will therefore examine how musical appropriation can refer not only to the explicit process by which musical material produced by others becomes ones own, but also how appropriation does not have to be an objective or conscious process. I will also argue that musical appropriation does not need to refer exclusively to musical sounds and abstractions notated in a score, but can also include the appropriation of physical objects like instruments or bodies, as well as methods of composition and performance. In addition, I will also argue that within the process of musical appropriation lies a musician's relationship to the past and the appropriated Other, which at the same time discloses a deep connection to the musician's ideology and the way s/he relates to tradition. Consequently, I will attempt to point at the potential implicit in using appropriation explicitly as a creative strategy. For that reason, I will first look at certain appropriation strategies within the fine arts that not only have accomplished something within their discipline, but have already influenced musical thought and can inspire new musical strategies. In doing so, I will look at specific artistic strategies like Duchamp's use of \emph{readymades} and Debord's notion of \emph{d\'{e}tournement}, as well as how during the eighties and nineties artists approached appropriation in their own practice. Later, I will come back to music and look at musicians who explicitly appropriate from others, focusing however on specific musical strategies developed during the last half of the twentieth-century that have had considerable influence on my own work. After examining and scrutinizing strategies of appropriation used by musicians of what recently \emph{postmodern music} has become,\footnote{See \hyperlink{postmodernmusic}{pp. 17--20}.} I will consider other options that in my view approach the process of appropriation in more significant and innovative ways and which have the potential of achieving something new within music. I will therefore examine Clarence Barlow's notion of \emph{musica derivata} and John Oswald's \emph{plunderphonics} as musical strategies of appropriation that I believe have the potential of inspiring new forms of music-making. Finally, I will attempt to elaborate both of these concepts taking into account certain considerations specific to today's cultural configurations and aesthetic concerns as well as recent technological advancements which have had a considerable impact in the way in which we create and experience music. The ideas, concepts and propositions this chapter elaborates actively influenced and informed the creative process that resulted in the submitted work.

\section{Appropriation, Ideology and the Past}

Appropriation is certainly nothing new to music. Musicians have used, borrowed and stolen from the work of others for centuries, either explicitly or without deliberation. The very notion of what it means to make music is implicitly related to the act of appropriating existing sounds, structures, actions and thoughts. As with any other type of artistic production, music depends on already existing materials and actions to produce something new. In the case of musical appropriation, the existing materials used not only refer to the physical materials or objects utilized to create music (the instrument, the strings of the piano, the stage, the musician's own body, etc.), but also to its content (already existing sounds, rhythms, structures, gestures, timbres, etc.). Nevertheless, I will argue that musical appropriation is not limited only to the act by which musicians choose their materials to kick-start the creative process. My position is that we should consider a wider notion of musical appropriation that also includes the process of determining the system of musical production itself, which is based on \emph{a priori} forms of knowledge, abstractions, deductions and subjective processes that as a whole constitute a set of creative frameworks and stratergies of production. Put briefly, not only does the musician appropriate existing materials to work with, but the method of working itself. Moreover, the methods by which the musician works may not only be appropriated from within the traditional conventions of music-making but also from the knowledge gained from other forms of thought and production that come from disciplines outside music (for instance from the visual arts, social sciences, physics, philosophy or mathematics, to mention just a few). Furthermore, musicians curiously enough might not be aware of all of the sources that they appropriate from (this also includes the music they `derive inspiration from')---they might just casually absorb and appropriate from their own experience by \emph{living} and dwelling within their surrounding culture. Consequently, for as long as human beings have produced music, there have been strategies of musical appropriation. The notions of creativity and originality in music are therefore not independent from that of appropriation and a debate about music is only futile if we start from the premise that musicians either appropriate or not. A much more fruitful discussion would stem from the presupposition that all musicians appropriate---they all use existing materials and methods, they borrow and steal from each other and from other existing cultural and historical modes of production---and the questions that are more relevant are about what musicians appropriate, who and where they appropriate from, how and why they appropriate and what they accomplish through appropriation. What the musician does with the appropriated materials and methods and with what purpose is therefore much more relevant to the notions of creativity and originality than questions of authorship, copyright or legitimacy concerning an individual been responsible for the action of creating music. 

What's more, whether deliberate or not, within the act of appropriation lies in essence the musician's relationship to the past. This relationship may be established through the materials and methods the musician appropriates as well as how they have been appropriated. For instance, a clear relationship could be established with the work of an individual composer from the past---either systematically or intuitively---by imitating, copying, interpreting or modifying materials and methods from his/her body of work. The relationship between the musician and the past can be one that is initiated by a personal interest (which might be conscious or unconscious) in the musical practices and output of a specific historical figure. The process of appropriation from the work of historical figures that a musician admires or relates to is, I will claim, how many of them start to form their (musical) identity. However, I believe that musicians should not only rely on relationships with the past based on personal or subjective impressions on the musical practices and output of individual historical figures but should also consider their work within a wider cultural and historical context. When musicians appropriate from another composer's work, they are not merely referring to an isolated body of work but accessing a complex set of relationships in connection to how the composer relates to his/her surrounding culture and what is the significance of his/her work in a wider historical and cultural context. That is to say, even when we appropriate from just one musical source from the past, we are in fact making reference to a panoply of historical and cultural symbols that today might have a complex order of significance. Nevertheless, the relationship between the musician and the past is not only reflected by who creates the appropriated materials and methods of music-making and what those are, but also on how the musician might appropriate them. Thus, we may deduce more (depending on the nature of the \emph{musical result}) about the musicians's relationship with the past through the way in which s/he appropriates an existing piece of music, than who possesses the original authorship of the composition and what cultural/historical significance it might have. Similarly, a musician's relationship with the past might be contingent on how s/he uses different types of performance practice within his/her own, rather than what these practices are or what they signify.

Additionally, within the act of appropriation lies a deep relationship between the musician who appropriates and the appropriated Other. How the appropriator selects the material (whether it is a specific score by a composer from western music history or an `ethnic' instrument from a non-western culture) and acts towards it reveals a type of relationship established between him/herself and the Other. These types of relationships disclose an attitude towards the Other that display his/her position in reference to the struggle of cultural appropriation. By attempting to establish a certain type of relationship with the Other, the appropriator may in reality be (either consciously or unconsciously) imposing a different one. That is the case today for instance with a common attitude in western countries towards non-western cultures that is reflected through musical appropriation and can be associated with the notion of \emph{multiculturalism}.\footnote{See \hyperlink{multiculturalism}{pp. 42--44} for a wider discussion about the concept of \emph{multiculturalism} and  Slavoj \v{Z}i\v{z}ek's interpretation of it.} Even though the initial intention of a composer who incorporates instruments and performers from different cultures is to establish an attitude that reflects equality and tolerance, by treating the Other with `respect' and assuming a superfluous position of `openness', in reality what this type of appropriation might exhibit is a patronizing and condescending distance that situates the western composer in a position of superiority in the struggle that is an intercultural exchange. Therefore, determining the appropriator's attitude towards the appropriated Other requires a critical evaluation and reflection of the complex processes that take place during musical appropriation. Put briefly, the appropriator should not infer that the initial attitude s/he hopes to assume regarding the Other (without putting it to through rigorous scrutiny) will be reflected in the type of relationship that is eventually established. The musician who appropriates cannot simply rely on `first impressions' concerning his/her preconceived notions of the Other or the conviction that he/she will decipher the Other's otherness through the process of acculturation. I believe that, as musicians, we can only take a sensible stance when we accept the fundamental impossibility of truly understanding the Other (and what the radical alterity of the Other spells for us, namely the split within ourselves). Thus, a `constructive' exchange regarding musical appropriation only starts when we agree to disagree with the Other in some aspects of his/her musical judgements, taste and traditions. Moreover, we should not forget that the act of appropriation in-itself implies a form of violence and confrontation between different opinions, cultures, traditions and social strata.

The way in which musicians appropriate also reflects a deep connection to their ideology and the way they relate to tradition. Even though music rarely has a \emph{raison d'\^{e}tre} that is simply ideological, it nevertheless reflects ideology (paradoxically not always aware that it is doing so). Furthermore, because of music's non-conceptual and non-objective nature, ideology is not revealed within music as clearly and directly as it is for instance through language. However, ideological positions may be traced through what happens within music, including how and where it is created, presented, perceived and consumed. Music's subjective characteristics nevertheless make the task of examining ideology in music an analytical and interpretative process. Therefore, a way of demonstrating ideological positions in music is through critical analysis and reflection, taking as a premise that music has a large context of meaning and that it can be examined in connexion to broader cultural phenomena. That said, I believe that by critically analyzing the way in which musicians appropriate from tradition and how those traditions relate to their own background we might learn something about the prevailing ideologies that dwell within their work. Through this process we can also grasp the wider ideological context in which the music is created---in other words, we should not only take into consideration the ideological position of the musicians themselves but think about the bigger cultural picture, which includes the ideas and opinions of all the people involved in the creation and reception of the music (the institutions involved in a musical performance, the type of audience, the type of venue, the way in which the music is disseminated and consumed, etc.). Consequently, the relationship between ideology and musical appropriation at once makes music a discipline embedded within ideology and at the same time displays a significant link between musical and cultural appropriation. Slavoj \v{Z}i\v{z}ek has argued that cultural appropriation and ideology imply an act of violence and friction that is created through the appropriation of past traditions. According to \v{Z}i\v{z}ek, in today's global society, ideology is more relevant than ever before, considering the violence implied in the act of cultural appropriation.
\begin{quote}
The contemporary era constantly proclaims itself as post-ideological, but this denial of ideology only provides the ultimate proof that we are more than ever embedded in ideology. Ideology is always a field of struggle---among other things, the struggle for appropriating past traditions.\footnote{\hyperlink{zizektragedy}{\v{Z}i\v{z}ek (2009)}, p. 37.}
\end{quote}
As with other types of traditions, the appropriation of musical traditions also reflects an ideological struggle. Not only is music evidence of an ideological battleground, but the act of musical appropriation in itself is a form of cultural violence. It is my position that musicians, instead of denying this irrepressible fact, should be aware of the consequences inherent in the inevitability of appropriating existing music.

Given the relationship that exists between music and culture, and the struggle inherent  in appropriating musical traditions, we may deliberately use musical appropriation as an ideological tool to convey thoughts, opinions and feelings that may be associated with other forms of human knowledge and action. Therefore, through the intentional and formalized use of appropriation as a musical strategy, musicians may convey their own subjective and objective views about the appropriated Other, which at the same time may reflect a wider commentary on their relationship with the past, as well as with other cultures and traditions. By using appropriation as a strategy, musicians may also express their opinion, contemplate and denounce specific forms of music-making (musical genres, styles, performance practices, compositional techniques, etc.), as well as people (specific composers, performers, types of audience, etc.) and objects (instruments, scores, stages, halls, etc.) involved in music-making that have symbolic significance in our present culture. That is to say, through strategies of formalized musical appropriation, musicians may convey ideology through musical aesthetics. Finally, by manipulating, rearranging, modifying and re-contextualizing the appropriated musical references, musicians may also construct a symbolic space where they can `play with culture and signification' by imagining alternative conditions, meanings and forms of visibility for these musical references, including the social and economic struggles they may represent. 

%Maybe here introduce Ed s idea of dividing appropriation terminology
As has been mentioned before, the deliberate and conscious use of appropriation as a creative strategy is nothing new to music. However, I will also claim that the strategies of musical appropriation---regarding the means by which musicians appropriate as well as the types of materials that are appropriated---have drastically changed during the twentieth- and twenty-first-centuries due to technological advancements that have redefined (and continue to redefine) the way in which we appropriate music. During this time some prominent theories within the fine arts discourse dealing with specific types of appropriation have influenced the practice of musical appropriation.\footnote{That is not to say however that musical theories and practices have not influenced appropriation in the fine arts. Certain musical practices (for instance, DJing) have had a strong influence on recent art practices and theories dealing with appropriation.} I will continue by examining a number of specific strategies and approaches that have been used within fine arts during the twentieth-century that not only have been influential and important in defining current musical discourses and attitudes but also have had considerable influence on (and has serve as inspiration to) the submitted work. 

\section{Appropriation Art}

The following discussion does not aim to give a comprehensive survey of the artistic theories and strategies that use appropriation, nor it is an effort to make an original contribution on this subject of research. Rather, my aim is to simply describe certain concepts and ideas that are prominent in art theory that I think are relevant to musical appropriation and to the submitted work. Although I might be taking the risk of appearing to oversimplify what is an exhaustive subject, I will not elaborate too much on the significance of the described artistic notions within fine arts practice as my concern is mainly focused on the repercussions these ideas might have on music and how these artistic notions might contribute to the practices of musical appropriation. This discussion is aimed at a reader who is interested in musical appropriation but is not necessarily aware of the discourse developed in the fine arts regarding appropriation.

\hypertarget{appartists}{}
\emph{Appropriation art} is a term used within art theory to describe practices by artists that deliberately use appropriation as a strategy to take possession of usually unauthorized materials (images, objects, etc.) whose authorship is widely acknowledged to be by others (other artists, filmmakers, companies, advertising agencies, etc.).\footnote{See \hyperlink{evans}{Evans (2009)}.} Even though \emph{appropriation art} is sometimes used to refer to a specific group of artists who in the late seventies and eighties were associated to certain New York galleries (these artists include Richard Prince, Cindy Sherman, Ashley Bickerton, Peter Halley, Jeff Koons and Meyer Vaisman), its theoretical and historical foundation starts at the beginning of the twentieth-century with the formulation of \emph{montage}, the development of photography and the influence of Marcel Duchamp. Moreover, \emph{appropriation art} is not only limited to this specific group of artists as these strategies by now have been used widely by many artists who quite often have also developed their own idiosyncratic approaches towards appropriation. The advancements and increasing accessibility of computer technologies and the internet has also contributed to the way in which appropriation artists experience, consume, plunder and transform existing cultural objects. I will attempt to briefly examine some of the notions and theories associated with \emph{appropriation art} with the aim of later investigating their possible implementation and relevance to musical practices.

\subsection {Readymades and D\'etournement}

\hypertarget{readymade}{}
The notion of appropriation within the arts during the twentieth-century was deeply influenced by Marcel Duchamp's concept of \emph{readymades}, which simply refers to an object that is `already made' and is then presented as an `art work'. 
\begin{quote}
In 1913 I had the happy idea to fasten a bicycle wheel to a kitchen stool and watch it turn. A few months later I bought a cheap reproduction of a winter evening landscape, which I called `Pharmacy' after adding two small dots, one red and one yellow, in the horizon. In New York in 1915 I bought at a hardware store a snow shovel on which I wrote `in advance of the broken arm'. It was around that time that the word `readymade' came to mind to designate this form of manifestation. . . . At another time---wanting to expose the basic antinomy between art and readymades---I imagined a `reciprocal readymade': use a Rembrandt as ironing board!\footnote{\hyperlink{duchamp}{Duchamp (2009)}, p. 40.}
\end{quote}
This notion implies that the artist does not need to produce an `art work' but only choose a `found object'. The artistic process therefore might not imply the fabrication of an object but the selection of already existing ones---the artist's task is to look at his/her surroundings, choose an object and give it meaning through its re-contextualization. Duchamp's \emph{reciprocal readymade} is the reversal of the former: an existing object considered to be an `art work'  that is given a new context by using it as an `ordinary' object that is normally not regarded to be art. By taking an object that is regarded as a `masterpiece' (a Rembrandt) and using it as an object that usually performs a different function and is not appreciated for its aesthetic qualities (the ironing board), the \emph{reciprocal readymade} is a gesture that serves as criticism to the mystification behind such canonic artworks and at the same time questions the traditional western values of artistic appreciation. By taking an `art work' and using it as an object usually not considered to be art, and using an existing `ordinary' object as an `art work', Duchamp contributes to the idea of art as more than just a `specialized' production of highly praised objects. The concept of \emph{readymades} opens up a new debate on \emph{what should the function of the artist be} (if not that of producing objects), and at the same time questions notions of uniqueness and authorship in art.\hypertarget{lhooq}{} Moreover, within Duchamp's categories there is yet another type of \emph{readymade} which implies an alteration or adjustment by the artist (often very small) on the original `found object'. He calls this type of work \emph{assisted readymade}, from which \emph{L.H.O.O.Q.} (1919) is one of his most famous. \emph{L.H.O.O.Q.} consists of a reproduction (a commercial print) of Leonardo's \emph{Mona Lisa} on which Duchamp draws a mustache and goatee in pencil. By doing this simple gesture, Duchamp criticizes established and authoritative notions of western art by banalizing a canonic work that represents symbolically the ideal of beauty. The vulgarization of the image of the \emph{Mona Lisa} in a commercial reproduction raises questions about gender and the value of art, as well as making a comment on the reproduction, commercialization and popularization of art works.\footnote{\hyperlink{judovitz}{Judovitz (1995)}, pp. 139--143.}

\hypertarget{si}{}
The concepts of non-assisted (original without modification), \emph{assisted} and \emph{reciprocal readymades} were later embraced within the notion of \emph{d\'{e}tournement}, first introduced by the avant-garde movement the Situationist International in the late 1950s.\footnote{See \hyperlink{ford}{Ford (2005)} and \hyperlink{mcdonough}{McDonough (2002)} for more information on the Situationist International (SI).} \emph{D\'{e}tournement} (which means `diversion' and can also mean hijacking, embezzlement and corruption) is an artistic and/or political gesture that consists of the use or hijacking of existing works of art and other cultural forms within the artist's own work.\footnote{\hyperlink{postproduction}{Bourriaud (2005)}, pp. 35--37.} This notion might also imply an intervention, re-contextualization or alteration to be exerted by the artist on the existing works or forms to provoke their depreciation, degeneration and devalorization. Furthermore, Guy Debord argues that \emph{d\'{e}tournement} is ``not an enemy of art. The enemies of art are those who have not wanted to take into account the positive lessons of the `degeneration of art' ''.\footnote{\hyperlink{debord2}{Debord (2009)}, p. 66.} The violence implied in the gesture of \emph{d\'{e}tournement} is not to oppose or denounce art but to provoke an action that interrupts and displaces an existing (artistic) theory---which reminds us that theories are truly faithful to themselves only if they allow `history's corrective judgment' upon themselves. Debord sees appropriation that specifically maintains a certain distance from accepted conventions as a form of improving upon and correcting ideas from the past.
\begin{quote}
The defining characteristic of this use of \emph{d\'{e}tournement} is the necessity for \emph{distance} to be maintained towards whatever has been turned into an official verity. . . . Ideas Improve. The meaning of words has a part in the improvement. Plagiarism is necessary. Progress demands it. Staying close to an author's phrasing, plagiarism exploits his expressions, erases false ideas, replaces them with correct ideas.\footnote{\hyperlink{debord}{Debord (1994)}, p. 145.}
\end{quote}
In other words, he claims that through the act of plagiarism one may improve, correct and update ideas and theories (about art), and at the same time in the process of doing so distance them from their authority and alleged truth. Moreover, Debord argues that \emph{d\'{e}tournement} of existing works may also be used as a correction of the `artistic inversion of life'.\footnote{\hyperlink{debord2}{Debord (2009)}, p. 66.} By appropriating material (images, films, etc.) from the `spectacle' that self-evidently signifies a reversal of what happens in real life, the artist may in actuality represent real life, meaning and emotions through \emph{d\'{e}tournement}---consequently rectifying the inversion of reality implicit in the original material. An example of this strategy would be if an artist uses a \emph{d\'{e}tourned} Hollywood romantic film---which in its original form does not accurately portray `love' in real life---to express and convey `love' in reality. Furthermore, \emph{d\'{e}tournement} denounces the separation (within the arts) between production and consumption that characterizes the `society of the spectacle', encouraging the appropriation of lived experience over the fabrication of work that contributes to the division between art and the spectator.\footnote{\hyperlink{postproduction}{Bourriaud (2005)}, p. 36.} It is also important to consider that there are different types of \emph{d\'{e}tourned} elements which may affect the strategies of appropriation. In `Directions for the Use of D\'{e}tournement', Debord and Wolman make a difference between two main categories of \emph{d\'{e}tournement}: minor and deceptive \emph{d\'{e}tournement}. Minor \emph{d\'{e}tournement} is the use of elements that have no considerable signification within a given political or social context, but only obtain meaning through their re-contextualization. Deceptive \emph{d\'{e}tournement} on the other hand, relies on elements that have a strong cultural/political meaning that acquires a different dimension of significance from the new context.\footnote{\hyperlink{debord3}{Debord and Wolman (2009)}, p. 35.} The authors attempt to give an analysis of certain considerations regarding \emph{d\'{e}tournement} that they considered important for their own artistic/political purposes. In their analysis, they also warn of less effective forms as well as negative uses of \emph{d\'{e}tournement}, which in some cases might include the use of \emph{d\'{e}tourned} elements for right-wing propaganda\footnote{Here, one could argue that a recent example that illustrates this point is the use of \emph{d\'{e}tournement} on Barack Obama's photograph (a mustache is superimposed on Obama's image) by the American `Tea-party' movement to suggest that a relationship exists between Obama and Hitler, which at the same time attempts to imply that Obama has `totalitarian impulses' that are tied to his `socialist agenda' of expanding the federal government.} or strategies that instead of condemning the `spectacle', often reassures its hegemony by using these strategies within consumer society and with the purpose of the production of commodities for individual profit and gain.\footnote{Ibid., pp., 36--37.} One could argue that some manifestations of \emph{d\'{e}tournement} establish a cynical distance between the individual and the appropriated material that---instead of performing a subversive response to the `spectacle'---in reality conceals the individual's complicity with the prevailing system of commodification. According to Slavoj \v{Z}i\v{z}ek, in our contemporary `post-ideological'  era, artistic strategies that incorporate this cynical or ironic distance have lost their capability to be transgressive and have instead become a form of conformism.\footnote{See \hyperlink{zizekuniv}{\v{Z}i\v{z}ek (2006)}, `Why are Laibach and the \emph{Neue Slowenische Kunst} not Fascists?', pp. 63--66.}

\subsection {(Post)-postproduction in the Digital Age}

During the eighties, the interest in deliberately using appropriated material as a creative tool reemerged within the visual arts world. The concepts of \emph{readymades} and \emph{d\'{e}tournement} were reconsidered (and at the same time considerably expanded) as strategies of appropriation by artists who could sense that the tactics of \emph{modernism} (particularly those involving anti-mimetic principles) were approaching a dead-end. Nicolas Bourriaud, in his book \emph{Postproduction}, argues that for artists like Jeff Koons, Sherrie Levine and Heim Steinbach, Jean Baudrillard's ideas of \emph{simulation} and \emph{simulacra} supplied a theoretical foundation for much of their artistic practices at the time, which involved using appropriated objects that embodied the `subject of desire' within the western capitalist system. These objects become virtual, neutralized by their presentation as a product that is inaccessible to the viewer: the artist becomes the consumer of objects, at once \emph{instead of} and \emph{for} the viewer. Bourriaud argues that these artists' work becomes \emph{simulacra}, the artist simply appropriates from the market place, s/he chooses mainstream products that are most desirable as a  consumer to be exhibited and/or recreated as their work.\footnote{\hyperlink{postproduction}{Bourriaud (2005)},  p. 35.} This type of appropriation as a form of consumption was later adapted by a new generation of artists in the nineties, but nevertheless what became apparent is that the new generation consumed \emph{different} products \emph{differently}. In other words, even though both generations followed a model by which the artist acts more as a consumer than a manufacturer, what they consumed and how they consumed it was quite contrasting between generations. According to Bourriaud, one of the main features of this new generation of artists which includes Rirkit Tiravanija, James Jones, Thomas Hirschorn and Michel Henochsberg, to mention just a few, is that they started to create work that appropriates products that are not those that the mainstream consumer would buy (like the former generation of artists did during the eighties), but articles one would find in more specialized or unusual shops, in the back of the shelves or in the `flea-market.'\footnote{Ibid., p. 28.} The objects this new generation appropriates come from a diverse set of origins, circumstances, places, cultures, different lived stories and histories. Moreover, these artists display material differently from their predecessors: they present objects in arrangements that promote the formation of relationships between the people who are observing. The arrangement of the objects is often slightly chaotic, sometimes the materials are presented with the intention of escaping formal unity, the objects are displayed randomly as a cluster of non-assisted \emph{readymades} with the purpose of presenting an ensemble of disjunct objects that retain their autonomy, giving the impression that they resist categorization and structure. On the other hand, sometimes these arrangements also display some sort of `order within chaos' and within their apparent randomness, one can find intention, order, structure and homogeneity. \hypertarget{claudelevi}{}Additionally, these artists use and manipulate objects in different ways, where the degree of transformation of material fluctuates between the extremely altered and the completely unaltered. Claude L\'{e}vi-Strauss's opposition between ``the raw and the cooked'' has been used to describe these practices in the transformation of appropriated material: `the cooked' representing the objects that have been radically transformed (sometimes beyond recognition) and `the raw' constituting the appropriated objects that have not been altered at all by the artist.\footnote{Ibid., p. 29.} The description of how `raw' or `cooked' this material is may characterize one single `art work' (where the amount of transformation of all its elements is consistent through the whole work). Conversely, elements with different levels of transformation may exist within the same work, creating a wide palette of `recycled' objects with varying degrees of alteration. The artistic strategies of appropriation that emerged during the eighties and nineties still constitute an important practice in the visual arts today. The work of the appropriation artist also seems to point towards an intensification in the amount of transformation of the appropriated material as well as a concern with finding new forms in which this material can be presented. What started as `straightforward' appropriation (non-assisted or just slightly assisted) has gradually turned into (post)-postproduction\footnote{Here, I have added the extra `post' to \emph{postproduction} in brackets to differentiate this type of artistic strategy from its standard definition in the audio/video industry (I also want to avoid misunderstandings that for instance Bourriaud's use of the word might generate if compared to that of the industry. (Post)-postproduction therefore refers to a type of artistic production that relies on already produced art work, sound or music---even if in some cases this existing material has already been subjected to what the media industries call \emph{postproduction}.}---artists gradually further their use, manipulation and transformation of the material they seize and continue to increase the amount and diversity of sources they appropriate from.

The effect of digital technology in the way in which artists consume, create and present their work has also had a significant influence over the artistic strategies of (post)-postproduction. The increasing use and accessibility of computers and the internet has transformed the way artists use, experience and process cultural information. Digital technologies first brought with them unprecedented possibilities to reproduce information---copying and transferring data has become a powerful tool for artists. With the increasing power of computers, not only has the artist hugely increased his/her potential to reproduce, transfer and access information, but also s/he has mastered overwhelming control over the appropriated data. Moreover, through the development of the internet, sharing information has become a common practice which poses interesting questions regarding the ownership and copyrights of artistic forms. Once information becomes digitalized, transferred and manipulated, its right of ownership becomes a complex and perplexing legal, ethical and aesthetic problem. Bourriaud has argued that for the reasons mentioned above, contemporary art tends towards an abolishment of the ownership of artistic forms. 
\begin{quote}
Throughout the eighties, the democratization of computers and the appearance of sampling allowed for the emergence of a new cultural configuration, whose figures are the programmer and DJ. The remixer has become more important than the instrumentalist, the rave more exciting than the concert hall. The supremacy of cultures of appropriation and the reprocessing of forms calls for an ethics: to paraphrase Philippe Thomas, artworks belong to everyone. Contemporary art tends to abolish the ownership of forms, or in any case to shake up the old jurisprudence. Are we heading toward a culture that would do away with copyright in favor of a policy allowing free access to works, a sort of blueprint for a communism of forms?\footnote{\hyperlink{postproduction}{Bourriaud (2005)},  p. 35.}
\end{quote}
The challenge that digital technology poses towards the ownership of forms has triggered new social configurations of artists, hackers, musicians and other people who take part in shareware and open source communities, where they openly share all of their digital productivity. In these communities, the notion of `creative output' has drifted away from the idea of individual intellectual property, in favor of a practice based on creating by openly using other individuals' contributions. Social re-appropriation has become a model by which reciprocated copying and sharing is evaluated for its potential to make the flow of information faster, smarter and more effective for the common gain of the community. However, we are far from Bourriaud's `communism of forms'. Although these communities might share their production with other members of the community, they do not directly oppose capitalism. Many of the individuals involved in these practices are mostly looking only for their own individual gain within our society of global capitalism. The `communism of forms' only reaches a small privileged group of artists and musicians, hackers and entrepreneurs who have been fortunate enough to have the time, training and resources to access these materials and information. The programmer, artist or DJ who works by borrowing, stealing or sharing forms does not accurately represent `real existing communism', but rather a kind of figure closer to \v{Z}i\v{z}ek's interpretation of a \emph{liberal communist}. According to \v{Z}i\v{z}ek, \emph{liberal communists} are new entrepreneurs who believe in (occasionally) giving their products for free (but at the same time making money from related services like advertisements on their websites, donations, etc.), being aware of society and trying to change it through charity so that it is fairer, using smart and dynamic communications and avoiding traditional notions of labor, being creative but at the same time sharing their creativity with others, promoting education, philanthropy and voluntary work.\footnote{\hyperlink{zizekviolence}{\v{Z}i\v{z}ek (2008)}, pp. 15--16.} However, \emph{liberal communists}---epitomized by the figure of Bill Gates (`the ex-hacker who made it')---at the same time are primarily entrepreneurs whose main pursuit is to make more profit, even if in some cases that might imply engaging in `cruel' practices like destroying or buying their competition, engaging in dubious financial speculation, indirectly exploiting employees and attempting to monopolize the market.\footnote{Ibid., pp. 14--19.} \v{Z}i\v{z}ek has argued that this dual behavior discloses an avoidance of their complicity with the system.
\begin{quote}
In liberal communist ethics, the ruthless pursuit of profit is counteracted by charity. Charity is the humanitarian mask hiding the face of economic exploitation. In a superego blackmail of gigantic proportions, the developed countries `help' the undeveloped with aid, credits and so on, and thereby avoid the key issue, namely their complicity in and co-responsability for the miserable situation of the undeveloped.\footnote{Ibid., p. 19.}
\end{quote}
%Transitional sentence
Therefore, one should be cautious about the optimism associated with digital sharing and hacking, cyber-communities, shareware, open source and other digital practices. The idea of giving away digital information for free and sharing creativity with others, as well as other proclaimed liberal communist principles might at a first glance seem positive stances but underneath they might carry a complicity with the same ruthless attitude that may be associated with today's global capitalism. Even though digital technology has the potential for people to organize themselves in cyber-communities and gives opportunities to subvert traditional forms of capitalism (intellectual property that is digitalized is harder to regulate), we are far from Bourriaud's utopia---what we are approaching is not a blueprint for a `communism of forms' but a new configuration of digitized production with adapted capitalistic values. Having said that, I am still convinced that these digital practices and communities have considerable artistic and musical potential if they are used positively as tools to create new \emph{aesthetic} forms. However, in doing so, I believe artists should try to avoid the naivety and dishonesty behind the liberal communist enthusiasm over digital sharing.

Having briefly described and examined some of the prominent theories and strategies of appropriation that emerged during the twentieth-century in the fine arts, I will now engage in a discussion about appropriation as it pertains to recent developments in music. My aim is to use some of the theory related to the notions of \emph{readymades}, \emph{d\'{e}tournement} and (post)-postproduction to explain musical strategies that deal explicitly with appropriation. I will also elaborate on already existing strategies developed during the twentieth-century that deal with musical appropriation which I have attempted to implement in my own work.

\section{Musical Appropriation}

Musical appropriation is a very broad subject that includes topics as diverse as the use of chansons as \emph{cantus firmus} in polyphonic masses of the fifteenth- and sixteenth-centuries, Webern's orchestration of Bach's Ricercata from \emph{Musical Offering}, sampling practices in recent pop music, quotation in Charles Ives' music, Handel's operatic borrowings and the `dropping' of quotes in bebop solos. For that reason, I will try to narrow this discussion to a more limited subject, that is, how musical aesthetics and technological developments during the twentieth- and twenty-first-centuries have influenced the way in which musicians appropriate already existing music. I will not attempt to produce a detailed study of twentieth-century music that explicitly uses appropriation,\footnote{There are already good examples of such studies, see for instance \hyperlink{metzer}{Metzer (2003)}.} but instead I will concentrate briefly on a few composers and musical strategies that have influenced my own creative output. I will venture to describe some ideas of how these musical strategies of appropriation could be expanded further through recent developments in computer technology. Finally, I will also endeavor to clarify to a greater extent the rationale behind the more idiosyncratic practices I have developed in my own work and the potential of using appropriation as a creative strategy to produce a \emph{musical result} that aims to accomplish something new within music.

\subsection{Appropriation and Postmodern Music}

In \hyperlink{chapter1}{Chapter 1}, I have already described how \emph{postmodern music} at first started as a reaction to the confusion and misunderstandings ascribed to the notion of \emph{modernism} in music. For that reason, the first composers who later became associated with the label of \emph{postmodern music} abandoned modernism's anti-mimetic principles by amongst other strategies, explicitly appropriating other music. Hence, the appropriation of existing music became a mechanism by which musicians could differentiate themselves from modernism, which was not only becoming institutionalized, but also had been weakened by its association with the `fall of communism' as well as its own inability to see beyond its self-imposed anti-mimetic principles and ideals of `purity' and `authenticity'.\footnote{See \hyperlink{modernismstuff}{pp. 8--17}.} Musical appropriation also became a vehicle by which these composers rediscovered the semiotic potential of using existing music to convey meaning and thus, allowing them to make wider cultural/historical references within their work. Musical strategies such as quotation, transcription and transformation of already existing music became a form of re-contextualizing meaning, imagining new symbolic spaces for appropriated cultural/historical objects and a way of constructing narrative by creating wider cultural associations and metaphors.\footnote{Luciano Berio's oeuvre, for instance, contains many good examples of how these musical strategies can be used for the above-mentioned objectives.} However, as I have argued in previous chapters, the notion of \emph{postmodern music} in recent years has also started to signify something more than the musical strategies of appropriation that have just been described.\footnote{See \hyperlink{postmodernmusicnow}{pp. 18--20}.} Today, a good portion of the music labeled as \emph{postmodern} for the most part approaches appropriation by adopting a false notion of `openness', resulting in a mixture of musical styles and genres that has as its only purpose to serve as commodified entertainment. Musical appropriation---which once opposed the spectacle---has now become an accomplice to it.\footnote{Here, one could consider as an example of this argument, Mark-Anthony Turnage's recent appropriation of Beyonc\'{e}'s song `Single Ladies' that appears in his composition \emph{Hammered Out} (2010), premiered at the BBC Proms 2010. The way in which Turnage appropriates Beyonce's popular hit, musically speaking, accomplishes nothing new, adding not even an apparent commentary about the original. The Turnage appropriation even looses the rhythmic drive and timbreal characteristics of the original pop song through its orchestral arrangement (it is very difficult for orchestral musicians to accurately reproduce this kind of pop music because it is virtually impossible to translate the compressed and rhythmically precise sound of the studio production that characterizes Beyonce's music to an orchestral language). The result is a narrowed down version of the original (not being able to even render the sexuality and drive of Beyonce's version) which accomplishes nothing but commodified entertainment for a relatively conservative audience of concert goers, which to this day still seem to `get a kick' from the (to their minds) still `transgressive' and `controversial' gesture of introducing pop styles into concert music.} The main problem of what \emph{postmodern music} has become is that its musical `games' of aimlessly borrowing, mixing and remixing from a plurality of musics (classical, pop, rock, jazz, world music, etc.) have ceased to accomplish anything new within music. At the same time, as I have argued at the beginning of this chapter, if one does not subscribe to this kind of permissive appropriation, the solution can not be to avoid the act of appropriation itself, as this would only become a futile gesture that only would take us back to the misunderstandings brought by modernism's anti-mimetic ideals. Therefore, the problem we are faced with today is not whether we should or should not appropriate existing music, but more precisely, how can we accomplish something musically significant and new through music appropriation. For that reason, I will now attempt to first examine several musical strategies that I consider approach the process of appropriation in significant and innovative (and sometimes controversial) ways and which, in my opinion, have the potential of inspiring new forms of music-making. Later, I will also undertake the task of proposing new ideas (based on the creative work submitted) of how these musical strategies might be further developed.

\subsection{Musica Derivata}

The first musical strategy I will discuss is Clarence Barlow's concept of \emph{musica derivata} (derivative music) which simply refers to `music that is compositionally based on other music'.\footnote {See \href{http://users.skynet.be/P-ART/P-ARTWEB/1BARLOW/BARLOW.htm}{\texttt{http://users.skynet.be/P-ART/P-ARTWEB/1BARLOW/BARLOW.htm}}.} Thus, \emph{musica derivata} simply describes a method of composition that aims at deriving new material from existing music (originally not written by the composer) to generate a new composition. However, Barlow's implementation of \emph{musica derivata} in his own work is more specific than one could draw from this simple definition. What one could conclude from Barlow's compositions that can be labeled as \emph{musica derivata} is that they deal with the borrowed material in such a way that it reveals something new about the appropriated music. At the same time, these compositions not only seem to be about the appropriated music but about music itself---they are concerned with disclosing subjectivities within the original music through very precise, exhaustive, and at times obsessive compositional processes that surgically reveal the `music within the music'. For instance, in his piano trio \emph{1981} (1981),\footnote{See \hyperlink{barlowcd}{Barlow (2000)}.} Barlow combines other piano trios by Clementi, Schumann and Ravel through a rigorously preconceived compositional process. \emph{1981} is a trio about three trios that travel through three performers by means of interpolation between the selected compositions. Through a spiral structure, Barlow manages to statistically morph between the notated material from each trio, which at first frenetically cycles from one trio to the other---each rotation gradually becoming slower and progressively unveiling the original material which at the very end becomes clearly recognizable. By dissecting and re-contextualizing the piano trios through a rigorous method, Barlow discloses the essence of this ensemble's mode of playing as well as it dispassionately discloses the emotional \emph{ethos} of the appropriated compositions. \emph{`Spright the Diner' by Nib Writer} (1986),\footnote{See Ibid.} Barlow's following piano trio, in turn uses material from \emph{1981}---thereby simultaneously appropriating his own composition and further transforming the already modified trios by Clementi, Schumann and Ravel. \emph{Spright the Diner}, curiously enough seems to be more of a self-commentary (or a commentary on Barlow by his alter ego \emph{Nib Writer}) on his own personality and compositional practices than a reflexion on the three original piano trios.

\emph{Musica derivata} can therefore also be a mechanism whereby the composer can make comments about the appropriated music, the composer or the tradition it might represent, how the music relates to other practices in performance and composition and how it might be connected to the appropriator's personal experience. Barlow's own rationale for the use of other music within his work, varies in almost all of his \emph{musica derivata} compositions. 
\begin{quote}
Sometimes I use other musics to pay homage to or even to ridicule them, sometimes just to point out certain processes in music, in musical practice. There are hardly any two pieces which have the same motivation in using other musics. I frequently do a whole lecture on my derived pieces and I would say that almost all the examples there have different reasons for being.\footnote{\hyperlink{barlowbob}{Barlow (2007)}.}
\end{quote}
For instance, in his series of thirteen preludes and fugues for piano \emph{Ludu ragalis} (1974-2003), Barlow's reason to appropriate thirteen indian ragas and treat them as material for baroque style counterpoint is to point at the similarities between classical indian and western music (for instance, northern indian music and western classical music share very similar ways of dealing with pitch material and their scales are almost identical).\footnote{\hyperlink{barlowpiano}{Barlow (2008)}.} On the other hand, in \emph{Variazioni e un pianoforte meccanico} (1986) the reason that Barlow appropriates Beethoven's Opus 111 \emph{Arietta}  seems to be that the improvisational and generative nature of the original perfectly fits with Barlow's scheme of creating real-time variations through an algorithmic computer progamme that produces generative variations. The form of `theme and variation' also seems fitting to Barlow's idea of the mechanical piano gradually `taking over' after the pianist has played the theme and starting to generate algorithmic/mechanical variations until the pianist is unable to continue playing. Once the mechanical piano seems to have overtaken the pianist, each variation becomes more elaborate than the previous one until the computer system seems to exhaust itself at the very end, when the pianist `regains control' over the piano by once more playing the original Beethoven. Barlow's appropriation of the \emph{Arietta} therefore creates a poetic association between Beethoven's  creativity and skill as a composer to progressively produce new variations (each one more ingenuous than the former) and the increasingly generative and complex nature of the music produced by the computer that gradually becomes `superhuman'---at some point the music becomes `artificial' in nature and too fast and difficult for an ordinary human to perform. Put briefly, Barlow makes a link between the generative nature of both Beethoven's creativity as a composer and the computer's output. While listening to \emph{Variazioni e un pianoforte meccanico}, one cannot avoid noticing that at some point the variations that result from the theme can only be produced through mechanical and computer iteration; the composition seems also to be a commentary on the relationship between human beings and technology and the difference between human and mechanical (re)production. Regardless of the specific motivation for appropriating other music, Barlow always seems to have a personal relationship with the music he has appropriated---for instance, he has a deep interest in Indian music since he was born and raised in Calcutta and he has a fascination with Opus 111 because he was trained as a pianist and has previously performed this work.

Another important element about \emph{musica derivata} is the use of computer technology in the process of derivation. Barlow's \emph{musica derivata} compositions are characterized by the use of \emph{Computer Aided Composition} (CAC) tools to create compositional processes that require complex computation. The type of calculations involved in the compositional process of most of Barlow's \emph{musica derivata} works would be strenuous if not impossible for the composer to perform without the power of the computer. In his \emph{musica derivata} compositions, Barlow usually uses notated information (information coming from the original score of the composition or a transcription in the case that there is no available score) as an input to a computer programme that is designed to modify or transform this data according to his compositional scheme. Barlow's compositional schemes are also carefully conceived as algorithms such that they can be implemented within a computer programme. The finished composition is sometimes limited to the results that the computer gives, sometimes even without any type of intervention from the composer after the detailed pre-compositional system has been formulated. However, sometimes Barlow is less strict about following through the pre-conceived process from beginning to end and his compositional practice sometimes involves selecting from various computer results and at some points even leaving some space for more `traditional' compositional decisions.\footnote{\hyperlink{barlowbob}{Barlow (2007)}.} 
\hypertarget{spectactics}{}

In addition to score information, Barlow sometimes uses other types of inputs for his compositional processes. For instance, in his compositions that he labels as \emph{musica linguistica} (music that is based on speech and language), he extensively (but not exclusively)\footnote{In some of his \emph{music linguistica} compositions, Barlow derives musical structures from language rules and syntax, instead of spectral characteristics of speech.} uses information derived from spectral analysis of speech recordings to generate pitch and rhythmic material to produce a score for acoustic instruments. One of the techniques he uses to produce instrumental music from speech is what he calls \emph{Synthrumentation},\footnote{See \hyperlink{barlowspec}{Barlow (1998)}.} which takes as a starting point the spectral information gathered from an FFT (Fast Fourier Transform) analysis and through a method analogous to \emph{additive synthesis} in electronic music (where sinusoidal waves are superimposed to create timbre), attempts to reproduce speech by using acoustic instruments instead of sinusoidal waves. The resulting instrumental harmonies are derived from the harmonic structure of speech sounds by mapping their fundamental and upper partials to notated pitches performed by acoustic instruments. However, each acoustic instrument produces a pitch that has its own harmonic structure and is not equivalent to a sinusoidal wave, and the result therefore retains the instrument's timbre but at the same time keeps some characteristics similar to the original speech recording. Barlow in some occasions deals with the discrepancy between the distinctive instrumental timbre and the result of added individual frequencies by selecting instruments and instrumental techniques that produce sounds that have similar characteristics to that of a sine wave (for example clarinets, flutes, string harmonics, etc.). This is the case in Barlow's composition \emph{Im Januar am Nil} (1982), where he bases the whole first three minutes of the composition on a recording of him speaking a text in German that is \emph{synthrumentized} using natural string harmonics, just intonation and scordatura-tuned strings. Barlow carefully composes the German text to avoid noise spectra, relying only on vowels and laterals (L, M, N and `NG' sounds) so that it is more suitable to be \emph{synthrumentized} through string instruments playing natural harmonics tuned to approximate the desired frequencies.\footnote{\hyperlink{barlowbob}{Barlow (2007)}.} Another example of the use of \emph{Synthrumentation} can be found in his composition \emph{Orchideae Ordinariae} (1986), where he uses this technique in an attempt to create the illusion of a symphony orchestra uttering the phrase ``Why me no money?". Barlow uses this technique to support the wider context for which \emph{Orchideae Ordinariae} was composed, which is mainly to make a commentary about the institutionalization of the musical avant-garde and the complacency that became endemic at that time (and up to a certain point remains today) regarding the commissioning of new works.\footnote{Ibid.} Another type of spectral technique used by Barlow is what he calls \emph{Spectastics} (or `spectral stochastics'), which takes as a starting point a melody based on a sequence of pitches determined through the probability distribution of spectral data.\footnote{See \hyperlink{barlowspec}{Barlow (1998)}.} When the melody is sped up, at some point it is no longer perceived as a melody but as cloud of notes or as timbre. Spectral techniques similar to the ones just described however are not unusual in contemporary western music and were exhaustively used during the seventies and eighties by the \emph{Groupe l'Itin\'{e}raire}, whose most prominent figures included composers Gerard Grisey, Tristan Murail and Hugues Dufourt. These composers however were interested in other types of spectra, focusing more for instance on instrumental timbre and its gradual transformation. Other composers, including Jonathan Harvey, Kaija Saariaho, Philippe Hurel and Marco Stroppa, to mention just a few, since then have utilized IRCAM released software (mainly \emph{AudioSculpt} and \emph{Open Music})\footnote{See \href{http://forumnet.ircam.fr/}{\texttt{http://forumnet.ircam.fr/}} for information about IRCAM software.} to extract information from a spectrogram (visualization of an FFT) and then transfer it to a notated score. However, most of these spectral techniques have been used to analyze isolated sounds as their main interest has focused on the spectral characteristics of sounds and not their wider cultural/historical meaning. Barlow's own use of spectral techniques has been limited to the analysis of speech and curiously enough, have not been widely used in his \emph{musica derivata} compositions. Nevertheless, I will claim that a significant potential exists in using spectral techniques such as \emph{Synthrumentation} and \emph{Spectastics} as a tool for musical appropriation to produce \emph{musica derivata} based on analysis of recorded music.\footnote{In my own earlier instrumental compositions (composed during my studies at the Royal Conservatory, The Hague) I have derived notated material from spectral analysis of recorded music. For example, in \href{http://www.federicoreuben.com/media/pdf/scores/esferica/Esferica.pdf}{\emph{Esf\'{e}rica Cant\'{a}ndote} (2005)}, I combine and transform material that I derived from both spectral analysis and transcriptions of five recordings of popular music. See \href{http://www.federicoreuben.com/media/audio/mp3/radio/esferica/}{\texttt{http://www.federicoreuben.com/media/audio/mp3/radio/esferica/}} for an interview about how \emph{Esf\'{e}rica Cant\'{a}ndote} was composed.} 

The final result of the process of derivation in Barlow's music, whether it involves deriving from notated or spectral information, is usually either a score for classically trained musicians to perform or pitch information (usually in Midi or other previous formats) that may be used for instance to trigger a mechanical instrument. The process of transferring the computer results into either the instrumental or mechanical mediums is crucial: the process of transcription needs to consider social/cultural elements of performance practice, issues that computers at the moment cannot adequately address. I will therefore claim that in \emph{musica derivata} the transcription of the computer results into the medium of performance (and its eventual realization) is a critical part of the creative process, as in it may lay some of the composer's most important subjective decisions that may be absent from the impartiality of the computer calculations. Put briefly, in \emph{musica derivata} the process of transcription and realization of the computer output might give us a clue into the composer's musical vision and character.

\emph{Musica derivata} remains a musical strategy that deals with appropriation in such fashion that the musician---by creating a precise, inquisitive, elaborate and exhaustive method of derivation---strives to simultaneously accomplish new \emph{aesthetic} results and at the same time reconsider already existing music (which in itself implies contemplating the appropriated music's larger context of meaning). The creativity and critical scrutiny involved in conceiving the process of derivation and the imagination required for the re-contextualization of the reworked musical material not only encourage self-reflection regarding how one as a musician relates to the appropriated music and surrounding culture but also might inspire new forms of perception and thought in other areas of human endeavor. Christopher Fox, in his article `Where the river bends: the Cologne School in retrospect' argues that Clarence Barlow, as well as other composers associated with what he calls \emph{The Cologne School},\footnote{A group of four composers: Clarence Barlow, Gerald Barry, Kevin Volans and Walter Zimmerman, who during the seventies and early eighties were based in Cologne and shared similar musical interests. See \hyperlink{fox}{Fox (2007)}.} use methods of appropriation that are unlike those used by other composers in that there is an absence of irony and a sense of commitment involved in their strategies of appropriation. Barlow's own use of appropriation also distances itself from the cynical and permissive attitude towards appropriation that has recently become recurrent in a lot of music under the \emph{postmodern} label. For these reasons, I believe that \emph{musica derivata} is a musical strategy that can be expanded and developed further to inspire new ways of creating and experiencing music. 

\subsection{Plunderphonics}

The second musical strategy I would like to put forward is John Oswald's \emph{plunderphonics}. \emph{Plunderphonics} refers to the practice of appropriating audio recordings of existing music and using them as material to create a new musical result. Thus, both  \emph{plunderphonics} and \emph{musica derivata} use as a starting point the explicit appropriation of existing music. However, in order to avoid confusion between these two strategies, I will attempt to draw a theoretical distinction between them. I will therefore propose that the main difference between \emph{plunderphonics} and \emph{musica derivata} lies in the sounding result: while \emph{musica derivata} culminates in an instrumental or mechanical performance resulting from notated material (often involving classically trained musicians reading from a score or a mechanical/computer realization of notation), \emph{plunderphonics} rather relies on recorded material to produce sound. That is to say, while \emph{plunderphonics} produces an audible result through sonic material derived from recorded music, \emph{musica derivata} produces it through the instrumental/mechanical realization of notated material. Having made a distinction between the two, I will now examine the potential (as well as the challenges) of \emph{plunderphonics} as a strategy of musical appropriation. 

The term \emph{plunderphonics} was initially coined by John Oswald in his 1985 article ``Plunderphonics, or Audio Piracy as Compositional Prerogative'' as a justification of his own practices of appropriating previously released commercial recordings of existing music. In this article, Oswald argues that devices that play audio recordings can be used with the purpose of reproducing music but at the same time may be used creatively to produce new and unique sounds. 
\begin{quote}
Musical instruments produce sounds. Composers produce music. Musical instruments reproduce music. Tape recorders, radios, disc players, etc., reproduce sound. A device such as a wind-up music box produces sound and reproduces music. A phonograph in the hands of a hip hop/scratch artist who plays a record like an electronic washboard with a phonographic needle as a plectrum, produces sounds which are unique and not reproduced---the record player becomes a musical instrument. A sampler, in essence a recording, transforming instrument, is simultaneously a documenting device and a creative device, in effect reducing a distinction manifested by copyright.\footnote{\hyperlink{oswald}{Oswald (1985)}.}
\end{quote}
Oswald's position here is that devices that reproduce sound can also be used as `musical instruments'---audio recordings are not only a form of documentation and reproduction, but also a way of creating new music. From phonographs and gramophones to the latest digital devices (computers, MP3 playes, etc.), the opportunity to reproduce sound becomes at the same time the possibility of finding a new musical context for it either through sonic quotation or transformation. Creativity through the medium of sound reproduction therefore blurs the line of distinction between original and new sounds, questioning the traditional notions of originality and authorship imposed by copyright laws. The \emph{plunderphonic} artist creates new work by modifying and re-contextualizing recordings that are available to him/her, which includes the commercial recordings that s/he consumes as a listener. As previously discussed in our examination of appropriation art, consumption is at the same time production---artists rely on consumption (of materials, objects, other artworks, etc.) to produce new work. Consumption is therefore not only the motivating force behind the production of art but the action by which new art is produced. The consumer of records not only consumes them by listening but uses them to produce new music (either through inspiration or actually by using their sound). Oswald's own practices as a consumer of recordings is eventually what leads him to produce new music through experimentation while listening.
\begin{quote}
As a listener my own preference is the option to experiment. My listening system has a mixer instead of a receiver, an infinitely variable speed turntable, filters, reverse capability, and a pair of ears. An active listener might speed up a piece of music in order to perceive more clearly it�s macrostructure, or slow it down to hear articulation and detail more precisely.\footnote{Ibid.}
\end{quote}
Here, Oswald's argument exposes the creative potential an active listener has to experiment with recordings. By being able to transform a recording of existing music, the listener not only may experience music differently, but may also stumble upon new sounds through this process of exploration---listening to a recording by modifying it might reveal something unheard before by the listener. Audio recordings---while conventionally considered to be a fixed reproduction of sound---in actuality may disclose the unfamiliar within the familiar through the process of transformation, thus becoming `uncharted territories' where new musical discoveries can take place. The active listener and consumer of records becomes the creator of new music through this process of discovery. 

\subsubsection{Early plunderphonics}

Using existing audio recordings creatively as material for a new musical result predates Oswald's term by many years. Nevertheless, we can still consider the use of this musical strategy before Oswald's work as early examples of \emph{plunderphonics}.  Chris Cutler has argued that the history of \emph{plunderphonics} is partially ``the history of the self-realization of the recording process; its coming, so to speak, to consciousness''.\footnote{\hyperlink{cutler}{Cutler (2006)}, p. 143.} Starting with the popularization of the earliest sound reproduction devices, the phonograph and gramophone, the possibilities of sound reproduction as a form of production gradually became self-evident. By the 1920s sound reproduction devices using discs were becoming reasonably cheaper (as opposed to the more expensive cylinder system) and therefore became a platform for experimentation---artists and musicians began to experiment with disc manipulation as creative and performative strategy.\footnote{See Ibid., pp. 143--146, for a detailed history about early \emph{plunderphonic} practices, starting with early experiments involving disc manipulation by Stephan Wolpe (1920), Darius Milhaud (1922),  L\'{a}szlo\'{o} Moholy-Nagy (1923), Ottorino Respighi (1924) and later Edgard Var\`{e}se (1936). Also, see \hyperlink{moholy}{Moholy-Nagy (2006)} for a fascinating early artistic statement about the potential of using the phonograph as an instrument for sound production instead of reproduction.} John Cage's \emph{Imaginary Landscape \mbox{No. 1}} (1939) is one of the first compositions that gives instructions for a gramophone to be operated as a musical instrument in a concert performance---the gramophone record containing test tones is `played' by actively varying the playback speed of the devise as indicated in a notated score.\footnote {\hyperlink{cutler}{Cutler (2006)}, p. 145.} \hypertarget{landscape4}{} One of the most intriguing compositions that can be considered as an early example of \emph{plunderphonics} is Cage's \emph{Imaginary Landscape No. 4} (1951), scored for twelve performers operating radios as musical instruments. In this composition, Cage manages to re-contextualize the sonic material produced by the radios (whether it is music, speech or other types of sound) through indeterminacy---the sounds from the radio broadcast are not intended to be fixed but on the contrary, they are left to chance. Moreover, the plundered sounds from the radio are appropriated as ever-changing \emph{readymades} that depend on the immediacy of what is being broadcasted at the precise moment of the performance. The result is a type of \emph{plunderphonics} that is at the same time current and generative (it generates new sounds depending on the varying trends of music and \hypertarget{landscape5}{}radio programmes that are broadcast) as well as indeterminate (the sounds are not predetermined). \emph{Imaginary Landscape No. 5} (1961) is another interesting \emph{plunderphonics} composition by Cage, where he specifies `empty containers' through time in a graphic score for forty-two unspecified phonograph records. The `empty containers' are specific amounts of time (represented in the score as rectangles of different lengths specifying their precise duration) when the records should be playing. Cage also gives instructions in the score for constant and varying dynamics and when the records should be changed. Even though the records are not specified, Cage indicates that for the performance of the composition he used exclusively jazz records.\footnote{\hyperlink{cage}{Cage (1961)}.} 

During the late fifties and early sixties, Nam June Paik composed works that are compelling early examples of \emph{plunderphonics}. Nam June Paik's collage tape pieces \emph{Hommage \`{a} Cage: Music for Tape Recorder and Piano} (1958-1959) and \emph{Etude for Pianoforte} (1960), which he created while working with Stockhausen at the WDR (Westdeutscher Rundfunk) Studio for Electronic Music in Cologne, are groundbreaking in their crude but effective use of appropriated recordings of existing music. To create these collage pieces, Paik uses simple tape manipulation involving cutting and splicing different recordings together as well as transforming them with different simple techniques like varying speeds, reverse playback and distortion. \emph{Hommage \`{a} Cage} includes music that originates from a fixed tape part, sounds created by the audience, a hen, a motorcycle and other objects.\footnote{\hyperlink{liggett}{Liggett (2004)}.} The tape part is made up of a mix of sounds, music, screams, cries and radio broadcasts including a congregation reciting the `Hail Mary' prayer, a sped-up sonic quote of Teresa Brewer's 1959 version of the hit song `Hula Hoop', test tones, and what sounds like a fragment of Beethoven's ninth symphony.  In \emph{Etude for Pianoforte}, Paik uses similar collage techniques resulting in a combination of unrecognizable sounds and music, screams, cries and very clearly recognizable quotes from Beethoven's fifth and ninth symphonies and Stravinsky's \emph{Petrushka}. These two surviving tape pieces however were part of much more elaborate actions/performances which included theatrical elements that were often seen as unpredictable, absurd, aggressive and sometimes even `terrifying'.\footnote{These two works included theatrical elements such as a poet reading from a toilet paper sitting on top of a ladder, Paik playing Chopin on a piano and breaking into tears, sawing through the piano strings with a kitchen knife, throwing himself onto a mutilated piano, cutting John Cage's shirt and tie with big scissors and shampooing the hair of both Cage and David Tudor who were sitting in the audience during the performance. See \hyperlink{meidenkunst}{Daniels, Frieling and Helfert (2003)} and  \hyperlink{paik}{Paik (2001)} for accounts about the performances of these two works.} Paik's use of \emph{plunderphonics} in both of these works seem to embrace the sporadic and violent nature of the actions/performances they formed part of. What makes this type of \emph{plunderphonics} exciting to the listener is that---through the overlap of plundered music with visceral sound (screams, cries, etc) and the use of abrupt cuts and distortion---Paik manages to convey feelings of brutality, suffering, irreverence and cruelty that arouse and stimulate certain strong and basic instincts that closely relate to human nature and experience. In what appears to be a nonsensical collage of sounds, he manages to expose a certain fragility inherent in the plundered music (through its re-contextualization), as one gets the impression that the recorded material may be cut or mutilated at any moment (even the Beethoven symphonies or Stravinsky's \emph{Petrushka} are not exempt from being vandalized despite their status as `masterpieces'). By conveying these heightened emotions in such aggressive and unpredictable fashion, Paik manages to recreate the thrill of experience by making the audience feel vulnerable through sound and performance. 

A later example of a completely different approach towards \emph{plunderphonics} is Karlheinz Stockhausen's \emph{Hymnen} (1966-67), where he appropriates recordings of different national anthems from around the world.\footnote{See \hyperlink{metzer}{Metzer (2003)}, pp. 139--159, for a detailed analysis of \emph{Hymnen}.} Stockhausen's approach to \emph{plunderphonics} in \emph{Hymnen} could not be more different than Paik's---while Paik plunders different existing music to seek discontinuity and expose violence, Stockhausen appears to have the desire to impart notions of unity and congruence. In \emph{Hymnen}, Stockhausen attempts to present the national anthems not as disconnected fragments but on the contrary, as a heterogeneity of parts. Moreover, Stockhausen not only seeks a union between the different recorded anthems through sound transformation, but also aims at blending them with other electronically generated sounds. In addition to the formal effort to integrate sound, Stockhausen also makes his wider call towards unity more explicit first, by after modulating through a variety of anthems from around the world, introducing an anthem from the fictitious land of ``Hymnunion'' and later, by ending the composition with a recording of his own breathing to symbolize `the respiration of humankind'.\footnote{Ibid., p. 152.} This plea for unity and universality has been interpreted both as \emph{utopian} and \emph{dystopian}---it could be understood as a utopian claim for human equality and unity or a vision of a totalitarian and authoritative union with imperialistic values which seeks universality through homogeneity and supremacy as one nation, instead of a diversity of nations.\footnote{See Ibid., pp. 149--154 and \hyperlink{}{Boehmer (1970)}.} Regardless of which interpretation is more valid, \emph{Hymnen} represents an early example of \emph{plunderphonics} where appropriated existing music is treated in a sophisticated way both technically and semantically to sonically enact the composer's fantasy.

\subsubsection{Oswald, Sampling and Copyrights}

Despite the fact that one can find early examples of \emph{plunderphonics} which appropriate recordings of popular music,\footnote{For instance, Nam June Paik's use of Teresa Brewer's `Hula Hoop' in \emph{Hommage a Cage (1958-1959)} and James Tenney's manipulation of Elvis Presley's `Blue Suede Shoes' is \emph{Collage No. 1} (1961). See \hyperlink{cutler}{Cutler (2006)}, p. 145.} it was not until John Oswald's work during the seventies, eighties and early nineties that it was done so extensively and deliberately. In these works, Oswald appropriates commercially released recordings of well known pop songs by celebrated pop musicians such as Michael Jackson (\emph{dab} (1989)), Madonna (\emph{madmod} (1993)), The Beatles (\emph{btls} (1989) and \emph{sfield} (1980)), Led Zeppelin (\emph{power} (1975)), The Doors (\emph{o'hell} (1990)), James Brown (\emph{brown} (1989)), Dolly Parton (\emph{pretender} (1988)), Metallica and Queen (\emph{2net} (1990)).\footnote{\hyperlink{oswaldcd}{Oswald (2001)}.} In these works one can find that, even though Oswald usually transforms the appropriated recordings, he does so such that they are still clearly recognizable (a consistent feature in Oswald's \emph{plunderphonics}). Moreover, the way in which Oswald alters the recordings is usually deliberate and calculated to express an opinion about the appropriated music and the musicians who perform it, which usually also refers to a wider commentary about (consumer) culture and the creative process itself. Oswald's practice of taking an already finished product, appropriating and modifying it, and consequently giving it a new meaning through its re-contextualization fits perfectly with Duchamp's notion of an \emph{assisted readymade}.\footnote{See \hyperlink{readymade}{pp. 64--66}.} Oswald, like Duchamp before him, radically changes the meaning of already fabricated products by modifying them through a simple (however significant) gesture. That is the case, for example in Oswald's \emph{pretender} (1988), where he plunders Dolly Parton's 1984 version of \emph{The Platters} 1955 song `The great pretender'. By gently slowing down the speed of the playback of the recording, Oswald gradually changes the pitch of the voice of the country music singer, resulting in the transposition of the original female voice a fourth lower, which makes it sound more like a male voice.\footnote{Ibid., p. 22.} Through this simple gesture, Oswald raises questions about gender in a similar way that Duchamp does, when he adds a mustache and goatee to the image of the Mona Lisa in \emph{L.H.O.O.Q.} (1919).\footnote{See \hyperlink{lhooq}{pp. 65--66}.} However, Oswald also seems to be influenced here by ideas about gender transformation and ambiguity that other visual artist of the same generation also deal with and that have been theorized by Jean Baudrillard. One can find a similar concern for instance in \emph{dab} (1989), where Oswald, by picking Michael Jackson's 1987 song `Bad' and exaggerating its features (for instance by looping and slowing down sonic idiosyncrasies that point towards Jackson's style and identity) again raises questions about gender definition but also about racial alterity. In addition to plundering Jackson's song, \emph{dab} was released as part of Oswald's \emph{Plunderphonic CD} (1989) which featured as a cover a \emph{d\'{e}tourned} image of Jackson's \emph{Bad} (1987) album. The altered image consists of the famous photograph of Jackson but his body (as it appears in the original image) is replaced with a white women's naked body. By modifying the original image, Oswald confirms his interest in Jackson's altered racial and gender identity---through this simple juxtaposition of images Oswald explicitly exposes Jackson's physical transformation (his facial and body features `thinning out' and  `whitening' through plastic surgeries and other medical interventions).

Oswald also seems to be particularly interested in appropriating recordings straight from consumer culture and the music industry. Owald's deliberate plundering of hit-songs can also be compared with the practices of visual artists of the same generation like Jeff Koons and Richard Prince, who appropriate objects that embody mainstream capitalist consumption. In other words, Oswald as well as the generation of \emph{appropriation artists} who became prominent in the eighties,\footnote{See \hyperlink{appartists}{p. 64}.} appropriate from the `shopping mall' or the `top ten list in the record store', rather than from less known or obscure sources. Other artists who shared Oswald's philosophy and plundered music straight from the mainstream of mass media and the music industry, are the members of the experimental music and art collective Negativland and multimedia group the Tape-beatles. In their work, these groups use tactics close to the ones used by the Situationist International (SI).\footnote{For example, the style of the Tape-beatles ``The Grand Delusion'' (1993) bares striking resemblance to Guy Debord's cinematic work ``The Society of the Spectacle'' (1973) in its use of \emph{d\'{e}tourned} images and sounds from the mass media.} By appropriating hit songs, radio programmes and other mass media products and transforming and mixing them together, they create new work using \emph{d\'{e}tournement} as a creative and denouncing strategy.\footnote{See \hyperlink{si}{pp. 66--67}.} For instance, in \emph{U2} (1990), Negativland ``fuses recitation of the lyrics to the Irish band's ``I Still Haven't Found What I'm Looking For'' with obscene outtakes of the radio broadcaster Casey Kasem's ``Top 40'' program, thirty seconds of the original recording, and a host of other musical and non musical materials'',\footnote{\hyperlink{sanjek}{Sanjek (2003)}, p. 359.} as a mechanism to denounce the manipulation that the mass media exerts over its public, the fake nature and emptiness behind the products of record companies, the sole motivation towards individual profit behind the music business and the authoritative for-profit reasoning which dominates record companies and copyright laws. 

However, one of the downsides of explicitly plundering from mainstream media and the music industry is that eventually, the industry can take legal action against artists who use \emph{plunderphonics} as a musical and artistic strategy. Unfortunately, that was the case for both John Oswald and Negativeland, who were respectively sued for copyright infringement. In the case of Owsald, it happened after a representative of the record industry got a copy of Oswald's \emph{Plunderphonic CD} (1989) and legally challenged him for copyright infringement of Michael Jackson's \emph{Bad} album. The copyright lawsuit was based on moral claims (they could not sue Oswald for financial circumstances since he had distributed the record free of charge) since according to the plaintiff, Jackson's photograph and music had been mutilated, damaging his image and misleading his fans.\footnote{\hyperlink{oswaldcd}{Oswald (2001)}, pp. 23--26.} As a consequence of this legal dispute, the recording was recalled and destroyed (including the master tapes) by Jackson's record company.\footnote{Ibid., p. 27.} A similar outcome came out of Negativeland's release of their \emph{U2} (1990) single: Island Records, the company that released U2's original song sued Negativeland not only for their unauthorized use of samples by U2 and radio broadcaster Casey Kasem, but also for the packaging which featured the letter U and the number 2 on the cover---the legal action eventually led to the destruction of all existing copies and a 25,000 dollar fine for the collective.\footnote{\hyperlink{metzer}{Metzer (2003)}.} In both cases, projects that had only artistic motivations and no lucrative ambitions were impaired because of their creative appropriation of copyrighted material. The creative use of appropriation as a musical strategy may therefore be jeopardized by copyright laws. By restricting artists in their creative use of appropriation (including strategies like \emph{d\'{e}tournement}, the use of readymades and \emph{plunderphonics}), copyright laws---instead of protecting the artists---might handicap their creativity and restrict their freedom to express themselves and voice their opinion. Chris Cutler, has argued that these type of incidents calls for a long overdue rethinking of copyright laws. 
\begin{quote}
Current copyright law differs from country to country, but in general follows international accords. It certainly allows `fair use' which would include parody, quotation and reference, though these may need to be argued and defended. This is a minefield in which only lawyers profit. So where The Beatles had to pay up for quoting `In the mood' at the end of `All You Need is Love', and Oswald had his work destroyed, Two Live Crew's parody of Roy Orbison's `Pretty Woman' got off free as `fair use'. . . . The rethinking of copyright law is long overdue. Recording has been with us now for more than 100 years.\footnote{\hyperlink{cutler2}{Cutler(1994)}.}
\end{quote}
Cutler's argument springs out from a fierce debate that has been taken place ever since sampling recordings has become a common practice amongst musicians. This debate has also intensified with the development of digital technology, since sampling has become a practice that is easier to perform and is more accessible and affordable to anyone who owns a computer. Additionally, the amount of recorded music that can be easily obtained, shared and copied through the internet, gives musicians a huge palette from which they can sample. Ethical and legal questions therefore have emerged regarding the intellectual property of recordings and how sampling might be considered either as a form of stealing or as a creative practice: On the one hand, a musician can use an existing recording to avoid or simplify the creative process---for example, within the music industry, sampling sometimes is used to save time and effort, minimize costs and maximize profit (to avoid paying a drummer, they might use a recording of a drummer or instead of `coming up' with a good bass line, they might just `steal' it from another recording). On the other hand, a musician may use a recording as part of the creative process or to enhance it---for instance, to re-contextualize it and give it a new meaning, to make a symbolic or cultural reference or to transform it and create something new. The first argument for sampling in my opinion is ethically dubious, while the latter one I believe should be considered morally acceptable. \emph{Plunderphonics} as a practice I propose lies within the second argument (sampling as a creative strategy) as is the case with both of Oswald and Negativeland's sampling practices---not only have these artists clearly used the appropriated material creatively but also through the re-contextualization of these music/images they rise interesting concerns about culture and our condition within it. However important these moral questions are, I also believe that if one considers \emph{plunderphonics} as a musical strategy, one should constantly scrutinize oneself with the simple aesthetic question of whether one is accomplishing something new and original through the process of appropriating recordings or whether one is not only making reference to past notions and definitions of music.
\hypertarget{elaborationapprop}{}

\subsection{Elaborations on Musical Strategies of Appropriation}

Having examined various musical strategies that in my view deal with the process of appropriation in significant and creative ways, I will now attempt to put forward some ideas about how to elaborate them. I will also present some personal suggestions and thoughts about how musicians may approach the practice of using creative work by others to feed into their own creative process. In doing so, I will consider recent technological developments that have taken place since the musical strategies of appropriation
previously discussed emerged. Also, I will suggest that some aesthetic considerations since then have changed and that today we can approach some of the vital questions surrounding musical appropriation differently. Furthermore, the ideas I will elaborate here not only influenced and informed the submitted creative work, but also stem from aesthetic and musical concerns that I have contemplated during the research period.

\hypertarget{appropstrat}{}
The tendency of today's musical strategies of appropriation (as it has also been happening within the fine arts) is towards expanding the palette of appropriated mateiral to include music that comes from less familiar and more obscure sources. In contrast to Negativland or John Oswald's approach to \emph{plunderphonics}, where appropriated recordings exclusively come straight from consumer culture and from `the top ten list' in the record store, today we can expand the sources we appropriate from to include music that comes from less known, more exotic, hidden or forgotten places. Moreover, if we have access to the internet, we have at our disposal an unprecedented amount of sound recordings and musical data (Midi files, synthetic sounds, videos, computer applications, code, etc.). We can draw from all music that is available to us instead of limiting ourselves to familiar music by pop stars and famous composers. Our musical `vocabulary' therefore could increase if we appropriate from a wider palette of music: from songs by lesser known bands, smaller musical scenes, unknown composers and musicians distributed by small independent labels, to virtually unknown korean traditional music, obscure early music or exotic music from pygmy villages. By extending the sources we appropriate from, the musical results of appropriation strategies can be more idiosyncratic and personal. In other words, musicians can disclose the idiosyncrasies of their personality by selecting more distinct and peculiar musical sources. That is not to say however, that musicians should stop appropriating from mainstream music industry if they want to make a reference to, or commentary about, consumer culture or a particular pop artists.

At the same time, today appropriated music can be transformed and modified more extensively as a consequence of the development of digital technologies. Our ability as musicians to process digital data has increased drastically during the last forty years, and I believe we are becoming masters of musical information and sound transformation. Musicians today can, for instance, transform an appropriated sound recording so drastically that the original is no longer recognized as the source. Following the same logic, it is possible to algorithmically modify the information of an appropriated score to such extremes that the resulting music is no longer associated with the original composition. In other words, the amount of computer processing may affect a person's ability to recognize the appropriated musical source (whether it is a recording, score or some other type of musical information that can be digitalized). We have the possibility of applying different degrees of processing to the appropriated material, resulting in a wide palette of musical outcomes available to us: from the radically processed result whose appropriated original is less recognizable as the process takes over the source (possibly perceived as more `abstract'), to the less processed result whose original source remains recognizable (therefore perceived as a `reference' or a `quotation' of the original).\footnote{Here, Claude L\'{e}vi-Strauss's opposition between ``the raw and the cooked'' could be used to describe the two extremes of unprocessed (``raw'') and heavily processed (``cooked'') sounds. See \hyperlink{claudelevi}{pp. 68--69}.} Consequently, we can use sources that are radically processed at the same time as sources that are less processed (or not processed at all), resulting in a type of musical result where `referential' sources co-exist with music that could be perceived as `abstract'. What's more, we can make the process of transformation of the appropriated source dynamic (the amount of transformation can constantly change in time). We can gradually (or more abruptly) change the degrees of processing applied to the original sources so that for instance, the appropriated material gradually could morph from a musical outcome that is perceived as `abstract' to one that sounds like a quotation.\footnote{This happens to an extent in the last movement of Richard Barrett's \emph{Vanity} (1990-1994) for orchestra, where highly complex and abstract musical material gradually transforms into more recognizable music, until at the very end of the composition it is possible to clearly identify a quotation of Schubert's \emph{Der Tod und das M\"{a}dchen}. See \hyperlink{barrett}{Barrett (1996)}.} Of course, it would also be feasible to morph between appropriated sources, like Clarence Barlow does in \emph{1981} with notated material, or by morphing  two appropriated recordings of existing music through sound transformations (and therefore `treating music as sound') using techniques similar to the ones Trevor Wishart uses in his work.\footnote{See \hyperlink{wishart}{Wishart (1996)}.}
\hypertarget{macroplunder}{}

Additionally, with the power of digital processing today we can also decide what particular features, structures, characteristics or parameters we wish to appropriate from music. Whether we appropriate score information, sound recordings, live audio/video or physical gestures, we can analyze the musical sources as long as the resulting information is digitalized. Furthermore, we can focus only on the \emph{micro} or \emph{macro} elements of the musical sources. In other words, it is not necessary to appropriate all of the elements of an existing piece of music, but only extract the desired details from musical data. Therefore, when appropriating notated information, we can consider only specific elements from a composition. Through the analysis of score or Midi information (or similar types of formats) we can for instance extract only rhythmic or pitch material, dynamics, harmonic progressions or tempo from the original composition to use as material for a new musical result. We can easily use more than one musical source and extract different parameters from different sources and then combine them. It is also possible to process and reorder the derived data from the original sources for our desired musical result. We can derive shapes, contours, phrasing, timing, orchestration and other musical abstractions. We could even map a musical parameter from one appropriated source to a different parameter of another source, therefore generating new information from nonidentical data types (this way we could for instance, generate rhythm from harmonic structures or notes from dynamics). Moreover, when dealing with score information, through its analysis and abstraction we can decide to appropriate only the \emph{macro} elements (larger structures) of music. Consequently, it is possible for example to only appropriate harmonic or phrase structures from an original composition and from them derive a structural blueprint for a new composition. In this way, we can appropriate the musical form of a source, only \emph{macro} plundering the original composition.

In addition to using information derived from notation, we can consider appropriating particular features and data from recordings of existing music. With the power of computers, we may concentrate on desired features and parameters of sound. By using different types of computer analysis, we can extract specific information from particular characteristics of sound, such as pitch, rhythm, harmonic structure, noisiness, amplitude, etc.  Therefore, we are able to select only the information we want to use from a recording. For example, we could select pitch information from the spectral data derived from an FFT of one appropriated recording, while just selecting the rhythm from the results of onset detection from another recording. Furthermore, if we consider Curtis Road's notion of \emph{microsound}, we could also contemplate the potential of \emph{microplunderphonics}. \emph{Microsound} as described by Roads refers to \emph{micro} elements of sound (or `sound particles' as he calls them) both in the \emph{window frequency-domain} (spectral domain) and \emph{time-domain}.\footnote{See \hyperlink{roads}{Roads (2001)}.} \emph{Microplunderphonics}, therefore could be a practice whereby musicians use the \emph{microelements} of sound coming from appropriated recordings. However, I am not suggesting that \emph{microplunderphonics} should be limited to a specific time or frequency scale (Roads considers \emph{microsound} sound phenomena lasting less than one tenth of a second).\footnote{Ibid., p. viii.} What I am proposing is that \emph{microplunderphonics} is the practice whereby musicians extract information from appropriated recordings that is somewhat restricted in the frequency or time-domain. The original recording might still be recognizable but only through specific sound characteristics (either by timbre, pitch or other sonic properties). An example of \emph{microplunderphonics} on a time-domain would be to isolate each separate piano attack from a recording of Glenn Gould\footnote{Taking as a starting point its attack and ending point its release or yet another attack.} and reorder each segment such that it sounds like another piece of music. We would still hear that it is Gould playing the piano (the timbre would sound familiar to us) but the result would be so far removed from the original recording that it would have to be considered as a new composition. An example of \emph{microplunderphonics} on a frequency-domain would be if we just extract the fundamental frequency of a vocal melody and re-synthesize it using a different sound---even though we would loose the timbreal characteristics of the voice, we might still be able to recognize the original melody with its characteristic intonation. \emph{Microplunderphonic} elements do not always need to be recognizable and can also be used as pure sound, to generate clouds of `sound-particles', to be heavily processed to produce new sonic material, to generate noise for sound synthesis models (instead of white or pink noise) or as any other type of \emph{micro} element in the creation of sound. 

Moreover, we can combine musical strategies of appropriation such as \emph{plunderphonics} and \emph{musica derivata} within the same \emph{musical result}. That is to say, we can at the same time have a sounding result that combines appropriated sonic material derived from recordings and sounds produced by an instrumental/mechanical realization of notated material derived from existing music. Additionally, we can also fabricate samplers from appropriated recordings and therefore be able to realize notated material with plundered sounds (one could make a sampler for instance by \emph{microplundering} a recording of an instrument, isolating each single note and tuning it according to the pitch of each segment). In other words, we can derive notated material from either scores or analysis/transcriptions of recordings to be realized by either acoustic/synthetic instruments or by samplers made-up from plundered recordings. The distinction between \emph{plunderphonics} and \emph{musica derivata} therefore can be blurred.
\hypertarget{realtimeplunderfuck}{}

Another consequence of the development of digital technologies and the increasing processing speed and power of computers is that today, the process of appropriation can take place in real-time.\footnote{See \hyperlink{realtimepos}{pp. 54--56} for an examination of some of the consequences of real-time computation in music composition.} Music can be appropriated, transformed, manipulated, analyzed and processed within the immediacy of the performance. Not only can we derive and transform preexisting appropriated material in real-time (for example by using records and transforming them in a live performance), but we can also appropriate material that is produced within milliseconds of its plundered result. In other words, we can use live performances of existing music (compositions, songs, specific styles, etc., originally not written by the appropriator) as source material for musical appropriation. In addition to plundering live audio signals, we can use live Midi signals and other types of musical data in real-time. That is to say, we can combine different types of data derived from various live performances of existing music simultaneously to create \emph{real-time plunderphonics} and \emph{live musica derivata}. Furthermore, if one plunders live performances from electronic instruments which produce no significant acoustic sound (the musicians could wear headphones to monitor themselves), the audience would only be able to hear the result of the process of appropriation---consequently creating a cognitive dissonance between audio and visuals: given that the live performances would not be audible in their original form, and only in their re-contextualized/transformed/processed result, what the audience would see would differ from what they would hear. In this type of performance the amount of processing of the audio signals is clearly exposed to the audience through the perception of the audio/visual relationship: the more processed the performances are, the more contrasting they will look in relationship to what is heard through the speakers. In contrast to the \emph{acousmatic} tradition, this type of real-time musical appropriation makes the process of appropriation transparent to the audience through the cognitive association between audio and visuals. Even though the appropriated musical source is not recognizable only through listening, it is visually exposed, disclosing visually not only the source itself, but also the amount of processing that is taking place at a particular moment (the original audible source however remains hidden from the audience). This type of real-time musical appropriation also changes the relationship with the appropriated Other: the performer becomes an accomplice in the process of appropriation (of themselves). 
\indent

As we consider musical appropriation today, we might want to rethink some of the key issues that are relevant to musicians who use these strategies explicitly. First, we may want to reconsider what we as musicians appropriate---this includes appropriated musical objects (instruments, bodies, etc), materials (scores, recordings) and methods (compositional methods, performance practices).\hypertarget{codeapprop}{} I suggest that we should not only attempt to plunder musical scores and recordings, but also consider appropriating performative and compositional systems, processes, schemes and algorithms explicitly. Given that some of these methods are implemented within computer technologies, we may consider sharing, plundering, using and modifying computer code, programmes and applications written by others.\footnote{See \hyperlink{codingcons}{pp. 56-57} for a brief consideration on how coding practices may affect the creative process and the act of composition.} This includes for instance open source software, Max/MSP patches, computer code in high and low-level computer languages and programme libraries. I believe appropriating and sharing code (or patches) is vital to the continuing development of computer music applications and within code that is produced to create music there is both musical and compositional knowledge that is valuable. While in a score or recording only the notated or aural result of music is disclosed, computer code may reveal within it the compositional/performative process itself. In addition to rethinking what we as musicians appropriate, we could also reflect on where we appropriate from: Do we appropriate from the `flea market' or the `shopping mall'? From the music industry or independent labels? From the internet or from our physical surroundings? From western or non-western cultures? Also, we may ask ourselves who are we appropriating from: Are we plundering music by a historical or living composer/performer? A specific pop star or an unknown and obscure musician? Someone we know personally or someone we have not met? Furthermore, I believe we should ask ourselves why we are appropriating these sources: is it for their cultural significance or do they have a personal meaning to us? Do we want to a suggest a metaphor through the symbolism implied by the sources or are we only interested in them for their use or sonic qualities? Are we using these sources to showcase them in different configurations or in an attempt to create with them something significantly new? Finally, I believe that the last question that we should ask ourselves regarding musical appropriation is how we as musicians appropriate---the process by which we borrow or plunder and how we modify, transform or re-contextualize the appropriated objects, materials and methods, as well as how this affects the perception of the musical result. I hope that through this chapter I have opened up some possibilities and ideas of how this question might be answered. I consider that reflecting about these important issues regarding the process of appropriation is vital for the musician's creative process as they can be revealed within the \emph{musical result}. The relevance of the answers to these key questions is especially significant today if we seek to create something new through musical appropriation.

\label{ch:approp}