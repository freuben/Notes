\hypertarget{chapter4}{}
\chapter{Technology and Strategy}

In this chapter, I will critically examine recent technological developments that I consider had considerable impact on the musical strategies implemented during the creation of the submitted work. Technological advancements will be discussed not for their scientific value nor for their implementation in the creation of already existing models of music-making. My interest rather lies in how preconceived notions of \emph{what music is} may be redefined through strategies that use technology to challenge fundamental aspects of how we create and experience music. I will particularly focus on how technology may have an impact on musical strategies that are concerned with the relationships established in music-making. I will therefore examine musical strategies that use technology to alter relationships established traditionally through well-known conventions in performance-practice and composition. In addition, I will analyze how recent technology brings new opportunities to imagine new ways in which music may be performed and presented. Considering the particularities of the electronic medium, the causality between human action and sounding result found in traditional music for acoustic instruments may not be apparent to the audience in live electronic music. Therefore, new opportunities exist in live electronic performance to form new types of perceptual relationships between an agent's action and a sounding result as well as between the human performer and the technological object. Moreover, technology might serve as tool to encourage new ways of thinking about the relationship between musicians and audience. Musical strategies mediated through technology might be utilized to consider new conditions of exchange and new types of audience participation. However, I will argue that these strategies by themselves do not guarantee an active involvement from the listener and for this reason, I will advocate for an approach to these strategies that promotes audience engagement through critical thinking and reflection. Furthermore, I will explore how musical strategies that employ technology can foster new approaches to composition and how real-time computation and interactive systems pose challenges to the conventional distinctions between performance/composition and composition/improvisation. Finally, considering the emergence of new compositional practices through the use of computers, I will suggest that the role of the composer in live electronic performance is changing and his/her practice is becoming more varied to include activities as diverse as composition, performance, improvisation, musical direction, instrument building, postproduction and computer programming.

\section{Technological vs. Musical Innovation}

Before discussing my views on how technology might have an important function in rethinking musical strategies, I would like to examine some problems that might arise regarding the use of recent technology in music. As a musician, one of my concerns regarding the relationship between technology and music is that on many occasions scientific innovation and technological curiosity are given priority over musical creativity and aesthetics. Luciano Berio has eloquently expressed the same position:
\begin{quote}
If in the past---even the distant past---music was often the testing bench and the stimulus for scientific research, and thus music tended to draw scientific knowledge to it, in more recent years you get the impression that it's now science that draws music to it and takes possession of it. Indeed, you often get the impression that a scientific creativity applicable to music has substituted itself for musical creativity, and that musical thought has regressed to the level of the (invariably squalid) opinions that an electronic engineer from Bell Telephone or a Stanford ``software man'' may have about music.\footnote{\hyperlink{berio}{Berio (1985)}, p. 121.} 
\end{quote}
The attitude of giving more importance to technological (as opposed to musical) innovation while creating music has also increased with the complexity and development of the tools themselves. Scientists and technologists often create music with the sole purpose of demonstrating new developments in music technology. Additionally, musicians that are interested in using technology to a higher level of sophistication very often need to immerse themselves in intricate technological subjects. These circumstances can be misleading for the musician if his priorities shift from a position in which technology is researched and developed for its creative potential in music, to a position in which technological innovation becomes the driving force behind musical creativity.  The shift of attention might even happen without the musician's awareness as a consequence of the effort one needs to go through in understanding the complexity of the technological tools and research developed in this field. This can be deceiving and even `dangerous' if music becomes just a showcase of new technological advancements. 

The experience gained by musicians during the second half of the twentieth century who worked closely with technology can also be very valuable to us today as a warning of the possible problems that working with technology might lead to. Looking back at Berio's account of his experience on this issue, one can grasp how the notion that new technological developments lead to important musical progress is erroneous. On his account, Berio describes how the advancements which permitted the creation of new sounds with electronic means did not by themselves produce any meaningful musical results. 
\begin{quote}
Thus many of the more sensitive musicians quickly realized that it was as easy as it was superfluous to produce new sounds that were not the product of musical thought, just as it's easy nowadays to develop and `improve'' the technologies of electronic music when these are devoid of any real and profound \emph{raison d'\^{e}tre}.\footnote{Ibid., p. 122.}
\end{quote}
He goes on to describe how music that was motivated by technological developments instead of musical thought resulted in a display that did not address the complex set of relationships and conventions that take place in music.
\begin{quote}
It was recognized, for example, that the spectacle of a public gathered together to listen to loudspeakers was not a particularly cheerful one, and that, yet again, the experience of public musical listening was made up of many different conventions, and was rooted in many different aspects of social and cultural life: it was not made up merely of a piece, a musical object to listen to, even if it proposed ``new sounds''. By its very nature, a piece of music by itself cannot easily transform listening conventions and socio-musical relations in general.\footnote{Ibid., pp. 122,123.}
\end{quote}
The lesson to learn from Berio's statement is clear: musical and technological innovation are inherently different from each other and if one's interest lies in creating music, one needs to guide technological interests and development with priorities that will be relevant to the desired \emph{musical result}. That is not to say of course, that scientific research or technological development regarding music is not valuable. On the contrary, my position is that technology can have a vital role in musical innovation if it is developed with a critical approach and considering the complex social, cultural and philosophical aspects inherent in music's definition. Moreover, if technology is developed imaginatively with the purpose of creating new musical strategies for the future, it might help reshaping the way in which we make and experience music.

\section{Reshaping Relationships in Music Through Technology}

Even though technology may play a key role in rethinking many aspects that form part of a \emph{musical result}, here I will focus specifically on new strategies concerning the relationships between composer, performer and audience. Therefore, I am not going to go into detail into subjects that are not related to this specific area of interest as this would be out of the scope of this commentary. Nevertheless, I belive that there is huge potential and work to be done in these areas, which include concerns such as how technology may radically change the way in which musical institutions operate; the visual elements related to the performative aspects of music; how music is recorded, distributed, advertised and consumed. However, what I will concentrate on here is how technology brings a unique opportunity to envision new compositional and performative strategies based on reshaping relationships that have been established traditionally through compositional and performance-practice conventions. I will therefore start by examining the possibilities technology could bring in revising the way in which musical knowledge is transfered by imagining a new type of score that would combine oral and visual traditions within a multimedia experience.

\subsection{The Score in the Digital Age}

By now, much has been written about the limitations and advantages of the traditional score as a form of communication between composer and performer in western music.\footnote{See, for instance \hyperlink{goer}{Goehr (2007)}, \hyperlink{emmersoncross}{Emmerson (2000)}, \hyperlink{small}{Small (1998)}, \hyperlink{wishart}{Wishart (1996)} and \hyperlink{hamilton}{Hamilton (2008)}.} Through research in \mbox{ethnomusicology} and other music practices that incorporate improvisation, an increasing attention has been given to other forms of knowledge transfer in performance-practice that do not utilize a written score. These might include oral traditions that incorporate practices such as transferring music from one generation to another through a master-apprendice relationship or the now common convention of studying recordings as a method of learning a particular song, style, genre or performance-practice. It has also been argued that the score is a medium that is highly individual and `isolates' the performer not only from the audience but also when playing within a group of musicians.\footnote{See \hyperlink{emmersoncross}{Emmerson (2000)}, p. 121.} On the other hand, the idea of using notation has been defended as well for its capacity of capturing complex musical ideas and thoroughly worked structures, establishing a particular relationship between composer and performer, providing points of reference for the performer and making certain types of ensemble playing possible (for example, facilitating certain types of synchronicity and group playing within an ensemble).\footnote{See, for example \hyperlink{ferneyhough}{Ferneyhough (1995)}, for an in depth discussion not only about the difficulties implicit in the practice of notation (the impossibility of depicting sound as visual representation), but its potential as a vehicle to express ideological concerns and to achieve auto-instrospection,  as well as the role it might have as a common denominator in different fields of musical interests. According to Ferneyhough, the score contributes to the \emph{act of composing} as an exercise in self-analysis through the process of notation, and to the \emph{act of performance} by establishing the (social and contextual) conditions of its realization.} My position regarding this matter is that the score is still a valuable tool for communicating with musicians trained within western tradition and it is worth expanding the notion of the score to include new strategies that can be developed through technology that might enhance or facilitate communication between composer and performer. In this respect, I completely agree with Simon Emmerson, who argues that there is still a need to develop new forms of notation through technology that are flexible enough to in encapsulate different strategies of transferring musical knowledge. 
\begin{quote}
But we have one new invention which may hinder and help our endeavor: the computer. Its power was rapidly applied to western music . . . Composition, analysis, transcription, sound production, processing, storage and distribution are all now in one way or another within its domain. . . . An unaddressed need remains: the development of more flexible notation systems; these may also be stimulated by the development of a new generation of music interfaces. . . . We should dream of a technology which bypasses some of theses constraints: a combination of ear and eye---a new `superscore'.\footnote{\hyperlink{emmersoncross}{Emmerson (2000)}, pp. 121-122.} 
\end{quote}
Emmerson's idea of a `superscore' combines oral and visual forms of communication within a multimedia object combining traditional notation, extended notation, recordings of example material from the live performer, electroacoustic materials, software for performance, patches for live electronic treatment, examples of live electronic treatment, an example recorded performance, written and spoken commentary, video performance material, video example material and graphical material.\footnote{Ibid., pp. 128-129.} 

Taking Emmerson's idea further, one could easily imagine the `superscore' as a package that combines performance materials with documentation---including video tutorials, audio examples (sampled mock-up performances or real performances), recordings, interviews, \emph{etc.}---residing on the internet. Additionally, with the increasing accessibility of laptops, one could easily imagine replacing a score that is printed on paper, with one that is displayed on a computer monitor. This would bring the opportunity of exploring the potential to communicate musical meaning through a computer display, which would add movement to the expressive palette of a conventional score. By using animated graphics, scores, pictures, as well as other types of visual cues and timed written directions, the composer could enhance the way in which he communicates musical ideas and knowledge through the computer display. In addition, the performer could receive other types of audio information through headphones complementing the visual input with an `aural score'. This could comprise from spoken directions and sounding cues (click tracks, reference pitches,  \emph{etc.}) to recordings of acoustic or electroacoustic music that the performer would have to react to or improvise with. Moreover, with the development of real-time processing technologies and generative algorithms, the notion of a \emph{fixed} score could also be contested by a score that is \emph{dynamic}, thereby creating a composition that may change its content (pitches, rhythms, \emph{etc.}) each time it is performed. Real-time scoring could be explored further by combining elements of real-time animation and graphics display with new advancements in machine listening and interactive technologies, thereby generating a score that not only responds to the sonic and acoustic context of a specific performance, but also reacts to the audience's immediate participation and response. The possibility of creating a network including several computers could also provide instant communication between performers and the option for the composer or conductor to send directions that would be specific to a particular performance. With the increasing popularity of wireless networks and new types of interfaces and gadgets, portable devices like the iPad or iPhone could be used to implement the `superscore', making it easier to carry and even place on a music stand. 

In addition to enhancing communication with musicians trained within the western tradition, the `superscore' could also foster new collaborative possibilities between performers of different cultures.\footnote{It could also open up collaborations with artists from other disciplines (actors, film makers, dancers, choreographers and visual artists, \emph{etc.}). However, these possibilities will not be examined in detail as this subject is outside of the main focus of this commentary. Nevertheless, I consider that there is significant potential to develop this subject for research that focuses on music and cross-arts collaboration.} By sending information that is specifically devised and customized for a particular type of performer, the `superscore' could provide the opportunity for musicians from different backgrounds and traditions to share the stage simultaneously in a computer-mediated performance. A group of performers from mixed backgrounds could therefore play together within a predetermined structure by receiving different types of visual and aural stimuli.\footnote{Here, I want to emphasize the importance of incorporating an aural element to the score, particularly for some non-western musicians that rely on oral traditions and are not used to performing with any type of notation. Nevertheless,  at the same time one should not overlook that some musicians from non-western traditions actively seek alternative strategies that not only depend on oral traditions. In my experience, this constructive attitude is also sometimes contingent on the personal relationships involved in the `intercultural' exchange.} The collaborative opportunities this could bring are considerable as technology could facilitate and even solve problems that until now have made it difficult (if not impossible) for musicians from different backgrounds to play together.  

\subsection{Crossing Cultural Borders?}

Given the opportunities technology brings for a diverse group of musicians to share the stage despite previous incompatible performance conventions, important questions arise concerning the types of relationships established during collaboration. These relationships might become particularly sensitive if one is collaborating with musicians from different cultures. In his article \emph{Crossing Cultural Boundaries through Technology}, Simon Emmerson already expresses some concerns as a composer when dealing with cross-cultural collaborations and `ensembles with ethnic instruments'. He argues that the western composer often appropriates music from different cultures through `strongly filtered sources' and cultural misunderstandings, frequently resulting in `cultural murder'. 
\begin{quote}
There are plenty of examples of composers killing stone dead the spontaneity and vitality which they themselves admire in non-western music through insensitive appropriation of surface technique (usually, once again, through an inadequate notation system and inadequate formalized `rules'). Too simple an understanding of acculturation may hinder the very process we aim to foster.\footnote{Ibid., p. 126-127.} 
\end{quote}

Emmerson suggests the western composer should undergo a process that surpasses the initial first impression of the other culture's music---which is solely based on our previous expectations and experience---to develop a process where `new measures of significance' are created. According to Emmerson, this stage is crucial: if the western composer declares intentions to define the meaning of the musical result (based on misconceptions and misunderstandings of the other culture), he might reinforce ``the purely western basis for the evaluation of such projects thus defeating much of their object''.\footnote{Ibid., p. 126.} He therefore promotes a positive attitude towards `successful acculturation' through education, practical experience, mutual understanding and respect.\footnote{Ibid., pp. 115-134.} 

\hypertarget{multiculturalism}{}
Even though Emmerson's position appears to be sincere and well-intentioned, a danger exists if it lends itself to an attitude analogous to the notion of \emph{multiculturalism}, which Slavoj \v{Z}i\v{z}ek has rightfully criticized. According to \v{Z}i\v{z}ek, \emph{multiculturalism} is a tendency that has spread in western nations through globalization that treats local (other) cultures with `respect' and displays an interest in studying, understanding and preserving their traditions. Nevertheless, this arrangement is established through a hegemonic relationship---imposed by western nations and from a western perspective---by maintaing a condescending distance between the dominant and repressed cultures.
\begin{quote}
Multiculturalism involves patronizing Eurocentrist distance and/or respect for local cultures without roots in one's own particular culture. In other words, multiculturalism is a disavowed, inverted, self-referential form of racism, a `racism with a distance'---it `respects' the Other's identity, conceiving of the Other as a self-enclosed `authentic' community towards which he, the multiculturalist, maintains a distance rendered possible by his privileged universal position. Multiculturalism is a racism which empties its own position of all positive content (the multiculturalist is not a direct racist, he doesn't oppose to the Other the \emph{particular} values of his won culture), but nonetheless retains this position as the privileged \emph{empty point of universality} from which one is able to appreciate (and depreciate) properly other particular cultures---the multiculturalist respect for the Other's specificity is the very form of asserting one's own superiority.\footnote{\hyperlink{zizekuniv}{\v{Z}i\v{z}ek (2006), \emph{The Universal Exception}}, `Multiculturalism, or, the cultural logic of multinational capitalism', p.170-172.}
\end{quote}
Emmerson's approach towards intercultural projects might become misleading if it is assumed that through a process of education and experience with music/musicians from `other' cultures, these projects will lose their western basis and become productive or successful cultural exchanges. Moreover, this process of study and practical exchange might in itself become the basis of establishing a relationship of power and an attitude that reflects---as \v{Z}i\v{z}ek would say---the way `the colonizer treats colonized people'.\footnote{Ibid., p. 170.} I will therefore suggest that a more `honest' form of exchange is to approach intercultural projects with skepticism and self-awareness; without distancing oneself from the musicians from `other cultures' by treating them with special respect or with a fake notion of open-mindedness.\footnote{Here it is also important to mention that many `intercultural' collaborations happen with non-western musicians who are nevertheless often second or third generation immigrants and therefore are already enculturated within western social relationships. It is also important to notice that some musicians who grew up or still live in non-western countries might also be extremely familiar and at ease and with western social relationships due to globalization and the prominence of western influence in some of these countries.} I would propose dealing with these musicians as one would deal with other musicians within our own culture (we are not usually particularly concerned with treating people within our own culture with special `respect' or distance), by collaborating with them (without assuming a patronizing distance) towards one's desired musical result. One should also assume that there will be a struggle involved in the process of intercultural collaboration as there are always different types of violence and relationships of power that emerge during cultural exchanges. 

The way in which we deal with music and musicians from different cultures underlines a wider problem, that is, how should we as creative musicians approach the act of appropriation. Nevertheless, before engaging in such discussion,\footnote{See \hyperlink{chapter5}{Chapter 5}, for a discussion about appropriation in music.} I would first like to consider how technology---and more specifically real-time computer processing---may offer new applications that challenge the conventional notion of a musical performance and the relationships established traditionally in music-making.

\section{Live Electronic Music Performance}

The introduction of the computer to live performance offers the possibility to establish new relationships regarding the way in which we perceive a musical performance. The causality inherent in traditional music produced with mechanical means,\footnote{This includes traditional means of producing vocal, instrumental and mechanical music.} which follows `well-understood Newtonian mechanics of action and reaction, motion, energy, friction and damping,'\footnote{\hyperlink{emmersonliving}{Emmerson (2009)}, p. xiv.} does not need to apply to live electronic music performance. In electronic music, the causal relationships found in our acoustic surroundings are usually not clearly revealed, given that sound may be produced with little evidence of mechanical production (with the exception of the vibrating cone of the loudspeaker). Nevertheless, considering that most of our sonic experience lies within our acoustic environment, we usually seek to form causal relationships, even within the electronic medium. Therefore, many efforts have been made to reestablish these causal relationships by suggesting that the human performer is clearly the agent producing the sound by `playing' a technological object as one would an acoustic instrument. This has been attempted through the continuing development of interfaces that attempt to reestablish an instrumental approach to electronic music (for example synthesizers, Midi samplers, electric guitars, \emph{etc.}). Nevertheless, electronic music performance also offers new opportunities to form other types of relationships as perceived by the listener. This specific feature of the electronic medium may challenge conventional notions of what a musical performance is as it may form new types of relationships that go beyond the traditional instrumental approach. Therefore, when dealing with electronic music performance, the composer may decide what types of relationships he/she wants to establish---for instance, how different sonic and visual aspects of a performance may relate with each other or how the human body and movement may be associated to sound. 

Simon Emmerson, in his book \emph{Living electronic music}, describes different approaches the musician may take towards electronic music performance based on how the audience may perceive the actions of the human performer in relationship to the sounding result. First, he describes what he calls the `Local/Field Distinction', in an attempt to conceptualize differently relationships that seem to have a perceived causality between a human performer's action and the sounding result, and those that don't.
\begin{quote}
\emph{Local} controls and functions seek to extend (but not to break) the perceived relation of human performer action to sounding result. \emph{Field} functions create a context, a landscape or an environment within which \emph{local} activity may be found. It is important to emphasize that the \emph{field} as defined above \emph{can contain other agencies}, in other words, it is not merely a `reverberant field' in the crude sense but an area in which the entire panoply of both pre-composed and real-time electro-acoustic music may be found. . . . This definition aims to separate out the truly live element as clearly the `local agency' in order to re-form more coherently the relationship with this open stage area, which may surround the audience and extend outside.\footnote {\hyperlink{emmersonliving}{Emmerson (2009)}, p. 92.}
\end{quote}
This distinction is useful to the musician as it encourages reflection on how the presentation of electronic music performance---particularly aural/visual relationships concerning causality and human presence---might influence the listener's perception of the overall \emph{musical result}. Additionally, given the particularities of the medium, the electronic musician is encouraged to rethink important aspects about performance (for instance, how it might look, what functions the musicians might perform onstage, what types of human/machine interaction might be established, \emph{etc.}). This distinction can also be helpful if it is considered creatively as a parameter within a composition: the distinction between \emph{local} and \emph{field} could be emphasized or blurred according to the desired musical moment, the extremes could be alternated or even morphed between each other, an extreme might be embraced as the other is sublimated, \emph{etc}. In addition, Emmerson also makes a difference between \emph{real} and \emph{imaginary} relationships that may be \emph{local} or \emph{field}. According to Emmerson, \emph{real} relationships are also `real-time' and have direct relation with the \emph{real} cause as perceived by the audience (a sonic result that can be followed by the listener). This may include processing the `live' sound, abstracting a gesture through an interface or sensor, or through other types of analysis (audio or video). \emph{Imaginary} relationships, on the other hand, are `prepared in advance (soundfiles, control sequences,  \emph{etc.}) in such a way as to \emph{imply} a causal link of sound to performer action in the \emph{imagination} of the listener'.\footnote{Ibid, p. 93.} Emmerson also emphasizes that the difference between \emph{real} and \emph{imaginary} relationships might be different for the listener as they are for the composer (or as they are in reality). Even though I find Emmerson's terminology slightly confusing,\footnote{His distinction I don't find particularly useful as it seems to make a link between \emph{real} relationships with `real-time' processing and \emph{imaginary} relationships with `fixed' or prepared material. I think this is misleading, as `real-time' processes usually contain large amount of prepared or `fixed' elements (for instance, computer programs, patches, data bases,  \emph{etc.}, that have been prepared in advance) that also create what Emmerson calls \emph{imaginary} relationships and an \emph{illusion} of causality. That is to say, his terminology might lead to misunderstandings as it equates types of relationships the listener makes to whether an electroacoustic part is influenced by a performer or is autonomous.} I think it points towards an issue that I think is important to anyone dealing with electronic music performance, that is, what should concern us is what \emph{appears} to be real or not to the listener, and not whether technological processes are taking place `in reality' (real-time) or have been prepared before hand. Consequently, the question of whether to use `real-time' processing or not should stem from aesthetic concerns in relationship to the listener's perception of the performance and not from `technical authenticity', or to cling to a set of technological concepts.

The opportunities that electronic music gives in forming new relationships between the performer's action and the sounding result, gives the composer the option of thinking creatively about how a performance might be presented. The cognitive dissonance that might arise between aural and visual elements of the performance could be used as a performative element, creating meaning out of the apparent sensorial disjunction. This approach could even be exaggerated, for instance, by suggesting causal relationships that may be only observed and have no corresponding sounding result, or by creating a visually static performance while having a sounding result that would suggest frenetic activity. This slightly more idiosyncratic approaches towards the presentation of electronic music may also encourage people to reflect on the subject of how the performer relates to a technological object, which at the same time may prompt two important questions, mainly, what types of relationships we form with technological objects and how we as human beings relate with each other through technology.

\subsection{Interactivity or Interpassivity?}
 
The conventional viewpoint regarding the relationship between human beings and technological objects is that we relate with them through interaction. However, we should also consider the notion of \emph{interpassivity} to describe a different type of relationship that might be formed between humans and technology. \emph{Interpassivity} is a concept first coined by philosopher and cultural theorist Robert Pfaller\footnote{See \hyperlink{pfaller}{Pfaller (2003)}.} to describe the opposite of \emph{interactivity}.
\begin{quote}
Obviously, the concept of interpassivity is opposed to that of interactivity. Interactivity in the arts means that observers must not only indulge in observation (``passivity"), but also have to contribute creative ``activity" for the completion of the artwork. The interactive artwork is a work that is not yet finished, but ``waits" for some creative work that has to be added to it by the observer. What could be the inverse structure of that? The artwork, then, would already be more than finished. Not only no activity, but also no passivity would have to be added to it. Observers would be relieved from creating as well as from observing. The artwork would be an artwork that observes itself.\footnote{Ibid. p. 2.}
\end{quote}
An example of \emph{interpassive music} would then be a concert music performance in which the performers create, listen, observe, reflect on and enjoy the performance in place of the audience. The performers therefore have fun and take pleasure from the concert instead of the spectators, who delegate their pleasure to the performance itself. According to Pfaller, \emph{interpassivity} therefore consists of a relationship in which ``passivity'' is delegated---someone or something enjoys or consumes \emph{instead of me} and at the same time \emph{for me}. 
\begin{quote}
Interpassivity is delegated ``passivity"---in the sense of delegated pleasure, or delegated consumption. Interpassive people are those who want to delegate their pleasures or their consumptions. And interpassive media are all the agents---machines, people, animals etc.---to whom interpassive people can delegate their pleasures.\footnote{Ibid. p. 3.}
\end{quote}
Consequently, if we are to think of how \emph{interpassive} relationships might be formed between humans and technological objects, we need to think of a technological object that serves as an agent which people can delegate their enjoyment or consumption to. A behavior that exemplifies this type of relationship with a technological object, is one common amongst people who take pleasure in downloading mp3s to make their collection more interesting and complete, without taking the time to actually listen to the music. They delegate the act of listening to the computer (the technological object) through the act of downloading the computer files. The mp3 collector therefore acts as if the computer listened for them---it is not necessary to listen to the music as the computer already did that for us through the act of downloading and storing the files. By substituting the act of listening by the figurative or symbolic act of downloading, a link is formed between the human and the technological object through representation. 

Furthermore, Slavoj \v{Z}i\v{z}ek, has pointed out that \emph{interpassive} relationships may be formed through the opposition between active and passive roles---were passivity is delegated to one party through the others' (hyper)activity. 
\begin{quote}
Interpassivity, like interactivity, thus subverts the standard opposition between activity and passivity: if in interactivity (or the `cunning of Reason'), I am passive while being active through another, in interpassivity, I am active while being passive through another. More precisely, the term `interactivity' is currently used in two senses: (1) \emph{interacting with} the medium---that is, not being just a passive consumer; (2) \emph{acting through} another agent, so that my job is done, while I sit back and remain passive, just observing the game. While the opposite of the first mode of interactivity is also a kind of interpassivity, the mutual passivity of two subjects, like two lovers passively observing each other and merely enjoying each others presence, the proper notion of interpassivity aims at the reversal of the second meaning of interactivity: the distinguishing feature of interpassivity is that, in it, the subject is incessantly---frenetically even---active, while displacing on to another the fundamental passivity of his or her being.\footnote{\hyperlink{zizekreader}{\v{Z}i\v{z}ek (2006), \emph{The \v{Z}i\v{z}ek Reader}}, `The Fantasy in Cyberspace', p. 105.}
\end{quote}
\v{Z}i\v{z}ek argues that for the subject to be relentlessly active---and thus to delegate passivity---the Other needs to make excessive demands from the subject to produce this incessant behavior. That is to say, this type of \emph{interpassivity} presupposes that the Other consistently demands from us, consequently arousing in us relentless activity. At this point, it is interesting to think what the impact of this type of relationship might be regarding the way in which we relate to technological objects. While the ordinary stance regarding the way in which we relate to technological objects is that we use them---with the purpose of producing or creating something (either \emph{with} or \emph{for me})---what this notion suggests is that some technological objects might in actuality demand something from us. It would also be relevant to think for example how this idea might impact the concept of the `user' (a term commonly employed by developers of computer technologies) as it might have to be expanded not only to refer to a person that \emph{uses} technology, but also to include a person that is \emph{used} by technology. Furthermore, \v{Z}i\v{z}ek agrees with Pfaller in that \emph{interpassivity} implies delegating passivity---while we are obsessively active (by the Other's demands), we also rely on the Other to be passive for us. The act of being obsessively active (and thus displacing passivity to the Other) therefore creates an `empty' ritual, in which we go through a set of gestures and feelings precisely so that we do not have to experience them in reality (thus avoiding exposure to the \emph{true} feelings and emotions). Moreover, \v{Z}i\v{z}ek claims that through our activity we attempt to symbolically subvert the Other's desire, to put off our recognition that enjoyment cannot be achieved in full. According to \v{Z}i\v{z}ek, it is precisely for this reason that the notion of \emph{interpassivity} is vital in understanding the artistic possibilities of digital technology.\footnote{See Ibid., pp. 104-110.}

The notions of \emph{interpassive} and \emph{interactive} relationships as reflecting an opposition between active and passive roles may also be applied to the way in which a performer might relate to a technological object in a live electronic musical performance. The performer therefore might form \emph{interactive} relationships with a technological object if he/she seems to remain passive while technology appears to be active. For instance, when a performer plays a note or presses a button that sets an active chain of sound, or the `typical' laptop performer's role of sitting behind the computer appearing to be passive while triggering musical events that suggest activity. The performer might also form \emph{interpassive} relationships with technology, when a technological object appears to remain passive while making the performer appear franticly active. This concept hasn't been explored thoroughly by composers using technology but can be found for example in cases where the performer receives directions from a technological object (through a computer display or through headphones) directing the performer towards frenetic activity---the point of the activity being to give the semblance of reality to the illusion that one is controlling the technological object, when in reality one is just following its demands. I think a potential exists to further develop musical applications that establish \emph{interpassive} relationships between performer and technological objects through computer-mediated performances or more experimental methods, such as involuntary bodily movement\footnote{See \hyperlink{stelarc}{Stelarc (2004)}.} or by radically altering the sound of a performer playing an electronic instrument (which produces no considerable audible sound that is not generated electronically) such that the initial physical effort of the performer is reduced or stripped to silence by computer processing (giving the impression of human activity being subverted through technology). 

Additionally, the musician might also establish \emph{interactive} as well as \emph{interpassive} relationships with the audience through technology.\footnote{However, this is not the same as forming relationships between the audience and a technological object. These types of \emph{interactive} and \emph{interpassive} relationships are nevertheless possible. For example, the audience could form \emph{interactive} relationships with a technological object if for example, the audience would give a few instructions to a computer that would generate and perform a composition for them. \emph{Interpassive} relationships on the other hand, could be formed for instance, if an audio/visual recording device coming with a complex set of instructions (of how and when to operate it to get the best results) is given to the members of the audience during the time of the performance (having to return the device and recording at the end of the performance), keeping them busy but not having to pay attention to the performance itself as they rely on the recording device to do the listening and observation for them.} \emph{Interpassive} relationships might be formed if the audience remains passive through technology, while the performer displays activity. This would be the case for example if one would follow a model whereby the performer displays intense activity by playing an electronic instrument without amplification and at the same time monitoring himself through headphones, while the audience remains passively seated (without hearing what the musician is playing) and therefore delegating the pleasure of listening to the musician. On the other hand, \emph{interactive} relationships may be formed if the audience becomes active through technology, while the performer seems to remain passive. This is the case for example of the laptop performer or DJ---which play music that encourages the audience to become active by dancing frenetically---while staying behind their computer offering no clue that they are in actuality producing the sound. However, it could be argued that this apparent activity displayed by the audience through bodily motion at the same time might shift the audience's attention away from certain musical content, resulting in a type of passivity. In other words, by becoming active through movement, the audience might stop focusing on certain aspects of the music as their attention shifts towards physical activity, resulting in a reversal of \v{Z}i\v{z}ek's first definition of \emph{interactivity}. A reversal could also be applied regarding the engagement of the audience within the concert hall model: while the audience seemingly takes the role of passive spectator and the musicians displays activity through their virtuosity; in actuality, the audience might be the one engaging with the music, rendering emotional and intellectual activity through the performance, while the performers `stop listening' as they concentrate on bodily movement.  

\subsection{New Relationships with the Audience}

The particularities of live electronic performance also encourage new ways of thinking about the relationships that may be established between musicians and audience during a performance. Due to the increasing development of technologies that have an impact on the performer/audience relationship, today we have at our disposal a considerable amount of tools that enable us to reconsider and rethink this exchange. The traditional forms by which the audience experiences music may therefore be expanded by establishing new conditions for exchange mediated by technological tools. Thus, with the creative use of new technologies we are able to change the traditional way in which the audience participates in a musical performance. What's more, through technology composers can devise a piece of music partially based on these conditions of exchange, which could evolve and change during the performance.\footnote{For instance, the composer could devise a performance that changes how the audience participates during the duration of an event: at the beginning of the event the audience could interact directly with the performance by actively giving directions to the musicians through a computer system on how the music should sound. After some time, the performers would gradually stop receiving directions and start taking their own as the audience would gradually assume a more conventional position as listeners/observers. Finally, at the end of the event, the performers could delegate their complete responsibility to the members of the audience, who would  now have to produce all of the sounds by `playing together' through the computer system (this could be achieve for instance, by generating sounds through the analysis of their gesture and movement), forcing them to interact and work with each other to create a performance for the musicians who now would become the spectators.}

Seeking to form new types of relationships between musicians and audience through technology could also lead to developing new interactive possibilities in a musical performance. The audience could take an active role making decisions as far as how they want to experience the performance. These decisions could go as far as what type of content a composition may have and how it might unravel. Interactive elements usually associated with installation work, could be incorporated within a musical performance: the audience could explore the performing space triggering and modifying musical events by interacting in different ways with each other, the space and performers. In other words, the performance could become immersive and the audience could directly influence its outcome. Another musical strategy that could be developed through technology is something close to what Nicolas Bourriaud has called \emph{Relational Art}, which refers to ``a set of artistic practices which take as their theoretical and practical point of departure the whole of human relations and their social context, rather than an independent and private space.''\footnote{\hyperlink{relational}{Bourriaud (2002)}, p. 113.} In other words, a musical performance could be contemplated for its potential as a collective space in which members of the audience could engage with each other in a variety of ways. Technology could mediate this platform of exchange by facilitating tools by which individuals within the audience could communicate with each other and establish new kinds of transactions.

However, these strategies alone do not guarantee a type of activity from the audience that suggests reflection and encourages critical thinking and creativity. As I suggested earlier, the illusion of activity and passivity might be deceiving---the appearance of this opposition might in reality be its reversal. What is conventionally associated with passivity, in actuality could suggest a different type of activity and \emph{vice versa}. Jacques Ranci\`{e}re addresses these issues in his book \emph{The Emancipated Spectator}, where he challenges preconceived notions that associate listening and observation to passivity and identifies the audience as inactive. Ranci\`{e}re therefore proposes a vision for a spectator that, while seated and listening, is active---fabricating his/her own interpretation and understanding of the performance, associating it with his/her own ideas about the world and the future. This kind of spectator is emancipated in as much as he/she is not manipulated by the performance, but maintains a critical distance and independence from what he/she experiences as an observer.

\begin{quote}
Emancipation begins when we challenge the opposition between viewing and acting; when we understand that the self-evident facts that structure the relations between saying, seeing and doing themselves belong to the structure of domination and subjection. It begins when we understand that viewing is also an action that confirms or transforms this distribution of positions. The spectator also acts, like the pupil or scholar. She observes, selects, compares, interprets. She links what she sees to a host of other things that she has seen on other stages, in other kinds of places. She composes her own poem with the elements of the poem before her. She participates in the performance by refashioning it in her own way---by drawing back, for example, from the vital energy that it is supposed to transmit in order to make it a pure image and associate this image with a story which she has read or dreamt, experienced or invented. They are thus both distant spectators and active interpreters of the spectacle offered to them.\footnote{\hyperlink{ranspec}{Ranci\`{e}re (2009), \emph{The Emancipated Spectator}}, p. 13.}
\end{quote}

Ranci\`{e}re's positive image of an emancipated spectator\footnote{The notion of an emancipated spectator also relates to Roland Barthes' concept of \emph{musica practica}, which describes an idea of a type of music in which the spectator acquirers skills to `put oneself in the position or, better, in the activity of an operator, who knows how to displace, assemble, fit together'. Barthes argues that to compose \emph{musica practica} is `to give to do'---one does not \emph{receive} this type of music but \emph{operates it}, encouraging an active and creative role of the spectator. See \hyperlink{barthes}{Barthes (1977)}, pp. 149-154.} therefore resists the notion of a spectator that is passive, submissive and dominated by the stimuli that is thrown at him/her. Furthermore, Ranci\`{e}re questions strategies that attempt to break with the hegemony established by what Debord describes as `the society of the spectacle'\footnote{Where the `spectacle' (entretainment, mass media, \emph{etc.}) serves as a tool for power and domination of the masses. See \hyperlink{debord}{Debord (1994)}.} through practices that for instance invert the position between audience and performers (or blurs the line between the two), change the space of the performance (presenting it outside traditional spaces) and blur the line between everyday occasions and the performance.\footnote{See \hyperlink{ranspec}{Ranci\`{e}re (2009), \emph{The Emancipated Spectator}}, p. 15.} Even though these practices might be interesting for aesthetic purposes, many times in the way that they are carried out, they do not deal with the idea of a performance as having as its principal aim to assemble spectators that end the hegemonic relationships of the `spectacle'. That is to say, these strategies as carried out by many artists only `redistribute positions and spaces' and do not by themselves encourage the audience to actively listen, observe or think critically during a musical performance.\footnote{That is not to say however, that these strategies could not be carried out effectively to contribute to the type of engagement that would break with the hegemony of the `spectacle'.} 

\subsection{New Ensemble Dynamics}

In addition to serving as a tool to establish new relationships between musicians and audience, technology may be used to create new forms of exchange between performers within an ensemble. In other words, technology might provide new ways by which musicians may collaborate with each other as they perform within a group. With the aid of interactive systems, new modes of computer-mediated performance can be developed to foster new types of group communication, cooperation and causality. The way in which performers relate with each other through technology could go beyond the performance-practice conventions and common undestandings established in ensembles for acoustic instruments. New ways of thinking about the notion of `playing together'---as well as the types of interaction musicians might have within an ensemble---could emerge as a consequence of interfacing an ensemble through computer systems. Musicians within an ensemble may be able to influence each other's sounding result through computer systems that utilize and analyze audio signals and data coming from each performer and combine them in ways that new types of causality between performers can be established. Furthermore, the data gathered from the musician's actions would not be limited to audio signals, but could also include information coming from Midi instruments or controllers, sensors and other human interface devices (HIDs), as well as visual data. The information collected from performers could be combined, for instance, by altering the pitch of the signal of one instrument through the amplitude of the signal of another, or by changing the timbre of one instrument through another instrument's timbral characteristics, or by deriving Midi information from the activity of one instrument to be used as control structures to apply different types of digital effects on the other instruments, \emph{etc}. Consequently, by gathering, mapping and combining data from each musician's actions---in a way such that what they do, not only influences their own sound, but rather has an effect on certain parameters in the sounding result of the other musicians---new types of synergy between performers could emerge, as well as new understandings of what it means to `play together'. These new forms of cooperation, communication and action undoubtedly would be reflected in the \emph{musical result} and certainly represent a considerable potential for musical innovation in performance, improvisation and composition.

\section{Technology and New Practices in Composition}

Recent developments in computer technologies have already had direct implications in the practices of music composition. Today composers use computers for a wide variety of purposes, from score editing and Midi sequencing (to create `mock-ups' of instrumental compositions) to algorithmic composition and electroacoustic (`fixed' or real-time) compositions. Moreover, computers now provide an amount of processing power for musical computation never imagined before by composers. Modern computation provides the composer the possibility of executing calculations that for a human would be extremely tedious or impossible in a matter of milliseconds. Composers can take advantage of capabilities provided by digital technology to save time otherwise devoted to calculations that computers perform more efficiently than humans (e.g. iteration), and focus on tasks that humans excel at and that are impossible for computers to perform (e.g. analytical/aesthetic decisions  within a cultural framework).\footnote{That is not to say, however, that a possibility exists that at some point in the future computers will be able to make these decisions as competently as humans.} If a critical evaluation of technology becomes part of the composer's criteria---taking into consideration the possibilities and limitations of the tools implemented---the repercussions these new innovations may bring to music composition will be significant. The use of computers to compose music not only is changing the way composers write music, but also rises new questions regarding the notion of what it means to compose music.

One of the consequences of the increasing processing speed developed for computers in the last two decades has been the ability to execute complex algorithms within the immediacy of a musical performance. The speed at which these calculations are processed brings a whole new set of possibilities in the practices of musical composition. Robert Rowe has enthusiastically described the possibilities real-time computation brings to music composition and human-machine interaction in music-making:
\begin{quote}
Composers have used algorithms in the creation of music for centuries. The speed with which such algorithms can now be executed by digital computers, however, eases their use during the performance itself. Once they are part of a performance, they can change their behavior as a function of the musical context going on around them. For me, this versatility represents the essence of interaction and an intriguing expansion of the craft of composition. An equally important motivation for me, however, is the fact that interactive systems require the participation of humans making music to work.\footnote{\hyperlink{rowe}{Rowe (2001)}, p. 4.}
\end{quote}
Not only can the computer make calculations in real-time that formerly would have been made by the composer prior to the performance, but the results of these calculations can change each time the composition is performed. The immediacy of real-time computation therefore brings new possibilities for the composer, who can---instead of imagining a `fixed' composition (at least regarding the set of musical events described by a traditional score)---formulate an algorithmic system which generates different possible musical outcomes.\footnote{This type of real-time composition is sometimes referred to as \emph{generative music}. See \hyperlink{eno}{Eno(1996)}, for a brief introduction to ideas behind \emph{generative music}.} In addition, the possible outcomes may vary according to the sonic environment and musical situation of each specific performance. An important point in Rowe's statement is also that interactive systems reestablish human performers at the center of computer music. The role of the human performer is not limited to triggering and manipulating computer generated/processed sound but can be extended to interacting with computer systems with a traditional acoustic instrument or voice. Not only may human action influence the computer's response in a wide variety of ways---by using different types of audio and visual analysis as well as physical gesture tracking\footnote{Physical gesture tracking has gained attention in the computer music community recently and the development of sophisticated interfaces and sensors has considerably increased in the past few years.}---but the human performer himself/herself may be influenced by the computer's reaction. In addition, another human agent in computer music can be the composer himself/herself taking part in the performance, leaving some compositional decisions to be made during the performance itself. Xenakis' vision of a composer as a sort of pilot who presses buttons, directs the morphology of sound and general form of a composition by making global decisions and leaving certain details to the computer,\footnote{See \hyperlink{xenakis}{Xenakis (1992)}, p. 144.} today is possible within the immediacy of a live performance. The composer's role therefore may broaden to include decision-making during the performance itself. As a result, the action of deciding which elements of the composition are `pre-composed' and which ones are taken during the performance may become central to the compositional process. Real-time computation consequently opens-up the possibility for the composer to explore and formalize the relationship between premeditated and spontaneous decisions and the dialog between improvisation and composition in live electronic performance. Live electronic music that uses real-time computation and interactive systems, I will propose, can be characterized as \emph{generative}, \emph{temporal} and \emph{relational}---notions that are at the same time at the core of improvisation. For this reason, there are many common interests and concerns between the two fields and today they also often overlap. This explains why musicians dealing with improvisation have become attracted to using real-time computation, as well as composers using these technologies have increasingly started to collaborate with improvisers, incorporate improvisatory elements in their work and to improvise themselves. Furthermore, the inclusion of composers using computers within the performance, as well as the type of decisions they make regarding details in sound production,\footnote{For instance, details regarding timing, articulation, timberal changes, \emph{etc.}, formerly made mostly by performers.} also questions the separation between performer and composer that emerged during the twentieth century. In other words, as a consequence of the possiblility of making `compositional' decisions within the immidiecy provided by real-time computation and because of the nature of these decisions, the clear division between composition and performance has been blurred in live electronic performance. 

I believe that a model of a composer who is involved in performance, improvises, directs musicians, builds instruments, records, produces and is actively engaged in all aspects of music-making\footnote{This vision of the composer is closer to that of the baroque and classical periods in contrast to the twenthieth century model of the composer as a specialist who only writes scores.} will reemerge partly as a consequence of the inclusion of computers in live performance. The notion of the composer as a musician who specializes only in writing scores will be replaced in favor of a concept of a creative musician who, through technology, engages in a diverse set of practices. I am convinced that amongst these practices, programming computer applications will become a crucial activity for the composer. Computer code not only is becoming a new way of documenting music, but also has a similar function to the musical score in terms of its influence over the creative process: the act of writing computer code may serve as a process of self-reflection and critical examination regarding the musical output. Additionally, coding may become a collaborative practice through the internet: by using version control systems (VCS),\footnote{Version control systems are used in software development to manage changes that are made to computer code, documents and other computer files. Revisions are made to files at different points in their development and can be altered by many people. The changes to the files are tracked and the system provides control over the changes by locking, backtracking, merging, duplicating, branching and cloning different versions. Revision control software such as \href{http://subversion.tigris.org/}{Subversion (2000)} and \href{http://git-scm.com/}{Git (2005)} encourage social coding and are open-source (free). See \href{http://en.wikipedia.org/wiki/Revision_control} {\texttt{http://en.wikipedia.org/wiki/Revision\_control}}, for more information on VCS.} musicians may be able to share compositional ideas with each other, as well as strategies on how to implement them technically. The act of composition might become a collective practice in itself if composers collaborate as a group on the creation of a single piece of music through the internet. Instant feedback from the performers could influence the result of the composition as well, by giving valuable information regarding the practicality of the performative aspects of the music at an early stage in the creative process.\footnote{This information could include specialized information regarding instrumentation and performance practice, for example, whether the desired sounds sit comfortably in the instrument or whether they are impossible to perform.} Therefore, sharing information would be facilitated at all stages of the creative process through the use of the internet as a collaborative platform. I am confident that new composition practices will soon emerge reflecting the agile forms of communication and collaboration fostered by the development of digital technologies and the internet. Increasingly, the notion of a single composer making all decisions regarding the creation of one piece of music, in my view, will be replaced by other models of collaboration in which a team of musicians with different specialities\footnote{Possibly, this team could also include experts in other areas as well, including for example computer science, engineering, architecture, acoustics, mathematics, anthropology, \emph{etc.}, that could give valuable feedback into the compositional process.} will work towards the same \emph{musical result}.

In this chapter,  we have discussed an approach to using technology in music that is driven by musical concerns and creativity instead of scientific research and technological curiosity. 
take it back to how it relates to the submitted work.
Conclusion . . .

\label{ch:strategies}