\hypertarget{chapter4}{}
\chapter{Musical Strategies}

This chapter...

\section{Technology and Strategy}

New technologies may have a vital role in the creative use of the \emph{strategic} aspects of music-making and consequently in the creation of new \emph{aesthetic} forms. However, technology is not often used with the purpose of redefining \emph{what music is}, its own rules and subject matter. Today, technology is more often used to create---going back to Ranci\`{e}re's terminology---music that lies within the \emph{ethical} and \emph{poetic} regimes. Put briefly, new technologies are often produced as tools that facilitate the creation of music that fits with previous models of \emph{what music is supposed to be} and preconceived notions of the functions it performs. It would be pointless, however, to ignore the efforts that have been previously attempted to use and create technology with the purpose of breaking with already-existing-models of music-making and therefore contributing with the establishment of the \emph{aesthetic regime} in music. The technological developments of the twentieth and twenty-first centuries have also inspired and motivated the musical avant-grade to get involved and work with these new technologies. Nevertheless, the same misunderstandings and misconceptions that are ascribed to the notions of \emph{modernism} and \emph{postmodernism} have been embraced in thinking about and implementing technology in music.\footnote{A survey of how these notions have influenced the thought behind the use of technology in music is beyond the scope of this commentary. However, I think this topic deserves an extended study of its own.}  For this reason, I will attempt to put forward some ideas of how to think about, create and use technology in music, considering the aesthetic preoccupations I have previously elaborated.

\subsection{Technological vs. Musical Innovation}

Before discussing my views on how technology might have an important function in rethinking musical strategies, I would like to examine some problems that might arise regarding the use of recent technology in music. As a musician, one of my concerns regarding the relationship between technology and music is that on many occasions scientific innovation and technological curiosity are given priority over musical creativity and aesthetics. Luciano Berio has eloquently expressed the same position:
\begin{quote}
If in the past---even the distant past---music was often the testing bench and the stimulus for scientific research, and thus music tended to draw scientific knowledge to it, in more recent years you get the impression that it's now science that draws music to it and takes possession of it. Indeed, you often get the impression that a scientific creativity applicable to music has substituted itself for musical creativity, and that musical thought has regressed to the level of the (invariably squalid) opinions that an electronic engineer from Bell Telephone or a Stanford ``software man'' may have about music.\footnote{\hyperlink{berio}{Berio (1985)}, p. 121.} 
\end{quote}
The attitude of giving more importance to technological (as opposed to musical) innovation while creating music has also increased with the complexity and development of the tools themselves. Scientists and technologists often create music with the sole purpose of demonstrating new developments in music technology. Additionally, musicians that are interested in using technology to a higher level of sophistication very often need to immerse themselves in intricate technological subjects. These circumstances can be misleading for the musician if his priorities shift from a position in which technology is researched and developed for its creative potential in music, to a position in which technological innovation becomes the driving force behind musical creativity.  The shift of attention might even happen without the musician's awareness as a consequence of the effort one needs to go through in understanding the complexity of the technological tools and research developed in this field. This can be deceiving and even `dangerous' if music becomes just a showcase of new technological advancements. 

The experience gained by musicians during the second half of the twentieth century who worked closely with technology can also be very valuable to us today as a warning of the possible problems that working with technology might lead to. Looking back at Berio's account of his experience on this issue, one can grasp how the notion that new technological developments lead to important musical progress is erroneous. On his account, Berio describes how the advancements which permitted the creation of new sounds with electronic means did not in-itself produce any meaningful musical results. 
\begin{quote}
Thus many of the more sensitive musicians quickly realized that it was as easy as it was superfluous to produce new sounds that were not the product of musical thought, just as it's easy nowadays to develop and `improve'' the technologies of electronics music when there are devoid of any real and profound \emph{raison d'\^{e}tre}.\footnote{Ibid., p. 122.}
\end{quote}
He goes on to describe how music that was motivated by technological developments instead of musical thought resulted in a spectacle that did not address the complex set of relationships and conventions that take place in music.
\begin{quote}
It was recognized, for example, that the spectacle of a public gathered together to listen to loudspeakers was not a particularly cheerful one, and that, yet again, the experience of public musical listening was made up of many different conventions, and was rooted in many different aspects of social and cultural life: it was not made up merely of a piece, a musical object to listen to, even if it proposed ``new sounds''. By its very nature, a piece of music by itself cannot easily transform listening conventions and socio-musical relations in general.\footnote{Ibid., pp. 122,123.}
\end{quote}
The lesson to learn from Berio's statement is clear: musical and technological innovation are inherently different from each other and if one's interest lies in creating music, one needs to guide technological interests and development with priorities that will be relevant to the desired \emph{musical result}. That is not to say of course, that scientific research or technological development regarding music is not valuable. On the contrary, my position is that technology can have a vital role in musical innovation if it is developed with a critical approach and considering the complex social, cultural and philosophical aspects inherent in music's definition. Moreover, if technology is developed imaginatively with the purpose of creating new musical strategies for the future, it might help reshaping the way in which we make and experience music.

\section{Reshaping Musical Strategies Through Technology}

Even though technology may play a key role in rethinking many aspects that form part of a \emph{musical result}, here I will focus specifically on new strategies concerning the relationships between composer, performer and audience. Therefore, I am not going to go into detail into subjects that are not reated to this specific area of interest as this would be out of the scope of this commentary. Nevertheless, I belive that there is huge potential and work to be done in these areas, which include concerns such as how technology may radically change the way in which musical institutions operate; the visual elements related to the performative aspects of music; how music is recorded, distributed, advertised and consumed. However, what I will concentrate on here is how technology brings a unique opportunity to envision new compositional and performative strategies based on reshaping relationships that have been established traditionally through compositional and performance-practice conventions. I will start by examining the possibilities technology could bring in revising the way in which musical knowledge is transfered by imagining a new type of score that would combine oral and visual traditions within a multimedia experience. Then, I will...

\subsection{The Score in the Digital Age}

By now, much has been written about the limitations and advantages of the traditional score as a form of communication between composer and performer in western music.\footnote{See, for instance \hyperlink{goer}{Goehr (2007)}, \hyperlink{emmersoncross}{Emmerson (2000)}, \hyperlink{small}{Small (1998)}, \hyperlink{wishart}{Wishart (1996)} and \hyperlink{hamilton}{Hamilton (2008)}.} Through research in \mbox{ethnomusicology} and other music practices that incorporate improvisation, an increasing attention has been given to other forms of knowledge transfer in performance-practice that do not utilize a written score. These might include oral traditions that include such practices as transferring music from one generation to another through a master-apprendice relationship or the increasing convention of studying recordings as a method of learning a particular song, style, genre or performance-practice. It has also been argued that the score is a medium that is highly individual and `isolates' the performer not only from the audience but also when playing within a group of musicians.\footnote{See \hyperlink{emmersoncross}{Emmerson (2000)}, p. 121.} On the other hand, the idea of using notation has been defended as well for its capacity of capturing complex musical ideas and thoroughly worked structures, establishing a particular relationship between composer and performer, providing points of reference and facilitating synchronicity.\footnote{See, for example \hyperlink{ferneyhough}{Ferneyhough (1995)}, for an in depth discussion not only about the difficulties implicit in the practice of notation (the impossibility of depicting sound as visual representation), but its potential as a vehicle to express ideological concerns and to achieve auto-instrospection,  as well as the role it might have as a common denominator in different fields of musical interests. According to Ferneyhough, the score contributes to the \emph{act of composing} as an exercise in self-analysis through the process of notation, and to the \emph{act of performance} by establishing the (social and contextual) conditions of its realization.} My position regarding this matter is that the score is still a valuable tool for communicating with musicians trained within western tradition and it is worth expanding the notion of the score to include new strategies that can be developed through technology that might enhance or facilitate communication between composer and performer. In this respect, I completely agree with Simon Emmerson, who argues that technology can serve as a tool in generating new forms of notation that can encapsulate different forms of transferring musical knowledge. 
\begin{quote}
But we have one new invention which may hinder and help our endeavor: the computer. Its power was rapidly applied to wester music in all the forms we have discussed. Composition, analysis, transcription, sound production, processing, storage and distribution are all now in one way or another within its domain. . . . An unaddressed need remains: the development of more flexible notation systems; these may also be stimulated by the development of a new generation of music interfaces. . . . We should dream of a technology which bypasses some of theses constraints: a combination of ear and eye---a new `superscore'. . . .\footnote{\hyperlink{emmersoncross}{Emmerson (2000)}, pp. 121-122.} 
\end{quote}
Emmerson's idea of a `superscore' combines oral and visual forms of communication within a multimedia object combining traditional notation, extended notation, recordings of example material from the live performer, electroacoustic materials, software for performance, patches for live electronic treatment, examples of live electronic treatment, an example recorded performance, written and spoken commentary, video performance material, video example material and graphical material.\footnote{Ibid., pp. 128-129.} 

Taking Emmerson's idea further, one could easily imagine the `superscore' as a package that combines performance materials with documentation (including video tutorials, audio examples (sampled mock-up performances or real performances), recordings, interviews, \emph{etc.}) residing on the internet. Additionally, with the increasing accessibility of laptops, one could easily imagine replacing a score that is printed on paper, with one that is displayed on a computer monitor. This would bring the opportunity of exploring the potential to communicate musical meaning through a computer display, which would add movement to the expressive palette of a conventional score. By using animated graphics, scores, pictures, as well as other types visual cues and timed written directions, the composer could enhance the way in which he communicates musical ideas and knowledge through the computer display. In addition, the performer could receive other types of audio information through headphones complimenting the visual input with an `aural score'. This could comprise from spoken directions and sounding cues (click tracks, reference pitches, etc) to recordings of acoustic or electroacoustic music that the performer would have to react to or improvise with. Moreover, with the development of real-time processing technologies and generative algorithms, the notion of a \emph{fixed} score could also be contested by a score that is \emph{dynamic}, thereby creating a composition that may change its content (pitches, rhythms, etc) each time it is performed. Real-time scoring could be explored further by combining elements of real-time animation and graphics display with new advancements in machine listening technologies, thereby generating a score that responds to the sonic and acoustic context of a specific performance and space. The possibility of creating a network including several computers could also provide instant communication between performers and the option for the composer or conductor to send directions that would be specific to a particular performance. With the increasing popularity of wireless networks and new types of interfaces and gadgets, portable devises like the iPad or iPhone could be used to implement the `superscore', making it easier to carry and even place in a music stand . 

In addition to enhancing communication with musicians trained within the wester tradition, the `superscore' could also foster new collaborative possibilities between performers of different cultures. By sending information that is specifically devised and customized for a particular type of performer, the `superscore' could provide the opportunity for musicians from different backgrounds and traditions to share the stage simultaneously in a computer-mediated performance. A group of performers from mixed backgrounds could therefore play together within a predetermined structure by receiving different types of visual and aural stimuli. The collaborative opportunities this could bring are vast as technology could facilitate and even solve problems that until now have made it difficult (if not impossible) for musicians from different backgrounds to play together.  

\subsubsection{Crossing Cultural Borders?}

Given the opportunities technology brings for a diverse group of musicians to share the stage despite previous incompatible performance conventions, important questions arise concerning the types of relationship established during collaboration. These relationships might become particularly sensitive if one is collaborating with musicians from different cultures. In his article \emph{Crossing Cultural Boundaries through Technology}, Simon Emmerson already expresses some concerns as a composer when dealing with cross-cultural collaborations and `ensembles with ethnic instruments'. He argues that the western composer often appropriates music from different cultures through `strongly filtered sources' and cultural misunderstandings, frequently resulting in `cultural murder'. 
\begin{quote}
There are plenty of examples of composers killing stone dead the spontaneity and vitality which they themselves admire in non-western music through insensitive appropriation of surface technique (usually, once again, through an inadequate notation system and inadequate formalized `rules'). Too simple an understanding of acculturation may hinder the very process we aim to foster.\footnote{Ibid., p. 126-127.} 
\end{quote}

Emmerson suggests the western composer should undergo a process that surpasses the initial first impression of the other culture's music---which is solely based on our previous expectations and experience---to develop a process where `new measures of significance' are created. According to Emmerson, this stage is crucial: if the western composer declares intentions to define the meaning of the musical result (based on misconceptions and misunderstandings of the other culture), he might reinforce ``the purely western basis for the evaluation of such projects thus defeating much of their object''.\footnote{Ibid., p. 126.} He therefore promotes a positive attitude towards `successful acculturation' through education, practical experience, mutual understanding and respect.\footnote{Ibid., pp. 115-134.} 

Even though Emmerson's position appears to be sincere and well-intentioned, a danger exists if it lends itself to an attitude analogous to the notion of \emph{multiculturalism}, which Slavoj \v{Z}i\v{z}ek has rightfully criticized. According to \v{Z}i\v{z}ek, \emph{multiculturalism} is a tendency that has spread in western nations through globalization that treats local (other) cultures with `respect' and displays an interest in studying, understanding and preserving their traditions. Nevertheless, this arrangement is established through a hegemonic relationship---imposed by western nations and from a western perspective---by maintaing a condescending distance between the dominant and repressed cultures.
\begin{quote}
Multiculturalism involves patronizing Eurocentrist distance and/or respect for local cultures without roots in one's own particular culture. In other words, multiculturalism is a disavowed, inverted, self-referential form of racism, a `racism with a distance'---it `respects' the Other's identity, conceiving of the Other as a self-enclosed `authentic' community towards which he, the multiculturalist, maintains a distance rendered possible by his privileged universal position. Multiculturalism is a racism which empties its own position of all positive content (the multiculturalist is not a direct racist, he doesn't oppose to the Other the \emph{particular} values of his won culture), but nonetheless retains this position as the privileged \emph{empty point of universality} from which one is able to appreciate (and depreciate) properly other particular cultures---the multiculturalist respect for the Other's specificity is the very form of asserting one's own superiority.\footnote{\hyperlink{zizekuniv}{\v{Z}i\v{z}ek (2006), \emph{The Universal Exception}}, `Multiculturalism, or, the cultural logic of multinational capitalism', p.170-172.}
\end{quote}
Emmerson's approach towards intercultural projects might become misleading if it is assumed that through a process of education and experience with music/musicians from `other' cultures, these projects will loose their western basis and become productive or successful cultural exchanges. Moreover, this process of study and practical exchange might in itself become the basis of establishing a relationship of power and an attitude that reflects---as \v{Z}i\v{z}ek would say---the way `the colonizer treats colonized people'.\footnote{Ibid., p. 170.} I will therefore suggest that a more `honest' form of exchange is to approach intercultural projects with skepticism and self-awareness; without distancing oneself from the musicians from `other cultures' by treating them with special respect or with a fake notion of open-mindedness. I would propose dealing with these musicians as one would deal with other musicians within our own culture (we are not usually particularly concerned with treating people within our own culture with special `respect' or distance), by collaborating with them (without assuming a patronizing distance) towards ones desired musical result. One should also assume that there will be a struggle involved in the process of intercultural collaboration as there are always different types of violence and relationships of power that emerge during cultural exchanges. 

The way in which we deal with music and musicians from different cultures underlines a bigger problem, that is, how should we as creative musicians should approach the act of appropriation. Nevertheless, before engaging in such discussion,\footnote{See \hyperlink{appropriation}{pp. 44-52}, for  a discussion about appropriation in music and its relationship to ideology.} I would first like to consider how technology---and more specifically real-time computer processing---may offer new applications that challenge the conventional notion of a musical performance and the relationships established traditionally in music-making.

\subsection{Live Electronic Music Performance}

The introduction of the computer to live performance offers the possibility to establish new relationships regarding the way in which we perceive a musical performance. The causality inherent in traditional music produced with mechanical means,\footnote{This includes traditional means of producing vocal, instrumental and mechanical music.} which follows `well-understood Newtonian mechanics of action and reaction, motion, energy, friction and damping,'\footnote{\hyperlink{emmersonliving}{Emmerson (2009)}, p. xiv.} does not need to apply to live electronic music performance. In electronic music, the causal relationships found in our acoustic surroundings are usually not clearly revealed, given that sound may be produced with little evidence of mechanical production (with the exception of the vibrating cone of the loudspeaker). Nevertheless, considering that most of our sonic experience lies within our acoustic environment, we usually seek to form causal relationships (even within the electronic medium). Therefore, many efforts have been made to reestablish causal relationships that are characteristic of traditional music through mechanical means in electronic music performance. This as been attempted through the continuing development of interfaces that attempt to reestablish an instrumental approach to electronic music (for example synthesizers, Midi samplers, electric guitars, \emph{etc}). Nevertheless, electronic music performance also offers new opportunities to form other types of relationships as perceived by the listener. This specific feature of the electronic medium may challenge conventional notions of what a musical performance is as it may form new types of relationships that go beyond the traditional instrumental approach. Therefore, when dealing with electronic music performance, the composer may decide what types of relationships he/she wants to establish (for instance, how different sonic and visual aspects of a performance may relate with each other or how the human body and movement may be associated to sound). 

Simon Emmerson, in his book \emph{Living electronic music}, describes different approaches the musician may take towards electronic music performance based on how the audience may perceive the actions of the human performer in relationship to the sounding result. First, he describes what he calls the `Local/Field Distinction', in an attempt to conceptualize differently relationships that seem to have a perceived causality between a human performer's action and the sounding result, and those that don't.
\begin{quote}
\emph{Local} controls and functions seek to extend (but not to break) the perceived relation of human performer action to sounding result. \emph{Field} functions creat a context, a landscape or an environment within which \emph{local} activity may be found. It is important to emphasize that the \emph{field} as defined above \emph{can contain other agencies}, in other words, it is not merely a `reverberant field' in the crude sense but an area n which the entire panoply of both pre-composed and real-time electro-acoustic music may be found. . . . This definition aims to separate out the truly live element as clearly the `local agency' in order to re-form more coherently the relationship with this open stage area, which may surround the audience and extend outside.\footnote {\hyperlink{emmersonliving}{Emmerson (2009)}, p. 92.}
\end{quote}
This distinction is useful to the musician as it encourages reflection on how the presentation of electronic music performance---particularly aural/visual relationships concerning causality and human presence---might influence the listeners perception of the overall \emph{musical result}. Additionally, given the particularities of the medium, the electronic musician is encouraged to rethink important aspects about performance (for instance, how it might look like, what function might the musicians perform onstage, what types of human/machine interaction might be established, \emph{etc}.) This distinction can also be helpful if it is considered creatively as a parameter within a composition: the distinction between \emph{local} and \emph{field} could be emphasized or blurred according to the desired musical moment, the extremes could be alternated or even morphed between each other, an extreme might be embraced as the other is sublimated, \emph{etc}. In addition, Emmerson also makes a difference between \emph{real} and \emph{imaginary} relationships that may be \emph{local} or \emph{field}. According to Emmerson, \emph{real} relationships are also `real-time' and have direct relation with the \emph{real} cause as perceived by the audience (a sonic result that can be followed by the listener). This may include processing the `live' sound, abstracting a gesture through an interface or sensor, or through other types of analysis (audio or video). \emph{Imaginary} relationships, on the other hand, are `prepared in advance (soundfiles, control sequences, etc.) in such a way as to \emph{imply} a causal link of sound to performer action in the \emph{imagination} of the listener'.\footnote{Ibid, p. 93.} Emmerson also emphasizes that the difference between \emph{real} and \emph{imaginary} relationships might be different for the listener as they are for the composer (or as they are in reality). Even though I find Emmerson's terminology slightly confusing,\footnote{His distinction I don't find very useful as it seems to make a link between \emph{real} relationships with `real-time' processing and \emph{imaginary} relationships with `fixed' or prepared material. I think this is misleading, as `real-time' processes usually contain large amount of prepared or `fixed' elements (for instance, computer programs, patches, data bases, etc., that have been prepared in advance) that also create what Emmerson calls \emph{imaginary} relationships and an \emph{illusion} of causality. That is to say, his terminology might lead to misunderstandings as it equates types of relationships the listener makes to whether an electroacoustic part is influenced by a performer or is autonomous.} I think it points towards an issue that I think is important to anyone dealing with electronic music performance, that is, what should concern us is what \emph{appears} to be real or not to the listener, and not whether technological processes are taking place `in reality' (real-time) or have been prepared before hand. Consequently, the question of whether to use `real-time' processing or not should stem from aesthetic concerns in relationship to the listener's perception of the performance and not from `technical authenticity', or to cling to a set of technological concepts.

The opportunities that electronic music gives in forming new relationships between the performer's action and the sounding result, gives the composer the option of thinking creatively about how a performance might be presented. The cognitive dissonance that might arise between aural and visual elements of the performance could be used as a performative element, creating meaning out of the apparent sensorial disjunction. This approach could even be exaggerated, for instance, by suggesting causal relationships that may be only observed and have no corresponding sounding result, or by creating a visually static performance while having a sounding result that would suggest frenetic activity. This slightly more idiosyncratic approaches towards the presentation of electronic music may also encourage people to reflect on the subject of how the performer relates to a technological object, which at the same time may prompt a deeper question, mainly, how we as human beings relate through technology.

\subsubsection{Interactivity or Interpassivity?}
 
The conventional viewpoint regarding the relationship between human beings and technological objects is that we relate with them through interaction. However, Slavoj \v{Z}i\v{z}ek has proposed the alternative notion of  \emph{interpassivity} (to describe the opposite of \emph{interactivity}) regarding the duality between active and passive relationships that might be formed between a person and a technological object.
\begin{quote}
Interpassivity, like interactivity, thus subverts the standard opposition between activity and passivity: if in interactivity (or the `cunning of Reason'), I am passive while being active through another, in interpassivity, I am active while being passive through another. More precisely, the term `interactivity' is currently used in two senses: (1) \emph{interacting with} the medium---that is, not being just a passive consumer; (2) \emph{acting through} another agent, so that my job is done, while I sit back and remain passive, just observing the game. While the opposite of the first mode of interactivity is also a kind of interpassivity, the mutual passivity of two subjects, like two lovers passively observing each other and merely enjoying each others presence, the proper notion of interpassivity aims at the reversal of the second meaning of interactivity: the distinguishing feature of interpassivity is that, in it, the subject is incessantly---frenetically even---active, while displacing on to another the fundamental passivity of his or her being.\footnote{\hyperlink{zizekreader}{\v{Z}i\v{z}ek (2006), \emph{The \v{Z}i\v{z}ek Reader}}, `The Fantasy in Cyberspace', p. 105.}
\end{quote}
Therefore, \v{Z}i\v{z}ek implies that we not only form \emph{interactive}, but also \emph{interpassive} relationships with technological objects. That is to say, while the ordinary stance regarding the way we relate to technology is by using its objects for our own purpose, what \v{Z}i\v{z}ek suggests is that today technological objects might actually demand something from us instead. Therefore, the `user'\footnote{Here, I am referring to the term `user' as applied commonly in the development of computer technologies.} not only \emph{uses} technology, but is also \emph{used} through technology. Moreover, \v{Z}i\v{z}ek elaborates his argument further by using Lacanian psychoanalysis. He claims that \emph{interpassivity} implies that while we are obsessively active by the object's demands, we also rely on the object to be passive for us. This transfer implies a game that goes on in our minds, in which we imagine the object as the Other, whose desire we subvert through our activity in order to put off our recognition that enjoyment cannot be achieved in full. It is for this reason---according to \v{Z}i\v{z}ek---that the notion of \emph{interpassivity} is vital in understanding the artistic possibilities of digital technology.\footnote{See Ibid., pp. 104-110.} 

\v{Z}i\v{z}ek's definitions may also be applied to the way in which a performer might relate to technology in a live electronic musical performance. The performer therefore might form \emph{interactive} relationships with a technological object if he/she seems to remain passive while technology appears to be active. For instance, when a performer plays a note or presses a button that sets an active chain of sound, or the `typical' laptop performer's role of sitting behind the computer appearing to be passive while triggering musical events that suggest activity. The performer might also form \emph{interpassive} relationships with technology, when a technological object appears to remain passive while making the performer appear franticly active. This concept hasn't been explored thoroughly by composers using technology but can be found for example in cases where the performer receives directions from a technological object (through a computer display or through headphones) directing the performer towards frenetic activity. I think a potential exists to further develop musical applications that establish \emph{interpassive} relationships between performer and technological objects through computer-mediated performances or more experimental methods, such as involuntary bodily movement\footnote{See Stelarc} or by radically altering the sound of a performer playing an electronic instrument (which produces no considerable audible sound that is not generated electronically) such that the initial physical effort of the performer is reduced or striped to silence by the computer processing (giving the impression of human activity being subverted through technology). 

Additionally, the musician might also establish \emph{interactive} as well as \emph{interpassive} relationships with the audience through technology. \emph{Interpassive} relationships might be established by following a model whereby the performer displays intense activity through technology---for example, by using electronic instruments, interfaces, sensors or other forms of tracking human movement---following the concert hall format, where the audience remains seated as passive spectators of the action onstage. On the other hand, \emph{interactive} relationships may be formed if the audience becomes active through technology, while the performer seems to remain passive. This is the case for example of the laptop performer or DJ---which play music that encourages the audience to become active by dancing frenetically---while staying behind their computer offering no clue that they are in actuality producing the sound. However, it could be argued that this apparent activity displayed by the audience through bodily motion at the same time might shift the audience's attention away from certain musical content, resulting in a type of passivity. In other words, by becoming active through movement, the audience might stop focusing on certain aspects of the music as their attention shifts towards physical activity, resulting in a reversal of \v{Z}i\v{z}ek's first definition of \emph{interactivity}. A reversal could also be applied regarding the engagement of the audience within the concert hall model: while the audience seemingly takes the role of passive spectator and the musicians displays activity through their virtuosity; in actuality, the audience might be the one engaging with the music, rendering emotional and intellectual activity through the performance, while the performers `stops listening' as they concentrates on bodily movement. 

\subsubsection{New Relationships with the Audience}

The particularities of live electronic performance also encourage new ways of thinking about the relationships that may be established between musicians and audience during a performance. Due to the increasing development of technologies that have an impact on the performer/audience relationship, today we have at our disposal a considerable amount of tools that enable us to reconsider and rethink this exchange.\footnote{These tools include, for example, developments in sound amplification and diffusion (for instance, different types of microphones and microphone techniques, as well as varieties of loudspeakers, loudspeaker setups, sound projection systems and headphones), audio processing (real-time digital signal processing tools and other types of sound manipulation), human interaction devises (Midi instruments, interfaces, sensors, \emph{etc}) and real-time computation of musical control-structures (autonomous or interactive computer programs that trigger and control musical events) as well as technologies that have a visual impact on the performance (for instance, live video streaming and processing, automated lighting, smoke machines, \emph{etc}.) and technological tools that might facilitate communication between people involved in a performance (for example, computer networks, portable and wireless devices, gadgets such as iPods, iPhones and iPads).} The traditional forms by which the audience experiences music, may therefore be expanded by establishing new conditions for exchange mediated by technological tools. Thus, with the creative use of new technologies we are able to change the traditional way in which the audience participates in a musical performance. What's more, through technology composers can devise a piece of music partially based on these conditions of exchange, which could evolve and change during the performance. 

Seeking to form new types of relationships between musicians and audience through technology could also lead to developing new interactive possibilities in a musical performance. The audience could take an active role making decisions as far as how they want to experience the performance. These decisions could go as far as what type of content a composition may have and how it might unravel. Interactive elements usually associated with installation work, could be incorporated within a musical performance: the audience could explore the performing space triggering and modifying musical events by interacting in different ways with each other, the space and performers. In other words, the performance could become immersive and the audience could directly influence its outcome. Another musical strategy that could be developed through technology is something close to what Nicolas Bourriaud has called \emph{Relational Art}, which refers to ``a set of artistic practices which take as their theoretical and practical point of departure the whole of human relations and their social context, rather than an independent and private space.''\footnote{\hyperlink{relational}{Bourriaud (2002)}, p. 113.} In other words, a musical performance could be contemplated for its potential as a collective space in which members of the audience could engage with each other in different ways. Technology could mediate this platform of exchange by facilitating tools by which individuals within the audience could communicate with each other and establish new kinds of transactions.

However, these strategies by themselves do not guarantee a type of activity from the audience that suggests reflection and encourages critical thinking and creativity. As it was suggested earlier, the illusion of activity and passivity might be deceiving. The appearance of this opposition might in reality be its reversal. What is conventionally associated with passivity, actually could suggest a different form activity and \emph{vise versa}. These are some of the questions Ranci\`{e}re addresses in his book \emph{The Emancipated Spectator}, where he challenges preconceived notions that associate listening and observation to passivity, and identifies the audience as inactive. Ranci\`{e}re therefore proposes a vision for a spectator that, while seating and listening, is active---fabricating his/her own interpretation and understanding of the performance, associating it with his/her own ideas about the world and the future. This kind of spectator is emancipated in as much as he/she is not manipulated by the performance, but maintains a critical distance and independence from what he experiences as an observer.

\begin{quote}
Emancipation begins when we challenge the opposition between viewing and acting; when we understand that the self-evident facts that structure the relations between saying, seeing doing themselves belong to the structure of domination and subjection. It begins when we understand that viewing is also an action that confirms or transforms this distribution of positions. The spectator also acts, like the pupil or scholar. She observes, selects, compares, interprets. She links what she sees to a host of other things that she has seen on other stages, in other kinds of place. She composes her own poem with the elements of the poem before her. She participates in the performance by refashioning it in her own way---by drawing back, for example, from the vital energy that it is supposed to transmit in order to make it a pure image and associate this image with a story which she has read or dreamt, experienced or invented. They are thus both distant spectators and active interpreters of the spectacle offered to them.\footnote{\hyperlink{ranspec}{Ranci\`{e}re (2009), \emph{The Emancipated Spectator}}, p. 13.}
\end{quote}

Ranci\`{e}re's positive image of an emancipated spectator resists the decadent viewpoint of the spectator held by Guy Debord, which is described as . . . \footnote{See Debord (1994)}

\subsection{Technology and New Practices in Composition}

Code (as score?--- may serve as Brian F's suggestion of one of the `advantages' of the score: reflection, self-critisism...), no score, internet documentation (git)---collaborative opportunities, etc... interactive systems and what they might mean to the composer... generative music...

Iteration?


Interactive Systems:

Real-time computer technology in the last two decades has radically increased in processing speed, consequently allowing the execution of complex algorithms within the immediacy of a musical performance. The speed at which these calculations are processed brings a whole new set of possibilities for musical applications concerned with computer-human interaction. Robert Rowe has enthusiastically described the possibilities interactive systems bring to music composition and performance. 
\begin{quote}
Composers have used algorithms in the creation of music for centuries. The speed with which such algorithms can now be executed by digital computers, however, eases their use during the performance itself. Once they are part of a performance, they can change their behavior as a function of the musical context going on around them. For me, this versatility represents the essence of interaction and an intriguing expansion of the craft of composition. An equally important motivation for me, however, is the fact that interactive systems require the participation of humans making music to work.\footnote{\hyperlink{rowe}{Rowe (2001)}, p. 4.}
\end{quote}
The possibility of using, processing and analyzing audio signals from a live performance in real-time and using this information as building blocks for a new \emph{musical result} has become today a common practice amongst musicians dealing with technology. Additionally, human-computer interaction has become more sophisticated in recent years with the development of new interfaces and technologies with the specific purpose of tracking human gesture. The repercussions these innovations may bring to the way in which musicians interact and communicate with each other and to the composer-performer relationship, I believe are significant.  Maybe mention some of those implications?


\subsubsection{New Ensemble Dynamics}

New modes of performance through computer-mediated group `interaction' in which traditional relationships and performance-practice conventions within an ensemble may change. Thus, through the way in which we use technology within a musical performance, we can also make powerful associations to the way in which we as people relate with each other through technology. How do we relate with each other through technology?

%\hypertarget{appropriation}{}
%\section {Musical Appropriation}

%\begin{quote}
%The contemporary era constantly proclaims itself as post-ideological, but this denial of ideology only provides the ultimate proof that we are more than ever embedded in ideology. Ideology is always a field of struggle---among other things, the struggle for appropriating past traditions.\footnote{\hyperlink{zizektragedy}{\v{Z}i\v{z}ek (2009)}, `It's Ideology, Stupid!', p. 37.}
%\end{quote}
%Start on appropriation and past traditions...

%\begin{quote}
%Consumption is simultaneously also production, just as in nature the production of a plant involves the consumption of elemental forces and chemical material\footnote{Karl Marx}
%\end{quote}

%\begin{quote}
%Starting with the language imposed upon us (the system of production), we construct our own sentences (acts of everyday life), thereby reappropriating for ourselves, through these clandestine microbricolages, the last word in the productive chain.\footnote{\hyperlink{postproduction}{Bourriaud (2005)}, p.}
%\end{quote}

%\subsection{Appropriation and Postproduction in the Digital Age} 

%\begin{quote}
%By listening to music or reading a book, we produce new material, we become producers. And each day we benefit from more ways in which to organize this production: remote controls, VCRs, computers, MP3s, tools that allow us to select, reconstruct, and edit. Postproduction artists are agents of this evolution, the specialized workers of cultural reappropriation.\footnote{Ibid. p. ?}
%\end{quote}

%\begin{quote}
%Throughout the eighties, the democratization of computers and the appearance of sampling allowed for the emergence of a new cultural configuration, whose figures are the programmer and DJ. The remixer has become more important than the instrumentalist, the rave more exciting than the concert hall. The supremacy of cultures of appropriation and the reprocessing of forms calls for an ethics: to paraphrase Philippe Thomas, artworks belong to everyone. Contemporary art tends to abolish the ownership of forms, or in any case to shake up the old jurisprudence. Are we heading toward a culture that would do away with copyright in favor of a policy allowing free access to works, a sort of blueprint for a communism of forms?\footnote{Ibid. p. ?}
%\end{quote}

%\subsection{critisisms} 
%\subsection{The liberal-comunists: Open Source, etc.} 

%There is no music by John Oswald on the net free to download. Hypocrisy from the appropriator? Or does he fall into the logic of late-capitalism - �no communism of forms�? �I plunder but don�t plunder me. Or, at least not for free��? 

%I propose an attitude towards music appropriation similar to that of hacker communities and the open source initiative. Not with the purpose of suggesting a communist utopia, but of being consequent with my creative process. By giving away my music, recorded sounds and experiments, code, etc, through the net, I will hopefully instigate others to do so as well. If this attitude is followed, it could promote the organization of music cyber communities that would plunder, engage with and promote each other, hopefully producing more subversive types of music.

%We are far from the Bourriaud�s utopia. The only people how have access to (artistic) shareware are commoditized people, mostly in western countries. Isn�t the DJ approach towards plunderphonics one that appropriates to make more profit and diminish costs only to thereafter feed back their product into the music industry system?

%\section{Musical Appropriation through Technology}

%I will continue by examining different strategies and practices used in my work that use technology as means to appropriate, derive from and transform existing music by other musicians. It is only logical, considering that music is not an object but a complex set of actions, productions, perceptions and thoughts,\footnote{See pp 13-45 for a discussion regarding my preference of the notion of a \emph{musical result} versus the more widely use concept of \emph{musical work}.} that the act of appropriation of existing music can manifest itself in many different ways and take lots of unexpected guises. Therefore, I will propose that the appropriation of existing music \emph{does not} refer exclusively to `borrowing' or `stealing' from \emph{musical works} by other composers but to . . . . Moreover, when dealing with appropriation, I will claim that there are certain fundamental questions that both music creators and listeners should ask themselves. According to David Mezter, Stockhausen (while referring to \emph{Hymnen}) emphasized the importance of asking the questions of ``what'' and ``how''  regarding the practice of `borrowing' or `quoting' from other music.
%\begin{quote}
%According to him \emph{[Stockhausen]}, the practice involves a rich exchange between the ``what'' and the ``how'', that is, the gesture has us hear ``what'' music has been borrowed and ``how'' it has been changed. The more familiar and obvious the ``what'', the more we are drawn into the ``how'', and the more captivating the ``how'', the more we can appreciate anew the ``what''. It is the ways in which quotation handles the ``what'' and the ``how'' that make it so effective a  cultural agent.\footnote{\hyperlink{metzer}{Metzer (2003)}, p. 6.} 
%\end{quote}
%I agree with Stockhausen's claim because . . . 
%Nevertheless, I would also add : 
%the difference between ``what'' and ``who''
%also ``from where''. 
%but most importantly ``why'',  
%Why = motivations.
%The motivations regarding musical appropriation can be very varied and also reflect ideological positions that in many cases reflect more the beliefs and feelings of the appropriator that the appropriated. Therefore, I will attempt to explain my viewpoint regarding the motivations and ways in which I use other music within my own work. In doing so, I will also examine other composers work that deals with musical appropriation in ways that I consider valid, interesting and intriguing.

%Technology

%Will do so by examining other composers work dealing with this issues... that I find valid, interesting, intriguing, stimulating(?)...  

%\subsection{Musica Derivata and Plunderphonics}

%``A good composer does not imitate; he steals''       I. Stravinsky

%Musica Derivata:

%``music that is compositionally based on other music'' (K. Barlow) 

%
%\subsection{plunderphomes, ideology and the use of references}

%\begin{quote}
%While some start up a prolonged lamentation for the lost image, others reopen their albums to rediscover the pure enchantment of images- that is, the alterity of the \emph{was}, between the pleasure of pure presence and the bit of the absolute Other.
%\end{quote}
%\begin{quote}
%Evidence of exhibitions devoter to `images', but also the dialectic that affects each type of image and mixes its legitimations and powers with those of the other tow.
%\end{quote}
%Plunderphones reflect ideology . . . \v{Z}i\v{z}ek/Adorno but. . . . The artist can present their own view of these references by rearranging them modifying them. The plunderphonics artist doesn't necessarily adheres to the ideology of the appropriated material, but reflects it by the use of the plunderphones - how are they presented, modified, etc?  

%\subsection{On Musical Appropriation}

%What? 

%Code, compositional tecniques, what piece of music? 
%Do we plunder from the ``flea market or (the) airport shopping mall''? (N. Bourriaud). From the top 20 list - J. Oswald approach-, or from the hidden CDs at the back of the music store?

%Who?

%Music Industry? Pop/commercial? Historical (dead composers)? Music from different cultures? 

%Appropriation of the Other. What relationship do we want to establish with the Other? Impersonal like the 1st/3rd World relationships?

%Liberal multiculturalists approach? ``Other deprived of its Otherness (the idealized Other who dances fascinating dances and has an ecologically sound holistic approach to reality, while features like wife beating remain out of sight�)?'' (Slavoj \v{Z}i\v{z}ek, 2003)

%Why?

%For the meaning of the cultural object you are appropriating? For it�s symbolism? To suggest a metaphor?

%For it�s use? ``Don�t look for the meaning, look for the use'' - L. Wittgenstein - for example for the sonic qualities of the appropriation (intonation, groove, etc.)

%How? �

%\section{Typologies of Musical Appropriation}

%\subsection{Copyrights Violation}
% 
%\subsection{Scores}

%
%The first strategy considered is Clarence Barlow's concept of \emph{Musica Derivata}, which refers to the idea of transforming existing music with Computer Aided Composition (CAC) tools to create ``music that is compositionally based on other music''\footnote{\hyperlink{barlow}{Barlow (2000)}.} This approach seems to take as a starting point mostly notated material (but in some occasions spectral information from recordings) from music by other composers. 

%Midi

%\subsubsection{me}

%\subsection{Recordings}

%	
%\subsubsection{Plunderphonics}

%Plunderphonics:

%John Owald, 1985. ``Plunderphonics, or Audio Piracy as Compositional Prerogative''

%Use of audio samples as a technique for composition. 

%Different from Musica Derivata in that it appropriates the recording of the original musical source. Information from recording (tibre, rhythm, performace practice, etc) is plundered from the original source to create a new composition.

%``As a listener my own preference is the option to experiment. My listening system has a mixer instead of a receiver, an infinitely variable speed turntable, filters, reverse capability, and a pair of ears. An active listener might speed up a piece of music in order to perceive more clearly it�s macrostructure, or slow it down to hear articulation and detail more precisely''.\footnote{\hyperlink{oswald}{Oswald (1985)}.}

%\subsubsection{Sound Transformations} 					

%
%``With the power of the computer, we can transform sounds in such radical ways that we can no longer assert that the goal sound is related to the source sound merely because we have derived one from the other''. (T. Wishart)

%In my work, sound transformations are used for the transformation of existing music. 

%Why transformation of musical sources? Because they may carry complex cultural symbolism. 

%The amount of processing can affect our ability to recognize the source sound or musical sample. Therefore, there is a wide palette of derivative music available to us: from the radically processed � less recognizable source � more `abstract' extreme; to the less processed � more recognizable source � more `referential'  and quotation type music.

%Performance practice and other sonic characteristics of many original musical sources is lost in the transcription to a fully notated score for ensembles of western classically trained musicians. Many aspects of sound production (intonation, groove, spectral characteristics of instruments/voices, etc) is lost via this process.

%Process of derivation and sound transformation is not directly apparent to the audience. The act of appropriation is not transparent.

%
%\subsection{Spectral Information} 

%\subsubsection{To generate sc}

%\subsection{Computer Code}

%Max patches, Computer Code.

%\subsection{Real Performances} 

%
%\subsection{Real-Time Plunderphonics}

%Appropriation of audio signals from live music performances as material for a new composition

%Creates a cognitive dissonance between audio and visuals.

%The amount of processing of the audio signals is visible. The more processed the performances are, the more contrasting they will look in relationship with what is heard through speakers.

%In contrast to acousmatic tradition, Real-Time Plunderphonics makes the process of appropriation transparent to the audience through the cognitive association between audio and visuals.

%Changes relationship with the appropriated Other: The performer becomes an accomplice in the process of appropriation (or themselves). 

%Deals with the problematic of the lack of visual clues and theatrical elements in electronic music performance by introducing a dynamic group of live performers and an interesting and unusual visual scenario.  

%\subsubsection{Some ideas of how to plunder}

%Get to know what and who you are plundering and figure why your are doing so before you decide how to plunder.(Know your performers, their music and why you want to work with them)

%Appropriate and plunder yourself. 

%Plundering not as central purpose of the creative process, but rather a tool for creating new idiosyncratic audio/visual result. 

%Use ``from raw to cooked'' (L\'{e}vi-Strauss) techniques to create a narrative that navigates, in literary terms, between the �real� (actual performance) and the `surreal' (extreme processed audio).

%Combinations of Real-Time Plunderphonics, (Real-Time) Musica Derivata and Sound Transformations

%Use plunderphones as data: reprogram, not just remix.

%Micro and macro plundering.

%Use also Non Real-Time tools (Scores, Samples, etc.) if suitable. 

%Using plunderphones as data

%An example: Use FFT data of your plunderphone to trigger samples of recorded instruments.

%\subsubsection{Micro and Macro Plundering}

%Microplunderphonics

%Plundering just microelements of sound. Not the whole spectrum of the original sound file. 

%Generate noise with your plunderphones and use it instead of white noise for sound synthesis

%
%Macroplundering

%Appropriate a composition�s form. Use the structure as blueprint for a new composition. 

%Use variables of the appropriated piece (pitch, dynamics, etc.) as control structures for new output.


\label{ch:strategies}