\hypertarget{chapter4}{}
\chapter{Musical Strategies}

This chapter...

\section{The Role of Technology in New Musical Strategies}

New technologies may have a vital role in the creative use of the \emph{strategic} aspects of music-making and consequently in the creation of new \emph{aesthetic} forms. However, technology is not often used with the purpose of redefining \emph{what music is}, its own rules and subject matter. Today, technology is more often used to create music---going back to Ranci\`{e}re's terminology---that lies within the \emph{ethical} and \emph{poetic} regimes. Put briefly, new technologies are often produced as tools that facilitate the creation of music that fits with previous models of \emph{what music is supposed to be} and preconceived notions of the functions it performs. It would be pointless, however, to ignore the efforts that have been previously attempted to use and create technology with the purpose of breaking with already-existing-models of music-making and therefore contributing with the establishment of the \emph{aesthetic regime} in music. The technological developments of the twentieth and twenty-first centuries have also inspired and motivated the musical avant-grade to get involved and work with these new technologies. Nevertheless, the same misunderstandings and misconceptions that are ascribed to the notions of \emph{modernism} and \emph{postmodernism} have been embraced in thinking about and implementing technology in music.\footnote{A survey of how these notions have influenced the thought behind the use of technology in music is beyond the scope of this commentary. However, I think this topic deserves an extended study of its own.}  For this reason, I will attempt to put forward some ideas of how to think about, create and use technology in music, considering the aesthetic preoccupations I have previously elaborated.

\subsection{Technological vs. Musical Innovation}

Before discussing my views on how technology might have an important function in rethinking musical strategies, I would like to examine some problems that might arise regarding the use of recent technology in music. As a musician, one of my concerns regarding the relationship between technology and music is that on many occasions scientific innovation and technological curiosity are given priority over musical creativity and aesthetics. Luciano Berio has eloquently expressed the same position:
\begin{quote}
If in the past---even the distant past---music was often the testing bench and the stimulus for scientific research, and thus music tended to draw scientific knowledge to it, in more recent years you get the impression that it's now science that draws music to it and takes possession of it. Indeed, you often get the impression that a scientific creativity applicable to music has substituted itself for musical creativity, and that musical thought has regressed to the level of the (invariably squalid) opinions that an electronic engineer from Bell Telephone or a Stanford ``software man'' may have about music.\footnote{Luciano Berio, \emph{Two Interviews with Rossana Dalmonte and B\'{a}lint Andr\'{a}s Varga},  Ed. and Trans. David Osmond-Smith, London: Marion Bowars, 1985, p. 121.} 
\end{quote}
The attitude of giving more importance to technological (as opposed to musical) innovation while creating music has also increased with the complexity and development of the tools themselves. Scientists and technologists often create music with the sole purpose of demonstrating new developments in music technology. Additionally, musicians that are interested in using technology to a higher level of sophistication very often need to immerse themselves in intricate technological subjects. These circumstances can be misleading for the musician if his priorities shift from a position in which technology is researched and developed for its creative potential in music, to a position in which technological innovation becomes the driving force behind musical creativity.  The shift of attention might even happen without the musician's awareness as a consequence of the effort one needs to go through in understanding the complexity of the technological tools and research developed in this field. This can be deceiving and even `dangerous' if music becomes just a showcase of new technological advancements. 

The experience gained by musicians during the second half of the twentieth century who worked closely with technology can also be very valuable to us today as a warning of the possible problems that working with technology might lead to. Looking back at Berio's account of his experience on this issue, one can grasp how the notion that new technological developments lead to important musical progress is erroneous. On his account, Berio describes how the advancements which permitted the creation of new sounds with electronic means did not in-itself produce any meaningful musical results. 
\begin{quote}
Thus many of the more sensitive musicians quickly realized that it was as easy as it was superfluous to produce new sounds that were not the product of musical thought, just as it's easy nowadays to develop and `improve'' the technologies of electronics music when there are devoid of any real and profound \emph{raison d'\^{e}tre}.\footnote{Ibid., p. 122.}
\end{quote}
He goes on to describe how music that was motivated by technological developments instead of musical thought resulted in a spectacle that did not address the complex set of relationships and conventions that take place in music.
\begin{quote}
It was recognized, for example, that the spectacle of a public gathered together to listen to loudspeakers was not a particularly cheerful one, and that, yet again, the experience of public musical listening was made up of many different conventions, and was rooted in many different aspects of social and cultural life: it was not made up merely of a piece, a musical object to listen to, even if it proposed ``new sounds''. By its very nature, a piece of music by itself cannot easily transform listening conventions and socio-musical relations in general.\footnote{Ibid., pp. 122,123.}
\end{quote}
The lesson to learn from Berio's statement is clear: musical and technological innovation are inherently different from each other and if one's interest lies in creating music, one needs to guide technological interests and development with priorities that will be relevant to the desired \emph{musical result}. That is not to say of course, that scientific research or technological development regarding music is not valuable. On the contrary, my position is that technology can have a vital role in musical innovation if it is developed with a critical approach and considering the complex social, cultural and philosophical aspects inherent in music's definition. Moreover, if technology is developed imaginatively with the purpose of creating new musical strategies for the future, it might help reshaping the way in which we make and experience music.

\subsection{Reshaping Musical Strategies Through Technology}

Even though technology may play a key role in rethinking many aspects that form part of a \emph{musical result}, here I will focus specifically on new strategies concerning the relationships between composer, performer and audience. Therefore, I am not going to go into detail in subjects that are not reated to this specific area of interest as this would be out of the scope of this commentary. Nevertheless, I belive that there is huge potential and work to be done in these areas, which include concerns such as how technology may radically change the way in which musical institutions operate; the visual elements related to the performative aspects of music; how music is recorded, distributed, advertised and consumed. However, what I will concentrate on here is how technology brings a unique opportunity to envision new compositional and performative strategies based on reshaping relationships that have been established traditionally through compositional and performance-practice conventions. I will start by examining the possibilities technology could bring in revising the way in which musical knowledge is transfered by imagining a new type of score that would combine oral and visual traditions within a multimedia experience. Then, I will...

\subsubsection{The Score in the Digital Age}

By now, much has been written about the limitations and advantages of the traditional score as a form of communication between composer and performer in western music.\footnote{Authors like Christopher Small, Trevor Wishart, Simmon Emmerson and many others have written extensively about the limitations of the western notation, questioning the traditional score as the only mode of communication between composer and performer and the notion that within the score lies the musical work's ontology.} Through research in \mbox{ethno-musicollogy} and other music practices that incorporate improvisation, an increasing attention has been given to other forms of knowledge transfer in performance-practice that do not utilize a written score. These might include oral traditions that include such practices as transferring music from one generation to another through a master-apprendice relationship or the increasing convention of studying recordings as a method of learning a particular song, style, genre or performance-practice. It has also been argued that the score is a medium that is highly individual and `isolates' the performer not only from the audience but also when playing within a group of musicians.\footnote{See Simon Emmerson, `Crossing cultural boundaries through technology?'  in \emph{Music, Electronic Media and Culture}, Aldershot: Ashgate, 2000, p. 121.} On the other hand, the idea of using notation has been defended as well for its capacity of capturing complex musical ideas and thoroughly worked structures, establishing a particular relationship between composer and performer, providing points of reference and facilitating synchronicity.\footnote{See for example Brian Ferneyhough, `Aspects of Notational and Compositional Practice,' in \emph{Collected Writings}, ed. James Boros and Richard Toop, Amsterdam: Harwood Academic Publishers, 1995, for an in depth discussion not only about the difficulties implicit in the practice of notation (the impossibility of depicting sound as visual representation), but its potential as a vehicle to express ideological concerns and to achieve auto-instrospection,  as well as the role it might have as a common denominator in different fields of musical interests. According to Ferneyhough, the score contributes to the \emph{act of composing} as an exercise in self-analysis through the process of notation, and to the \emph{act of performance} by establishing the (social and contextual) conditions of its realization.} My position regarding this matter is that the score is still a valuable tool for communicating with musicians trained within western tradition and it is worth expanding the notion of the score to include new strategies that can be developed through technology that might enhance or facilitate communication between composer and performer. In this respect, I completely agree with Simon Emmerson, who argues that technology can serve as a tool in generating new forms of notation that can encapsulate different forms of transferring musical knowledge. 
\begin{quote}
But we have one new invention which may hinder and help our endeavor: the computer. Its power was rapidly applied to wester music in all the forms we have discussed. Composition, analysis, transcription, sound production, processing, storage and distribution are all now in one way or another within its domain. . . . An unaddressed need remains: the development of more flexible notation systems; these may also be stimulated by the development of a new generation of music interfaces. . . . We should dream of a technology which bypasses some of theses constraints: a combination of ear and eye---a new `superscore.' . . .\footnote{Simon Emmerson, `Crossing cultural boundaries through technology?', pp. 121-122.} 
\end{quote}
Emmerson's idea of a `superscore' combines oral and visual forms of communication within a multimedia object combining traditional notation, extended notation, recordings of example material from the live performer, electroacoustic materials, software for performance, patches for live electronic treatment, examples of live electronic treatment, an example recorded performance, written and spoken commentary, video performance material, video example material and graphical material.\footnote{Ibid., pp. 128-129.} 

Taking Emmerson's idea further, one could easily imagine the `superscore' as a package that combines performance materials with documentation (including video tutorials, audio examples (sampled mock-up performances or real performances), recordings, interviews, \emph{etc.}) residing on the internet. Additionally, with the increasing accessibility of laptops, one could easily imagine replacing a score that is printed on paper, with one that is displayed on a computer monitor. This would bring the opportunity of exploring the potential to communicate musical meaning through a computer display, which would add movement to the expressive palette of a conventional score. By using animated graphics, scores, pictures, as well as other types visual cues and timed written directions, the composer could enhance the way in which he communicates musical ideas and knowledge through the computer display. In addition, the performer could receive other types of audio information through headphones complimenting the visual input with an `aural score.' This could comprise from spoken directions and sounding cues (click tracks, reference pitches, etc) to recordings of acoustic or electroacoustic music that the performer would have to react to or improvise with. Moreover, with the development of real-time processing technologies and generative algorithms, the notion of a \emph{fixed} score could also be contested by a score that is \emph{dynamic}, thereby creating a composition that may change its content (pitches, rhythms, etc) each time it is performed. Real-time scoring could be explored further by combining elements of real-time animation and graphics display with new advancements in machine listening technologies, thereby generating a score that responds to the sonic and acoustic context of a specific performance and space. The possibility of creating a network including several computers could also provide instant communication between performers and the option of sending directions that would be specific to a particular performance. With the increasing popularity of wireless networks and new types of interfaces and gadgets, portable devises like the iPad or iPhone could be used to implement the `superscore,' making it easier to carry and even place in a music stand . 

In addition to enhancing the communication with musicians trained within the wester tradition, the `superscore' could also foster new collaborative possibilities between performers of different cultures. By sending information that is specifically devised and customized for a particular type of performer, the `superscore' could provide the opportunity for musicians from different backgrounds and traditions to share the stage simultaneously in a computer-mediated performance. A group of performers from mixed backgrounds could therefore play together within a predetermined structure by receiving different types of visual and aural stimuli. The collaborative opportunities this could bring are vast as technology could facilitate and even solve problems that until now have made it difficult (if not impossible) for musicians from different backgrounds to play together.  

\subsubsection{Crossing Cultural Borders?}

Given the opportunities technology brings for a diverse group of musicians to share the stage despite previous incompatible performance conventions, important questions arise concerning the types of relationship established during collaboration. These relationships might become particularly sensitive if one is collaborating with musicians from different cultures. In his article \emph{Crossing Cultural Boundaries through Technology}, Simon Emmerson already expresses some concerns as a composer when dealing with cross-cultural collaborations and `ensembles with ethnic instruments.' He argues that the western composer often appropriates music from different cultures through `strongly filtered sources' and cultural misunderstandings, frequently resulting in `cultural murder.' He therefore promotes a positive attitude towards `successful acculturation' of ` highly developed traditions' through education and practical experience.\footnote{Ibid., p. 115-134.} 

``There are plenty of examples of composers killing stone dead the spontaneity and vitality which they themselves admire in non-western music through insensitive appropriation of surface technique (usually, once again, through an inadequate notation system and inadequate formalized `rules'). Too simple an understanding of acculturation may hinder the very process we aim to foster.''\footnote{Ibid., p. 126-127.} 

\begin{quote}
What at first hearing we perceive as significant will be based on our previous experience and an expectation based on circumstance. But this creation of `new measures of significance' is precisely the task of any such intercultural enterprise. We may have to acknowledge that our new expressiveness cannot be defined `within' the culture we observe. We have no automatic right to an emic interpretation in this case. We are in a truly experimental situation. If our western composer invokes the `right to be wrong' at this stage and declares emic intentions for the work---possibly based on complete misreading and misunderstandings---he/she strongly reinforces the purely western basis for the evaluation of such projects thus defeating much of their object.\footnote{Ibid., p. 125-126.} 
\end{quote}


Musicians of different backgrounds (improvisation and notated music) and traditions (Western and non-Western) may now share the stage simultaneously and productively through technology; in spite of previously incompatible performance conventions.


A discussion of Simon Emmerson's Crossing Cultural Boundaries through Technology. 
\v{Z}i\v{z}ek's view of Multiculturalism. 

Link to appropriation . . .

\subsubsection{New Collaborative Possibilities: Real-time Appropriation and Exchange}
 
Real-Time computer processing allows the possibility of using the audio signal (as well as other information - like MIDI) from several live performances simultaneously as building blocks for a composition.

\subsubsection{Interactivity and Interpassivity}

\begin{quote}
Interpassivity, like interactivity, thus subverts the standard opposition between activity and passivity: if in interactivity (or the cunning of Reason), I am passive while being active through another, in interpassivity, I am active while being passive through another. More precisely, the term interactivity is currently used in two senses: (1) interacting with the medium, that is, not being just a passive consumer: (2) acting through another agent, so that my job is done, while I sit back and remain passive, just observing the game. While the opposite of the first mode of interactivity is also a kind of interpassivity, the mutual passivity of two subjects, like two lovers passively observing each other and merely enjoying each others presence, the proper notion of interpassivity aims at the reversal of the second meaning of interactivity: the distinguishing feature of interpassivity is that, in it, the subject is incessantly (frenetically even) active, while displacing on to another the fundamental passivity of his or her being.\footnote{From The Fantasy in Cyberspace by Slavoj \v{Z}i\v{z}ek}
\end{quote}



This can be achieved for example through new ways of engaging in the act of composition, new dynamics in improvisation, new ways of presenting music and new spaces in which it can be presented, new ways of documenting the act of composition and performance, new forms of scoring

How? Appropriation, Technology

How to reestablish the agreement? Appropriation/Ideology 

\section {Appropriation and Ideology in Music}

\begin{quote}
The contemporary era constantly proclaims itself as post-ideological, but this denial of ideology only provides the ultimate proof that we are more than ever embedded in ideology. Ideology is always a field of struggle---among other things, the struggle for appropriating past traditions.\footnote{Slavoj Zizek, ``It's Ideology, Stupid!'', in \emph{First As Tragedy, Then as Farce}, London: Verso, 2009, p. 37.}
\end{quote}
Start on appropriation and past traditions...

My approach to appropriation??
\begin{quote}
Characters on stage should be flat, like clothes in a fashion show: what you get should be no more than what you see. Psychological realism is repulsive, because it allows us to escape unpalatable reality by taking shelter in the ``luxuriousness'' of personality, losing ourselves in the depth of individual character. The writer's task is to block this manoeuvre, to chase us off to a point from which we can view the horror with a dispassionate eye.\footnote{Elfriede Jelinek, quoted in Slavoj Zizek, \emph{First As Tragedy, Then as Farce}, p. 40.}
\end{quote}

\begin{quote}
Consumption is simultaneously also production, just as in nature the production of a plant involves the consumption of elemental forces and chemical material\footnote{Karl Marx}
\end{quote}

\begin{quote}
Starting with the language imposed upon us (the system of production), we construct our own sentences (acts of everyday life), thereby reappropriating for ourselves, through these clandestine microbricolages, the last word in the productive chain.\footnote{Nicolas  Bourriaud, \emph{Postproduction. Culture as Screenplay: How Art Reprograms the World}, New York: Lukas and Sternberg, 2005.}
\end{quote}

\subsection{Appropriation and Postproduction in the Digital Age} 

\begin{quote}
By listening to music or reading a book, we produce new material, we become producers. And each day we benefit from more ways in which to organize this production: remote controls, VCRs, computers, MP3s, tools that allow us to select, reconstruct, and edit. Postproduction artists are agents of this evolution, the specialized workers of cultural reappropriation.\footnote{Ibid. p. ?}
\end{quote}

\begin{quote}
Throughout the eighties, the democratization of computers and the appearance of sampling allowed for the emergence of a new cultural configuration, whose figures are the programmer and DJ. The remixer has become more important than the instrumentalist, the rave more exciting than the concert hall. The supremacy of cultures of appropriation and the reprocessing of forms calls for an ethics: to paraphrase Philippe Thomas, artworks belong to everyone. Contemporary art tends to abolish the ownership of forms, or in any case to shake up the old jurisprudence. Are we heading toward a culture that would do away with copyright in favor of a policy allowing free access to works, a sort of blueprint for a communism of forms?\footnote{Ibid. p. ?}
\end{quote}

\subsection{critisisms} 
\subsection{The liberal-comunists: Open Source, etc.} 

There is no music by John Oswald on the net free to download. Hypocrisy from the appropriator? Or does he fall into the logic of late-capitalism - �no communism of forms�? �I plunder but don�t plunder me. Or, at least not for free��? 

I propose an attitude towards music appropriation similar to that of hacker communities and the open source initiative. Not with the purpose of suggesting a communist utopia, but of being consequent with my creative process. By giving away my music, recorded sounds and experiments, code, etc, through the net, I will hopefully instigate others to do so as well. If this attitude is followed, it could promote the organization of music cyber communities that would plunder, engage with and promote each other, hopefully producing more subversive types of music.

We are far from the Bourriaud�s utopia. The only people how have access to (artistic) shareware are commoditized people, mostly in western countries. Isn�t the DJ approach towards plunderphonics one that appropriates to make more profit and diminish costs only to thereafter feed back their product into the music industry system?

\subsection{Musical Appropriation through Technology}

I will continue by examining different strategies and practices used in my work that use technology as means to appropriate, derive from and transform existing music by other musicians. It is only logical, considering that music is not an object but a complex set of actions, productions, perceptions and thoughts,\footnote{See pp 13-45 for a discussion regarding my preference of the notion of a \emph{musical result} versus the more widely use concept of \emph{musical work}.} that the act of appropriation of existing music can manifest itself in many different ways and take lots of unexpected guises. Therefore, I will propose that the appropriation of existing music \emph{does not} refer exclusively to `borrowing' or `stealing' from \emph{musical works} by other composers but to . . . . Moreover, when dealing with appropriation, I will claim that there are certain fundamental questions that both music creators and listeners should ask themselves. According to David Mezter, Stockhausen (while referring to \emph{Hymnen}) emphasized the importance of asking the questions of ``what'' and ``how''  regarding the practice of `borrowing' or `quoting' from other music.
\begin{quote}
According to him \emph{[Stockhausen]}, the practice involves a rich exchange between the ``what'' and the ``how,'' that is, the gesture has us hear ``what'' music has been borrowed and ``how'' it has been changed. The more familiar and obvious the ``what,'' the more we are drawn into the ``how,'' and the more captivating the ``how,'' the more we can appreciate anew the ``what.'' It is the ways in which quotation handles the ``what'' and the ``how'' that make it so effective a  cultural agent.\footnote{David Metzer, \emph{Quotation and Cultural Meaning in Twentieth-Centure Music}, Cambridge: Cambridge University Press, 2003, p. 6.} 
\end{quote}
I agree with Stockhausen's claim because . . . 
Nevertheless, I would also add : 
the difference between ``what'' and ``who''
also ``from where''. 
but most importantly ``why,''  
Why = motivations.
The motivations regarding musical appropriation can be very varied and also reflect ideological positions that in many cases reflect more the beliefs and feelings of the appropriator that the appropriated. Therefore, I will attempt to explain my viewpoint regarding the motivations and ways in which I use other music within my own work. In doing so, I will also examine other composers work that deals with musical appropriation in ways that I consider valid, interesting and intriguing.

Technology

Will do so by examining other composers work dealing with this issues... that I find valid, interesting, intriguing, stimulating(?)...  

\subsubsection{Copyrights Violation}
 
\subsection{Scores}


The first strategy considered is Clarence Barlow's concept of \emph{Musica Derivata}, which refers to the idea of transforming existing music with Computer Aided Composition (CAC) tools to create ``music that is compositionally based on other music''\footnote{Clarence Barlow, \emph{Musica Derivata} [CD], hat[now]ART 126, Hat Hut Records Basle, 2000.} This approach seems to take as a starting point mostly notated material (but in some occasions spectral information from recordings) from music by other composers. 

MIDI

\subsubsection{me}

\subsection{Recordings}

	
\subsubsection{Plunderphonics}

Plunderphonics:

John Owald, 1985. ``Plunderphonics, or Audio Piracy as Compositional Prerogative''

Use of audio samples as a technique for composition. 

Different from Musica Derivata in that it appropriates the recording of the original musical source. Information from recording (tibre, rhythm, performace practice, etc) is plundered from the original source to create a new composition.

``As a listener my own preference is the option to experiment. My listening system has a mixer instead of a receiver, an infinitely variable speed turntable, filters, reverse capability, and a pair of ears. An active listener might speed up a piece of music in order to perceive more clearly it�s macrostructure, or slow it down to hear articulation and detail more precisely.''\footnote{John Oswald, ``Plunderphonics, or Audio Piracy as a Compositional Prerogative,'' in \emph{Wired Society Electro-Acoustic Conference}, Toronto, 1985. URL: \href{http://www.plunderphonics.com/xhtml/xplunder.html}{http://www.plunderphonics.com/xhtml/xplunder.html}.}

\subsubsection{Sound Transformations} 					


``With the power of the computer, we can transform sounds in such radical ways that we can no longer assert that the goal sound is related to the source sound merely because we have derived one from the other.'' (T. Wishart)

In my work, sound transformations are used for the transformation of existing music. 

Why transformation of musical sources? Because they may carry complex cultural symbolism. 

The amount of processing can affect our ability to recognize the source sound or musical sample. Therefore, there is a wide palette of derivative music available to us: from the radically processed � less recognizable source � more `abstract' extreme; to the less processed � more recognizable source � more `referential'  and quotation type music.

Performance practice and other sonic characteristics of many original musical sources is lost in the transcription to a fully notated score for ensembles of western classically trained musicians. Many aspects of sound production (intonation, groove, spectral characteristics of instruments/voices, etc) is lost via this process.

Process of derivation and sound transformation is not directly apparent to the audience. The act of appropriation is not transparent.


\subsection{Spectral Information} 

\subsubsection{To generate sc}

\subsection{Computer Code}

Max patches, Computer Code.

\subsection{Real Performances} 


\subsection{Real-Time Plunderphonics}

Appropriation of audio signals from live music performances as material for a new composition

Creates a cognitive dissonance between audio and visuals.

The amount of processing of the audio signals is visible. The more processed the performances are, the more contrasting they will look in relationship with what is heard through speakers.

In contrast to acousmatic tradition, Real-Time Plunderphonics makes the process of appropriation transparent to the audience through the cognitive association between audio and visuals.

Changes relationship with the appropriated Other: The performer becomes an accomplice in the process of appropriation (or themselves). 

Deals with the problematic of the lack of visual clues and theatrical elements in electronic music performance by introducing a dynamic group of live performers and an interesting and unusual visual scenario.  

\subsubsection{Some ideas of how to plunder}

Get to know what and who you are plundering and figure why your are doing so before you decide how to plunder.(Know your performers, their music and why you want to work with them)

Appropriate and plunder yourself. 

Plundering not as central purpose of the creative process, but rather a tool for creating new idiosyncratic audio/visual result. 

Use ``from raw to cooked'' (L\'{e}vi-Strauss) techniques to create a narrative that navigates, in literary terms, between the �real� (actual performance) and the `surreal' (extreme processed audio).

Combinations of Real-Time Plunderphonics, (Real-Time) Musica Derivata and Sound Transformations

Use plunderphones as data: reprogram, not just remix.

Micro and macro plundering.

Use also Non Real-Time tools (Scores, Samples, etc.) if suitable. 

Using plunderphones as data

An example: Use FFT data of your plunderphone to trigger samples of recorded instruments.

\subsubsection{Micro and Macro Plundering}

Microplunderphonics

Plundering just microelements of sound. Not the whole spectrum of the original sound file. 

Generate noise with your plunderphones and use it instead of white noise for sound synthesis


Macroplundering

Appropriate a composition�s form. Use the structure as blueprint for a new composition. 

Use variables of the appropriated piece (pitch, dynamics, etc.) as control structures for new output.



\subsection{Musical postmodernism in the digital age}

resurgence of image / music quotations/references - first as reaction to the anti-mimetic
later with digital technology, easy reproduction, etc, etc => the use of images becomes the same as before the establishment aestetic regime : commodification, capitalism, DJ culture, digital quotations (in hip-hop, sound libaries, etc, etc)

I propose an attitude towards music appropriation similar to that of hacker communities and the open source initiative. Not with the purpose of suggesting a communist utopia, but of being consequent with my creative process. By giving away my music, recorded sounds and experiments, code, etc, through the net, I will hopefully instigate others to do so as well. If this attitude is followed, it could promote the organization of music cyber communities that would plunder, engage with and promote each other, hopefully producing more subversive types of music.

We are far from the Bourriaud�s utopia. The only people how have access to (artistic) shareware are commoditized people, mostly in western countries. Isn�t the DJ approach towards plunderphonics one that appropriates to make more profit and diminish costs only to thereafter feed back their product into the music industry system?

\subsection{Musica Derivata and Plunderphonics}

``A good composer does not imitate; he steals''       I. Stravinsky

Musica Derivata:

``music that is compositionally based on other music'' (K. Barlow) 


\subsection{plunderphomes, ideology and the use of references}

\begin{quote}
While some start up a prolonged lamentation for the lost image, others reopen their albums to rediscover the pure enchantment of images- that is, the alterity of the \emph{was}, between the pleasure of pure presence and the bit of the absolute Other.
\end{quote}
\begin{quote}
Evidence of exhibitions devoter to `images', but also the dialectic that affects each type of image and mixes its legitimations and powers with those of the other tow.
\end{quote}
Plunderphones reflect ideology . . . \v{Z}i\v{z}ek/Adorno but. . . . The artist can present their own view of these references by rearranging them modifying them. The plunderphonics artist doesn't necessarily adheres to the ideology of the appropriated material, but reflects it by the use of the plunderphones - how are they presented, modified, etc?  

\subsection{On Musical Appropriation}

What? 

Code, compositional tecniques, what piece of music? 
Do we plunder from the ``flea market or (the) airport shopping mall''? (N. Bourriaud). From the top 20 list - J. Oswald approach-, or from the hidden CDs at the back of the music store?

Who?

Music Industry? Pop/commercial? Historical (dead composers)? Music from different cultures? 

Appropriation of the Other. What relationship do we want to establish with the Other? Impersonal like the 1st/3rd World relationships?

Liberal multiculturalists approach? ``Other deprived of its Otherness (the idealized Other who dances fascinating dances and has an ecologically sound holistic approach to reality, while features like wife beating remain out of sight�)?'' (Slavoj \v{Z}i\v{z}ek, 2003)

Why?

For the meaning of the cultural object you are appropriating? For it�s symbolism? To suggest a metaphor?

For it�s use? ``Don�t look for the meaning, look for the use'' - L. Wittgenstein - for example for the sonic qualities of the appropriation (intonation, groove, etc.)

How? �

\label{ch:strategies}