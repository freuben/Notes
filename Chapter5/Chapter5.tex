\hypertarget{chapter5}{}
\chapter{Appropriation as Strategy}

This chapter examines musical practices that take an explicit and formalized approach to the use of appropriation as a conceptual, performative and compositional strategy.
Not a survey...

\section{Ideology, Appropriation and the Past}

Appropriation is certainly nothing new to music or the other arts. Artists have used, borrowed and stolen from the work of others for centuries, either explicitly or without deliberation. The very notion of what it means to create art is implicitly related to the act of appropriating existing forms, actions and thoughts. As with any other type of production, art depends on already existing materials and actions to produce something new. In the case of artistic appropriation, the existing materials used not only refer to the physical materials or objects utilized to create art (the canvas, the strings of the piano, the stage, the artist's own body, \emph{etc.}), but also to its subject matter (already existing shapes, sounds, symbols, movements, \emph{etc.}). Nevertheless, artistic appropriation is not limited only to the act by which artists choose their materials to kick-start the artistic process. The notion of artistic appropriation also includes the process of determining the system of artistic production itself, which is based on a priori forms of knowledge, abstractions, deductions and subjective processes that as a whole constitute a set of creative frameworks and stratergies of production. Put briefly, not only does the artist appropriate existing materials to work with, but the method of working itself. Furthermore, the act of appropriation is not always deliberate. One, as an artist (and therefore as cultural agent), paradoxically might not be aware of all of the sources that one drives inspiration from---one might just causally absorb and appropriate from ones own experience by \emph{living} and dwelling within ones surrounding culture. Consequently, for as long as human beings have been producing art, there has been strategies of artistic appropriation. The notions of creativity and originality in art are therefore not independent from that of appropriation and a debate about art is only futile if one starts from the premise that artists either appropriate or not. A much more fruitful discussion would stem from the presupposition that all artists appropriate---they all use existing methods and materials, they borrow and steal from each other and from other existing cultural and historical modes of production---and the questions that are more relevant are what one as an artist appropriates, who and where one appropriates from, how and why one appropriates and what one accomplishes through appropriation. What the artist does with the appropriated methods and materials and with what purpose is therefore much more relevant to the notions of creativity and originality than questions of authorship, copyright or legitimacy concerning an individual been wholly responsible for the action of producing an `art work'.

What's more, whether deliberate or not, within the act of appropriation lies in essence the artist's relationship to the past. 

More on Past?

Non-ideological appropriation.

Adorno ---``music is not ideology'' . .  .  .

The way in which one as an artist appropriates, what one appropriates and where one appropriates from, reflects a deep connection to ones ideology and the way one relates to past traditions. The relationship between ideology and artistic appropriation at once makes art an ideological discipline---ideology is reflected in the creation and perception of art and different types of ideology may be identified in different types of art---and displays a significant link between artistic and cultural appropriation. Slavoj \v{Z}i\v{z}ek has argued that cultural appropriation and ideology imply an act of violence and friction that is created through the appropriation of past traditions. According to \v{Z}i\v{z}ek, in today's global society, ideology is more relevant than ever before, considering the violence implied in the act of cultural appropriation.
\begin{quote}
The contemporary era constantly proclaims itself as post-ideological, but this denial of ideology only provides the ultimate proof that we are more than ever embedded in ideology. Ideology is always a field of struggle---among other things, the struggle for appropriating past traditions.\footnote{\hyperlink{zizektragedy}{\v{Z}i\v{z}ek (2009)}, `It's Ideology, Stupid!', p. 37.}
\end{quote}
Given the relationship that exists between artistic and cultural appropriation, artistic appropriation may be deliberately used as an ideological tool and as a strategy to convey thoughts, opinions and feelings associated with the struggle of cultural appropriation. 

Past traditions . . .

Moreover, through the intentional and specific use of artistic appropriation as a an act of violence, the artists may denounce 

In today's globalized society . . . Given the fact that 

Appropriated material reflects ideology . . . \v{Z}i\v{z}ek/Adorno but. . . . The artist can present their own view of these references by rearranging them modifying them. The appropriation artist doesn't necessarily adheres to the ideology of the appropriated material, but reflects it by the use of the appropriation - how are they presented, modified, etc?  



\section{Appropriation Art}

Appropriation artists deliberatly/consciously use appropriation - formalization of appropriation. 

\subsection {Readymades and D\'etournement}

- Duchamp \\
- Reciprocal Readymade.\\
- D\'etournement \\
- Guy Debord \\

\subsection {Postproduction}

\begin{quote}
Starting with the language imposed upon us (the \emph{system} of production), we construct our own sentences (\emph{acts} of everyday life), thereby reappropriating for ourselves, through these clandestine microbricolages, the last word in the productive chain. Production thus becomes a lexicon of a practice, which is to say, the intermediary material from which new utterances can be articulated, instead of representing the end result of anything. What matters is what we make of the elements placed at our disposal. . . . By listening to music or reading a book, we produce new material, we become producers. And each day we benefit from more ways in which to organize this production: remote controls, VCRs, computers, MP3s, tools that allow us to select, reconstruct, and edit. Postproduction artists are agents of this evolution, the specialized workers of cultural reappropriation.\footnote{\hyperlink{postproduction}{Bourriaud (2005)}, pp. 24-25.}
\end{quote}

\subsection{Appropriation in the Digital Age}

\begin{quote}
Throughout the eighties, the democratization of computers and the appearance of sampling allowed for the emergence of a new cultural configuration, whose figures are the programmer and DJ. The remixer has become more important than the instrumentalist, the rave more exciting than the concert hall. The supremacy of cultures of appropriation and the reprocessing of forms calls for an ethics: to paraphrase Philippe Thomas, artworks belong to everyone. Contemporary art tends to abolish the ownership of forms, or in any case to shake up the old jurisprudence. Are we heading toward a culture that would do away with copyright in favor of a policy allowing free access to works, a sort of blueprint for a communism of forms?\footnote{Ibid. p. 35.}
\end{quote}

\subsubsection {Copying and Sharing: Social (Re)appropriation}

- Hackers \\
- Open Source \\

\subsubsection{Liberal-communists} 

- Zizek, Bill Gates model \\

\subsection {Postmodernism and Appropriation}

Relationship between Postmodern art and appropriation.
\\
Criticisms of Postmodern art practices?

\subsubsection{Considerations} 

What? 

Code, compositional tecniques, what piece of music? 
Do we plunder from the ``flea market or (the) airport shopping mall''? (N. Bourriaud). From the top 20 list - J. Oswald approach-, or from the hidden CDs at the back of the music store?
\\
Who?

Music Industry? Pop/commercial? Historical (dead composers)? Music from different cultures? 

Multiculturalism, Globalization., etc.

Appropriation of the Other. What relationship do we want to establish with the Other? Impersonal like the 1st/3rd World relationships?

Liberal multiculturalists approach? ``Other deprived of its Otherness (the idealized Other who dances fascinating dances and has an ecologically sound holistic approach to reality, while features like wife beating remain out of sight�)?'' (Slavoj \v{Z}i\v{z}ek, 2003)
\\
Why?

For the meaning of the cultural object you are appropriating? For it�s symbolism? To suggest a metaphor?

For it�s use? ``Don�t look for the meaning, look for the use'' - L. Wittgenstein - for example for the sonic qualities of the appropriation (intonation, groove, etc.)
\\
How?

\section{Musical Appropriation}

This is not a survey. For that ``Quotation and Cultural Meaning in Twentieth-Century Music.''\footnote{See \hyperlink{metzer}{Metzer (2003)}.}

\subsection{Postmodern Music}

Start with Berio--- transcriptions, etc. Then to more recent trends...

\subsection{Musica Derivata}

``music that is compositionally based on other music'' (K. Barlow) 

The first strategy considered is Clarence Barlow's concept of \emph{Musica Derivata}, which refers to the idea of transforming existing music with Computer Aided Composition (CAC) tools to create ``music that is compositionally based on other music''\footnote{\hyperlink{barlow}{Barlow (2000)}.} This approach seems to take as a starting point mostly notated material (but in some occasions spectral information from recordings) from music by other composers. 

Appropriation of Score Info: Performance practice and other sonic characteristics of many original musical sources is lost in the transcription to a fully notated score for ensembles of western classically trained musicians. Many aspects of sound production (intonation, groove, spectral characteristics of instruments/voices, etc) is lost via this process.

\subsubsection{Spectral Information} 

\subsection{Plunderphonics}

a little history... The Gramophone (ideas by Moholy-Magy\footnote{See \hyperlink{moholy}{Moholy-Nagy (2006)}}), John Cage, Stockhausen, Oswald, Negativeland etc.

John Oswald and Chris Cutler's articles.

\begin{quote}
As a listener my own preference is the option to experiment. My listening system has a mixer instead of a receiver, an infinitely variable speed turntable, filters, reverse capability, and a pair of ears. An active listener might speed up a piece of music in order to perceive more clearly it�s macrostructure, or slow it down to hear articulation and detail more precisely.\footnote{\hyperlink{oswald}{Oswald (1985)}.}
\end{quote}

\subsubsection{Copyrights}

\subsubsection{Sampling Culture}

\subsection{Other Ideas on Musical Appropriation}

\subsubsection{Sound Transformations} 					

``With the power of the computer, we can transform sounds in such radical ways that we can no longer assert that the goal sound is related to the source sound merely because we have derived one from the other''. (T. Wishart)

Palette: Recognizable (quotation) - to non-recognizable (``abstract'').

The amount of processing can affect our ability to recognize the source sound or musical sample. Therefore, there is a wide palette of derivative music available to us: from the radically processed � less recognizable source � more `abstract' extreme; to the less processed � more recognizable source � more `referential'  and quotation type music.

\subsubsection{Micro and Macro Plundering}

Microplunderphonics

Plundering just microelements of sound. Not the whole spectrum of the original sound file. 

Generate noise with your plunderphones and use it instead of white noise for sound synthesis

Macroplundering

Appropriate a composition�s form. Use the structure as blueprint for a new composition. 

Use variables of the appropriated piece (pitch, dynamics, etc.) as control structures for new output.

\subsubsection{Using Data and Mapping }

Using plunderphones as data

An example: Use FFT data of your plunderphone to trigger samples of recorded instruments.

\subsubsection{Sharing Code}

Max patches, Computer Code.

\subsubsection{Real-Time Plunderphonics}

Real-time possibilities of appropriation. Appropriation of Live Performances.

Appropriation of audio signals from live music performances as material for a new composition

Creates a cognitive dissonance between audio and visuals.

The amount of processing of the audio signals is visible. The more processed the performances are, the more contrasting they will look in relationship with what is heard through speakers.

In contrast to acousmatic tradition, Real-Time Plunderphonics makes the process of appropriation transparent to the audience through the cognitive association between audio and visuals.

Changes relationship with the appropriated Other: The performer becomes an accomplice in the process of appropriation (or themselves). 

\label{ch:approp}