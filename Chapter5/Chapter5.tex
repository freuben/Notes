\hypertarget{chapter5}{}
\chapter{Appropriation as Strategy}

This chapter examines musical practices that take an explicit and formalized approach to the use of appropriation as a conceptual, performative and compositional strategy.
Not a survey...

\section{Appropriation, Ideology and the Past}

Appropriation is certainly nothing new to music. Musicians have used, borrowed and stolen from the work of others for centuries, either explicitly or without deliberation. The very notion of what it means to make music is implicitly related to the act of appropriating existing sounds, structures, actions and thoughts. As with any other type of artistic production, music depends on already existing materials and actions to produce something new. In the case of musical appropriation, the existing materials used not only refer to the physical materials or objects utilized to create music (the instrument, the strings of the piano, the stage, the musician's own body, etc.), but also to its content (already existing sounds, rhythms, structures, gestures, timbres, etc.). Nevertheless, I will argue that musical appropriation is not limited only to the act by which musicians choose their materials to kick-start the creative process. My position is that we should consider a wider notion of musical appropriation that also includes the process of determining the system of musical production itself, which is based on a priori forms of knowledge, abstractions, deductions and subjective processes that as a whole constitute a set of creative frameworks and stratergies of production. Put briefly, not only does the musician appropriate existing materials to work with, but the method of working itself. Moreover, the methods by which the musician works may not only be appropriated from within the traditional conventions of music-making but also from the knowledge gained from other forms of thought and production that come from disciplines outside music (for instance from the visual arts, social sciences, physics, philosophy or mathematics, to mention just a few). Furthermore, one as a musician curiously enough might not be aware of all of the sources that one appropriates from (this also includes the music one `drives inspiration from')---one might just causally absorb and appropriate from ones own experience by \emph{living} and dwelling within ones surrounding culture. Consequently, for as long as human beings have produced music, there has been strategies of musical appropriation. The notions of creativity and originality in music are therefore not independent from that of appropriation and a debate about music is only futile if one starts from the premise that musicians either appropriate or not. A much more fruitful discussion would stem from the presupposition that all musicians appropriate---they all use existing materials and methods, they borrow and steal from each other and from other existing cultural and historical modes of production---and the questions that are more relevant are what one as a musician appropriates, who and where one appropriates from, how and why one appropriates and what one accomplishes through appropriation. What the musician does with the appropriated materials and methods and with what purpose, is therefore much more relevant to the notions of creativity and originality than questions of authorship, copyright or legitimacy concerning an individual been responsible for the action of creating music. 

What's more, whether deliberate or not, within the act of appropriation lies in essence the musician's relationship to the past. This relationship may be established through the materials and methods the musician appropriates as well as how they have been appropriated. For instance, a clear relationship could be established with the work of an individual composer from the past---either systematically or intuitively---by imitating, copying, interpreting or modifying materials and methods from his/her body of work. Therefore, the relationship between the musician and the past can be one that is initiated by a personal interest (which might be objective or subjective) in the musical practices and output of a specific historical figure. The process of appropriation from the work of historical figures that a musician admires or relates to is, I will claim, how many musicians start to form their (musical) identity. However, I believe that one as a musician should not only rely on relationships with the past based on personal or subjective impressions on the musical practices and output of individual historical figures without considering their work within a wider cultural and historical context. Therefore, when one is appropriating from another composer's work, one is not merely referring to an isolated body of work but accessing a complex set of relationships in connection to how the composer relates to his/her surrounding culture and what is the significance of his/her work in a wider historical and cultural context. That is to say, even when we appropriate from just one musical source from the past, we are in fact making reference to a panoply of historical and cultural symbols that today might have a complex order of significance. Nevertheless, 
the relationship between the musician and the past is not only reflected by who creates the appropriated materials and methods of music-making and what those are, but also on how the musician might appropriate them. Thus, one may deduce more (depending on the nature of the \emph{musical result}) about the musicians's relationship with the past through the way in which s/he appropriates an existing piece of music, than who possesses the original authorship of the composition and what cultural/historical significance it might have. Similarly, a musician's relationship with the past might be contingent to how s/he uses different types of performance practice within his/her own, rather than what these practices are or what they signify.

Additionally, within the act of appropriation lies a deep relationship between the musician who appropriates and the appropriated Other. How one as appropriator selects the material (whether it is a specific score by a composer from western music history or an `ethnic' instrument from a non-western culture) and acts towards it, reveals a type of relationship established between oneself and the Other. These types of relationships disclose (either objectively or subjectively) an attitude towards the Other that displays ones position in reference to the struggle of cultural appropriation. This attitude however might not be the one intended by the appropriator. That is to say, even though the intention of the musician who appropriates from the Other is to convey a specific type of attitude, in reality it might show a rather different one. That is the case today for instance, with a common attitude in western countries towards non-western cultures that is reflected through musical appropriation and can be associated with the notion of \emph{multiculturalism}.\footnote{See \hyperlink{multiculturalism}{pp. 43-44} for a wider discussion on the concept of \emph{multiculturalism} and  Slavoj \v{Z}i\v{z}ek's interpretation of what it means in actuality.} Even though the initial intent of a composer who incorporates instruments and performers from different cultures is to establish an attitude that reflects equality and tolerance, by treating the Other with `respect' and assuming a superfluous position of `openness', in reality what this type of appropriation might exhibit is a patronizing and condescending distance that situates the western composer in a position of superiority in the struggle that is an intercultural exchange. Therefore, determining the appropriator's attitude towards the appropriated Other requires a critical evaluation and reflection of the complex processes that take place during musical appropriation. Put briefly, one should not infer that the initial attitude one hopes to assume regarding the Other (without putting it to through rigorous scrutiny) will be reflected in the type of relationship that is eventually established. Furthermore, one cannot simply rely on `first impressions' concerning ones preconceived notions of the Other or the conviction that one will decipher his/her otherness through a process of acculturation. I believe that one can only take a sensible stance towards appropriation when one accepts the fundamental impossibility of fully understanding the Other. Thus, a `constructive' exchange regarding musical appropriation only starts when one agrees to disagree with the Other in some aspects of his/her musical judgements, taste and traditions. Moreover, one should not forget that the act of appropriation in-itself implies a form of violence and confrontation between different opinions, cultures, traditions and social strata.

The way in which one as a musician appropriates also reflects a deep connection to ones ideology and the way one relates to tradition. Even though music rarely has a \emph{raison d'\^{e}tre} that is simply ideological, it nevertheless reflects ideology (paradoxically not always been aware that it is doing so). Additionally, because of music's non-conceptual and non-objective nature, ideology is not revealed within music as clearly and directly as it is for instance through language. However, ideological positions maybe traced through what happens within music, including how and where it is created, presented, perceived and consumed. Music's subjective characteristics nevertheless make the task of examining ideology in music, an analytical and interpretative process. Therefore, a way of demonstrating ideological positions in music is through critical analysis and reflection, talking as a premise that music has a large context of meaning and that it can be examined in connexion to broader cultural phenomena. That been said, I believe that by critically analyzing the way in which musicians appropriate from tradition---as well as which traditions they appropriate from considering their own background---we might learn something about the prevailing ideologies that dwell within their work. Also, through this process we can also grasp the wider ideological context in which the music is created---in other words, we should not only take into consideration the ideological position of the musicians themselves but think about the bigger cultural picture, which includes the ideas and opinions of all the people involved in the creation and reception of the music (the institutions involved in a musical performance, the type of audience, the type of venue, the way in which the music is disseminated and consumed, etc.). Consequently, the relationship between ideology and musical appropriation at once makes music a discipline embedded within ideology and at the same time displays a significant link between musical and cultural appropriation. Slavoj \v{Z}i\v{z}ek has argued that cultural appropriation and ideology imply an act of violence and friction that is created through the appropriation of past traditions. According to \v{Z}i\v{z}ek, in today's global society, ideology is more relevant than ever before, considering the violence implied in the act of cultural appropriation.
\begin{quote}
The contemporary era constantly proclaims itself as post-ideological, but this denial of ideology only provides the ultimate proof that we are more than ever embedded in ideology. Ideology is always a field of struggle---among other things, the struggle for appropriating past traditions.\footnote{\hyperlink{zizektragedy}{\v{Z}i\v{z}ek (2009)}, `It's Ideology, Stupid!', p. 37.}
\end{quote}
As with other types of traditions, the appropriation of musical traditions also reflects an ideological struggle. Not only is music evidence of an ideological battleground, but the act of musical appropriation in itself is a form of cultural violence. It is my position that we as musicians, instead of denying this irrepressible fact, should be aware of the consequences inherent in the inevitability of appropriating existing music.

Given the relationship that exists between music and culture, and the struggle inherent  in appropriating musical traditions, we may deliberately use musical appropriation as an ideological tool to convey thoughts, opinions and feelings that may be associated with other forms of human knowledge and action. Therefore, through the intentional and formalized use of appropriation as a musical strategy, one as a musician may convey ones own subjective and objective views about the appropriated Other, which at the same time may reflect a wider commentary on ones relationship with the past, as well as with other cultures and traditions. By using appropriation as a strategy, one may also contemplate, express ones opinion and denounce specific forms of music-making (musical genres, styles, performance practices, compositional techniques, etc.), as well as people (specific composers, performers, types of audience, etc.) and objects (instruments, scores, stages, halls, etc.) involved in music-making that have symbolic significance in our present culture. That is to say, through strategies of formalized musical appropriation, one as a musician may convey ideology through musical aesthetics. Finally, by manipulating, rearranging, modifying and re-contextualizing the appropriated musical references, one as a musician may also construct a symbolic space where one can `play with culture and signification' by imaging alternate conditions, meanings and forms of visibility for these musical references, including the social and economic struggles they may represent. 

As it has been mentioned before, the deliberate and conscious use of appropriation as a creative strategy is nothing new to music. However, I will also claim that the strategies of musical appropriation---regarding the means by which one appropriates as well as the types of materials that are appropriated---have drastically changed during the twentieth- and twenty-first-centuries due to technological advancements that have redefined (and continue to redefine) the way in which we appropriate music. Additionally, during this time some prominent theories within the fine arts discourse dealing with specific types of appropriation have influenced the practice of musical appropriation.\footnote{That is not to say however that musical theories and practices haven't influenced appropriation in the fine arts. As we will see later on, certain musical practices (for instance, deejaying) have had a strong influence on art practices and theories dealing with appropriation.} Therefore, I will continue by examining a number of specific theories, strategies and approaches that have been used within fine arts during the twentieth-century that not only have been influential and important in defining current musical discourses and attitudes but also have had considerable influence on---and has serve as inspiration to---the submitted work. 

\section{Appropriation Art}

The following discussion has not as its aim to attempt to give a comprehensive survey of the artistic theories and strategies that use appropriation, nor it is an effort to make an original contribution on this subject of research. On the contrary, my aim is to simply describe certain concepts and ideas that are prominent in art theory that I think are relevant to musical appropriation and to the submitted work. Therefore, although I might be taking the risk of appearing to oversimplify in what is an exhaustive subject, I will not elaborate too much on the significance of the described artistic notions within fine arts practice, as my concern is mainly focused on the repercussions these ideas might have on music and how these artistic notions might contribute to the practices of musical appropriation. Therefore, this discussion is aimed at people who are interested in musical appropriation but are not necessarily aware of the discourse developed in the fine arts regarding appropriation.

\emph{Appropriation art} is a term used within art theory to describe practices by artists that deliberately use appropriation as a strategy to take possession of---usually unauthorized---materials (images, objects, etc.) whose authorship is widely acknowledged to be by others (other artists, filmmakers, companies, advertising agencies, etc.).\footnote{See \hyperlink{evans}{Evans (2009)}.} Even though \emph{appropriation art} is sometimes used to refer to a specific group of artists in the late seventies and eighties\footnote{Starting with the five artists (Brauntuch, Goldstein, Levine, Longo and Smith) featured in the `Pictures' exhibition in 1977, which showed re-used and assaulted mass-media photography.}  that were associated to certain New York galleries (these artists include Richard Prince, Cindy Sherman, Ashley Bickerton, Peter Halley, Jeff Koons and Meyer Vaisman), its theoretical and historical foundation starts at the beginning of the twentieth-century with the formulation of the montage, the development of photography and the influence of Marcel Duchamp. Moreover, \emph{appropriation art} is not only limited to refer to this specific set of artists as these strategies by now have been used widely by many artists who quite often have also developed their own idiosyncratic approaches towards appropriation. The advancements and increasing accessibility of computer technologies and the internet has also contributed to the way in which appropriation artists experience, consume, plunder and transform existing cultural objects. I will therefore attempt to briefly examine some of the notions and theories associated with \emph{appropriation art} with the aim of later investigating their possible implementation and relevance to musical practices.

\subsection {Readymades and D\'etournement}

The notion of appropriation within the arts during the twentieth-century was deeply influenced by Marcel Duchamp's concept of \emph{readymades}, which simply refers to an object that is `already made' that is presented as an `art work'. 
\begin{quote}
In 1913 I had the happy idea to fasten a bicycle wheel to a kitchen stool and watch it turn. A few months later I bought a cheap reproduction of a winter evening landscape, which I called `Pharmacy' after adding two small dots, one red and one yellow, in the horizon. In New York in 1915 I bought at a hardware store a snow shovel on which I wrote `in advance of the broken arm'. It was around that time that the word `readymade' came to mind to designate this form of manifestation. . . . At another time---wanting to expose the basic antimony between art and readymades---I imagined a `reciprocal readymade': use a Rembrant as ironing board!\footnote{\hyperlink{duchamp}{Duchamp (2009)}, p. 40.}
\end{quote}
This notion implies that the artist does not need to produce an `art work' but only choose a `found object'. The artistic process therefore might not imply the fabrication of an object but the selection of already existing ones---the artist's task is therefore to look at his/her surroundings and choose an object, and to give it meaning through its re-contextualization. Furthermore, Duchamp's \emph{reciprocal readymade} is the reversal of the former: an existing object considered to be an `art work'  that is given a new context by using it as an `ordinary' object that is normally not regarded to be art. By taking an object that is regarded as a `masterpiece' (a Rembrant) and using it as an object that usually performs a different function and is not appreciated for its aesthetic qualities (the ironing board), the \emph{reciprocal readymade} is a gesture that serves as criticism to the mystification behind such canonic artworks and at the same time questions the traditional western values of artistic appreciation. Moreover, by taking an `art work' and using it as an object usually not considered to be art, and using an existing `ordinary' object as an `art work', Duchamp contributes to the idea of art been more than just a `specialized' production of highly praised objects. The concept of \emph{readymades} therefore opens up a new debate on \emph{what should the function of the artist be} (if not that of producing objects), and at the same time questions notions of uniqueness and authorship in art. Additionally, within Duchamp's categories there is yet another type of \emph{readymade} which implies an alteration or adjustment by the artist (often very small) of the original `found object'. He calls this type of work \emph{assisted readymade}, from which \emph{L.H.O.O.Q.} (1919) is one of his most famous. \emph{L.H.O.O.Q.} consists of a reproduction (a commercial print) of Leonardo's \emph{Mona Lisa} on which Duchamp draws a mustache and goatee in pencil. By doing this simple gesture, Duchamp criticizes established and authoritative notions of western art by banalizing a canonic work that represents symbolically the ideal of beauty. The vulgarization of the image of the \emph{Mona Lisa} in a commercial reproduction rises questions about gender and the value of art, as well as it makes a comment on the reproduction, commercialization and popularization of art works.\footnote{\hyperlink{judovitz}{Judovitz (1995)}, pp. 139-143.}

The concepts of non-assisted (original without modification), \emph{assisted} and \emph{reciprocal readymades} were later embraced within the notion of \emph{d\'{e}tournement}, first introduced by the avant-garde movement the Situationist International in the late nineteen-fifties.\footnote{See \hyperlink{ford}{Ford (2005)} and \hyperlink{mcdonough}{McDonough (2002)} for more information on the Situationist International (SI).} \emph{D\'{e}tournement} (which means `diversion' as well as it can also mean hijacking, embezzlement and corruption) is an artistic and/or political gesture that consists of the use or hijacking of existing works of art and other cultural forms within the artist's own work.\footnote{\hyperlink{postproduction}{Bourriaud (2005)}, pp. 35-37.} This notion might also imply an intervention, re-contextualization or alteration to be exerted by the artist on the existing works or forms to provoke their depreciation, degeneration and devalorization. Furthermore, Guy Debord argues that \emph{d\'{e}tournement} is ``not an enemy of art. The enemies of art are those who have not wanted to take into account the positive lessons of the `degeneration of art'''.\footnote{\hyperlink{debord2}{Debord (2009)}, p. 66.} Therefore, the violence implied in the gesture of \emph{d\'{e}tournement} is not to oppose or denounce art but to provoke an action that interrupts and displaces an existing (artistic) theory---which reminds us that theories are truly faithful to themselves only if they allow `history's corrective judgment' upon themselves. Therefore, Debord sees appropriation that specifically maintains a certain distance from accepted conventions as a form of improving upon and correcting ideas from the past.
\begin{quote}
The defining characteristic of this use of \emph{d\'{e}tournement} is the necessity for \emph{distance} to be maintained towards whatever has been turned into an official verity. . . . Ideas Improve. The meaning of words has a part in the improvement. Plagiarism is necessary. Progress demands it. Staying close to an author's phrasing, plagiarism exploits his expressions, erases false ideas, replaces them with correct ides.\footnote{\hyperlink{debord}{Debord (1994)}, p. 145.}
\end{quote}
In other words, he claims that through the act of plagiarism one may improve, correct and update ideas and theories (about art), and at the same time in the process of doing so restraining them from their authority and alleged truth. Additionally, Debord argues that \emph{d\'{e}tournement} of existing works may also be used as a correction of the `artistic inversion of life'.\footnote{\hyperlink{debord2}{Debord (2009)}, p. 66.} By appropriating material (images, films, etc.) from the `spectacle' that self-evidently signifies a reversal of what happens in real life, the artist may in actuality represent real life, meaning and emotions through \emph{d\'{e}tournement}---consequently rectifying the inversion of reality implicit in the original material. An example of this strategy would be if an artist uses a \emph{d\'{e}tourned} hollywood romantic film---which in its original form does not accurately portray `love' in real life---to express and convey `love' in reality. Furthermore, \emph{d\'{e}tournement} denounces the separation (within the arts) between production and consumption that characterizes the `society of the spectacle', encouraging the appropriation of lived experience over the fabrication of work that contributes to the division between art and the spectator.\footnote{\hyperlink{postproduction}{Bourriaud (2005)}, p. 36.} In addition, it is important to take in consideration that there are different types of \emph{d\'{e}tourned} elements which may affect the strategies of appropriation. In `Directions for the Use of D\'{e}tournement', Debord and Wolman make a difference between two main categories of \emph{d\'{e}tournement}: minor and deceptive \emph{d\'{e}tournement}. Minor \emph{d\'{e}tournement} is the use of elements that have no considerable signification within a given political or social context, they only obtain meaning through their re-contextualization. Deceptive \emph{d\'{e}tournement} on the other hand, relies on elements that have a strong cultural/political meaning that acquires a different dimension of significance from the new context.\footnote{\hyperlink{debord3}{Debord and Wolman (2009)}, p. 35.} In addition, the authors attempt to give an analysis on certain considerations regarding \emph{d\'{e}tournement} that they considered important for their own artistic/political purposes. In their analysis, they also warn of less effective forms as well as negative uses of \emph{d\'{e}tournement}, which in some cases might include the use of \emph{d\'{e}tourned} elements for right-wing propaganda\footnote{Here, one could argue that a recent example that illustrates this point is the use of \emph{d\'{e}tournement} on Barack Obama's photograph (a mustache is superimposed on Obama's image) by the American `Tea-party' movement to suggest that a relationship exists between Obama and Hitler, which at the same time attempts to imply that Obama has `totalitarian impulses' that are tied to his `socialist agenda' of expanding the federal government.} or strategies that instead of condemning the `spectacle', often reassures its hegemony by using these strategies within consumer society and with the purpose of the production of commodities for individual profit and gain.\footnote{Ibid. pp., 36-37.} Additionally, one could argue that some manifestations of \emph{d\'{e}tournement} establish a cynical distance between the individual and the appropriated material that---instead of performing a subversive response to the `spectacle'---in reality conceals the individual's complicity with the prevailing system of commodification. According to Slavoj \v{Z}i\v{z}ek, in our contemporary `post-ideological'  era, artistic strategies that incorporate this cynical or ironic distance have lost their capability to be transgressive and have instead become a form of conformism.\footnote{See \hyperlink{zizekuniv}{\v{Z}i\v{z}ek (2006)}, `Why are Laibach and the \emph{Neue Slowenische Kunst} not Fascists?', pp. 63-66.}

\subsection {(Post)-postproduction in the Digital Age}

During the eighties, the interest of deliberately using appropriated material as a creative tool reemerged within the visual arts world. The concepts of \emph{readymades} and \emph{d\'{e}tournement} were reconsidered (and at the same time considerably expanded) as strategies of appropriation by artists who could sense that the tactics of \emph{modernism} (particularly those involving anti-mimetic principles) were approaching a dead-end. Nicolas Bourriaud, in his book \emph{Postproduction}, argues that for artists like Jeff Koons, Sherrie Levine and Heim Steinbach, Jean Baudrillard's ideas of \emph{simulation} and \emph{simulacra} supplied a theoretical foundation for much of their artistic practices at the time, which involved using appropriated objects that embodied the `subject of desire' within the western capitalist system. These objects become virtual, neutralized by their presentation as a product that is inaccessible to the viewer: the artist becomes the consumer of objects, at ones \emph{instead of} and \emph{for} the viewer. Bourriaud argues that these artists' work becomes \emph{simulacra}, the artist just appropriates from the market place, s/he chooses mainstream products that are most desirable as a  consumer to be exhibited and/or recreated as their work.\footnote{\hyperlink{postproduction}{Bourriaud (2005)},  p. 35.} This type of appropriation as a form of consumption was latter adapted by a new generation of artists in the nineties, but nevertheless what became apparent is that the new generation consumed \emph{different} products \emph{differently}. In other words, even though both generations followed a model by which the artist acts more as a consumer than a manufacturer, what they consumed and how they consumed it was quite contrasting between generations. According to Bourriaud, one of the main features of this new generation of artists which includes Rirkit Tiravanija, James Jones, Thomas Hirschorn and Michel Henochsberg, to mention just a few, is that they started to create work that appropriates products that are not those that the main-stream consumer would buy (like the former generation of artists did during the eighties), but articles one would find in more specialized or unusual shops, in the back of the selves or in the `flea-market.'\footnote{Ibid., p. 28.} The objects this new generation appropriates come from a diverse set of origins, circumstances, places, cultures, different lived stories and histories. Moreover, these artists display material differently from their predecessors: they present objects in arrangements that promote the formation of relationships between the people who are observing. The arrangement of the objects is often slightly chaotic, sometimes the materials are presented with the intention of escaping formal unity, the objects are displayed randomly as a cluster of non-assisted \emph{readymades} with the purpose of presenting an ensemble of disjunct objects that retain their autonomy, giving the impression that they resist categorization and structure. On the other hand, sometimes these arrangements also display some sort of `order within chaos' and within their apparent randomness, one can find intention, order, structure and homogeneity. Additionally, these artists use and manipulate objects in different ways, where the degree of transformation of material fluctuates between the extremely altered and the completely unaltered. Claude L\'{e}vi-Strauss's opposition between ``the raw and the cooked'' has been used to describe these practices in the transformation of appropriated material: `the cooked' representing the objects that have been radically transformed (sometimes beyond recognition) and `the raw' constituting the appropriated objects that have not been altered at all by the artist.\footnote{Ibid., p. 29.} The description of how `raw' or `cooked' this material is may characterize one single `art work' (where the amount of transformation of all its elements is consistent through the whole work). Conversely, elements with different levels of transformation may exist within the same work, creating a wide palette of `recycled' objects with varying degrees of alteration. The artistic strategies of appropriation that emerged during the eighties and nineties still constitute an important practice in the visual arts today. The tendency of how the appropriation artist works also seems to point towards an intensification in the amount of transformation of the appropriated material as well as a concern with finding new forms in which this material can be presented. What started as `strait-forward' appropriation (non-assisted or just slightly assisted) has gradually turned into (post)-postproduction\footnote{Here, I have added the extra `post' to \emph{postproduction} in brackets to differentiate this type of artistic strategy from its standard definition in the audio/video industry (I also want to avoid misunderstandings that for instance Bourriaud's use of the word might generate if compared to that of the industries. (Post)-postproduction therefore refers to a type of artistic production that relies on already produced art work, sound or music---even if in some cases this existing material has already been subjected to what the media industries call \emph{postproduction}.}---artists gradually further their use, manipulation and transformation of the material they seize and continue to increase the amount and diversity of sources they appropriate from.

The effect of digital technology in the way in which artists consume, create and present their work has also had a significant influence over the artistic strategies of (post)-postproduction. The increasing use and accessibility of computers and the internet has transformed the way artists use, experience and process cultural information. Digital technologies first brought with them unprecedented possibilities to reproduce information---copying and transferring data has become a powerful tool for artists. With the increasing power of computers, not only has the artist hugely increased his/her potential to reproduce, transfer and access information, but also s/he has mastered overwhelming control over the appropriated data. Moreover, through the development of the internet, sharing information has become a common practice which poses interesting questions regarding the ownership and copyrights of artistic forms. Ones information becomes digitalized, transferred and manipulated, its right of ownership becomes a complex and perplexing legal, ethical and aesthetic problem. Bourriaud has argued that for the reasons mentioned above, contemporary art tends towards an abolishment of the ownership of artistic forms. 
\begin{quote}
Throughout the eighties, the democratization of computers and the appearance of sampling allowed for the emergence of a new cultural configuration, whose figures are the programmer and DJ. The remixer has become more important than the instrumentalist, the rave more exciting than the concert hall. The supremacy of cultures of appropriation and the reprocessing of forms calls for an ethics: to paraphrase Philippe Thomas, artworks belong to everyone. Contemporary art tends to abolish the ownership of forms, or in any case to shake up the old jurisprudence. Are we heading toward a culture that would do away with copyright in favor of a policy allowing free access to works, a sort of blueprint for a communism of forms?\footnote{\hyperlink{postproduction}{Bourriaud (2005)},  p. 35.}
\end{quote}
The challenge that digital technology poses towards the ownership of forms has triggered new social configurations of artists, hackers, musicians and other people who take part in shareware and open source communities, where they openly share all of their digital productivity. In these communities, the notion of `creative output' has drifted away from the idea of individual intellectual property, in favor of a practice based on creating by openly using other individuals' contributions. Social re-appropriation has become a model by which reciprocated copying and sharing is evaluated for its potential to make the flow of information faster, smarter and more effective for the common gain of the community. However, we are far from Bourriaud's `communism of forms'. These communities although they might share their production with other members of the community, they do not directly oppose capitalism. Many of the individuals involved in these practices are mostly looking only for their own individual gain within our society of global capitalism. The `communism of forms' also does not reach but a small privileged group of artists and musicians, hackers and entrepreneurs who have been fortunate enough to have the time, training and resources to access these materials and information. Additionally, the programmer, artist or DJ who works by borrowing, stealing or sharing forms does not accurately represent `real existing communism', but rather a kind of figure closer to \v{Z}i\v{z}ek's interpretation of self-declared \emph{liberal communists}. According to \v{Z}i\v{z}ek, \emph{liberal communists} are new entrepreneurs who believe in (occasionally) giving their products for free (but at the same time making money from related services  like advertisements on their websites, donations, etc.), being aware of society and trying to change it through charity so that is fairer, using smart and dynamic communications and avoiding traditional notions of labor, being creative but at the same time sharing their creativity with others, promoting education, philanthropy and voluntary work.\footnote{\hyperlink{zizekviolence}{\v{Z}i\v{z}ek (2008)}, pp. 15-16.} However, \emph{liberal communists}---epitomized by the figure of Bill Gates (`the ex-hacker who made it')---at the same time are cold-hearted entrepreneurs whose main pursuit is to make more profit, even if in some cases that might imply engaging in `cruel' practices like destroying or buying their competition, engaging in dubious financial speculation, indirectly exploiting employees and attempting to monopolize the market.\footnote{Ibid. pp. 14-19.} \v{Z}i\v{z}ek has argued that this dual behavior discloses an avoidance of their complicity with the system.
\begin{quote}
In liberal communist ethics, the ruthless pursuit of profit is counteracted by charity. Charity is the humanitarian mask hiding the face of economic exploitation. In a superego blackmail of gigantic proportions, the developed countries `help' the undeveloped with aid, credits and so on, and thereby avoid the key issue, namely their complicity in and co-responsability for the miserable situation of the undeveloped.\footnote{Ibid. p. 19.}
\end{quote}
Therefore, one should be cautious about the optimism associated with digital sharing and hacking, cyber-communities, shareware, open source and other digital practices. The idea of giving away digital information for free and sharing creativity with others, as well as other proclaimed liberal communist principles might at a first glance seem positive stances but underneath they might carry a complicity with the same ruthless attitude that may be associated with today's global capitalism. Even though digital technology has the potential for people to organize themselves in cyber-communities and gives opportunities to subvert against traditional forms of capitalism (intellectual property that is digitalized is harder to regulate), we are far from Bourriaud's utopia---what we are approaching is not a blueprint for a `communism of forms' but a new configuration of digitalized production with adapted capitalistic values. Having said that, I am still convinced that these digital practices and communities have considerable artistic and musical potential if they are used positively as tools to create new \emph{aesthetic} forms. However, in doing so, I believe one should try to avoid naive political and artistic assumptions and attempt to reject the liberal communist hypocrisy. 

Having briefly described and examined some of the prominent theories and strategies of appropriation that emerged during the twentieth-century in the fine arts, I will now engage in a discussion about appropriation as it pertains to recent developments in music. My aim is to utilize some of the theory related to the notions of \emph{readymades}, \emph{d\'{e}tournement} and (post)-postproduction to explain musical strategies that deal explicitly with appropriation. I will also elaborate on already existing strategies developed during the twentieth-century that deal with musical appropriation and that I have attempted to implement in my own work.

\section{Musical Appropriation}

Musical appropriation is a very broad subject that includes topics as diverse as the use of chansons as \emph{cantus firmus} in polyphonic masses of the fifteenth- and sixteenth-centuries, Webern's orchestration of Bach's Ricercata from \emph{Musical Offering}, sampling practices in recent pop music, quotation in Charles Ives' music, Handel's operatic borrowings and the `dropping' of quotes in bebop solos. For that reason, I will try to narrow this discussion to a more limited subject, that is, how musical aesthetics and technological developments during the twentieth- and twenty-first-centuries have influenced the way in which musicians appropriate from already existing music. Nevertheless, I will not attempt to produce a detailed study of twentieth-century music that explicitly uses appropriation,\footnote{There are already good examples of such studies, see for instance \hyperlink{metzer}{Metzer (2003)}.} but instead I will only concentrate briefly on a few composers and musical strategies that I consider have influenced my own creative output. Additionally, I will venture to describe some ideas of how these musical strategies of appropriation could be expanded further through recent developments in computer technology. Finally, I will also endeavor to clarify to a greater extent the rationale behind the more idiosyncratic practices I have developed in my own work and the potential of using appropriation as a creative strategy to produce a \emph{music result} that aims to accomplish something new within music.

\subsection{Appropriation and Postmodern Music}

In \hyperlink{chapter2}{Chapter 2}, I have already described how \emph{postmodern music} at first started as a reaction to the confusion and misunderstandings ascribed to the notion of \emph{modernism} in music. For that reason, the first composers who later became associated with the label of \emph{postmodern music} abandoned modernism's anti-mimetic principles by amongst other strategies, explicitly appropriating from other music. Hence, the appropriation of existing music became a mechanism by which musicians could differentiate themselves from modernism, which was not only becoming institutionalized, but also had been weakened by its association with the `fall of communism' as well as its own avoidance to see beyond its self-imposed anti-mimetic principles and ideals of `purity' and `authenticity'. In addition, musical appropriation also became a vehicle by which these composers rediscovered the semiotic potential of using existing music to convey meaning (which was not possible if one would follow modernism's anti-mimetic principles) and thus, allowing them to make wider cultural/historical references within their work. Musical strategies such as quotation, transcription and transformation of already existing music became a form of re-contextualizing meaning, imagining new symbolic spaces for appropriated cultural/historical objects and a way of constructing narrative by creating wider cultural associations and metaphors.\footnote{Luciano Berio's oeuvre, for instance, contains many good examples of how these musical strategies can be used for the above-mentioned objectives.} However, as it was stated before, postmodern music in recent years has also become something more than the musical strategies of appropriation that have just been described. Today, much music labeled as postmodern for the most part approaches appropriation by adopting---in my view---a false notion of permissiveness resulting in a mixture of musical styles and genres that has as its only purpose to serve as commodified entertainment. Musical appropriation---which ones used to oppose the spectacle---has now become an accomplice to it.\footnote{Here, one could consider as an example of this stance, Mark-Anthony Turnage's recent appropriation of Beyonc\'{e}'s `Hammered Out' that appears in his composition `blabla', premiered at the Proms in /08/2010. The way in which Turnage appropriates Beyonce's popular hit, musically speaking, accomplishes nothing new but a narrowed down version of the original, adding not even an apparent commentary about the nature of the original. The Turnage appropriation even looses the rhythmic drive and timbreal characteristics of the original pop song through its orchestral arrangement (it is very difficult for orchestral musicians to accurately reproduce this kind of pop music because it is virtually impossible to translate the compressed and rhythmically precise sound of the studio production that characterizes Beyonce's music to an orchestral language). The result is a narrowed down version of the original (not being able to even render the sexuality and drive of Beyonce's version) which accomplishes nothing but commodified entertainment for a relatively conservative audience of concert goers, which to this day seem to `get a kick' from the (to their minds) still `transgressive' and `controversial' gesture of introducing pop styles into concert music.} The main problem of what postmodern music has become is that its musical `games' of aimlessly borrowing, mixing and remixing from a plurality of musics (classical, pop, rock, jazz, world music, etc.) have ceased to accomplish something new within music. At the same time, as I have argued at the beginning of this chapter, if one does not subscribe to this kind of `purposeless' appropriation, the solution can not be to avoid the act of appropriation itself, as this would only become a futile gesture that only would take us back to the misunderstandings brought by modernism's anti-mimetic principles. Therefore, the problem we are faced with at present time is not whether we should or shouldn't appropriate existing music, but more precisely, how can we accomplish something musically significant and new today through music appropriation. For that reason, I will now attempt to first examine several musical strategies that I consider approach the process of appropriation in significant and innovative (and sometimes controversial) ways and which, in my opinion, have the potential of inspiring new forms of music-making. Later, I will also undertake the task of proposing new ideas (based on the creative work submitted) of how these musical strategies might be further developed.

\subsection{Musica Derivata}

The first musical strategy I will discuss is Clarence Barlow's concept of \emph{musica derivata} which simply refers to ``music that is compositionally based on other music''.\footnote {\hyperlink{barlowcd}{Barlow (2000)}.} Thus, \emph{musica derivata} simply describes a method of composition that aims at deriving new material from existing music originally not written by the composer, to generate a new composition. However, Barlow's implementation of \emph{musica derivata} in his own work is more specific than one could draw from his own simple definition. What one could conclude from Barlow's compositions that can be labeled as \emph{musica derivata}, is that they deal with the borrowed material in such a way that it reveals something new about the appropriated music. At the same time, these compositions not only seem to be about the appropriated music but about music itself---they are concerned with disclosing subjectivities within the original music through very precise, exhaustive, and at times obsessive compositional processes that surgically reveal the `music within the music'. For instance, in his piano trio \emph{1981} (1981)\footnote{See Ibid.}, Barlow combines other piano trios by Clementi, Schumann and Ravel through a rigorously preconceived compositional process. \emph{1981} is a trio about three trios that travel through the three performers by means of interpolation between the selected compositions. Through a spiral structure, Barlow manages to morph between the notated material from each trio, which at fist frenetically cycles from one trio to the other, each rotation gradually becoming slower and progressively unveiling the original material which at the very end becomes clearly recognizable. By dissecting and re-contextualizing the piano trios through a rigorous method, Barlow discloses the essence of this ensemble's mode of playing as well as it dispassionately discloses the emotional \emph{ethos} of the appropriated compositions. \emph{`Spright the Diner' by Nib Writer} (1986),\footnote{See Ibid.} Barlow's following piano trio, in turn uses material from \emph{1981}---thereby simultaneously appropriating his own composition and further transforming the already modified trios by Clementy, Schumann and Ravel. \emph{Spright the Diner}, curiously enough seems to be more of a self-commentary (or a commentary by his alter ego \emph{Nib Writer}) on Barlow's own personality and compositional practices than a reflexion on the three original piano trios.

Therefore, \emph{musica derivata} can also be a mechanism whereby the composer can make comments about the appropriated music, the composer or the tradition it might represent, as well as how the music relates to other practices in performance and composition and how it might be connected to the appropriators' personal experience. Barlow's own rationale for the use of other music within his work, varies almost on all of his \emph{musica derivata} compositions. 
\begin{quote}
Sometimes I use other musics to pay homage to or even to ridicule them, sometimes just to point out certain processes in music, in musical practice. There are hardly any two pieces which have the same motivation in using other musics. I frequently do a whole lecture on my derived pieces and I would say that almost all the examples there have different reasons for being.\footnote{\hyperlink{barlowcd}{Barlow (2007)}.}
\end{quote}

Another important element about \emph{musica derivata} is the use of technology in the process of derivation.

Additionally, technology plays a vital role in Barlow's own application of \emph{musica derivata} in his own music---usually involving Computer Aided Composition (CAC) tools in the process of derivation. 


In some occasions spectral information from recorded material.

\subsubsection{Spectral Information} 

\emph{Musica Linguistica}

Spectralism . . .

\emph{Synthrumentation}\footnote{See {\hyperlink{barlowspec}{Barlow (1998)}}.}

\emph{Im Januar am Nil} (1982),  The orchestra saying "Why me no money?" in \emph{Orchideae Ordinariae}

\emph{Spectastics}

%This text is to draw your attention to two techniques I have developed for the approximated reproduction of speech sounds solely by acoustic instruments. In both, a Fast Fourier Analysis of sound waves is the starting point. Eight diagrammes are supplied as illustration. The first of these techniques, Additive Synthesis through Musical Instruments, which I will call Synthrumentation, creates a polyphonic note score in which all pitches at any given time are upper partials of a (not necessarily explicitly played) melody, acoustically coalescing to project the melody, if physically absent, as a residual tone or simply to enhance its timbre if it is physically present ; the second technique, Spectral Stochastics, here called Spectastics, redefines time-variant spectral amplitude envelope as a probability distribution curve for a monographic sequence of pitches, which if performed rapidly enough, can dissolve a cloud of notes reflecting the input timbres.

Conclusion:

Christopher Fox, in his article `Where the river bends: the Cologne School in retrospect' argues that Clarence Barlow, as well as other composers associated with what he calls \emph{The Cologne School}\footnote{Which consists of a group of four composers: Clarence Barlow, Gerald Barry, Kevin Volans and Walter Zimmerman, who during the seventies and early eighties were based in Cologne and shared similar musical interests. See \hyperlink{fox}{Fox (2007)}.}

I this case, his approach involves taking as a starting point mostly notated material from music by other composers and resulting as well in the creation of a score for classically trained musicians to perform. 

%Appropriation of Score Info: Performance practice and other sonic characteristics of many original musical sources is lost in the transcription to a fully notated score for ensembles of western classically trained musicians. Many aspects of sound production (intonation, groove, spectral characteristics of instruments/voices, etc) is lost via this process.

\emph{Musica Derivata} I will say, mostly results in notated material for performers trained in western `classical' tradition. Also, pretty pitched based, notation and MIDI. 

\subsection{Plunderphonics}

a little history... The Gramophone (ideas by Moholy-Magy\footnote{See \hyperlink{moholy}{Moholy-Nagy (2006)}}), John Cage, Nam June Paik, Stockhausen, Oswald, Negativeland, etc.

John Oswald and Chris Cutler's articles.

\begin{quote}
As a listener my own preference is the option to experiment. My listening system has a mixer instead of a receiver, an infinitely variable speed turntable, filters, reverse capability, and a pair of ears. An active listener might speed up a piece of music in order to perceive more clearly it�s macrostructure, or slow it down to hear articulation and detail more precisely.\footnote{\hyperlink{oswald}{Oswald (1985)}.}
\end{quote}

\subsubsection{Copyrights}

\subsubsection{Sampling Culture}

\subsection{Other Ideas on Musical Appropriation}

\subsubsection{The Impact of Sound Transformations} 					

``With the power of the computer, we can transform sounds in such radical ways that we can no longer assert that the goal sound is related to the source sound merely because we have derived one from the other''. (T. Wishart)

Palette: Recognizable (quotation) - to non-recognizable (``abstract'').

The amount of processing can affect our ability to recognize the source sound or musical sample. Therefore, there is a wide palette of derivative music available to us: from the radically processed � less recognizable source � more `abstract' extreme; to the less processed � more recognizable source � more `referential'  and quotation type music.

\subsubsection{Micro and Macro Plundering}

Microplunderphonics

Plundering just microelements of sound. Not the whole spectrum of the original sound file. 

Generate noise with your plunderphones and use it instead of white noise for sound synthesis

Macroplundering

Appropriate a composition�s form. Use the structure as blueprint for a new composition. 

Use variables of the appropriated piece (pitch, dynamics, etc.) as control structures for new output.

\subsubsection{Using Data and Mapping }

Using plunderphones as data

An example: Use FFT data of your plunderphone to trigger samples of recorded instruments.

Peter Ablinger...

\subsubsection{Sharing Code}

Max patches, Computer Code.

\subsubsection{Real-Time Plunderphonics}

Real-time possibilities of appropriation. Appropriation of Live Performances.

Appropriation of audio signals from live music performances as material for a new composition

Creates a cognitive dissonance between audio and visuals.

The amount of processing of the audio signals is visible. The more processed the performances are, the more contrasting they will look in relationship with what is heard through speakers.

In contrast to acousmatic tradition, Real-Time Plunderphonics makes the process of appropriation transparent to the audience through the cognitive association between audio and visuals.

Changes relationship with the appropriated Other: The performer becomes an accomplice in the process of appropriation (or themselves). 

\subsubsection{Other considerations} 

What? 

Code, compositional tecniques, what piece of music? 
Do we plunder from the ``flea market or (the) airport shopping mall''? (N. Bourriaud). From the top 20 list - J. Oswald approach-, or from the hidden CDs at the back of the music store?
\\
Who?

Music Industry? Pop/commercial? Historical (dead composers)? Music from different cultures? 

Multiculturalism, Globalization., etc.

Appropriation of the Other. What relationship do we want to establish with the Other? Impersonal like the 1st/3rd World relationships?

Liberal multiculturalists approach? ``Other deprived of its Otherness (the idealized Other who dances fascinating dances and has an ecologically sound holistic approach to reality, while features like wife beating remain out of sight�)?'' (Slavoj \v{Z}i\v{z}ek, 2003)
\\
Why?

For the meaning of the cultural object you are appropriating? For it�s symbolism? To suggest a metaphor?

For it�s use? ``Don�t look for the meaning, look for the use'' - L. Wittgenstein - for example for the sonic qualities of the appropriation (intonation, groove, etc.)
\\
How?

\label{ch:approp}