\hypertarget{chapter5}{}
\chapter{Revising Appropriation}

This chapter examines musical practices that take an explicit and formalized approach to the use of appropriation as a conceptual, performative and compositional strategy.
Not a survey...

\section{Ideology, Appropriation and the Past}

Appropriation is certainly nothing new to music or the other arts. Artists have used, borrowed and stolen from the work of others for centuries now, either explicitly or without deliberation. The very notion of what it means to create art is implicitly related to the act of appropriating existing forms, actions and thoughts. As with any other type of production, art depends on already existing materials and actions to produce something new. In the case of artistic appropriation, the existing materials used not only refer to the physical materials or objects utilized to create art, but also to their subject matter (already existing shapes, sounds, symbols, \emph{etc.}). Nevertheless, artistic appropriation is not limited only to the act by which artists choose their materials to kick-start the artistic process. The notion of artistic appropriation also should be expanded to include the process of determining the system of artistic production itself that is based on a priori forms of knowledge, abstractions and deductions as well as subjective process that as a whole constitute a set of creative frameworks and stratergies of production. Put briefly, not only does the artist appropriate existing materials to work with, but the method of working itself. Therefore, artists not only 


The act of appropriation is not always deliberate!

to the act of labor The effort
of past notions, knowledge and traditions. 

past traditions

\begin{quote}
The contemporary era constantly proclaims itself as post-ideological, but this denial of ideology only provides the ultimate proof that we are more than ever embedded in ideology. Ideology is always a field of struggle---among other things, the struggle for appropriating past traditions.\footnote{\hyperlink{zizektragedy}{\v{Z}i\v{z}ek (2009)}, `It's Ideology, Stupid!', p. 37.}
\end{quote}



\section{Appropriation Art}


Types:

\subsection {Postproduction}

\begin{quote}
Starting with the language imposed upon us (the \emph{system} of production), we construct our own sentences (\emph{acts} of everyday life), thereby reappropriating for ourselves, through these clandestine microbricolages, the last word in the productive chain. Production thus becomes a lexicon of a practice, which is to say, the intermediary material from which new utterances can be articulated, instead of representing the end result of anything. What matters is what we make of the elements placed at our disposal. . . . By listening to music or reading a book, we produce new material, we become producers. And each day we benefit from more ways in which to organize this production: remote controls, VCRs, computers, MP3s, tools that allow us to select, reconstruct, and edit. Postproduction artists are agents of this evolution, the specialized workers of cultural reappropriation.\footnote{\hyperlink{postproduction}{Bourriaud (2005)}, pp. 24-25.}
\end{quote}

\begin{quote}
Throughout the eighties, the democratization of computers and the appearance of sampling allowed for the emergence of a new cultural configuration, whose figures are the programmer and DJ. The remixer has become more important than the instrumentalist, the rave more exciting than the concert hall. The supremacy of cultures of appropriation and the reprocessing of forms calls for an ethics: to paraphrase Philippe Thomas, artworks belong to everyone. Contemporary art tends to abolish the ownership of forms, or in any case to shake up the old jurisprudence. Are we heading toward a culture that would do away with copyright in favor of a policy allowing free access to works, a sort of blueprint for a communism of forms?\footnote{Ibid. p. 35.}
\end{quote}

\subsection {D\'etournement}

\subsection {Appropriation of Social Forms}

\label{ch:approp}