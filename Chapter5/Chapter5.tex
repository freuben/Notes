\hypertarget{chapter5}{}
\chapter{Appropriation as Strategy}

This chapter examines musical practices that take an explicit and formalized approach to the use of appropriation as a conceptual, performative and compositional strategy.
Not a survey...

\section{Appropriation, Ideology and the Past}

Appropriation is certainly nothing new to music. Musicians have used, borrowed and stolen from the work of others for centuries, either explicitly or without deliberation. The very notion of what it means to make music is implicitly related to the act of appropriating existing sounds, structures, actions and thoughts. As with any other type of artistic production, music depends on already existing materials and actions to produce something new. In the case of musical appropriation, the existing materials used not only refer to the physical materials or objects utilized to create music (the instrument, the strings of the piano, the stage, the musician's own body, \emph{etc.}), but also to its content (already existing sounds, rhythms, structures, gestures, timbres, \emph{etc.}). Nevertheless, I will argue that musical appropriation is not limited only to the act by which musicians choose their materials to kick-start the creative process. My position is that we should consider a wider notion of musical appropriation that also includes the process of determining the system of musical production itself, which is based on a priori forms of knowledge, abstractions, deductions and subjective processes that as a whole constitute a set of creative frameworks and stratergies of production. Put briefly, not only does the musician appropriate existing materials to work with, but the method of working itself. Moreover, the methods by which the musician works may not only be appropriated from within the traditional conventions of music-making but also from the knowledge gained from other forms of thought and production that come from disciplines outside music (for instance from the visual arts, social sciences, physics, philosophy or mathematics, to mention just a few). Furthermore, one as a musician curiously enough might not be aware of all of the sources that one appropriates from (this also includes the music one `drives inspiration from')---one might just causally absorb and appropriate from ones own experience by \emph{living} and dwelling within ones surrounding culture. Consequently, for as long as human beings have produced music, there has been strategies of musical appropriation. The notions of creativity and originality in music are therefore not independent from that of appropriation and a debate about music is only futile if one starts from the premise that musicians either appropriate or not. A much more fruitful discussion would stem from the presupposition that all musicians appropriate---they all use existing materials and methods, they borrow and steal from each other and from other existing cultural and historical modes of production---and the questions that are more relevant are what one as a musician appropriates, who and where one appropriates from, how and why one appropriates and what one accomplishes through appropriation. What the musician does with the appropriated materials and methods and with what purpose, is therefore much more relevant to the notions of creativity and originality than questions of authorship, copyright or legitimacy concerning an individual been responsible for the action of making music. 

What's more, whether deliberate or not, within the act of appropriation lies in essence the musician's relationship to the past. This relationship may be established through the materials and methods the musician appropriates as well as how they have been appropriated. For instance, a clear relationship could be established with the work of an individual composer from the past---either systematically or intuitively---by imitating, copying, interpreting or modifying materials and methods from his/her body of work. Therefore, the relationship between the musician and the past can be one that is initiated by a personal interest (which might be objective or subjective) in the musical practices and output of a specific historical figure. The process of appropriation from the work of historical figures that a musician admires or relates to is, I will claim, how many musicians start to form their (musical) identity. However, I believe that one as a musician should not only rely on relationships with the past based on personal or subjective impressions on the musical practices and output of individual historical figures without considering their work within a wider cultural and historical context. Therefore, when one is appropriating from another composer's work, one is not merely referring to an isolated body of work but accessing a complex set of relationships in connection to how the composer relates to his/her surrounding culture and what is the significance of his/her work in a wider historical and cultural context. That is to say, even when we appropriate from just one musical source from the past, we are in fact making reference to a panoply of historical and cultural symbols that today might have a complex order of significance. Nevertheless, 
the relationship between the musician and the past is not only reflected by who creates the appropriated materials and methods of music-making and what those are, but also on how the musician might appropriate them. Thus, one may deduce more (depending on the nature of the \emph{musical result}) about the musicians's relationship with the past through the way in which he/she appropriates an existing piece of music, than who possesses the original authorship of the composition and what cultural/historical significance it might have.
%\footnote{Maybe some examples?} 
Similarly, a musician's relationship with the past might be contingent to how he/she uses different types of performance practice within his/her own, rather than what these practices are or what they signify.
%\footnote{Maybe more examples?}

Additionally, within the act of appropriation lies a deep relationship between the musician who appropriates and the appropriated Other. How one as appropriator selects the material (whether it is a specific score by a composer from western music history or an `ethnic' instrument from a non-western culture) and acts towards it, reveals a type of relationship established between oneself and the Other. These types of relationships disclose (either objectively or subjectively) an attitude towards the Other that displays ones position in reference to the struggle of cultural appropriation. This attitude however might not be the one intended by the appropriator. That is to say, even though the intention of the musician who appropriates from the Other is to convey a specific type of attitude, in reality it might show a rather different one. That is the case today for instance, with a common attitude in western countries towards non-western cultures that is reflected through musical appropriation and can be associated with the notion of \emph{multiculturalism}.\footnote{See \hyperlink{multiculturalism}{pp. 43-44} for a wider discussion on the concept of \emph{multiculturalism} and  Slavoj \v{Z}i\v{z}ek's interpretation of what it means in actuality.} Even though the initial intent of a composer who incorporates instruments and performers from different cultures is to establish an attitude that reflects equality and tolerance, by treating the Other with `respect' and assuming a superfluous position of `openness', in reality what this type of appropriation might exhibit is a patronizing and condescending distance that situates the western composer in a position of superiority in the struggle that is an intercultural exchange. Therefore, determining the appropriator's attitude towards the appropriated Other requires a critical evaluation and reflection of the complex processes that take place during musical appropriation. Put briefly, one should not infer that the initial attitude one hopes to assume regarding the Other (without putting it to through rigorous scrutiny) will be reflected in the type of relationship that is eventually established. Furthermore, one cannot simply rely on `first impressions' concerning ones preconceived notions of the Other or the conviction that one will decipher his/her otherness through a process of acculturation. I believe that one can only take a sensible stance towards appropriation when one accepts the fundamental impossibility of fully understanding the Other. Thus, a `constructive' exchange regarding musical appropriation only starts when one agrees to disagree with the Other in some aspects of his/her musical traditions. Moreover, one should not forget that the act of appropriation in-itself implies a form of violence and confrontation between different cultures, traditions and social strata.

The way in which one as a musician appropriates also reflects a deep connection to ones ideology and the way one relates to tradition. Even though music rarely has a \emph{raison d'\^{e}tre} that is simply ideological, it nevertheless reflects ideology (paradoxically not always been aware that it is doing so). Additionally, because of music's non-conceptual and non-objective nature, ideology is not revealed within music as clearly and directly as it is for instance through language. However, ideological positions maybe traced through what happens within music, including how and where it is created, presented, perceived and consumed. Music's subjective characteristics nevertheless make the task of examining ideology in music, an analytical and interpretative process. Therefore, a way of demonstrating ideological positions in music is through critical analysis and reflection, talking as a premise that music has a large context of meaning and that it can be examined in connexion to broader cultural phenomena. That been said, I believe that by critically analyzing the way in which musicians appropriate from tradition---as well as which traditions they appropriate from considering their own background---we might learn something about the prevailing ideologies that dwell within their work. Also, through this process we can also grasp the wider ideological context in which the music is created---in other words, we should not only take into consideration the ideological position of the musicians themselves but think about the bigger cultural picture, which would includes the ideas and opinions of all the people involved in the creation and reception of the music (the institutions involved in a musical performance, the type of audience, the type of venue, the way in which the music is disseminated and consumed, \emph{etc}.). Consequently, the relationship between ideology and musical appropriation at once makes music a discipline embedded within ideology and at the same time displays a significant link between musical and cultural appropriation. Slavoj \v{Z}i\v{z}ek has argued that cultural appropriation and ideology imply an act of violence and friction that is created through the appropriation of past traditions. According to \v{Z}i\v{z}ek, in today's global society, ideology is more relevant than ever before, considering the violence implied in the act of cultural appropriation.
\begin{quote}
The contemporary era constantly proclaims itself as post-ideological, but this denial of ideology only provides the ultimate proof that we are more than ever embedded in ideology. Ideology is always a field of struggle---among other things, the struggle for appropriating past traditions.\footnote{\hyperlink{zizektragedy}{\v{Z}i\v{z}ek (2009)}, `It's Ideology, Stupid!', p. 37.}
\end{quote}
As with other types of traditions, the appropriation of musical traditions also reflects an ideological struggle. Not only is music evidence of an ideological battleground, but the act of musical appropriation in itself is a form of cultural violence. It is my position that we as musicians, instead of denying this irrepressible fact, should be aware of the consequences inherent in the inevitability of appropriating existing music.

Given the relationship that exists between music and culture, and the struggle inherent  in appropriating musical traditions, we may deliberately use musical appropriation as an ideological tool to convey thoughts, opinions and feelings that may be associated with other forms of human knowledge and action. Therefore, through the intentional and formalized use of appropriation as a musical strategy, one as a musician may convey ones own subjective and objective views about the appropriated Other, which at the same time may reflect a wider commentary on ones relationship with the past, as well as with other cultures and traditions. By using appropriation as a strategy, one may also contemplate, express ones opinion and denounce specific forms of music-making (musical genres, styles, performance practices, compositional techniques, \emph{etc}.), as well as people (specific composers, performers, types of audience, \emph{etc}.) and objects (instruments, scores, stages, halls, \emph{etc.}) involved in music-making that have symbolic significance in our present culture. That is to say, through strategies of formalized musical appropriation, one as a musician may convey ideology through musical aesthetics. Finally, by manipulating, rearranging, modifying and re-contextualizing the appropriated musical references, one as a musician may also construct a symbolic space where one can `play with culture and signification' by imaging alternate conditions, meanings and forms of visibility for these musical references, including the social and economic struggles they may represent. 

As it has been mentioned before, the deliberate and conscious use of appropriation as a creative strategy is nothing new to music. However, I will also claim that the strategies of musical appropriation---regarding the means by which one appropriates as well as the types of materials that are appropriated---have drastically changed during the twentieth and twenty-first centuries due to technological advancements that have redefined (and continue to redefine) the way in which we appropriate music. Additionally, during this time some prominent theories within the fine arts discourse dealing with specific types of appropriation have influenced the practice of musical appropriation.\footnote{That is not to say however that musical theories and practices haven't influenced appropriation art. As we will see later on, certain musical practices (for instance, deejaying) have had a strong influence on art practices and theories dealing with appropriation.} Therefore, I will continue by examining a number of specific theories, strategies and approaches that have been used within fine arts during the twentieth century that not only have been influential and important in defining current musical discourses and attitudes but also have had considerable influence on---and has serve as inspiration to---the submitted work. 

\section{Appropriation Art}

The following discussion has not as its aim to attempt to give a comprehensive survey of the artistic theories and strategies that use appropriation, nor it is an effort to make an original contribution to the academic research on the subject. On the contrary, my aim is to simply describe certain concepts and ideas that are prominent in art theory that I think are relevant to musical appropriation and to the submitted work. Therefore, although I might be taking the risk of appearing to oversimplify in what is an exhaustive subject, I will not elaborate too much on the significance of the described artistic notions within fine arts practice, as my concern is mostly focused on the repercussions these ideas might have on music and how these artistic notions might contribute to the practices of musical appropriation. Therefore, this discussion is aimed at musicians who are interested in musical appropriation but are not necessarily aware of the discourse developed in the fine arts regarding appropriation.

\emph{Appropriation art} is a term used within art theory to describe practices by artists that deliberately use appropriation as a strategy to take possession of---usually unauthorized---materials (images, objects, \emph{etc.}) whose authorship is widely acknowledged to be by others (other artists, filmmakers, companies, advertising agencies, \emph{etc.}).\footnote{See \hyperlink{evans}{Evans (2009)}.} Even though \emph{appropriation art} is sometimes used to refer to a specific group of artists in the late seventies and eighties\footnote{Starting with the `Pictures' exhibition in 1977 which featured art by five artists (Brauntuch, Goldstein, Levine, Longo and Smith) dealing with re-used and assaulted mass-media photography.}  that were associated to certain New York galleries (these artists include Richard Prince, Cindy Sherman, Ashley Bickerton, Peter Halley, Jeff Koons and Meyer Vaisman), its theoretical and historical foundation starts at the beginning of the twentieth century, with the formulation of the montage, the development of photography and the influence of Marcel Duchamp. Art theorists and artists associated with \emph{appropriation art} incorporated ideas---as a justification to their own discourse---by thinkers like Walter Benjamin, Roland Barthes, Guy Debord, Claude L\'{e}vi-Strauss and Jean Baudrillard, to mention just a few.\footnote{Ibid. p. 12-15.} I will therefore briefly examine some of the theories behind these artistic strategies with the purpose of later investigating their possible implementation and relevance to past, current and future musical practices.

\subsection {Readymades and D\'etournement}

The notion of appropriation within the arts during the twentieth century was deeply influenced by Marcel Duchamp's concept of `readymades', which simply refers to an object that is `already made' that is presented as an `art work'. 
\begin{quote}
In 1913 I had the happy idea to fasten a bicycle wheel to a kitchen stool and watch it turn. A few months later I bought a cheap reproduction of a winter evening landscape, which I called `Pharmacy' after adding two small dots, one red and one yellow, in the horizon. In New York in 1915 I bought at a hardware store a snow shovel on which I wrote `in advance of the broken arm'. It was around that time that the word `readymade' came to mind to designate this form of manifestation. . . . At another time---wanting to expose the basic antimony between art and readymades---I imagined a `reciprocal readymade': use a Rembrant as ironing board!\footnote{\hyperlink{duchamp}{Duchamp (2009)}, p. 40.}
\end{quote}
This notion implies that the artist does not need to produce an `art work' but only choose a `found object'. The artistic process therefore might not imply the fabrication of an object but the selection of already existing ones---the artist's task is therefore to look at his surroundings and choose an object, and to give it meaning through its re-contextualization. Furthermore, Duchamp's `reciprocal readymade' is the reversal of the former: an existing object considered to be an `art work'  that is given a new context by using it as an `ordinary' object that is normally not regarded to be art. By taking an object that is regarded as a `masterpiece' (a Rembrant) and using it as an object that usually performs a different function and is not appreciated for its aesthetic qualities (the ironing board), the `reciprocal readymade' is a gesture that serves as a criticism to the mystification behind such canonic artworks and at the same time questions the traditional western values of artistic appreciation. Moreover, by taking an `art work' and using it as an object usually not considered to be art, and using an existing `ordinary' object as an `art work', Duchamp contributes to the idea of art being more than just a `specialized' production of highly praised objects. The concept of `readymades' therefore opens up a new debate on \emph{what should the function of the artist be} (if not that of producing objects), and at the same time questions notions of uniqueness and authorship in art. Additionally, within Duchamp's categories there is yet another type of `readymade' which implies an alteration or adjustment by the artist (often very small) of the original `found object'. He calls this type of work `assisted readymade', from which \emph{L.H.O.O.Q.} (1919) is one of his most famous. \emph{L.H.O.O.Q.} consists of a reproduction (a commercial print) of Leonardo's \emph{Mona Lisa} on which Duchamp draws a mustache and goatee in pencil. By doing this simple gesture, Duchamp criticizes established and authoritative notions of western art by banalizing a canonic work that represents symbolically the ideal of beauty. The vulgarization of the image of the \emph{Mona Lisa} in a commercial reproduction rises questions about gender and the value of art, as well as it makes a comment on the reproduction, commercialization and popularization of art works.\footnote{\hyperlink{judovitz}{Judovitz (1995)}, pp. 139-143.}

The concepts of non-assisted (original without modification), assisted and reciprocal `readymades' were later embraced within the notion of \emph{d\'{e}tournement}, first introduced by the avant-garde movement the Situationist International in the late nineteen-fifties.\footnote{See \hyperlink{ford}{Ford (2005)} and \hyperlink{mcdonough}{McDonough (2002)} for more information on the Situationist International (SI).} \emph{D\'{e}tournement} (which means `diversion' as well as it can also mean hijacking, embezzlement and corruption) is an artistic and/or political gesture that consists of the use or hijacking of existing works of art and other cultural forms within the artist's own work.\footnote{\hyperlink{postproduction}{Bourriaud (2005)}, pp. 35-37.} This notion might also imply an intervention, re-contextualization or alteration to be exerted by the artist on the existing works or forms to provoke their depreciation, degeneration and devalorization. Furthermore, Guy Debord argues that \emph{d\'{e}tournement} is ``not an enemy of art. The enemies of art are those who have not wanted to take into account the positive lessons of the `degeneration of art'''.\footnote{\hyperlink{debord2}{Debord (2009)}, p. 66.} Therefore, the violence implied in the gesture of \emph{d\'{e}tournement} is not to oppose or denounce art but to provoke an action that interrupts and displaces an existing (artistic) theory---which reminds us that theories are truly faithful to themselves only if they allow `history's corrective judgment' upon themselves. Therefore, Debord sees appropriation that specifically maintains a certain distance from accepted conventions as a form of improving upon and correcting ideas from the past.
\begin{quote}
The defining characteristic of this use of \emph{d\'{e}tournement} is the necessity for \emph{distance} to be maintained towards whatever has been turned into an official verity. . . . Ideas Improve. The meaning of words has a part in the improvement. Plagiarism is necessary. Progress demands it. Staying close to an author's phrasing, plagiarism exploits his expressions, erases false ideas, replaces them with correct ides.\footnote{\hyperlink{debord}{Debord (1994)}, p. 145.}
\end{quote}
In other words, he claims that through the act of plagiarism one may improve, correct and update ideas and theories (about art), and at the same time in the process of doing so restraining them from their authority and alleged truth. Additionally, Debord argues that \emph{d\'{e}tournement} of existing works may also be used as a correction of the `artistic inversion of life'.\footnote{\hyperlink{debord2}{Debord (2009)}, p. 66.} By appropriating material (images, films, \emph{etc.}) from the `spectacle' that self-evidently signifies a reversal of what happens in real life, the artist may in actuality represent real life, meaning and emotions through \emph{d\'{e}tournement}---consequently rectifying the inversion of reality implicit in the original material. An example of this strategy would be if an artist uses a \emph{d\'{e}tourned} hollywood romantic film---which in its original form does not accurately portray `love' in real life---to express and convey `love' in reality. Furthermore, \emph{d\'{e}tournement} denounces the separation (within the arts) between production and consumption that characterizes the `society of the spectacle', encouraging the appropriation of lived experience over the fabrication of work that contributes to the division between art and the spectator.\footnote{\hyperlink{postproduction}{Bourriaud (2005)}, p. 36.} In addition, it is important to take in consideration that there are different types of \emph{d\'{e}tourned} elements which may affect the strategies of appropriation. In `Directions for the Use of D\'{e}tournement', Debord and Wolman make a difference between two main categories of \emph{d\'{e}tournement}: minor and deceptive \emph{d\'{e}tournement}. Minor \emph{d\'{e}tournement} is the use of elements that have no considerable signification within a given political or social context, they only obtain meaning through their re-contextualization. Deceptive \emph{d\'{e}tournement} on the other hand, relies on elements that have a strong cultural/political meaning that acquires a different dimension of significance from the new context.\footnote{\hyperlink{debord3}{Debord and Wolman (2009)}, p. 35.} In addition, the authors attempt to give an analysis on certain considerations regarding \emph{d\'{e}tournement} that they considered important for their own artistic/political purposes. In their analysis, they also warn of less effective forms as well as negative uses of \emph{d\'{e}tournement}, which in some cases might include the use of \emph{d\'{e}tourned} elements for right-wing propaganda\footnote{Here, one could argue that a recent example of this argument is the use of \emph{d\'{e}tournement} on Barack Obama's photograph (a mustache is superimposed on Obama's image) by the American `Tea-party' movement to suggest that a relationship exists between Obama and Hitler, which at the same time attempts to imply that Obama has `totalitarian impulses' that are tied to his `socialist agenda' of expanding the federal government.} or strategies that instead of condemning the `spectacle', often reassures its hegemony by using these strategies within consumer society and with the purpose of the production of commodities for individual profit and gain.\footnote{Ibid. pp., 36-37.} Additionally, one could argue (in \v{Z}i\v{z}ekian terms) that \emph{d\'{e}tournement} might sometimes imply a distance that is established between the individual and the appropriated elements of the `spectacle' that creates the illusion of a transgressive stance that in reality hides the individual's complicity with the prevailing system of global capitalism.

\subsection {Postproduction and the Digital Age}

During the eighties and nineties, the interest of deliberately using appropriated material as a creative tool reemerged within the visual arts world. The concepts of `readymades' and \emph{d\'{e}tournement} were reconsidered as strategies of appropriation and were expanded by artists who were under the impression that the tactics offered by modernism were approaching a dead-end.

\begin{quote}
Throughout the eighties, the democratization of computers and the appearance of sampling allowed for the emergence of a new cultural configuration, whose figures are the programmer and DJ. The remixer has become more important than the instrumentalist, the rave more exciting than the concert hall. The supremacy of cultures of appropriation and the reprocessing of forms calls for an ethics: to paraphrase Philippe Thomas, artworks belong to everyone. Contemporary art tends to abolish the ownership of forms, or in any case to shake up the old jurisprudence. Are we heading toward a culture that would do away with copyright in favor of a policy allowing free access to works, a sort of blueprint for a communism of forms?\footnote{\hyperlink{postproduction}{Bourriaud (2005)},  p. 35.}
\end{quote}

\subsubsection {Copying and Sharing: Social (Re)appropriation}

- Hackers \\
- Open Source \\

\subsubsection{Liberal-communists} 

criticism  of Bourriaud's `communism of forms' . . .

- Zizek, Bill Gates model \\

\subsection {Postmodernism and Appropriation}

Postmodernism - Institutionalization of Modernism.

Relationship between Postmodern art and appropriation.

Criticisms of Postmodern art practices?



\section{Musical Appropriation}

This is not a survey. For that ``Quotation and Cultural Meaning in Twentieth-Century Music.''\footnote{See \hyperlink{metzer}{Metzer (2003)}.}

\subsection{Postmodern Music}

Start with Berio--- transcriptions, etc. Then to more recent trends...
Criticisms. Do they achieve anything? How to achieve something new through the appropriation of already existing music?

\subsection{Musica Derivata}

``music that is compositionally based on other music'' (K. Barlow) 

The first strategy considered is Clarence Barlow's concept of \emph{Musica Derivata}, which refers to the idea of transforming existing music with Computer Aided Composition (CAC) tools to create ``music that is compositionally based on other music''\footnote{\hyperlink{barlow}{Barlow (2000)}.} This approach seems to take as a starting point mostly notated material (but in some occasions spectral information from recordings) from music by other composers. 

Appropriation of Score Info: Performance practice and other sonic characteristics of many original musical sources is lost in the transcription to a fully notated score for ensembles of western classically trained musicians. Many aspects of sound production (intonation, groove, spectral characteristics of instruments/voices, etc) is lost via this process.

\subsubsection{Spectral Information} 

`Im Januar am Nil'

Spectralism . . .

\subsection{Plunderphonics}

a little history... The Gramophone (ideas by Moholy-Magy\footnote{See \hyperlink{moholy}{Moholy-Nagy (2006)}}), John Cage, Stockhausen, Oswald, Negativeland etc.

John Oswald and Chris Cutler's articles.

\begin{quote}
As a listener my own preference is the option to experiment. My listening system has a mixer instead of a receiver, an infinitely variable speed turntable, filters, reverse capability, and a pair of ears. An active listener might speed up a piece of music in order to perceive more clearly it�s macrostructure, or slow it down to hear articulation and detail more precisely.\footnote{\hyperlink{oswald}{Oswald (1985)}.}
\end{quote}

\subsubsection{Copyrights}

\subsubsection{Sampling Culture}

\subsection{Other Ideas on Musical Appropriation}

\subsubsection{The Impact of Sound Transformations} 					

``With the power of the computer, we can transform sounds in such radical ways that we can no longer assert that the goal sound is related to the source sound merely because we have derived one from the other''. (T. Wishart)

Palette: Recognizable (quotation) - to non-recognizable (``abstract'').

The amount of processing can affect our ability to recognize the source sound or musical sample. Therefore, there is a wide palette of derivative music available to us: from the radically processed � less recognizable source � more `abstract' extreme; to the less processed � more recognizable source � more `referential'  and quotation type music.

\subsubsection{Micro and Macro Plundering}

Microplunderphonics

Plundering just microelements of sound. Not the whole spectrum of the original sound file. 

Generate noise with your plunderphones and use it instead of white noise for sound synthesis

Macroplundering

Appropriate a composition�s form. Use the structure as blueprint for a new composition. 

Use variables of the appropriated piece (pitch, dynamics, etc.) as control structures for new output.

\subsubsection{Using Data and Mapping }

Using plunderphones as data

An example: Use FFT data of your plunderphone to trigger samples of recorded instruments.

Peter Ablinger...

\subsubsection{Sharing Code}

Max patches, Computer Code.

\subsubsection{Real-Time Plunderphonics}

Real-time possibilities of appropriation. Appropriation of Live Performances.

Appropriation of audio signals from live music performances as material for a new composition

Creates a cognitive dissonance between audio and visuals.

The amount of processing of the audio signals is visible. The more processed the performances are, the more contrasting they will look in relationship with what is heard through speakers.

In contrast to acousmatic tradition, Real-Time Plunderphonics makes the process of appropriation transparent to the audience through the cognitive association between audio and visuals.

Changes relationship with the appropriated Other: The performer becomes an accomplice in the process of appropriation (or themselves). 

\subsubsection{Other considerations} 

What? 

Code, compositional tecniques, what piece of music? 
Do we plunder from the ``flea market or (the) airport shopping mall''? (N. Bourriaud). From the top 20 list - J. Oswald approach-, or from the hidden CDs at the back of the music store?
\\
Who?

Music Industry? Pop/commercial? Historical (dead composers)? Music from different cultures? 

Multiculturalism, Globalization., etc.

Appropriation of the Other. What relationship do we want to establish with the Other? Impersonal like the 1st/3rd World relationships?

Liberal multiculturalists approach? ``Other deprived of its Otherness (the idealized Other who dances fascinating dances and has an ecologically sound holistic approach to reality, while features like wife beating remain out of sight�)?'' (Slavoj \v{Z}i\v{z}ek, 2003)
\\
Why?

For the meaning of the cultural object you are appropriating? For it�s symbolism? To suggest a metaphor?

For it�s use? ``Don�t look for the meaning, look for the use'' - L. Wittgenstein - for example for the sonic qualities of the appropriation (intonation, groove, etc.)
\\
How?

\label{ch:approp}