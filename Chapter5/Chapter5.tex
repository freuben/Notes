\hypertarget{chapter5}{}
\chapter{Appropriation as Strategy}

This chapter examines musical practices that take an explicit and formalized approach to the use of appropriation as a conceptual, performative and compositional strategy.
Not a survey...

\section{Ideology, Appropriation and the Past}

Appropriation is certainly nothing new to music or the other arts. Artists have used, borrowed and stolen from the work of others for centuries, either explicitly or without deliberation. The very notion of what it means to create art is implicitly related to the act of appropriating existing forms, actions and thoughts. As with any other type of production, art depends on already existing materials and actions to produce something new. In the case of artistic appropriation, the existing materials used not only refer to the physical materials or objects utilized to create art (the canvas, the strings of the piano, the stage, the artist's own body, \emph{etc.}), but also to its subject matter (already existing shapes, sounds, symbols, movements, \emph{etc.}). Nevertheless, artistic appropriation is not limited only to the act by which artists choose their materials to kick-start the artistic process. The notion of artistic appropriation also includes the process of determining the system of artistic production itself, which is based on a priori forms of knowledge, abstractions, deductions and subjective processes that as a whole constitute a set of creative frameworks and stratergies of production. Put briefly, not only does the artist appropriate existing materials to work with, but the method of working itself. Furthermore, the act of appropriation is not always deliberate. One, as an artist (and therefore as cultural agent), paradoxically might not be aware of all of the sources that one drives inspiration from---one might just causally absorb and appropriate from ones own experience by \emph{living} and dwelling within ones surrounding culture. Consequently, for as long as human beings have been producing art, there has been strategies of artistic appropriation. The notions of creativity and originality in art are therefore not independent from that of appropriation and a debate about art is only futile if one starts from the premise that artists either appropriate or not. A much more fruitful discussion would stem from the presupposition that all artists appropriate---they all use existing methods and materials, they borrow and steal from each other and from other existing cultural and historical modes of production---and the questions that are more relevant are what one as an artist appropriates, who and where one appropriates from, how and why one appropriates and what one accomplishes through appropriation. What the artist does with the appropriated methods and materials and with what purpose is therefore much more relevant to the notions of creativity and originality than questions of authorship, copyright or legitimacy concerning an individual been wholly responsible for the action of producing an `art work'.

What's more, whether deliberate or not, within the act of appropriation lies the essence of the artist's relationship to the past. 

More on Past?

The way in which one as an artist appropriates, what one appropriates and where one appropriates from, reflects a deep connection to ones ideology and the way one relates to past traditions. The relationship between ideology and artistic appropriation at once makes art an ideological discipline---ideology is reflected in the creation and perception of art and different types of ideology may be identified in different types of art---and displays a significant link between artistic and cultural appropriation. Slavoj \v{Z}i\v{z}ek has argued that cultural appropriation and ideology imply an act of violence and friction that is created through the appropriation of past traditions. According to \v{Z}i\v{z}ek, in today's global society, ideology is more relevant than ever before, considering the violence implied in the act of cultural appropriation.
\begin{quote}
The contemporary era constantly proclaims itself as post-ideological, but this denial of ideology only provides the ultimate proof that we are more than ever embedded in ideology. Ideology is always a field of struggle---among other things, the struggle for appropriating past traditions.\footnote{\hyperlink{zizektragedy}{\v{Z}i\v{z}ek (2009)}, `It's Ideology, Stupid!', p. 37.}
\end{quote}
Given the relationship that exists between artistic and cultural appropriation, artistic appropriation may be deliberately used as an ideological tool and as a strategy to convey thoughts, opinions and feelings associated with the struggle of cultural appropriation. 

Past traditions . . .

Moreover, through the intentional and specific use of artistic appropriation as a an act of violence, the artists may denounce 

In today's globalized society . . . Given the fact that 

Appropriated material reflects ideology . . . \v{Z}i\v{z}ek/Adorno but. . . . The artist can present their own view of these references by rearranging them modifying them. The appropriation artist doesn't necessarily adheres to the ideology of the appropriated material, but reflects it by the use of the appropriation - how are they presented, modified, etc?  



\section{Appropriation Art}

Appropriation artists deliberatly/consciously use appropriation - formalization of appropriation. 

\subsection {Readymades and D\'etournement}

- Duchamp \\
- Reciprocal Readymade.\\
- D\'etournement \\
- Guy Debord \\

\subsection {Postproduction and Digitalization}

\begin{quote}
Starting with the language imposed upon us (the \emph{system} of production), we construct our own sentences (\emph{acts} of everyday life), thereby reappropriating for ourselves, through these clandestine microbricolages, the last word in the productive chain. Production thus becomes a lexicon of a practice, which is to say, the intermediary material from which new utterances can be articulated, instead of representing the end result of anything. What matters is what we make of the elements placed at our disposal. . . . By listening to music or reading a book, we produce new material, we become producers. And each day we benefit from more ways in which to organize this production: remote controls, VCRs, computers, MP3s, tools that allow us to select, reconstruct, and edit. Postproduction artists are agents of this evolution, the specialized workers of cultural reappropriation.\footnote{\hyperlink{postproduction}{Bourriaud (2005)}, pp. 24-25.}
\end{quote}

\subsubsection{Digital Technologies}

\begin{quote}
Throughout the eighties, the democratization of computers and the appearance of sampling allowed for the emergence of a new cultural configuration, whose figures are the programmer and DJ. The remixer has become more important than the instrumentalist, the rave more exciting than the concert hall. The supremacy of cultures of appropriation and the reprocessing of forms calls for an ethics: to paraphrase Philippe Thomas, artworks belong to everyone. Contemporary art tends to abolish the ownership of forms, or in any case to shake up the old jurisprudence. Are we heading toward a culture that would do away with copyright in favor of a policy allowing free access to works, a sort of blueprint for a communism of forms?\footnote{Ibid. p. 35.}
\end{quote}

\subsubsection{The liberal-comunists} 

- Zizek, Bill Gates model \\

\subsection {Appropriation of Social Forms}

- Hackers \\
- Open Source \\

\subsection {Postmodernism and Appropriation}

\section{Musical Appropriation}

\subsection{Postmodern Music}

\subsection{Musica Derivata}

\subsection{Plunderphonics}

\subsubsection{Copyrights}

\subsubsection{Sampling Culture}

\subsection{Other Ideas on Musical Appropriation}

\label{ch:approp}