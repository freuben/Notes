\hypertarget{chapter7}{}
\chapter{Compositions}

This chapter aims to briefly describe the submitted work. However, these are not detailed commentaries of the musical output as the previous chapters already serve as a meta-commentary on this work. 

%During the research period, I have engaged in work that is varied in its outcome: it includes results that maybe are difficult to categorize, but which involve a varying amount of pre-compositional work. Some of my creative output therefore may be categorized as compositions, improvisations, performances, studies or spontaneous experiments.

\section{E-tudes}

\emph{E-tudes} is a set of electronic \emph{\'{e}tudes} for six stage pianos, live electronics and mechanical piano.\footnote{In case a mechanical piano is not available, it is possible to use a sampler with piano sounds.} These compositions were written for the ensemble \textbf{piano}\emph{circus}\footnote{See \href{http://www.pianocircus.com/}{\texttt{http://www.pianocircus.com/}}} for a project that became a \mbox{two-year} collaboration and lead to two performances.\footnote{Enterprise 08 Festival, The Space, London, May, 2008, and The Sound Source, Kings Place, London, July, 2009.} What initially attracted me to this ensemble was its very particular instrumentation consisting of six electronic stage pianos. I thought this would be a suitable platform to experiment with the notions of \emph{real-time plunderphonics} and \emph{live musica derivata},\footnote{See \hyperlink{realtimeplunderfuck}{pp. 91--92}.} considering that these instruments are electronic and therefore produce no considerable audible acoustic sound.\footnote{The only acoustic sounds that can be heard are the keyclicks produced by the physical contact with the stage pianos while playing. This noise is slightly audible mostly when there are no sounds playing through the speakers (or they are very quiet).} Like a book of \emph{\'{e}tudes} from the repertoire, \emph{E-tudes} consists of a set of pieces that can be performed together at the same event or individually as separate short pieces. At present time, I have completed four `e-tudes', and as an ongoing project, I will continue adding new pieces to the collection. \emph{E-tudes} is modular in the way in which it can be presented: depending on the set of circumstances for a given event, they can be presented separately or as a whole, either as a concert performance or as an installation with perforative elements. In the installation version, the audience walks into, out of, and around the area surrounding the musicians and has creative control over how they want to experience the performance. By choosing between listening to the speakers in the room or to various headphones that are distributed through the performance space and generate different outputs, each member of the audience fabricates their own version of the piece. Therefore, in the installation version their are various possible outputs generated by the computer from the performance, which the audience can choose from. It is also possible to have a performance were the members of the audience are wearing wireless headphones that can receive multiple channels that are transmitted in the performance space, therefore allowing them to choose which channel they want to listen to.\footnote{This was the case in the performance at Kings Place.}

I use the same configuration for all of the pieces that comprise \emph{E-tudes}: the ensemble of six stage pianos is placed in hexagonal formation and divided into two subgroups. The first subgroup consisting of three pianists are asked to select \emph{\'{e}tudes} from the western piano repertoire at will---they can select the \emph{\'{e}tudes} they prefer to perform (for example, \emph{\'{e}tudes} by Chopin, Debussy or Ligeti, to mention just a few)---and are to play them in their chosen order during the duration of the performance. The second subgroup consisting of the remaining three pianists perform together from \emph{The Sixth Book of Madrigals} by Don Carlo Gesualdo da Venosa (1566-1613). 

The pianists playing the madrigals send Midi information to two laptops that transform the audio signal from the \emph{\'{e}tudes} and schedule the digital signal processing events. The audience is not be able to hear in the room what the pianists are playing as the stage pianos do not produce an acoustic sound. The seventh performer (performing the live-electronic part)\footnote{In the performances of \emph{E-tudes} I performed this part myself.} performs different tasks: at some points s/he speaks the Madrigals' text into a microphone and the spectral information from this signal is be used to process the final audio output and to trigger other sound events, at the same time playing Midi controllers. The live electronic part is not fixed, leaving space for improvisational elements within the human/computer interaction. Finally, through the analysis of all the inputs the computer sends Midi messages to the mechanical piano, adding yet another element to the performance. In the room the final result of the creative process of combining the simultaneous performances in diverse arrangements is diffused through the speakers. In the installation version, the headphones that are spread through the performance space portray the inner life of the performance sounding in the room and reveal the inner layers of computer processing and the appropriated compositions.

Computer programmes play a vital role in all the elements of \emph{E-tudes} and were written in SuperCollider---some of these prgrammes are discussed in \hyperlink{chapter5}{Chapter 5} but some were exclusively written for \emph{E-tudes}.\footnote{The code for these computer programmes can be found at \href{http://github.com/freuben/Etudes}{\texttt {http://github.com/freuben/Etudes}}.} These programmes are used to analyze incoming Midi data to schedule digital signal processing events. The digital signal processing of the live electronics come from two mayor audio sources: the input of the combined live audio of the sound generated by the three pianists playing \emph{\'{e}tudes} and \emph{micro} elements\footnote{See \hyperlink{macroplunder}{pp. 89--90}.} derived from various recordings of existing music which I choose to appropriate. The individual live audio signals coming from each pianist playing \emph{\'{e}tudes} are interpolated with one another (by altering the pitch and volume of the signals).\footnote{Each signal is interpolated with the other by gradually pitch-shifting one signal down four octaves and fading out its volume gradually, while at the same time introducing the next signal which would be pitch-shifted four octaves down and gradually transposing it up until its normal pitch and by gradually fading it in.} The live electronics performer can change the durations of the interpolation between \emph{\'{e}tudes} with a Midi controller. At the same time, the resulting signal is then pitch-shifted again through several pitch ratios generating multiple signals that are then mixed together to create yet another signal. The sounding result of this last signal is a very noisy signal which could be described as `piano noise' (it still retains a piano-like quality). I then utilize this `piano noise' as input in synthesis algorithms which filter it using several techniques. The `piano noise' however is very different to \emph{white noise}, \emph{pink noise} or any other types of noise used in classic synthesis techniques in that its spectral flux is constantly irregular and therefore more unpredictable. Additionally, the live electronics performer can change the sonic qualities of the `piano noise'---and therefore also control up to a certain point the spectral flux---by altering the interpolation time of the live signals coming from the pianists playing \emph{\'{e}tudes}. The the first two `e-tudes' in their final result (the final output diffused through the speakers) are composed exclusively using synthesis algorithms which use this `piano noise' as input. At the same time, in the installation version, the audience can listen through headphones to the different outputs at different degrees of processing---for instance, the output of one of the headphones is made out of material generated from the interpolation of \emph{\'{e}tudes}, while another one might reveal the `piano noise', etc. The original appropriated sources (the \emph{\'{e}tudes} and the madrigals by Gesualdo) are also displayed closer to their original form through certain headphone outputs. 

\subsection*{E-tude I}

\subsection*{E-tude II}

\subsection*{E-tude III}

\subsection*{E-tude IV}

- Computer Applications (partial tracking, midi triggering, generative, etc.)

- Relate it to other chapters (interpassivity)


%	E-tudes questions the traditional role and relationships between performer, composer and listener and gives a unique and innovative approach to the use of “found objects”. The composer in this piece does not communicate with the performers by writing a score or by teaching them the music ‘by ear’ as in previous performance practice conventions. He even lets the performers decide which pieces to play within a given repertoire. Therefore, the creative role of the composer is not to provide the music the performer should play but rather, in Oswaldian terms, to plunder their audio signal. On the other hand, E-tudes differentiates itself from John Oswald’s ‘Plunderphonics’ in that the plundering occurs in a live situation and that makes the performer an accomplice in the process of appropriation (of themselves). In a way, since E-tudes appropriates several live performances simultaneously, it proposes the notion of plundering in real-time, or ‘Real-Time Plunderphonics’. It is therefore important that the event take place in a live situation, as the theatrical effect of being plundered will be evident visually in relationship to the audio. Consequently, the amount of processing of the audio signals will be visible to the audience and the more processed the performances are, the more contrasting they will look in relationship to what is heard through the speakers. In E-tudes, this premise is consciously used to create a narrative that navigates, in literary terms, between the ‘real’ (actual performance) and the ‘surreal’ (more extreme processed audio). In contrast with the acousmatic tradition (music presented through loudspeakers in a fixed medium where the sound sources are not visible), the live performance makes the process of appropriation transparent to its audience as a result of the cognitive association between audio and visuals. In an acousmatic approach, a sound that is radically processed loses its characteristics and therefore the cognitive relationship between source and result may be lost. On the other hand, if the source is exposed visually in a live performance, the audience will have more audio/visual links and one may suppose that the audio processing could be even more extreme without losing the association with the source. 
%	Furthermore, E-tudes’ approach is atypical in relationship to ‘Plunderphonics’ or other music that borrows found material (for example, by musical quotation) in that plundering is not the central purpose of the creative process, but rather a tool for creating a new idiosyncratic audio/visual result. This difference is rather important since it addresses the question inherent in the ambivalence of plundering oneself to create something new as opposed to performing something new in an immediate and direct fashion. Therefore, the idiosyncratic result justifies the conscious participation of the performer in a piece in which what he or she plays is not directly heard by the audience.  This position proposes a new relationship between performer and composer and it also presents a new approach to composition. The composer’s role is not to establish direct communication with the performer (through a score or oral tradition) but rather to use live audio signals of existing music as building blocks to create a new work. All of this is achieved by writing computer software (using SuperCollider 3 – a programming language specialized in audio applications) specifically for the piece. Moreover, E-tudes takes a didactic attitude toward the process of appropriation by giving the listener access to the processed and unprocessed building blocks to show the different layers within the composition, not with the intention of being explicit, but to engage and establish a relationship with the listener. Finally, this composition combines the use of improvisation and generative music to have an unfixed output that changes for each performance of the work. This enables the piece to run in a loop during a long extended time frame without repeating itself. Every time the piece will be played not only will the audiences’ experiences differ, because of their own choices, but also the content of the piece itself will vary.
%	E-tudes takes many elements used before in electronic music and live performance such as improvisation, appropriation, generative music, installation and traditional performance practice, and by combining them points to a development in performing with live electronics. By introducing a dynamic group of live performers and an appealing and interesting visual scenario, this event deals with the problematic of the lack of visual clues and theatrical elements that live electronics performance has faced since its beginning. Hopefully, it will also encourage other creators that deal with live electronics to think seriously about the visual, theatrical and ritualistic aspects of performance. This composition will also contribute to instigating awareness within the contemporary music community on how the presentation of a piece can be as crucial as the sound. It also proposes that the creator is able to innovate by searching for new ways that the audience relates to the work. The event will also contribute to the creative development of the artists because it will give them the opportunity to try out and experiment on the various interactive and performative aspects of the piece and later examine and evaluate how these processes may be improved.

%Didactic attitude towards appropriation: listener will access processed and unprocessed building blocks, not with the intention of being explicit, but to engage and establish a relationship with the audience.

%Relational aspect: it proposes the idea that one may innovate by searching for new ways that the audience relates to the work.

%Elements of improvisation and generative music. Every time the piece will be played not only will the audience’s experience differ, because of their own choices, but also the content of the piece itself will vary.

\section{On Violence}

This composition attempts to explore the aesthetics of violence and reflect on different manifestations of violence. It is also inspired by slovenian philosopher Slavoj Zizek's ideas about violence. Zizek  categorizes violence into two main types: subjective and objective violence. Subjective violence is clearly identifiable by an agent, for example acts of terror or crime, and it is perceived as a clear interruption of the normal state of things. On the other hand, objective violence is  violence that is inherent in the social fabric and it is hard to see and experience for the advantaged classes or countries. What Zizek argues is that objective violence is inherent within the social "balance" and it is objective violence which triggers acts of subjective violence. Furthermore, Zizek identifies two types of objective violence: symbolic and systematic violence. Systematic violence is manifested through our economic and political systems that in order to give the idea of a normal smooth running of things, exert systematic violence on large groups of people. Symbolic violence is related to and included within systematic violence but it is specific to violence expressed through language and other symbolic systems (like music). Zizek goes further to argue that the forms of symbolic violence are actually based on and manifested by the symbolic systems as such. 

\section{\v{Z}i\v{z}ek!?}

Zizek? is a computer-mediated improvisation that gives a live alternative soundtrack to the Zizek! (2005) movie. Each performer has a laptop in front of them. The laptops are connected through a network by which the composer guides the improvisers by sending them written directions, animations (moving graphical notation) and through headphones, an aural score that consists of sound and music derived from the audio of the film. 

Alexander Hawkins (piano), Dominic Lash (double bass), Javier Carmona (drums)

\section{FreuPinta}

%\subsection{Simulation Series}
%\subsection{Occupation Series}
%\subsection{Transgression Series}

\section{Improvisations}

%\subsection{Horatio Oratorio}
%\subsection{Perro-chimp}
%\subsection{Too Hot to Handel}
%\subsection{Mowgli}

\label{ch:compositions}