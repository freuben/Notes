\chapter*{Preface}
\addcontentsline{toc}{chapter}{Preface}

%Give a few sentences in general introduction to why you have undertaken the research that you have � mention your reasons for embarking, your reasons for continuing and your hopes for how successful you have been.
%Thank your supervisor. Think about it, he or she has been your greatest support throughout the course of your thesis and you probably would not even have a thesis if it were not for them: the least they deserve is a �thank you� in black and white!
%Thank your family as without their support the journey would have been tougher.
%Thank your typist, if you used one, and/or those who helped and advised you along the way.

There are various reasons why I think writing about music can be a difficult, if not overwhelming challenge. First, I acknowledge that music can be examined from many different angles and by many disciplines given that music carries within itself a complex content of meaning. Furthermore, I accept that music can be analyzed by the elaborate---social, cultural, political, economic and historical---conditions in which it is created and received. These conditions at the same time are revealed within music and therefore music can also be studied by looking at how its wider context may be traced through what happens within music itself. At the same time, music can be studied as a vital human act that is deep-rooted in our evolutionary past and which performs important functions that are closely related to human behavior and emotions. I am convinced that it is for this reason that music has drawn a lot of interest from the scientific community, which has led to extensive research about the impact of music on human experience: how it affects the human body and the brain. I also recognize that within the act of music-making lies in essence a collective experience in which intricate human relationships are formed. Music can therefore be studied for its relational qualities---the types of relationships that are established between people involved in creating, performing and experiencing music. Furthermore, music is broadly considered as an autonomous discipline---music is studied independently as a subject that has developed its own music theory based on technical and aesthetic considerations (regarding how music is conceived, performed and received) from the musician's perspective. Therefore, music can be examined according to very different criteria: music can be studied for the (social, economic, evolutionary, etc.) functions it performs, for the type of relationships that are formed through music, as well as for the way in which it follows certain considerations and models that are exclusive to music. Moreover, I believe that during the last century it has become apparent that music can also be studied for its potential to inspire new forms of thought in other areas of human endeavor based on the reconfiguration and redefinition of its own considerations and models. I believe this criteria for examining music can be more effectively scrutinized through philosophical inquiry and critical reflection. In my view, writing about music can be a daunting task considering the diverse criteria by which music can be examined, the variety of disciplines that can be involved in studying music and the complexity of meanings that are revealed through music.

For me, this commentary involves an even greater challenge, mainly, writing about my own music. I am aware by been immersed within my own creative process of its highly associative, multilayered and subjective nature.  As a musician that is involved in music practice, I am also extremely aware that music can have multiple meanings---not only the definition of music is contingent to social, historical and philosophical conditions, but also music as it is perceived and experienced can convey a multiplicity of meanings and emotions---a self-evident fact that for me is too difficult to ignore. For that reason, I am not attempting to adopt a reductionist approach (which would focus only on a limited area of musical enquiry through rigorous empirical investigation) to writing this commentary, to avoid the risk of ignoring the intuitive character of my own creative process and the complex nature and meaning of music. Instead, I decided to write in a style that is multidisciplinary and combines \mbox{self-reflective} and speculative arguments with specialized practical and technical information regarding how the submitted music was created and performed. I consider this approach to writing better suited to describe the type of critical reflection that goes on as part of my creative process, which at some points deals exclusively with problems concerned with practical and technical aspects of how music is created and performed, however constantly trying to understand the creative work and the musical result within a larger context. I would like to think that my practice-based work is informed and inspired by wider---cultural, political and philosophical---concerns. At the same time, I also recognize that while attempting to write about these complex issues related to my creative process as well as to the interconnections between the multiple meanings of music, I am running the risk that my argumentation might loose accuracy, methodological rigor and empirical validity and that some of my claims at times may sound overambitious or as sweeping statements. Still I am convinced that this style of writing can tackle at best my aesthetic and musical concerns. Therefore, I am embracing the difficulty that lies in writing about my own music while simultaneously been aware of the impossibility of examining all aspects of my work. However, I also understand the importance of attempting to open up some of the concerns related to my creative work to others. I also believe that the process of writing itself might become self-reflective about my own creative process and might help me to better understand my own musical practice and aesthetic direction. 

Each chapter of this commentary deals with a different overarching theme that pertains (in part or as a whole) to the body of submitted work. I decided to take this approach partly due to the nature of the work submitted, which does not necessarily follow the characteristics of the work expected from a composer---the final output of my work does not involve ordinary scores or music that necessarily corresponds with the traditional notion of a `finished' composition. In my view, much of the submitted work is concerned with reworking certain musical strategies regarding the way in which we create, perform and experience music. I consider that this kind of work needs to embrace an experimental approach to music-making, which in my view should also include documenting failed and unfinished experiments. In my opinion, this type of experimental practice calls for a type of meta-commentary that explains the background and motivation of the submitted work as well as its underlying reasoning, preoccupations and concerns. Therefore, the first four chapters of this commentary serve as a meta-commentary to the submitted work. The first two chapters are introductory and deal with the background and motivation behind the musical output. \hyperlink{chapter1}{\mbox{Chapter 1}} gives a theoretical background based on Jacques Ranci\`{e}re's work on the relationship between aesthetics and politics, which in my view clarifies certain misunderstandings that are usually ascribed to the notions of so called `modernism' and `postmodernism' as it pertains to music. In \hyperlink{chapter2}{Chapter 2}, I explore the motivation behind the submitted work which embraces the notion that today it is possible to assume a positive and uncompromising attitude towards musical experimentation and innovation based on radically reworking the fundamental aspects of music-making. \hyperlink{chapter3}{Chapter 3} evaluates the potential of using recent technological developments to challenge how music is created, performed and experienced. In \hyperlink{chapter4}{Chapter 4}, I consider appropriation as a strategy, first by examining artistic and musical practices that explicitly and deliberately use appropriation and then by contemplating which strategies have the potential to achieve something new through appropriation. The last two chapters, are more descriptive in nature and deal more exclusively with specific technical and practical considerations regarding the submitted work. \hyperlink{chapter5}{Chapter 5} describes the computer applications that were developed as part of the creative work and discusses their implementation. \hyperlink{chapter6}{Chapter 6} aims at describing the submitted portfolio, going through the different musical outputs by briefly discussing them.

I hope that reading this commentary will give an inside into the submitted work and the concerns that surrounds it. I would certainly be pleased if I am able to share with the reader my deep fascination and commitment to music and at the same time explain the main aesthetic problems and questions that I strive to address in my work. I also hope that the reader is engaged with the music and attempts to understand its larger content of meaning, and if this commentary helps to do so in any way, I will be satisfied with its accomplishments.

\label{ch:preface}